%%%%%%%%%%%%%%%%%%%%%%%%%%%%%%%%%%%%%%%%%%%%%%%%%%%%%%%%%%%%%%%%%%%
%                                                                 %
%   MINUIT User Guide -- LaTeX Source                             %
%                                                                 %
%   Chapter 2                                                     %
%                                                                 %
%   The following external EPS files are referenced:              %
%                                                                 %
%   Editor: Michel Goossens / CN-AS                               %
%   Last Mod.:  1 Apr. 1992 10:10 mg                              %
%                                                                 %
%%%%%%%%%%%%%%%%%%%%%%%%%%%%%%%%%%%%%%%%%%%%%%%%%%%%%%%%%%%%%%%%%%%
 
\chapter{Minuit Installation.}

\section{Minuit Releases.}
 
Minuit has been extensively revised in 1989, but the usage is
largely compatible
with that of older versions which have been in use since before 1970.
Users familiar with older releases, who have not yet used releases
from 1989 or later, must however read this manual, in order to
adapt to the few changes as well as to discover the new features
and easier ways of using old features, such as free-field input.

\section{Minuit Versions.}

The program is entirely in standard portable Fortran 77, and requires
no external subroutines except those defined as part of the Fortran 77
standard and one logical function \Rind{INTRAC}
\footnote{\texttt{INTRAC} is available from the CERN Program Library
  for all common computers, and in the worst case can be replaced by a
  \texttt{LOGICAL FUNCTION} returning a value of \texttt{.TRUE.} or
  \texttt{.FALSE.} depending on whether or not Minuit is being used
  interactively.}.  The only difference between versions for different
computers, apart from \Rind{INTRAC}, is the floating point precision
(see heading below).
 
As with previous releases, Minuit does not use a memory manager.
This makes it easy to install and independent of other programs,
but has the disadvantage that both the memory occupation and the
maximum problem size (number of parameters) are fixed at compilation
time.
The old solution to this problem, which consisted of providing
``long'' and ``short'' versions, has proved to be somewhat clumsy and
anyway insufficient for really exceptional users, so it has been
abandoned in favour of a single ``standard'' version.
 
\index{parameters!number of}
\index{version}
\index{variable}
The currently``standard'' version of Minuit
will handle functions of up to 100
parameters, of which not more than 50 can be variable at one time.
Because of the use of the \texttt{PARAMETER} statement in the Fortran source,
redimensioning for larger (or smaller) versions is very easy
(although it will help to have a source code manager or a good editor
to propagate the modified \texttt{PARAMETER} statement through all the
subroutines, and of course it implies recompilation).
The definition of what is ``standard'' may well change in the light of
experience (it was 35 instead of 50 variable parameters for
release 89.05),
and it is likely that different installations will wish
to define it differently according to their own applications.
In any case, the dimensions used at compilation time are printed
in the program header at execution time, and the program is of course
protected against the user trying to define too many parameters.
The user who finds that the version available to him is too small
(or too big) must try to convince his computer manager to change the
installation default or to provide an additional special version,
or else he must obtain the source and recompile his own version.
\section{Interference with Other Packages}
The new Minuit has been designed to interfere as little as possible
with other programs or packages which may be loaded at the same
time. Thus it uses no memory manager or other external subroutines
(except \texttt{LOGICAL FUNCTION INTRAC}), all its own subroutine names
start with the letters \texttt{MN}
(except Minuit and the user written routines),
all \texttt{COMMON} block names start with the characters \texttt{MN7},
and the user should not need to use explicitly any
Minuit \texttt{COMMON} blocks.
 
In addition, more than one different functions can be minimized in the
same execution module, provided the functions have different names,
and provided one minimization and error analysis is completely
finished before the next one begins.

\section{Floating-point Precision}

It is recommended for most applications to use 64-bit floating point
precision, or the nearest equivalent on any particular machine.
This means that the standard Minuit installed on Vax, IBM
and Unix workstations will normally be the \texttt{DOUBLE PRECISION} version, 
while on CDC and Cray it will be \texttt{SINGLE PRECISION}.
 
The arguments of the
user's \Rind{FCN} must of course correspond in type to the declarations
compiled into the Minuit version being used.
The same is true of course for all floating-point arguments
to any Minuit routines called directly by the user in
Fortran-callable mode.
Furthermore, Minuit detects at execution time the precision
with which it was compiled, and expects that the calculations inside
\Rind{FCN} will be performed approximately to the same accuracy.
(This accuracy is called \texttt{EPSMAC} and is printed in the header produced
by Minuit when it begins execution.)
If the user fools Minuit by using a double precision version but
making internal \Rind{FCN} or \Rind{FUTIL}
computations in single precision, Minuit will
interpret roundoff noise as significant and will usually either fail
to find a minimum, or give incorrect values for the parameter errors.
It is therefore recommended, when using
double precision (\texttt{REAL*8}) Minuit, to make sure all computations
in \Rind{FCN} and \Rind{FUTIL} (if used), as well as all subroutines called
by \Rind{FCN} and \Rind{FUTIL}, are \texttt{REAL*8},
by including the appropriate \texttt{IMPLICIT} declarations in \Rind{FCN}
and all user subroutines called by \Rind{FCN}.
If for some reason the computations cannot be done to a precision
comparable with that expected by Minuit, the user {\bf must} inform Minuit
of this situation with the \Cind[SET EPSmachine]{SET EPS} command.
 
Although 64-bit precision is recommended in general,
the new Minuit is so careful to use all available precision that in
many cases, 32 bits will in fact be enough.
It is therefore possible now to envisage in some situations
(for example on microcomputers or when memory is severely limited,
or if 64-bit arithmetic is very slow) the use of Minuit with
32- or 36-bit precision.
With reduced precision, the user may find that certain features
sensitive to first and second differences 
(\Cind{HESse}, \Cind{MINOs}, \Cind{MNContour})
do not work properly, in which case the calculations must be 
performed in higher precision.
