%%%%%%%%%%%%%%%%%%%%%%%%%%%%%%%%%%%%%%%%%%%%%%%%%%%%%%%%%%%%%%%%%%%
%                                                                 %
%   HBOOK User Guide -- LaTeX Source                              %
%                                                                 %
%   This document needs the following external EPS files:         %
%   See the respective source files which are included            %
%                                                                 %
%   Editor: Michel Goossens / CN-AS                               %
%   Last Mod.: 12 Oct  1991   mg                                  %
%                                                                 %
%%%%%%%%%%%%%%%%%%%%%%%%%%%%%%%%%%%%%%%%%%%%%%%%%%%%%%%%%%%%%%%%%%%
\documentstyle[11pt,epsf,longtable,times]{cernman}
 
\setlongtables
\makeatletter
%\long\def\@makecaption#1#2{
% \vskip 10pt
% \setbox\@tempboxa\hbox{#1: #2}
% \ifdim \wd\@tempboxa >\hsize #1: #2\par \else \hbox
%to\hsize{\hfil\box\@tempboxa\hfil}
% \fi}
%       psboxes.sty
%
% This package enables to put a PostScript drawing behind a TeX box.
% The drawing is parametrized by the position and the size of the
% TeX box. To put a gray [rounded] box behind a word use
%
%       \PScommands % Once at the begining
%
%       ... text text \psboxit{25 cartouche}{THE WORD} text text
%       \psboxit{/box 0.5 setgray fill}{\spbox{ANOTHER WORD}}
%       text text ...
%
% WARNINGS : * This was written for dvips translator. You may want to
% change ``ps::'' to ``pstext='' to adapt it to others.
%
%            * If your boxes are ill sized try to change 16384 to
% something else (original code used 65536).
% % 65536 is the internal unit of TeX (scaled point, TeXBook page 57)
% Those macros were adapted from Tom Sheffler (CMU)'s psframe.sty style. In
% particular, the spacebox macro was just copied from his style.
%
%       Je'ro^me MAILLOT, INRIA
%       maillot@bora.inria.fr
%       August 1991
%
 
%%
%% PSBOXIT
%%
%% \psboxit{PS program}{TeX stuff}
%%
%% The bounding box of the TeX stuff is pushed on the PostScript stack
%% and then the program in the first argument is called
%%
%% EXAMPLE: set some text on a gray background, Use the SPBOX macro to
%% give some space around the text.
%%
%%      \psboxit{/box 0.5 setgray fill}{\spbox{Some Text}}
%%
%% See \PScommands for the \box definition
%%
 
\long\def\psboxit#1#2{%
\begingroup\setbox0=\hbox{#2}%
\dimen0=\ht0 \advance\dimen0 by \dp0%
    % Write out the PS code to set the current path using HEIGHT,
    % WIDTH , DEPTH of box0.
    \hbox{%
    \special{ps::gsave currentpoint translate
        0
        \number\dp0 \space 16384 div
        \number\wd0 \space 16384 div
        \number\ht0 \space -16384 div   % Bounding box
%        \number\dp0 \space 65536 div
%        \number\wd0 \space 65536 div
%        \number\ht0 \space -65536 div   % Bounding box
        #1 grestore}%
    \copy0%
}%HBOX
\endgroup%
}%
 
% SPACEBOX
%
% This macro simply takes some TeX stuff, and puts FOUR sides on it
% so that the box is the same size as the thing you'd get with
% an \fbox{} command.  (All I did was modify the code for \fbox{}
% so that all rules were replaced with struts).
%
% USAGE: \spbox{text} is just like \fbox{text} but makes no rules
%
% REASON: so that if using \pspath{...}{\fbox{stuff}}
%         there is a way to get another box the same size:
%         \pspath{...}{\spbox{stuff}}
%
\long\def\spbox#1{\leavevmode\setbox1\hbox{#1}%
    \dimen0\fboxrule \advance\dimen0 \fboxsep%
    \advance\dimen0 \dp1%
    \hbox{\lower \dimen0\hbox%
    {\vbox{\hrule height \fboxrule width 0pt%
          \hbox{\vrule width \fboxrule height 0pt \hskip\fboxsep%
          \vbox{\vskip\fboxsep \box1\vskip\fboxsep}\hskip%
                 \fboxsep\vrule width \fboxrule height 0pt}%
                 \hrule height \fboxrule width 0pt}}}}%
 
%
% A Few PostScript definitions to use with \psboxit
% Call \PScommands once at the begining of your program, this will
% define : box roundedbox rectcartouche and cartouche. They are 4
% PostScript programs. Change the values before setlinewidth end
% setgray to customize your boxes
%
%     Ex : \psboxit{25 cartouche}{blah blah}
%          \psboxit{rectcartouche}{blah blah}
%
\long\def\PScommands{\special{! TeXDict begin
/box{%                  Processes the path of a rectangle.
%                       Needs : x0 y0 x1 y1.
newpath 2 copy moveto 3 copy pop exch lineto 4 copy pop pop
lineto 4 copy exch pop exch pop lineto closepath } bind def
%
%
/roundedbox{%           Processes the path of a rounded rectangle.
%                       Needs : x0 y0 x1 y1 radius.
%       The bounding box is augmented by +/- radius to allow easily to
%       frame several rounded boxes around the same Texture box. Ex:
%  \psboxit{4 copy 15 roundedbox 25 roundedbox} {\spbox{Some Text}}
%       draws two scaled boxes arond the same word. Delete the `radius
%       sub' and `radius add' commands to suppress that enlargement.
%
/radius exch store
3 2 roll %              x0 x1 y1 y0
2 copy min radius sub /miny exch store max radius add /maxy exch store
2 copy min radius sub /minx exch store max radius add /maxx exch store
newpath
minx radius add miny moveto
maxx miny maxx maxy radius arcto
maxx maxy minx maxy radius arcto
minx maxy minx miny radius arcto
minx miny maxx miny radius arcto 16 {pop} repeat
closepath
}bind def
%
%
/rectcartouche{%        Draws a filled and framed box
%                       Needs : x0 y0 x1 y1
4 copy .9 setgray 5 setlinewidth box fill .25 setgray box stroke
}bind def
%
%
/cartouche{%            Draws a filled and framed rounded box
%                       Needs : x0 y0 x1 y1 radius
5 copy .9 setgray 5 setlinewidth roundedbox fill .25 setgray roundedbox stroke
}bind def
%
%
end }%                  Closes dictionnary
}%\newcommand{\action}{\Action}%
\renewcommand{\ttsc}{\tt\bf}

%%%%%%%%%%%% Command \Sbox     %%%%%%%%%%%%%%%%%%%%%%%%%%%%%%%%%%%%%

\renewcommand{\Sbox}[2]% #1 abbreviation #2 contents
{\par\leavevmode\vspace*{\baselineskip}
 \setbox\@tempboxa\hbox{\rule[-1ex]{0mm}{3ex}\quad{\ttsc#2}\quad}        
 \ifdim \wd\@tempboxa<\hsize \psboxit{box 0.85 setgray fill}{\the\wd\@tempboxa\box\@tempboxa}
 \else \psboxit{box 0.85 setgray fill}{\hbox to \hsize{\box\@tempboxa\hfil}}
 \fi
\par\label{#1}\index{{\em #1}}\Rkeep{#1}{#2}%
}% ***** end of \newcommand{\Sbox}
\newbox\mybox

%%%%%%%%%%%% Command \SBbox    %%%%%%%%%%%%%%%%%%%%%%%%%%%%%%%%%%%%%

\newcommand{\SBbox}[3]% #1 abbreviation #2 contents #3 external text
{\par\leavevmode\vspace*{\baselineskip}
 \setbox\@tempboxa\hbox{\rule[-1ex]{0mm}{3ex}\quad\ttsc#3\quad}        
 \begingroup\leftskip-\wd\@tempboxa% for leftskip
 \setbox\@tempboxa\hbox{\rule[-1ex]{0mm}{3ex}\quad\ttsc#3\qquad#2\quad}        
 \ifdim \wd\@tempboxa<\hsize \psboxit{box 0.85 setgray fill}{\box\@tempboxa}
 \else \psboxit{box 0.85 setgray fill}{\hbox to \hsize{\box\@tempboxa\hfil}}
 \fi
\label{#1}\index{{\em #1}}\Rkeep{#1}{#2}\par\endgroup% for leftskip
}% ***** end of \newcommand{\Sbox}

%%%%%%%%%%%% Command \SHbox    %%%%%%%%%%%%%%%%%%%%%%%%%%%%%%%%%%%%%

\newcommand{\SHbox}[2]% #1 abbreviation #2 contents 
{\par\@hangfrom{\bf CALL #1\quad}%
{\tt\hyphenchar\the\font=\defaulthyphenchar\interlinepenalty \@M #2\par
\hyphenchar\the\font=-1}
\label{#1}\index{{\em #1}}\Rkeep{#1}{#2}\par
}% ***** end of \newcommand{\Sbox}

%%%%%%%%%%%% Command \Sboxii   %%%%%%%%%%%%%%%%%%%%%%%%%%%%%%%%%%%%%

% Definition of two routines in same box separated by ``and''
\renewcommand{\Sboxii}[4]% #1 abb. 1 #2 abb. 2 #3 contents 1 #4 cont. 2
{\par\leavevmode\vspace*{\baselineskip}
 \setbox\@tempboxa\hbox{%
        \rule[-1ex]{0mm}{3ex}\quad{\ttsc\lowercase{#3}}\quad%
                    {\rm and}\quad{\ttsc\lowercase{#4}}\quad}
 \ifdim \wd\@tempboxa <\hsize \psboxit{box 0.85 setgray fill}{\box\@tempboxa}
 \else \psboxit{box 0.85 setgray fill}{\hbox to \hsize{\box\@tempboxa\hfil}}
 \fi
\par\label{#1}\index{{\em #1}}\label{#2}\index{#2!defined}%
\Rkeep{#1}{#3}\Rkeep{#2}{#4}%
}% ***** end of \newcommand{\Sboxii}
 
%%%%%%%%%%%% Command \Sboxz    %%%%%%%%%%%%%%%%%%%%%%%%%%%%%%%%%%%%%

\renewcommand{\Sboxz}[2]% #1 abbreviation #2 contents(no definition)
{\par\leavevmode\vspace*{\baselineskip}
 \setbox\@tempboxa\hbox{\rule[-1ex]{0mm}{3ex}\quad{\ttsc#2}\quad}        
 \ifdim \wd\@tempboxa<\hsize \psboxit{box 0.85 setgray fill}{\box\@tempboxa}
 \else \psboxit{box 0.85 setgray fill}{\hbox to \hsize{\box\@tempboxa\hfil}}
 \fi
\par\index{#1}
}% ***** end of \newcommand{\Sbox}
 
%%%%%%%%%%%% Environment DLtt   %%%%%%%%%%%%%%%%%%%%%%%%%%%%%%%%%%%%%

\renewenvironment{DLtt}[1]{% The parameter is the width of the term
\begin{list}{\phantom{\ttsc#1}}%
   {\settowidth{\labelwidth}{\ttsc#1\quad}% define global width
    \setlength{\leftmargin}{\labelwidth}% set global width
    \setlength{\itemindent}{0pt}% set global width
    \setlength{\labelsep}{0pt}% horizontal separation between term and item
    \setlength{\itemsep}{1pt}% vertical separation between two items
    \setlength{\parsep}{0pt}% vertical separation two paragraphs in an item
    \setlength{\topsep}{.2\baselineskip}% vertical separation text/list
    \renewcommand{\makelabel}[1]{\ttsc##1\hfil}}}% get parameter from item
{\end{list}}% ***** end of environment{DLtt}

%%%%%%%%%%%%%%%%%%%%%%%%%%%%%%%%%%%%%%%%%%%%%%%%%%%%%%%%%%%%%%%%%%%%%
\makeatother

\begin{document}
\PScommands
%*******************************************************************
%* HBOOK  Reference Guide - Chapter 2: Fundamentals                *
%*  Last Mod.   12 Oct 1991  18.30  mg                             *
%*******************************************************************
 
\chapter{\label{HFUNDAMS}Fundamentals}
%  ==================================================
\section{\label{HBOOKING}Booking}
\Sbox{HBOOK1}{CALL HBOOK1 (ID,CHTITL,NX,XMI,XMA,VMX)}
\index{one-dimensional histogram}
\index{histogram identifier}
\index{histrogram title}
\index{VMX}
\Action Books a one-dimensional histogram.
\Pdesc\begin{DLtt}{1234567}
\item[ID] histogram identifier, integer non zero
\item[CHTITL] histogram title (character variable or constant up to 80
characters)
\item[NX] number of channels
\item[XMI] lower edge of first channel
\item[XMA] upper edge of last channel
\item[VMX] upper limit of single channel content.(see below)\\
{\tt VMX=0} means 1 word per channel.
\end{DLtt}
\SBbox{HBOOK1}{CALL HBOOK1 (ID,CHTITL,NX,XMI,XMA,VMX)}{G}
{\bf Special values:}
\begin{UL}
\item If \ttbf{XMA<XMI}  then the  origin and binwidth are
calculated automatically, and one word is reserved per channel.
\item Zero (0) is an illegal histogram identifier
\item If the histogram ID already exists it will be deleted and
recreated with the new specifications. A warning message is printed.
\item \ttbf{VMX} is used to compute the number
of bits to be allocated per histogram channel.
If \ttbf{VMX<1} then one
full word is reserved per channel.
When filling a histogram with weights the weight is
truncated to the nearest integer unless one full word is
reserved per channel (i.e. \ttbf{VMX = 0}).
Filling with negative weights will give meaningless results
unless one word per channel has been allocated.\\
Automatic calculation of limits forces one word per channel.
\item
The title is a character variable or constant of up to 80 characters.
The \$ character required by the previous versions of \ttbf{HBOOK}
at the end of the title is not necessary. If it is specified, it
will not be printed.
\end{UL}
\Sbox{HBOOK2}{CALL HBOOK2 (ID,CHTITL,NX,XMI,XMA,NY,YMI,YMA,VMX)}
\index{two-dimensional histogram}
\Action Books a two-dimensional histogram.
\Pdesc\begin{DLtt}{1234567}
\item[ID] histogram identifier, integer
\item[CHTITL]histogram title  (character variable or constant up to 80
characters)
\item[NX] number of channels in X
\item[XMI] lower edge of first X channel
\item[XMA] upper edge of last X channel
\item[NY] number of channels in Y
\item[YMI] lower edge of first Y channel
\item[YMA] upper edge of last Y channel
\item[VMX] maximum population to store in 1 cell.
\end{DLtt}
\Remark
\begin{UL}
\item Similar to HBOOK1, apart from automatic binning.
\item By default, a 2-dimensional histogram will be
printed as a scatterplot
\item If the option \ttbf{TABL} is selected via
\ttbf{CALL~HIDOPT(ID,'TABL')}\Iind{TABL}
the 2-dimensional histogram will be printed as a table.
\item When editing the table, the number of columns \ttbf{NCOL} used to
write the content of one cell depends on the value of \ttbf{VMX}
as follows \ttbf{NCOL = ALOG10(VMX) + 2}.
When \ttbf{VMX} is zero,
the contents is printed in
10 columns in floating point format (including sign). If
necessary, all contents are multiplied by a power of 10,
this number being reported at the bottom of the table.
\end{UL}
\Sboxz{HBOOKN}{CALL HBOOKN (ID,CHTITL,NDIM,CHRZPA,NPRIME,TAGS)}
\index{Ntuple}
\Action Book an Ntuple. See chapter \ref{HNTUPLE} on Ntuples.
%  ==============================================
\section{\label{HFILLSEC}Filling}
\Sbox{HFILL}{CALL HFILL (ID,X,Y,WEIGHT)}
\index{weight}
\Action Fills a 1-dimensional or a 2-dimensional histogram.
The
channel which contains the value X and for two-dimensionals the cell that
contains the point (X,Y), gets its contents increased by
{\tt\bf WEIGHT}.
All booked projections, slices, bands, are filled as well.
\Pdesc\begin{DLtt}{1234567}
\item[ID] histogram identifier
\item[X] value of the abscissa
\item[Y] value of the ordinate
\item[WEIGHT] event weight (positive or negative)
\end{DLtt}
\Remark
\begin{UL}
\item If one full word per channel is reserved at booking time,
{\tt\bf WEIGHT} is taken
with its floating point value.
In case of packing (i.e. more than one channel
per word), \ttbf{WEIGHT} must be
positive and will be truncated to the nearest integer
(\ttbf{WEIGHT<0} will give meaningless results)
\item see \ref{HFILLSEC} for alternative filling entries.
\end{UL}
 
\Sboxz{HFN}{CALL HFN (ID,X)}
 
\index{Ntuple}
\Action
Fill an Ntuple. See chapter \ref{HNTUPLE} on Ntuples.
 
\section{\label{HEDITSEC}Editing}
 
\Sboxii{HISTDO}{HISTGG}{CALL HISTDO}{CALL HISTGG}
\index{print}
 
\Action
Outputs all booked histograms to the line printer.
An index is printed at
the beginning specifying the characteristics and storage use of each
histogram.
 
\Remark
\begin{UL}
\item
If a histogram is empty, a message declares this condition, and the
histogram is not printed.
\end{UL}
\Sbox{HBOOKV}{CALL HBOOKD (ID,CHTITL,NDIM,CHRZPA,NPRIME,TAGS
                            ID,CHTITL,NDIM,CHRZPA,NPRIME,TAGS)}
\SHbox{HBOOKV}{ID, CHTITL, NDIM, CHRZPA, NPRIME, TAGS,
                            ID, CHTITL, NDIM, CHRZPA, NPRIME, TAGS,
                            ID, CHTITL, NDIM, CHRZPA, NPRIME, TAGS}
\SHbox{HBOOK5}{ID, CHTITL, NX, XMI, XMA, VMX}
 

\end{document}
