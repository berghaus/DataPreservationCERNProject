%%%%%%%%%%%%%%%%%%%%%%%%%%%%%%%%%%%%%%%%%%%%%%%%%%%%%%%%%%%%%%%%%%%
%                                                                 %
%   ZEBRA RZ - Reference Manual -- LaTeX Source                   %
%                                                                 %
%   Appendices - System dependencies                              %
%              - Technical details                                %
%                                                                 %
%   Original: Michel Goossens (from SGML source)                  %
%   Additions: Jamie Shiers                                       %
%                                                                 %
%   Last Mod.:  3 June 1993 18:00 mg                              %
%                                                                 %
%%%%%%%%%%%%%%%%%%%%%%%%%%%%%%%%%%%%%%%%%%%%%%%%%%%%%%%%%%%%%%%%%%%

\Filename{H1rzappen-system-specific-details}
\chapter{System specific details}

\Filename{H2rzappen-VMCMS}
\section{VM/CMS systems}
\index{VM/CMS}

On VM/CMS systems, RZ files should always have mode 6, i.e. ``update
in place''. In addition, if a file is to be written by one user
and simultaneously read by another, the {\tt'F'} option of \Rind{RZMAKE} is 
recommended. If this is not used, users reading the file will
not see consistant information if the file is extended, unless
the file is first closed and they reaccess the appropriate disk.
\par
The \Rind{RZOPEN} routine should always be used to open an RZ file
on VM/CMS systems, to ensure that more than the VS Fortran default
of 50 records can be read and written.

\Filename{H2rzappen-VAXVMS}
\section{VAX/VMS systems}
\index{VAX/VMS}

Shared access (in the VAX Fortran sense of \Lit{OPEN(LUN,SHARED,..)})
can give rise to excessive RMS lock traffic in VAXcluster systems.
If shared read access with concurrent write is required, you
might wish to consider the use of a read-only copy of the
RZ file, updated at regular intervals from the ``master'' file,
to avoid such problems.

\subsection{NFS access}
\index{NFS}

Files residing on non-VMS systems that are accessed via NFS~\cite{bib-NFS}
should be processed using C I/O (option {\tt'C'} in \Rind{RZOPEN}).
This is because such files are seen as \Lit{STREAM\_LF} to VAX Fortran
and cannot be read with Fortran direct access.

Note that files created on most Unix systems are automatically
in exchange mode, as exchange mode corresponds to native on
such systems (IEEE floating point format). When reading these
files, the option 'X' must be given on both the \Rind{RZOPEN}
and \Rind{RZFILE} calls, as the file itself is not 'marked'
as exchange mode.

\newpage
\Filename{H2rzappen-MVS}
\section{MVS systems}
\index{MVS}

To enable shared access to an RZ (or any Fortran direct access) file
on systems running MVS, the file should be created as a VSAM
file using the following procedure.

\begin{XMPt}{Creating an RZ file on MVS systems}
         //CRVSAM    JOB  ,CLASS=E,TIME=(0,01),NOTIFY=F99ABC
         /*JOBPARM LINES=2
         //A EXEC PGM=IDCAMS,REGION=512K
         //SYSPRINT DD SYSOUT=*
         //SYSIN DD *
          DELETE                  /* delete old file, if exists */ -
          (F99ABC.KEYEDACC.FILE) -
          PRG    CLUSTER
 
          DEFINE CLUSTER          /* create new keyed access file */ -
          (NAME(F99ABC.KEYEDACC.FILE) -
          VOLUMES(STR012) -
          INDEXED -
          RECORDS(10 10)          /* primary secondary_space (in records) */ -
          KEYS(20 20)             /* length and offset (in bytes) */ -
           SHAREOPTIONS(4 3)       /* multi read + multi write */ -
          SPEED -
          SPANNED -
          REUSE -
          RECORDSIZE(23400 23400))      /* average maximum (in bytes) */ -
          DATA -
          (NAME(F99ABC.KEYEDACC.FILE.DATA)) -
          INDEX -
          (NAME(F99ABC.KEYEDACC.FILE.INDEX))
\end{XMPt}

\Filename{H2rzappen-automatic-determination-record-length}
\section{Automatic record length determination}

The routine \Rind{RZOPEN} can automatically determine the 
record length of existing RZ files. This is triggered
by specifying a value of zero for the parameter \Rarg{LRECL}
on input. The record length is determined as follows:

\begin{UL}
\item IBM systems (Fortran I/O only). 
      The file is first opened for sequential
      access and a Fortran unformatted read is issued,
      using the IBM extension NUM=nbytes, e.g.
\begin{XMP}
            READ(UNIT=LUNIT,NUM=LRECL) ITEST
\end{XMP}
      The file is then closed and reopened for direct-access
      I/O.
\item VAX/VMS and Apollo systems (SR9) (Fortran I/O only). 
      The file is first
      opened for sequential access and a Fortran inquire
      statment is issued, e.g.
\begin{XMP}
            INQUIRE(UNIT=LUNIT,RECL=LRECL)
\end{XMP}
      The file is then closed and reopened for direct-access
      I/O.
\item All other systems plus VAX/VMS systems using C I/O.
      The record length is determined from the data in the file
      itself. For this reason, the RZ package must know if the
      file is in native or exchange format.
      For this reason, the option \Lit{'X'} is recommended
      when processing exchange format files.
\end{UL}

\Filename{H1rzappen-RZ-technical-details}
\chapter{Technical details of the ZEBRA RZ system}

\Filename{H2rzappen-RZ-IO}
\section{RZ I/O}

The RZ package uses Fortran I/O, unless the option 'C' is specified
on the calls to \Rind{RZOPEN}, \Rind{RZMAKE} and \Rind{RZFILE}.
If C I/O is used, the C file pointer is {\bf returned} by the
routine \Rind{RZOPEN}. Thus the LUN parameter must not be a constant.

On most machines, the file pointer returned by the C library
is a small positive integer. To avoid possible conflicts with
Fortran logical units, the RZ package adds 1000 to such
pointers. This is handled internally by the RZ package and has
no impact on the user.

\Filename{H2rzappen-RZ-link-area}
\section{RZ link area}

A permanent reference link area, \Lit{RZCL}, is automatically created upon the first
call to \Rind{RZFILE} or \Rind{RZMAKE}. This link area is defined
in the patchy sequence \Lit{P=QCDE,D=QCDE,Z=RCL}.
This link area contains the following definitions:
\begin{DLtt}{123456}
\item[LTOP]Address of control bank for current RZ file
\item[LRZ0]Address of bank for memory files
\item[LCDIR]Address of directory structure for current working directory
\item[LRIN]Last record read 
\item[LROUT]Last record written
\item[LFREE]Pointer to list of free records
\item[LUSED]Pointer to list of used records
\item[LPURG]Pointer to list of purged records
\item[LTEMP]
\item[LCORD]Pointer to list of ordered cycles
\item[LFROM]Pointer to copied directory
\end{DLtt}
\Lit{LQRS} (system link 7) points to a linear chain containing
the control banks for all RZ files currently open.
A bank for a specific Fortran logical unit or C file pointer 
can be located using the following code.
(The numeric bank identifier is set to the Fortran logical
unit or C file pointer). 
\newpage
\begin{XMPt}{Locating the RZ control bank by logical unit}
*
*     Any RZ files open?
*
      IF(LQRS.EQ.0)GO TO 99
*
*     Yes, loop over linear chain looking for the bank
*     for logical unit LUN
*
      LRZ=LQRS
  10  IF(LRZ.EQ.0)GO TO 99
      IF(IQ(KQSP+LRZ-5).NE.LUN)THEN
         LRZ=LQ(KQSP+LRZ)
         GO TO 10
      ENDIF
*
*     LRZ now points to the control bank for LUN
*     LRECL is equal to # of data words of bank
*
      LRECL = IQ(KQSP+LRZ-1)
*
...
  99  CONTINUE
\end{XMPt}
\newpage

\Filename{H2rzappen-RZ-control-bank}
\section{Structure of the RZ control bank}

The variable \Lit{LTOP} points to the control bank
for the current RZ file, e.g. the one corresponding
to the current working directory. 
%\begin{XMPt}{Structure of LTOP bank}
%       - STATUS word description
%           - bit 1 =1  if no authorisation to modify directory
%           - bit 2 =1  if directory has been modified
%           - bit 3 =1  if file locked by 'RZFILE'
%           - bit 4 =1  if ORGANIZATION='RELATIVE' on VAX
%           - bit 5 =1  if C file access routine used
%           - bit 11-17 LOG level (default taken from MZ)
% 
%         LOGLV = JBYT(IQ(KQSP+LRZ),11,7)-10
% 
%       - LTOP description
% 
%         *********************************************************
%         * -10*       *  Free reference link                     *
%         * -9 *       *  Free                                    *
%         * -8 * LCORD *  Pointer to ordered cycles (RZCOPY)      *
%         * -7 * LRIN  *  Pointer to input buffer                 *
%         * -6 * LROUT *  Pointer to output buffer                *
%         * -5 * LPURG *  Pointer to list of purged records       *
%         * -4 * LFROM *  Pointer to copied directory             *
%         * -3 * LUSED *  pointer to list of used records         *
%         * -2 * LFREE *  Pointer to list of free records         *
%         * -1 * LSDIR *  Pointer to first subdirectory           *
%  LTOP==>*IIII*       *  Status word                             *
%         * +1 *       *  directory structure (see below)         *
%         * .. *       *                                          *
%         *LREC*       *                                          *
%         *********************************************************
%\end{XMPt}
%\newpage

The first data word of this control bank can be accessed by
\Lit{IQ(KQSP+LTOP+1)}, where 
\Lit{KQSP} is the offset of the primary store from \Lit{LOCF(LQ(0))}.

The record numbers for each subdirectory of a given
directory are in a separate structure. The subdirectory
structure of the top directory is reached by \Lit{IQ(KQSP+LTOP+KLS)}.

For the top directory, \Lit{IQ(KQSP+LTOP+KLB)}
points to a file descriptor structure,

\begin{XMPt}{RZ Directory structure (DZDOC format)}
\label{xmp:rztop}
 Bank IDH  RZ       RZ system top bank                                          
 Bank IDN           Numeric bank identifier
                    Fortran Unit
                    If CACCESS set: 1000 + file pointer
                    If HACCESS set: address of user routine
 Author             R.Brun                                                      
 Store     0                                                                    
 Division  SYSTEM                                                               
 NL              10                                                             
 NS               9                                                             
 ND        variable                                                             
 Next      RZ0                                                                  
 Up        NONE                                                                 
 IO-Charac          I                                                           
              ---------- Description of the links  ----------
 1         LSDIR    Pointer to first subdirectory                               
 2         LFREE    Pointer to list of free records                             
 3         LUSED    pointer to list of used records                             
 4         LFROM    Pointer to copied directory                                 
 5         LPURG    Pointer to list of purged records                           
 6         LROUT    Pointer to output buffer                                    
 7         LRIN     Pointer to input buffer                                     
 8         LCORD    Pointer to ordered cycles (RZCOPY)                          
 9         LSNUSED  Free                                                        
              ---------- Description of the Reference links  ----------
 1         LRNUSED  Free reference link *                                       
              ---------- Description of the status bits ----------
 1         NOAUTH   no authorisation to modify directory                        
 2         MODIFIED directory has been modified                                 
 3         LOCKED   file locked by 'RZFILE'                                     
 4         ORGREL   ORGANIZATION='RELATIVE' on VAX                              
 5         CACCESS  C file access routine used                                  
 6         HACCESS  Hook to user's I/O routine
 7-13      USERLUN  User unit number
 14                 Reserved for future use
 15-17     LOGLEVEL LOG-level+3  (default taken from MZ)                     
              ---------- Description of the data words   ----------
 1         Z:IDNAME Directory name (up to 16 characters)                        
 2         Z:IDNAME "                                                           
 3         Z:IDNAME "                                                           
 4         Z:IDNAME "                                                           
 5         RECPT1   Record number of the mother directory,                      
 6         RECPT2   or C file pointer (words 5 and 6)                           
 7         B:IWPW1  Write password (1st part)                                   
 8         B:IWPW2  (2nd part)                                                  
 9         NCHDRW   No. of char. DIR(1:5),WPW(6:10), and bit 12 eX mode         
 10        D:IDATEC Creation date/time                                          
 11        D:IDATEM Last mod date/time                                          
 12        NQUOTA   Maximum number of records QUOTA                             
 13        N:NRUSED Number of used records                                      
 14        NWUSED   Number of words used MOD 1000000                            
 15        NMEGA    Number of megawords used                                    
 16        RESERVED Reserved                                                    
 17        IRIN     Record number currently in LRIN                             
 18        IROUT    Record number currently in LROUT                            
 19        IRLOUT   Number of the last record written                           
 20        IP1      Pointer to first free word in IRLOUT                        
 21        ICONT    Record number continuation                                  
 22        NFREE    Number of words free in F                                   
 23        N:NSD    Number of subdirectories                                    
 24        P:LD     Pointer to directory records                                
 25        P:LB     Pointer to file descriptor (only for TOP)                   
 26        P:LS     Pointer to first subdirectory S                             
 27        P:LK     Pointer to first KEY K                                      
 28        P:LF     Pointer to free space F                                     
 29        LC       Pointer to last cycle C                                     
 30        LE       Pointer to end of directory                                 
 31        N:NKEYS  Number of keys in that directory                            
 32        N:NWKEY  Number of elements in one key                               
 --REP level=1  N:NWKEY + 9 / 10                                                
     1       B:KDES   KEYS descriptor (3 bits per el. ,10 keys per word)        
 --REP level=1 -- End --                                                        
 --REP level=1  N:NWKEY                                                         
     1       Z:TAG1   First part of CHTAG(1) 4 characters                       
     2       Z:TAG2   Second part                                               
 --REP level=1 -- End --                                                        
 1         L:LD     Directory records structure                                 
 1         N:NRD    Number of records to describe this dir.                     
 --REP level=1  N:NRD                                                           
     1       IREC(I)  Record number I of directory                              
 --REP level=1 -- End --                                                        
 1         L:LB     file descriptor (only for TOP)                              
 1         N:NWREC  Number of words for bitmap descriptor                       
 2         LREC     Physical record length (in words)                           
 3         D:IDATE  Creation date of the file                                   
 --REP level=1  N:NWREC                                                         
     1       B:BITMAP 1 bit per record on the file                              
 --REP level=1 -- End --                                                        
 1         L:LS     Subdirectory structure                                      
 --REP level=1  N:NSD                                                           
     1       Z:NAM1   Name of subdirectory                                      
     2       Z:NAM2   "                                                         
     3       Z:NAM3   "                                                         
     4       Z:NAM4   "                                                         
     5       NCHSD    Number of characters in subdirectory name                 
     6       IRECSD   Record number of this subdirectory                        
     7       D:IDTIME Date and Time of creation of subdirectory                 
 --REP level=1 -- End --                                                        
 1         L:LK     KEYS structure                                              
 --REP level=1  N:NKEYS ! I=1,NKEYS                                             
     1       P:LCYC   Pointer to highest cycle in C for key I                   
   --REP level=2  N:NWKEYS! K=1,NWKEYS                                          
         1     KEYS(I,K Element K of key I                                      
   --REP level=2 -- End --                                                      
 --REP level=1 -- End --                                                        
 1         L:LF     Start of free space                                         
 --REP level=1  NFREE                                                           
     1       EMPTY    Free space                                                
 --REP level=1 -- End --                                                        


For RZ version 0:

 1         L:LCYC   Cycles structure                                            
 1         PTOCYCLE                                                             
 1         P:BI0015 LCYC Pointer to prev cycle of KEY (0 if no)                 
 1         P:BI1631 SECREC Second record (if there) (bits 17 to 32)             
 2         D:CREATD Creation date/time relative to 1986 (bits 9 TO 32)          
 2         BITVAL04 RZKEEP (bit 5)                                              
 2         BITVAL03 Append mode (bit 4)                                         
 2         BITS0002 Vector format (if RZVOUT) (bits 1 to 3)                     
 3         PTODATA  Pointer to the data                                         
 3         BITS1631 Record number where data str. starts (bits 17 to 32)        
 3         BITS0015 Offset in record (bits 1 to 16)                             
 4         CYWORD4                                                              
 4         BITS0019 Number of words in data structure (bits 1 to 20)            
 4         BITS2031 Cycle number (bits 21 to 32)                                


For RZ version 1:


 1         L:LCYC   Cycles structure                                            
 1         PTOCYCLE LCYC Pointer to prev cycle of KEY (0 if no)                 
 2         D:CREATD Creation date/time relative to 1986 (bits 9 TO 32)          
 2         BITVAL04 RZKEEP (bit 5)                                              
 2         BITVAL03 Append mode (bit 4)                                         
 2         BITS0002 Vector format (if RZVOUT) (bits 1 to 3)                     
 3         PTODATA  Pointer to the data                                         
 3         PTODATA  Record number where data str. starts 
 4         OFFSET   (20 bits gives a long record!)
 4         BITS0015 Offset in record (bits 1 to 20)                             
 4         BITS2031 Cycle number (bits 21 to 32)                                
 5         NUMWORD  Number of words in data structure
 6         PTODATA  Pointer to the data                                         
 6         PTODATA  continuation record number
 7         KEYNUMB  Key number to which the cycle belongs


\end{XMPt}

%         *******************************************************************
%         * WORD  *   TAG   *                 CONTENT                       *
%         *******************************************************************
%         *  1-4  * IDNAME  *      Directory name  (up to 16 characters)    *
% KUP     *   5   *         *      Record number of the mother directory,   *
%         *   6   *         *      or C file pointer (words 5 and 6)        *
% KPW1    *   7   *  IWPW1  *      Write password (1st part)                *
%         *   8   *  IWPW2  *                     (2nd part)                *
%         *   9   * NCHDRW  *      No. of char. DIR(1:5),WPW(6:10), and     *
%         *       *         *      bit 12 eXchange mode                     *
% KDATEC  *  10   *  IDATEC *      Creation date/time                       *
% KDATEM  *  11   *  IDATEM *      Last mod date/time                       *
% KQUOTA  *  12   * NQUOTA  *      Maximum number of records QUOTA          *
% KRUSED  *  13   * NRUSED  *      Number of used records                   *
% KWUSED  *  14   * NWUSED  *      Number of words used MOD 1000000         *
% KMEGA   *  15   * NMEGA   *      Number of megawords used                 *
%         *  16   *         *               Reserved                        *
% KIRIN   *  17   *   IRIN  *      Record number currently in LRIN          *
% KIROUT  *  18   *   IROUT *      Record number currently in LROUT         *
% KRLOUT  *  19   *   IRLOUT*      Number of the last record written        *
% KIP1    *  20   *   IP1   *      Pointer to first free word in IRLOUT     *
%         *  21   *   ICONT *      Record number continuation               *
% KNFREE  *  22   *   NFREE *      Number of words free in F                *
% KNSD    *  23   *   NSD   *      Number of subdirectories                 *
% KLD     *  24   *   LD    *      Pointer to directory records             *
% KLB     *  25   *   LB    *      Pointer to file descriptor (only for TOP)*
% KLS     *  26   *   LS    *      Pointer to first subdirectory S          *
% KLK     *  27   *   LK    *      Pointer to first KEY   K                 *
% KLF     *  28   *   LF    *      Pointer to free space  F                 *
% KLC     *  29   *   LC    *      Pointer to last cycle  C                 *
% KLE     *  30   *   LE    *      Pointer to end of directory              *
% KNKEYS  *  31   *   NKEYS *      Number of keys in that directory         *
% KNWKEY  *  32   *   NWKEY *      Number of elements in one key            *
% KKDES   *  33   *   KDES  *      KEYS descriptor (3 bits per element)     *
%         *       *   ...   *      10 keys per word                         *
% KTAGS   *  34   *   TAG11 *      First part of CHTAG(1) 4 characters      *
%         *  35   *   TAG12 *      Second part                              *
%         *  ..   *         *      ....                                     *
%         *       *   TAGN1 *      First part of CHTAG(NWKEY)               *
%         *       *   TAGN2 *      Second part                              *
%  LD->   *  +0   *   NRD   *      Number of records to describe this dir.  *
%         *  +1   *   IREC  *      Record number 1 of directory             *
%         *  +2   *         *                    2                          *
%         *  ..   *         *                    ..                         *
%         *  +NRD *         *                   NRD                         *
%         *******************************************************************


\subsection*{Subdirectory descriptor structure}
The subdirectories below a given directory are described
by the part labelled \Lit{LS} in the description of the top bank.
%\begin{XMPt}{Subdirectory structure}
%         *******************************************************************
%         * WORD  *   TAG   *                 CONTENT                       *
%         *******************************************************************
%  LS->   *   1   *  NAM1   *    Name of 1st subdirectory                   *
%         *   2   *  NAM2   *    "                                          *
%         *   3   *  NAM3   *    "                                          *
%         *   4   *  NAM4   *    "                                          *
%         *   5   *  NCHSD  *    Number of characters in subdirectory name  *
%         *   6   *  IRECSD *    Record number of this subdirectory         *
%         *   7   *  IDTIME *    Date and Time of creation of subdirectory  *
%         *   8   *  NAM1   *    Name of 2nd subdirectory                   *
%         *   9   *  NAM2   *    "                                          *
%         *  10   *  NAM3   *    "                                          *
%         *  11   *  NAM4   *    "                                          *
%         *  12   *  NCHSD  *    Number of characters in subdirectory name  *
%         *  13   *  IRECSD *    Record number of this subdirectory         *
%         *  14   *  IDTIME *    Date and Time of creation of subdirectory  *
%         *   .   *         *               ..                              *
%         *       *         *    etc...                                     *
%         *******************************************************************
%\end{XMPt}
To loop over all subdirectories at a given level, the following code
could be used.
\begin{XMPt}{Looping over all subdirectories at a given level}
*
*     Loop over all subdirectories at this level
*
         DO 5 JJ=1,IQ(KQSP+LCDIR+KNSD)
            LS=IQ(KQSP+LCDIR+KLS)
            IH=LS+7*(JJ-1)
            CALL ZITOH(IQ(KQSP+LCDIR+IH),IHDIR,4)
            CALL UHTOC(IHDIR,4,CHDIR,16)
*
*     CHDIR now contains the name of subdirectory JJ in
*     CHARACTER format
*
...
   5        CONTINUE
\end{XMPt}

\finalnewpage
\subsection*{KEYS structure}

For a given directory, the keys structure is given
by the part labelled \Lit{LK} in the description of the top bank.
%\begin{XMPt}{KEYS structure}
%\small
%         *******************************************************************
%         * WORD  *   TAG   *                 CONTENT                       *
%         *******************************************************************
%  LK->   *   1   *  LCYC   *    Pointer to highest cycle in C for key 1    *
%         *   2   *  KEYS(1)*    First element of key 1                     *
%         *   3   *  KEYS(2)*    Second element of key 1 (if any)           *
%         *   .   *         *    ...........                                *
%         *NWKEY+1*  KEYS() *    NWKEYth element of key 1                   *
%         *   .   *  LCYC   *    Pointer to highest cycle in C for key 2    *
%         *   .   *         *    First element of key 2                     *
%         *   .   *         *    Second element of key 2 (if any)           *
%         *   .   *         *    ...........                                *
%         *   .   *         *    NWKEYth element of key 2                   *
%         *   .   *         *    ...........                                *
%         *   .   *  LCYC   *    Pointer to highest cycle in C for key NKEYS*
%         *   .   *         *    First element of key NKEYS                 *
%         *   .   *         *    Second element of key NKEYS (if any)       *
%         *   .   *         *    ...........                                *
%         *   .   *         *    NWKEYth element of key NKEYS               *
%         *******************************************************************
%
%C refers to the cycles structure, described below.
%\end{XMPt}
The following Fortran code could be used to process the keys 
in a given subdirectory.
\begin{XMPt}{Processing the RZ keys}
      NWK        = IQ(KQSP+LCDIR+KNWKEY)
      NKEYS      = IQ(KQSP+LCDIR+KNKEYS)
      LK         = IQ(KQSP+LCDIR+KLK)
      DO 50 I=1,NKEYS
*           Number of this key vector
         K=LK+(NWK+1)*(I-1)
         DO 40 J=1,NWK
            IKDES=(J-1)/10
            IKBIT1=3*J-30*IKDES-2
            IF(JBYT(IQ(KQSP+LCDIR+KKDES+IKDES),IKBIT1,3).LT.3)THEN
               KEYS(J)=IQ(KQSP+LCDIR+K+J)
            ELSE
               CALL ZITOH(IQ(KQSP+LCDIR+K+J),KEYS(J),1)
            ENDIF
  40     CONTINUE
*         Here we have KEY vector # I in KEYS()
      ...
  50  CONTINUE
\end{XMPt}
\subsection*{CYCLES structure}

The cycles structure is given by the part 
labelled \Lit{LCYC} in the description of the top bank.
%\begin{XMPt}{Cycles structure}
%     For each cycle
%
%      WORD 1   - Pointer to previous cycle of KEY (bits 1 to 16)(0 if no)
%               - Second record (if there)  (bits 17 to 32)
%
%      WORD 2   - Creation date relative to 1986   (bits 9 TO 32)
%               - Creation time. 1minute precision
%               - RZKEEP                           (bit 5)
%               - Append mode                      (bit 4)
%               - Vector format (if RZVOUT)        (bits 1 to 3)
%
%      WORD 3   - Record number where data str. starts (bits 17 to 32)
%               - Offset in record (bits 1 to 16)
%
%      WORD 4   - Number of words in data structure (bits 1 to 20)
%               - Cycle number  (bits 21 to 32)
%
%           The pointer LCYC in structure K points to WORD 1
%
%
%K refers to the cycles structure, described above.
%\end{XMPt}

\subsection*{File descriptor structure}

The file descriptor can be found in the
part labelled \Lit{LB} in the description of the top bank.
%\begin{XMPt}{File descriptor structure}
%         *******************************************************************
%         * WORD  *   TAG   *                 CONTENT                       *
%         *******************************************************************
%  LB->   *   1   *  NWREC  *    Number of words for bitmap descriptor      *
%         *   2   *  LREC   *    Physical record length (in words)          *
%         *   3   *  IDATE  *    Creation date of the file                  *
%         *   4   *         *             BITMAP                            *
%         *   5   *         *       1 bit per record on the file            *
%         *   .   *         *               ..                              *
%         *       *         *    etc...                                     *
%         *******************************************************************
%\end{XMPt}
%\newpage

\subsection*{Lock record}

The record lock information is encoded in the 
part labelled \Lit{LC} in the description of the top bank.
%\begin{XMPt}{Lock structure}
%         *******************************************************************
%         * WORD  *   TAG   *                 CONTENT                       *
%         *******************************************************************
%         *   1   *  NLOCK  *    Number of locks                            *
%         *   2   *  IFREE  *    Pointer to first free word in record       *
%         *   3   *  FLAG   *    LOCK flag                                  *
%         *   4   *  NWLOCK *    Number of words for 1st lock               *
%         *   5   *  LOCK1  *    First part of the lock ID                  *
%         *   6   *  LOCK2  *    Second part of the lock ID                 *
%         *   7   *  DATE/T *    Date and time of the lock                  *
%         *   8   *  IRECD  *    Record number of the locked directory      *
%         *   9   *  ND     *    Number of couples (first,last)             *
%         *  10   *  IR1    *    1st record locked                          *
%         *  11   *  IRL    *    last record locked                         *
%         *  12   *  IR1    *    "                                          *
%         *  13   *  IRL    *    "                                          *
%         *  14   *  ..     *                                               *
% NWLOCK+4*       *  NWLOCK *    Number of words for second lock            *
%         *   .   *         *               ..                              *
%         *       *         *    etc...                                     *
% IFREE   *   .   *    0    *    First free word (content=0)                *
%         *   .   *         *               ..                              *
%         *******************************************************************
%\end{XMPt}
%
