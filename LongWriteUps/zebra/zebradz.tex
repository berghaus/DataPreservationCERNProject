%%%%%%%%%%%%%%%%%%%%%%%%%%%%%%%%%%%%%%%%%%%%%%%%%%%%%%%%%%%%%%%%%%%
%                                                                 %
%   ZEBRA User Guide -- LaTeX Source                              %
%                                                                 %
%   Chapter DZ (Debugging tools)                                  %
%                                                                 %
%   The following external EPS files are referenced:              %
%   none                                                          %
%                                                                 %
%   Editor: Michel Goossens / AS-MI                               %
%   Last Mod.:  8 Dec. 1990   mg                                  %
%                                                                 %
%%%%%%%%%%%%%%%%%%%%%%%%%%%%%%%%%%%%%%%%%%%%%%%%%%%%%%%%%%%%%%%%%%%
\chapter{DZ: The debug and dump package}
\section{Display routines}
\subsection{Display of a bank or a data structure}
\par \Rind{DZSHOW} displays the contents of a bank or a data structure in a
store. The output format of the data part is controlled by the internal
or external I/O characteristic.
\Subr{CALL DZSHOW (CHTEXT,IXSTOR,LBANK,CHOPT,ILNK1,ILNK2,IDAT1,IDAT2)}
\index{display!data structure}
\index{display!bank}
\index{bank!display}
\index{data structure!display}
\Idesc
\begin{DL}{MMMM}
\item[CHTEXT]Character variable specifying the text to be printed
together with the display (truncated to 50 characters).
\item[IXSTOR]Index of the store containing the bank or data structure.
\item[LBANK]Address of bank or entry address to the data structure
which is to be displayed.
\item[CHOPT]Character variable specifying option desired.
\begin{DL}{MM}
\item['B']Print the single bank at {\tt LBANK} (default).
\item['D']Print the bank contents from top to bottom Downwards
with five elements per line.
\item['S']Print the bank contents from left to right Sideways
with up to ten elements per line (default).
\item['L']Print the linear structure supported by {\tt LBANK}.
\item['V']Print the vertical (down) structure supported by {\tt LBANK}.
\item['Z']Print the data part of each bank in hexadecimal format
(i.e. ignoring the I/O characteristic).
\index{bank!I/O characteristic}
\end{DL}
\item[ILNK1]Index of the first link in a bank which will be printed.
\item[ILNK2]Index of the last link in a bank which will be printed.
\item[IDAT1]Index of the first data word in a bank which will be printed.
\item[IDAT2]Index of the last data word in a bank which will be printed.
\end{DL}
\subsection{Example:}
\begin{verbatim}
      CALL DZSHOW ('Display banks',IXSTOR,LQMAIN,'BLV',3,7,0,0)
\end{verbatim}
The complete (both down and next links are to be followed)
\index{link!next}
\index{link!down}
structure supported by the bank
at address {\tt LQMAIN} in  store {\tt IXSTOR} is to
be displayed (both down and next links are to be followed).
\index{link!next}
\index{link!down}
Links 3 to 7 and all data words of each bank are to be printed.
\subsection{Notes:}
\par
\begin{OL}
\item When {\tt ILNK2<ILNK1} ({\tt IDAT2<IDAT1}) no links (data) are output.
\item When {\tt ILNK2=ILNK1=0} ({\tt IDAT2=IDAT1=0}) all links (data) are
output.
\item When {\tt ILNKi} or {\tt IDATi} are outside bounds for a given bank, the
actual values for the bank in question are taken.
\item The explanation of the
first output line printed for each bank is given in section "Bank information"
under the heading "First line (General information)" in the
description of routine \Rind{DZSNAP}.
\end{OL}
\subsection{Output examples}
\begin{Listing}\begin{verbatim}
 DZSHOW --- Dump EV structure                                                                      OPTIONS:                 BDLV
 
  DZSHOW  +++++ LEVEL     0++++++++++             Store nb. 0 = //       Division nb. 2 = QDIV2     ++++++++++
  EV  .     1     9580(00065734) SY/US/IO 0001/00000/2153 NL/NS/ND    7/    7/      10 N/U/O/@O       0/       0/       1/    9580
 STRUCTURAL links                                          --------------------
\index{link!structural}
           1    VX        9379     3                 0     5                 0     7                 0
           2                 0     4                 0     6                 0
 DATA part of bank                                         --------------------
           1     "        EV       3     "        VX       5                 5     7     7.0000000         9     9.0000000
           2     "        TK       4                 4     6                 6     8     8.0000000        10     10.000000
 DZSHOW  +++++ LEVEL     1++++++++++             Store nb. 0 = //       Division nb. 2 = QDIV2     ++++++++++
  VX  .     3     9379(00065410) SY/US/IO 0000/00000/01A3 NL/NS/ND    1/    1/      12 N/U/O/@O    9452/    9580/    9579/    9379
 STRUCTURAL links                                          --------------------
           1    TK        9328
 DATA part of bank                                         --------------------
           1                31     4     34.000000         7     37.000000        10     40.000000
           2                32     5     35.000000         8     38.000000        11     41.000000
           3                33     6     36.000000         9     39.000000        12     42.000000
 DZSHOW  +++++ LEVEL     2++++++++++             Store nb. 0 = //       Division nb. 2 = QDIV2     ++++++++++
  TK  .     2     9328(00065344) SY/US/IO 0000/00000/0003 NL/NS/ND    0/    0/      15 N/U/O/@O    9353/    9379/    9378/    9328
 DATA part of bank                                         --------------------
           1     231.00000         4     234.00000         7     237.00000        10     240.00000        13     243.00000
           2     232.00000         5     235.00000         8     238.00000        11     241.00000        14     244.00000
           3     233.00000         6     236.00000         9     239.00000        12     242.00000        15     245.00000
\end{verbatim}\end{Listing}
\begin{Listing}\begin{verbatim}
 DZSHOW --- Dump EV structure                                                                      OPTIONS:                 BSLV
 
  DZSHOW  +++++ LEVEL     0++++++++++             Store nb. 0 = //       Division nb. 2 = QDIV2     ++++++++++
  EV  .     1     9580(00065734) SY/US/IO 0001/00000/2153 NL/NS/ND    7/    7/      10 N/U/O/@O       0/       0/       1/    9580
 --------  LINK part of bank  --------
       1 /        9379           0           0           0           0           0           0
 --------  DATA part of bank  --------
       1 /       "EV         "TK         "VX             4           5           6   7.000       8.000       9.000       10.00
 DZSHOW  +++++ LEVEL     1++++++++++             Store nb. 0 = //       Division nb. 2 = QDIV2     ++++++++++
  VX  .     3     9379(00065410) SY/US/IO 0000/00000/01A3 NL/NS/ND    1/    1/      12 N/U/O/@O    9452/    9580/    9579/    9379
 --------  LINK part of bank  --------
       1 /        9328
 --------  DATA part of bank  --------
       1 /          31          32          33   34.00       35.00       36.00       37.00       38.00       39.00       40.00
      11 /   41.00       42.00
 DZSHOW  +++++ LEVEL     2++++++++++             Store nb. 0 = //       Division nb. 2 = QDIV2     ++++++++++
  TK  .     2     9328(00065344) SY/US/IO 0000/00000/0003 NL/NS/ND    0/    0/      15 N/U/O/@O    9353/    9379/    9378/    9328
 --------  DATA part of bank  --------
       1 /   231.0       232.0       233.0       234.0       235.0       236.0       237.0       238.0       239.0       240.0
      11 /   241.0       242.0       243.0       244.0       245.0
\end{verbatim}\end{Listing}
\subsection{Print the format of a bank}
\par \Rind{DZFORM} prints the format of the data part of a bank.
It uses the I/O characteristic stored in the bank, decodes the
information and prints it in a format which is compatible with the
input of \Rind{MZFORM}.
\Subr{CALL DZFORM(IXSTOR,LBANK)}
\index{bank!format display}
\Idesc
\begin{DL}{MMMM}
\item[IXSTOR]Index of the store where the bank resides.
\item[LBANK]Print the I/O characteristic for the bank at address {\tt LBANK}.
\newline If {\tt LBANK} = 0 all I/O characteristics declared with
\Rind{MZFORM} are printed
\end{DL}
\subsection{Display of a ZEBRA store}
\par DZSTOR displays the structure of the ZEBRA store identified by
{\tt IXSTOR}.
The routine outputs the parameters
characterizing the store, followed by a
list of all divisions and all link areas associated with the store in
question.
\Subr{CALL DZSTOR (CHTEXT,IXSTOR)}
\index{store!display}
\Idesc
\begin{DL}{MMMM}
\item[CHTEXT]Character variable specifying the text to be printed
together with the dump (truncated to 50 characters).
\item[IXSTOR]Index of store to be displayed.
\end{DL}
\subsection{Output example}
\par The store parameters give the store sequence number (identifier),
name and
absolute address, followed by the useful length, the number of fence,
structural, permanent and working space link words, the minimal and
actual number of words in the reserve area between divisions 1 and 2,
the minimal offset of the upper end of default division 2 and the
number of short term and long term user divisions.
\begin{Listing}\begin{verbatim}
 DZSTOR --- Dump of store //
  --- Store Parameters ---
 Id    Name    Abs.addr.  Length   Fence      NS      NL      WS  Min.Resv.  Act.Resv.   Min(1+2)   Low  High
  0  //        0005C188     9998       2       1       0       0       1600       9325       2000     2     0
  --- Division parameters ---
    DIVISION    START    END       MAX    KIND   MODE  WIPES  GARB.  GARB. PUSHES      LIVE BANKS  DROPPED BANKS    BANKS TOTAL
  NB.   NAME   ADDRESS ADDRESS  LENGTH                        SYST.   FREE          NUMB.   LENGTH NUMB.   LENGTH NUMB.   LENGTH
==================================================================================================================================
   1  QDIV1          2       1       0 U/EVENT  FORWD      0      0      0      0       0        0     0        0     0        0
   2  QDIV2       9327    9598       0 U/EVENT  REVRS      0      0      0      0      11      272     0        0    11      272
  20  QDIVSYST    9750    9998       0  SYSTEM  REVRS      0      0      0      0       6      249     0        0     6      249
  --- Link area parameters ---
 QWSP     PERMANENT LIST AREA      is at absolute 0005C188 NL/NS     1    1     status   ACTIVE
 QLASYST  PERMANENT LIST AREA      is at absolute 00079E50 NL/NS    20   10     status   ACTIVE
\end{verbatim}\end{Listing}
\subsection{Display of a link area}
\par \Rind{DZAREA} displays the contents of a ZEBRA link area.
\Subr{CALL DZAREA (CHTEXT,IXSTOR,CHLA,LLA,CHOPT)}
\index{link area!display}
\Idesc
\begin{DL}{MMMM}
\item[CHTEXT]Character variable with text to be printed with the
dump (truncated to 50 characters).
\item[IXSTOR]Index of store to which link area is associated.
\item[CHLA]Character variable specifying the name of the link area to be mapped.
\newline If {\tt CHLA=' '} all link areas associated with the store are printed
('N' option only)
\item[LLA]One of the links of the link area to be printed ('A' option only)
\item[CHOPT]Character variable specifying option desired
\begin{DL}{MM}
\item['A']Use the link parameter LLA to identify the link area (default)
\item['N']Use the name parameter CHLA to identify the link area
\end{DL}
\end{DL}
\subsection{Example}
\begin{verbatim}
      CALL DZAREA ('Display of link area TRACK',IXCOMM,'TRACK',0,'N')
\end{verbatim}
A list of the addresses in the link area TRACK associated
with store IXCOMM will be given.
\subsection{Survey of a ZEBRA data structure}
\par DZSURV displays the survey of a ZEBRA data structure.
All horizontal (NEXT) as well as all vertical (DOWN) structural
\index{link!next}
links of a ZEBRA (sub)structure are followed.
Illegal structural links cause transfer to ZFATAL.
\Subr{CALL DZSURV (CHTEXT,IXSTOR,LBANK)}
\index{data structure!survey}
\Idesc
\begin{DL}{MMMM}
\item[CHTEXT]Character variable specifying the text to be printed
together with the dump (truncated to 50 characters).
\item[IXSTOR]Index of store where the data structure resides.
\item[LBANK]Address of the bank supporting the data (sub)structure for which
the survey is desired.
\end{DL}
If the structure is legal, printed output is produced.
Each line contains the following information:
\begin{UL}
\item The cumulative number of words occupied by all banks so far
\item The total number of words occupied by all banks at this level
\item The length of the longest bank at this level
\item The number of banks at this level (any identifier)
\item Structural relation
\item Bank identifier(s)
\end{UL}
\subsection{Example:}
\begin{verbatim}
      CALL DZSURV ('Summary of the EV data structure',IXSTOR,LEV)
\end{verbatim}
\begin{Listing}\begin{verbatim}
DZSURV --- Survey of the EV data structure                    ST= //        LSTART=     9580
  NWCUM     NW   WBK  NBK    IDENTIFIER(S)
 
     27     27    27    1      EV
     96     69    23    3        -1 VX
    271    175    25    7             -1 TK
 
DZSURV --- The structure supported by bank EV   at       9580 in store //       occupies        271 words  in     11 banks
\end{verbatim}\end{Listing}
where:
\begin{DL}{MMMMMi}
\item[NWCUM]Cumulated NW to allow easy calculation of
the memory occupancy of the sub-structures
\item[NW]Total number of words occupied by these NBK banks,
including system words.
\item[WBK]Words per bank (if the NBK banks are not all
of the same length, the longest is given).
\item[NBK]Number of banks on this level
\item[IDENTIFIER]Name(s) of the banks at this level
\newline Several names are given if all names
in a linear structure are not identical
\end{DL}
\par In this example
the supporting bank {\tt'LEV'} is in common {\tt//} at address
9580 pointed to by the link {\tt LEV}.
It supports a linear structure of 3 {\tt VX} banks via link {\tt -7}.
The {\tt VX} banks all
have a length of 23 words, and thus occupy 69 words of storage.
Each {\tt VX} bank supports a linear chain of
{\tt TK} banks via link {\tt -1}.
There are 7 {\tt TK} banks in memory, all of 25 words
(i.e. they occupy 25x7=175 words).
The total structure contains 11 banks and occupies 271 words.
\section{Map and checks on the division level}
\subsection{Snap of one or more divisions}
\par \Rind{DZSNAP} provides a snapshot of one or more divisions in a ZEBRA
store. The kind of information provided is controlled by CHOPT.
\Subr{CALL DZSNAP (CHTEXT,IXDIV,CHOPT)}
\index{division!snap}
\Idesc
\begin{DL}{MMMM}
\item[CHTEXT]Character variable specifying the text to be printed
together with the snap (truncated to 50 characters).
\item[IXDIV]Index of the division(s) to be snapped.
\newline For combining division indices, function MZIXCO can be used.
\index{MZ!MZIXCO}
\newline If no explicit division identifiers are specified
all user divisions are snapped.
\item[CHOPT]Character variable specifying the snap options desired.
\begin{DL}{MMMMMM}
\item['C' ritical]Dump any active bank with status bit IQCRIT set;
bit IQCRIT will be reset to zero in each bank
(option C is implied by option T)
\item['D' ump]Dump any active bank with status bit IQMARK set;
bit IQMARK will be reset to zero in each bank
\item['E' xtend]Extend map entry to dump all links of each bank
(otherwise only as many links as will fit on a line)
\item['F' ull]Dump all active banks, links and data
\item['K' ill]Dropped banks to be treated as active
(dropped banks are not normally dumped under D or F option)
\item['L' ink]Dump all link areas associated with the store
\item['M' ap]Print map entry for each bank
\item['T' erminal]Terminal type dump, used for the post-mortem dump
mainly to mark "critical" directly accessible banks
\item['W' ork]Dump the working space, links and data
\item['Z']Dump the information in hexadecimal.
\end{DL}
\end{DL}
\subsubsection{Store information}
\par The first part of the output generated by DZSNAP refers to the store.
The following information is provided:
\begin{DL}{MMMMM}
\item[NAME]Name of the store
\item[lQSTOR]Absolute address -1 of the store
\item[NQSTRU]Number of structural links at the beginning of the store
\item[NQREF]Number of permanent links at the beginning of the store
\item[NQLINK]Number of permanent + working space links
\item[NQMINR]Minimum size of the reserve area between divisions 1 and 2
\item[LQ2END]Lower limit for the upper and of division 2
\item[JQDVLL]Index of the most recent short-range divisions
\item[JQDVSY]Index of the system division
\item[NQFEND]Number of fence words
\item[LOW-1/N]Start and end address of division 1
\item[HIGH-1/N]Start and end address of division 2
\item[SYST-1/N]Start and end address of the system division
\item[END]Address of the last user word in the store
\end{DL}
\subsubsection{Bank information}
\par A map output in \Rind{DZSNAP} (selected by the option letter M)
\index{division!MAP bank display}
\index{DZ!DZSNAP}
gives a comprehensive overview of all banks in
the one or more divisions in a ZEBRA dynamic store.
One or two (MAP-)line(s) are printed per bank.
They contain the following information:
\subsubsection{First line (General information)}
\begin{OL}
\item The 4 character Hollerith bank identifier preceded by a (
if the bank has been dropped.
\item The bank numeric identifier
\item The address of the bank (status word) relative to the beginning of
the store and as an absolute address (in octal or hexadecimal)
\item The contents of the system and user part of the status word of the
of the bank (bits 19-32 and 1-18) and of its I/O characteristic.
\item Number of links ({\tt NL})/ of structural links
({\tt NS})/ of data words ({\tt ND})
\item The contents of the next ({\tt N})/up ({\tt U})/and origin ({\tt O})
\index{link!next}
links of the bank,
as well as of the contents of the address pointed to by the origin link
\index{link!origin}
({\tt @O}), which should contain
the address of the bank itself (hence allowing an easy cross-check).
When an inconsistency is detected the
faulty address is preceded by a minus sign ({\tt-}).
\end{OL}
\subsubsection{Second line (Links) (present only when there are non-zero links)}
\begin{OL}
\item a two character flag:
\begin{DL}{MM}
\item[**]the bank is dropped (also signaled by a left parenthesis '('
on the first line)
\item[.]the bank is active, all non-zero links are printed
\item[+]the bank is active, not all non-zero links are printed
\item[F]in position 2 flags a bank with potentially dangerous
contents in the links printed. This could be either:
\begin{UL}
\item illegal link content
\item dropped bank supporting an active bank (not via NX link)
\item active bank pointing to a dropped bank
\end{UL}
\end{DL}
\item links 1,2....N are printed in this order with N the smaller of the
the following 2 numbers:
\begin{UL}
\item {\tt N1}, the last non-zero link of this bank;
\item {\tt N2}, the number of links which can be printed on one line
(typically 9)
\end{UL}
If the link points to a correct bank-address, the {\tt ID} of that
bank is also printed, preceded by '(' if this bank has been dropped.
If the link does not point to a status word, then a '-' or
{\tt'****'} is printed against it for legal or illegal link content.
\end{OL}
\par Normally, the map is at the same time a printout of the more
interesting links in the banks.
However, banks may have more than the {\tt N2} links,
the maximum printed in the map.
If it is desired to print all the links,
the option letter E should be given and
then an internal  call to \Rind{DZSHOW} is generated.
\par To avoid confusion about the format of a data word,
an extra symbol may be printed on its left:
\begin{DL}{MM}
\item[ O]for octal
\item[ Z]for hexadecimal,
\item[ "]for BCD.
\end{DL}
\begin{Listing}\begin{verbatim}
 DZSNAP --- Snap of //                                                                             OPTIONS:                    M
   NAME       LQSTOR NQSTRU  NQREF NQLINK LQMINR LQ2END JQDVLL JQDVSY NQFEND  LOW-1  LOW-N HIGH-1 HIGH-N SYST-1 SYST-N    END
  //      (0005C188)      1      1      1   1600   2001      2     20      2      2      1   9327   9598   9750   9998   9998
0DZSNAP.   -----  Store nb. 0 = //       Division nb. 2 = QDIV2                       --------------------
  TK  .     2     9328(00065344) SY/US/IO 0000/00000/0003 NL/NS/ND    0/    0/      15 N/U/O/@O    9353/    9379/    9378/    9328
  TK  .     1     9353(000653A8) SY/US/IO 0000/00000/0003 NL/NS/ND    0/    0/      15 N/U/O/@O       0/    9379/    9328/    9353
  VX  .     3     9379(00065410) SY/US/IO 0000/00000/01A3 NL/NS/ND    1/    1/      12 N/U/O/@O    9452/    9580/    9579/    9379
      . LINKS:    9328 TK
  TK  .     2     9401(00065468) SY/US/IO 0000/00000/0003 NL/NS/ND    0/    0/      15 N/U/O/@O    9426/    9452/    9451/    9401
  TK  .     1     9426(000654CC) SY/US/IO 0000/00000/0003 NL/NS/ND    0/    0/      15 N/U/O/@O       0/    9452/    9401/    9426
  VX  .     2     9452(00065534) SY/US/IO 0000/00000/01A3 NL/NS/ND    1/    1/      12 N/U/O/@O    9550/    9580/    9379/    9452
      . LINKS:    9401 TK
  TK  .     3     9474(0006558C) SY/US/IO 0000/00000/0003 NL/NS/ND    0/    0/      15 N/U/O/@O    9499/    9550/    9549/    9474
  TK  .     2     9499(000655F0) SY/US/IO 0000/00000/0003 NL/NS/ND    0/    0/      15 N/U/O/@O    9524/    9550/    9474/    9499
  TK  .     1     9524(00065654) SY/US/IO 0000/00000/0003 NL/NS/ND    0/    0/      15 N/U/O/@O       0/    9550/    9499/    9524
  VX  .     1     9550(000656BC) SY/US/IO 0000/00000/01A3 NL/NS/ND    1/    1/      12 N/U/O/@O       0/    9580/    9452/    9550
      . LINKS:    9474 TK
  EV  .     1     9580(00065734) SY/US/IO 0001/00000/2153 NL/NS/ND    7/    7/      10 N/U/O/@O       0/       0/       1/    9580
      . LINKS:    9379 VX
\end{verbatim}\end{Listing}
\subsection{Verify one or more ZEBRA divisions}
\par \Rind{DZVERI} checks the structure of one or more divisions in a ZEBRA
store. The verification detail depends on the settings in {\tt CHOPT}.
\Subr{CALL DZVERI (CHTEXT,IXDIV,CHOPT)}
\index{division!verification}
\Idesc
\begin{DL}{MMMM}
\item[CHTEXT]Character variable specifying the text to be printed
together with the verification (truncated to 50 characters).
\newline {\tt CHTEXT=' '}: No message is output unless an error is detected.
In the latter case a message detailing the problem is
{\bf\it always} output, irrespective of {\tt CHTEXT}.
\item[IXDIV]Index of the division(s) to be verified.
\newline For a combination of divisions the MZ function \Rind{MZIXCO} should
be used
\newline If no explicit division identifiers are specified
all user divisions are verified.
\item[CHOPT]Character variable specifying level of checks desired.
\begin{DL}{MM}
\item['C']Check chaining of banks only (default)
\item['F']Errors are considered fatal and generate a call to \Rind{ZFATAL}
\item['L']Check validity of the structural links in the banks
\index{link!structural}
(implies {\tt'C'})
\item['S']Check the store parameters
\item['U']Check the validity of the up and origin links in the banks
\index{link!up}
\index{link!origin}
(implies {\tt'C'})
\end{DL}
\end{DL}
\par
When an error is detected the variable {\tt IQUEST(1)} in common {\tt/QUEST/}
\index{QUEST!Error reporting}
will be non zero. When everything is correct it will contain zero.
\subsubsection{Example}
\begin{verbatim}
     CALL DZVERI('Check store layout',IXSTOR,'S')
\end{verbatim}
only checks the store parameters for store IXSTOR, while
\begin{verbatim}
     CALL DZVERI('Check everything',IXDIVI,'CFLSU')
\end{verbatim}
checks the store parameters of the store containing
the divisions {\tt IXDIVI}
verifies the chaining of the banks and the correctness of all
links (structural, next, up and origin links). When an error is
\index{link!structural}
\index{link!up}
\index{link!origin}
\index{link!next}
detected an exit is forced via \Rind{ZFATAL}.
\subsubsection{Error return codes}
When \Rind{DZVERI} detects an error then it fills the
{\tt IQUEST} vector as follows:
\index{QUEST!IQUEST}
\begin{DL}{MMMMMM}
\item[IQUEST(11)]{\tt JQSTOR}, the store identifier
\item[IQUEST(12)]{\tt JQDIVI}, the division identifier
\par
\item[ -----  'C' option only -- For each faulty bank]
\par
\item[IQUEST(13)]{\tt LN}, its start address
\item[IQUEST(14)]{\tt IQLS}, its status word
\item[IQUEST(15)]{\tt IQNL}, its total number of links
\item[IQUEST(16)]{\tt IQNS}, its number of structural links
\item[IQUEST(17)]{\tt IQNS}, its number of data words
\par
\item[ ----- 'L' option only (check structural links in banks)]
\par
\item[IQUEST(18)]{\tt L}, the address of the link being verified
\item[IQUEST(19)]{\tt LQ(L)}, its contents
\par
\item[ ----- 'U' option only (check origin and up links in banks)]
\par
\item[IQUEST(20)]{\tt LUP}, the value of the {\tt up} link
\item[IQUEST(21)]{\tt LORIG}, the value of the {\tt origin} link
\end{DL}
\section{Monitor changes inside a ZEBRA store or bank.}
\subsection{Calculate the checksum of a vector in a ZEBRA store}
\par
\Rind{DZCHVC} calculates the checksum of the vector (interval)
{\tt [LBEGIN,LEND]}
in a given ZEBRA store and returns the checksum result in a 2-word
user vector.
\Subr{CALL DZCHVC (CHTEXT,IXSTOR,LBEGIN,LEND,CHOPT,*ISUM*)}
\index{checksum!store}
\Idesc
\begin{DL}{MMMM}
\item[CHTEXT]Character variable specifying the text to be printed
(truncated to 50 characters).
\item[IXSTOR]Index of the store where the checksum has to be calculated.
\item[LBEGIN]First word of the interval for which the checksum
has to be calculated
which is to be displayed.
\item[LEND]Last word of the interval for which the checksum
has to be calculated
\item[CHOPT]Character variable specifying option desired.
\begin{DL}{MM}
\item['C']Calculate the checksum for the desired interval (default)
\item['V']Verify - Compare the newly calculated value of the checksum with the
one given on input in the array {\tt ISUM}
\end{DL}
\item[*ISUM*]{\tt'V'} option only -- Two word integer array containing the
checksum calculated by a previous call to \Rind{DZCHVC}. This value has to be
compared with the newly calculated checksum for the given interval.
\end{DL}
\Odesc
\begin{DL}{MMMM}
\item[*ISUM*]Two word integer array containing the calculated checksum.
\newline This vector can be used as input to a subsequent call to
\Rind{DZCHVC} for the same interval or can be tested for modifications by the
user himself.
\end{DL}
\par When the {\tt'V'} option is set and a difference is detected between the
input and output values of {\tt ISUM}, then a non-zero value will be
returned in {\tt IQUEST(1)}, otherwise {\tt IQUEST(1)} will be zero.
\index{QUEST!Error reporting}
\index{QUEST!IQUEST}
\subsection{Important remark}
\par Since the checksum algorithm
sums the contents of the vector {\tt[LBEGIN,LEND]}
bit by bit, not all possible changes can be detected (e.g. inversions
in the sequence of the elements will go undetected).
\subsection{Monitor changes in a ZEBRA bank}
\par {\Rind DZCHST} monitors changes in the banks of ZEBRA data structure.
It uses routine \Rind{DZCHVC}.
The data, system and link part of the banks are monitored separately.
\Subr{CALL DZCHST (CHTEXT,IXSTOR,LBANK,CHOPT,*ISUM*)}
\index{checksum!bank}
\Idesc
\begin{DL}{MMMM}
\item[CHTEXT]Character variable specifying the text to be printed
(truncated to 20 characters).
\item[IXSTOR]Index of the store where the bank to be monitored resides.
\item[LBANK]Address of the bank to be monitored.
\item[CHOPT]Character variable specifying option desired.
\begin{DL}{MM}
\item['B']Bank - Consider the checksum of the single bank at {\tt LBANK} (default)
\item['C']Calculate the checksum for the desired banks(s) (default)
 checksum for the desired bank parts
\item['D']Consider the checksum of the
structure supported by {\tt LBANK}, but do not follow the next link.
\index{link!next}
\item['L']Consider the checksum of the
structure supported by {\tt LBANK}, also following the next link.
\index{link!next}
\item['V']Verify - Compare the newly calculated value of the checksums with the
ones given on input in the array ISUM
\end{DL}
\item[*ISUM*]{\tt'V'} option only -- Six word integer array containing the
checksums calculated by a previous call to \Rind{DZCHST}. These values have to
be compared with the newly calculated checksums for the different
parts of the bank(s) in the data structure.
\end{DL}
\Odesc
\begin{DL}{MMMM}
\item[*ISUM*]Six word integer array containing the calculated
checksums for the different parts of the bank as follows:
\begin{DL}{MM}
\item[1-2]Checksum of the {\bf data} part
\item[3-4]Checksum of the {\bf link} part
\item[5-6]Checksum of the {\bf system} part
\end{DL}
This array should be used as input to a subsequent call to \Rind{DZCHST}
for the same bank or can be tested for modifications by the
user himself.
\end{DL}
\par
When the 'V'erify option is active and a difference is detected
then the variable IQUEST(1) in common /QUEST/
\index{QUEST!Error reporting}
will be non zero. When everything is correct it will contain zero.
\subsection{Error return codes}
\par
When DZCHST detects an error then it fills the IQUEST vector as follows:
\index{QUEST!IQUEST}
\begin{DL}{MMMMMi}
\item[IQUEST(11)]0 : Data part OK\\1 : Data part faulty
\item[IQUEST(12)]0 : Link part OK\\1 : Link part faulty
\item[IQUEST(13)]0 : System part OK\\1 : System part faulty
\end{DL}
