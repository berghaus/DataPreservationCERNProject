\Filename{H1-MZ-Data structure utilities}
\chapter{Data structure utilities}
\label{sec:H1-Data-structure-utilities}

\Filename{H2-MZFLAG-logical-walk-d-s}
\section{MZFLAG {\it et al.} - logical walk through a data-structure}

By following the structural links,
\Rind{MZFLAG} sets the selected status-bit into the status words
of all the banks of the data-structure supported
by the down-links of the specified start bank.
Optionally it can include into the marking
also the banks of the linear structure supported
by link 0 of the start bank and all their dependents.
The start bank itself may or may not be marked.

The request is

\Shubr{MZFLAG}{(IXSTOR,!L,IBIT,chOPT)}

with
\begin{verbatim}
      IXSTOR  index of the store or of any division in this store,
              zero for the primary store

          !L  address of the start bank supporting
              the partial data-structure; no action if L=0.

        IBIT  the bit-number of the status-bit to be set

       chOPT  character string of options:

         default  mark the bank at L (and its down dependents),
                  the 'next' link of this bank is not followed,
                  status-bit ITBIT is set to one

               L  mark the linear structure pointed to by L
                  ie. the 'next' link of the bank at L is followed

               V  mark only the partial data-structure
                  dependent vertically downwards from the bank at L,
                  but not the bank itself

               Z  set to zero bit IBIT in each bank to be marked
\end{verbatim} 

\Rind{MZFLAG} will store into the two words of the common \FCind{/ZLIMIT/}
the addresses of the lowest and of the highest bank marked
during the scan, ready for use by the table-building routines
of \Rind{FZOUT} for example.

\Rind{MZFLAG} is not a routine commonly called directly by the users;
its main current use is as a service routine to \Rind{MZDROP}.

Similarly, the routine \Rind{MZMARK} described below
is not normally needed by the users except for a special problem
mentioned there.
\Rind{MZMARK} is also used as a service routine by \Rind{FZOUT}.

The function \Rind{MZVOLM} walks through a data-structure to calculate
the space occupied, returning the number of words as the function value.

\Sfunc{MZVOLM}{NWORDS = MZVOLM (IXSTOR,!L,chOPT)}

with
\begin{verbatim}
      IXSTOR  index of the store or of any division in this store,
              zero for the primary store

          !L  address of the start bank supporting
              the partial data-structure; no action if L=0.

       chOPT  character string of options:

         default  the 'next' link of the bank at L is not followed

               L  the 'next' link of the bank at L is followed


\end{verbatim} 

\Examples

\begin{verbatim}
      CALL MZFLAG (0,LQMAIN,IQDROP,'L')
\end{verbatim} 
this will scan the banks of the data-structure supported by
the bank at \Rarg{LQMAIN} and its sisters (option L),
setting the system bit \Rarg{IQDROP} to be 'on' in each bank found.
This is equivalent to \Lit{CALL \Rind{MZDROP} (0,LQMAIN,'L')},
except that it does not set the contents of the word \Rarg{LQMAIN} to zero.

\begin{verbatim}
      PARAMETER  (NID=3)
      DIMENSION  IDLIST(NID)
      DATA       IDLIST  /  4HBGO , 4HTEC  , 4HMUC  /

      CALL MZMARK (0,LQMAIN,'L-',NID,IDLIST)
\end{verbatim} 
this will scan the banks of the data-structure supported by
the bank at \Lit{LQMAIN} and its sisters (option L),
but exclude (option \Ropt{-}) from the scan any lower level linear structure
starting with a bank whose \Lit{IDH} is any of \Lit{BGO}, 
\Lit{TEC}, \Lit{MUC} (and its dependents),
setting in each bank found system status bit \Lit{IQMARK} to be 'on'.

The primary purpose of \Rind{MZMARK} is to give the user a possibility
to select parts of a data-structure for output with \Rind{FZOUT}.
The selection works on \Lit{IDH}, the Hollerith \Lit{ID}, of the first bank
of each linear sub-structure of the full data-structure.
For convenience,
one may give to \Rind{MZMARK} either the list of the \Lit{IDH}'s to be included
into the scan, or the list of the \Lit{IDH}'s to be excluded from the scan;
hopefully one gets away with a short list by selecting the right
mode.

\Rind{MZMARK} is a modified version of \Rind{MZFLAG}, it is simpler in that
the bit-number and the bit value are not parameterized:
the bit is \Lit{IQMARK} and the value is 1, as needed by \Rind{FZOUT};
it is more complex in that linear structures can be selected
or anti-selected.

The request is

\Shubr{MZMARK}{(IXSTOR,!L,chOPT,NID,IDLIST)}

with
\begin{verbatim}
      IXSTOR  index of the store or of any division in this store,
              zero for the primary store

          !L  address of the start bank supporting
              the data-structure; no action if L=0.

       chOPT  character string of options:

         default  mark the bank at L (and its down dependents),
                  the 'next' link of this bank is not followed,
                  lower level linear structures are accepted only
                     if they start with a bank whose IDH appears in
                     the list IDLIST (or if NID=0)

               L  mark the linear structure pointed to by L
                  ie. the 'next' link of the bank at L is followed

               V  mark only the partial data-structure
                  dependent vertically downwards from the bank at L,
                  but not the bank itself

               -  accept a lower level linear structure only if
                  it starts with a bank whose IDH does
                  n o t  appear in IDLIST

      NID       number of elements in the list IDLIST,
                if =zero all banks are accepted ('-' option ignored)

      IDLIST    list of the Hollerith ID for selection
\end{verbatim} 

On return |lit{\IQUEST(2)} contains the total number of words
occupied by all the banks marked (unless L is zero on entry).

As for \Rind{MZFLAG}, the addresses of the lowest and the highest
bank are stored into \FCind{/ZLIMIT/}, ready for \Rind{FZOUT}.

\Filename{H2-LZHEAD}
\section{LZHEAD - find the first bank of a linear structure}

This routine will try to find the first bank of the linear
structure of which the bank at \Lit{LGO} is a member.
It does this by following the path indicated by the "origin"
link of the bank at \Lit{LGO}, and using its "up" link.


\Sfunc{LZHEAD}{!LF = LZHEAD (IXSTOR,!LGO)}


It returns the address of the first bank of the linear structure
as the function value; or zero if there is trouble.

If the linear structure is not a top-level structure,
ie. if the up-link \Lit{LUP} is non-zero, the path of origin-links should
end in the link region of the bank at \Lit{LUP}, at a word whose
off-set \Lit{JBIAS} can then be calculated. This is returned:
\begin{verbatim}
      IQUEST(1) negative:  = JBIAS
\end{verbatim}
ie. LQ(LUP+JBIAS) contains the address of the first bank of the
linear structure.

If LUP is zero, the origin-path should end at a word outside
the bank space of the store \Rarg{IXSTOR}, which word
should contain the address of the first bank of the linear structure.
In this case \Rind{LZHEAD} returns:
\begin{verbatim}
      IQUEST(1) = 1: top-level structure
      IQUEST(2) = LS, relative adr of the supporting link-area link,
                      ie. LQ(LS) contains LF
\end{verbatim}

If \Rarg{LUP} is zero, and if the origin-link in the last bank in the path
is zero, this is a stand-alone structure, in which case \Rind{LZHEAD} returns:
\begin{verbatim}
      IQUEST(1) = 2: stand-alone structure
\end{verbatim}

If there is trouble, \Rind{LZHEAD} will return the function value zero,
and set:
\begin{verbatim}
      IQUEST(3) = 1   if LGO is zero

                  2   if LUP non-zero and the last origin-link
                      points outside bank-space

                  3   if LUP non-zero and LQ(LUP+JBIAS) does not
                      point to the last bank in the origin-path

                  4   if LUP zero, and LQ(LS) does not point to
                      the last bank in the origin-path.
\end{verbatim}

\Filename{H2-ZSHUNT}
\section{ZSHUNT - change structural relation}

Unlike in HYDRA, and because of the reverse pointers,
the operation of moving a bank by re-linking from one data-structure
to another one is a non-trivial operation.
The routine \Rind{ZSHUNT} is provided to execute such an operation.

\Rind{ZSHUNT} may be used to extract either a single bank (\Lit{IFLAG=0})
or a whole linear structure (\Lit{IFLAG=1}) from the old context,
for insertion into the new context as described by the parameters
\Rarg{LSUP} and \Rarg{JB}, which have the same significance as in \Rind{MZLIFT}.

\Shubr{ZSHUNT}{(IXSTOR,!LSH, !LSUP,JB,IFLAG)}

with
\begin{verbatim}
      IXSTOR  index of the store, zero for the primary store;
              IXDIV, the index of the division containing
              the bank to be shunted, may be given instead

        !LSH  address of the bank or of the linear structure

       !LSUP  if JB < 1:  address of the new supporting bank
              if JB = 1:  the new supporting link*

          JB  if JB < 1:  link bias in the new supporting bank
              if JB = 1:  LSUP is the new supporting link,
                           the origin-link in the bank at LSH
                           will be made to point to it
              if JB = 2:  detach without insertion

       IFLAG  if IFLAG = 0:  shunt the one single bank at LSH
              if IFLAG = 1:  shunt the whole linear structure
                              pointed to by LSH
\end{verbatim} 
If the bank or the structure to be re-linked is in fact inserted
or added into an existing linear structure,
both must be contained in the same division.

\Examples

Suppose we have the following data-structures to start with:

\begin{verbatim}
       ______
      |      |                         up
      |  UA  | <---.-------------.-------------.
      |______|     |             |             |
         |         |             |             |
      -3 |         |             |             |
         |       ______        ______        ______
         |      |      |  <-- |      |  <-- |      |
         `----> |  A1  | ---> |  A2  | ---> |  A3  |
                |______|      |______|      |______|

and
       ______
      |      |                              up
      |  UN  | <---.-------------.-------------.
      |______|     |             |             |
         |         |             |             |
      -7 |         |             |             |
         |       ______        ______        ______
         |      |      |  <-- |      |  <-- |      |
         `----> |  N1  | ---> |  N2  | ---> |  N3  |
                |______|      |______|      |______|


and
                    ______        ______        ______
              <--- |      |  <-- |      |  <-- |      |
      LQMAIN  ---> |  X1  | ---> |  X2  | ---> |  X3  |
                   |______|      |______|      |______|
\end{verbatim} 
Any bank may support further dependent partial data-structures,
the corresponding structural down-links are not changed
by \Rind{ZSHUNT}.

In what follows the notation  \Lit{Lxx}  is used to designate
a link pointing to bank \Lit{xx}.

\Examples

\begin{verbatim}
         CALL ZSHUNT (0,LA2,LUN,-7,0)     gives:
       ______
      |      |
      |  UA  | <---.-------------.
      |______|     |             |
         |         |             |
      -3 |       ______        ______
         |      |      |  <-- |      |
         `----> |  A1  | ---> |  A3  |
                |______|      |______|
and
       ______
      |      |
      |  UN  | <---.-------------.-------------.-------------.
      |______|     |             |             |             |
         |         |             |             |             |
      -7 |       ______        ______        ______        ______
         |      |      |  <-- |      |  <-- |      |  <-- |      |
         `----> |  A2  | ---> |  N1  | ---> |  N2  | ---> |  N3  |
                |______|      |______|      |______|      |______|
\end{verbatim} 

This moves a single bank (with is dependents, if any) out of
a linear structure, and inserts it at the head of the linear
structure supported by link \Lit{-7} of the bank \Lit{UN}.

\begin{verbatim}
         CALL ZSHUNT (0,LA2,LUN,-7,1)     gives:
       ______
      |      |
      |  UA  |
      |______|
         |
      -3 |       ______
         |      |      |
         `----> |  A1  |
                |______|
and   ______
     |      |
     |  UN  |  <--.-------------.-------------.------------------.
     |______|     |             |             |                  |
        |         |             |             |                  |
     -7 |       ______        ______        ______             ______
        |      |      |  <-- |      |  <-- |      |  <- ... - |      |
        `----> |  A2  | ---> |  A3  | ---> |  N1  | -- ... -> |  N3  |
               |______|      |______|      |______|           |______|
\end{verbatim} 
This is the same as example 1, except that the (partial) linear
structure starting with bank \Lit{A2} is re-linked.


\begin{verbatim}
         CALL ZSHUNT (0,LA2,LN2,0,0)      gives:
       ______
      |      |
      |  UN  | <---.-------------.-------------.-------------.
      |______|     |             |             |             |
         |         |             |             |             |
      -7 |       ______        ______        ______        ______
         |      |      |  <-- |      |  <-- |      |  <-- |      |
         `----> |  N1  | ---> |  N2  | ---> |  A2  | ---> |  N3  |
                |______|      |______|      |______|      |______|
\end{verbatim} 
This is again like example 1, but the bank is inserted inside
the linear structure, rather than ahead of it.

\begin{verbatim}
          CALL ZSHUNT (0,LA2,LQMAIN,1,0)   gives:

                  0             0             0             0
                  ^             ^             ^             ^
                  |             |             |             |
                 ______        ______        ______        ______
         <----  |      |  <-- |      |  <-- |      |  <-- |      |
  LQMAIN -----> |  A2  | ---> |  X1  | ---> |  X2  | ---> |  X3  |
                |______|      |______|      |______|      |______|
\end{verbatim} 

This relinks bank A2 to be the first in the top-level linear
structure supported by \Lit{LQMAIN}.

\begin{verbatim}
          L = LQMAIN
          CALL ZSHUNT (0,LA2,L,1,0)
\end{verbatim} 
has exactly the same effect as Example 4 above because,
\Lit{LQMAIN} not being zero initially,
the origin-link of the bank pointed to by L
(and the up-link, but this is zero)
is used for the connection.


\begin{verbatim}
         CALL ZSHUNT (0,LA1,LHEAD,1,1)     gives:
       ______
      |      |
      |  UA  |
      |______|
         |
      -3 |
          ----> zero

and               0             0             0
                  ^             ^             ^
                  |             |             |
                 ______        ______        ______
         <----  |      |  <-- |      |  <-- |      |
  LHEAD  -----> |  A1  | ---> |  A2  | ---> |  A3  |
                |______|      |______|      |______|
\end{verbatim} 

supposing \Lit{LHEAD=0} initially; this connects the linear structure
to the (structural) link \Lit{LHEAD}, ie. the origin-link of the header bank \Lit{A1}
points back to the location of \Lit{LHEAD}.

\begin{verbatim}
         CALL ZSHUNT (0,LA1,LDUMMY,2,1)    gives:
       ______
      |      |
      |  UA  |
      |______|
         |
      -3 |
         `----> zero

and               0             0             0
                  ^             ^             ^
                  |             |             |
                 ______        ______        ______
         0 <--  |      |  <-- |      |  <-- |      |
  LA1    -----> |  A1  | ---> |  A2  | ---> |  A3  |
                |______|      |______|      |______|
\end{verbatim} 
This detaches the linear structure from its old context
without inserting it into a new one.
This should only be temporary, one should insert the floating
structure into a new context by a second call to \Rind{ZSHUNT}
not too much later.

\Filename{H2-ZSORT}
\section{ZSORT  - utility to sort the banks of a linear structure}

These routines re-arrange the horizontal linking
within a given linear structure such that the key-words contained in
each bank increase monotonically when moving through the linear
structure with \Lit{L=LQ(L)}.
For equal key-words the original order is preserved.

Key-words may be either floating, integer or Hollerith.
For Hollerith sorting a collating sequence
inherent in the representation is used,
thus the results will depend on the machine.

Sorting may be done either for a single key-word in every bank
or for a key vector in every bank:

\Shubr{ZSORT}{(IXSTOR,*!LLS*,JKEY)}

Sorts banks according to a single floating-point keyword

\Shubr{ZSORTI}{(IXSTOR,*!LLS*,JKEY)}

Sorts banks according to a single integer keyword

\Shubr{ZSORTH}{(IXSTOR,*!LLS*,JKEY)}

Sorts banks according to a single Hollerith keyword

\medskip

\Shubr{ZSORV}{(IXSTOR,*!LLS*,JKEY,NKEYS)}

Sorts banks according to a floating-point key vector

\Shubr{ZSORVI}{(IXSTOR,*!LLS*,JKEY,NKEYS)}

Sorts banks according to an integer key vector

\Shubr{ZSORVH}{(IXSTOR,*!LLS*,JKEY,NKEYS)}

Sorts banks according to a Hollerith key vector

\begin{verbatim}

    with the parameters

      IXSTOR  index of the store or of any division in this store,
              zero for the primary store;

      *!LLS*  the address of the first bank of the linear structure,
              reset on return to point to the new first bank;

        JKEY  in each bank at L, Q(L+JKEY) is the key word,
                                 or the first word of the key vector;

       NKEYS  the number of words in the key vector.
\end{verbatim}

The execution time taken by these routines is a function
of the re-ordering which needs to be done.
For perfect order the operation is a simple verification pass
through the structure.
The maximum time is taken if the banks are initially arranged with
decreasing key words.

Sorting re-links the banks such that the key-words are in
increasing order.
If one needs them in decreasing order on may use
\Lit{CALL \Rind{ZTOPSY} (IXSTOR,LLS)}
which reverses the order of the banks in the linear structure
pointed to be \Lit{LLS}.

\Filename{H2-ZTOPSY}
\section{ZTOPSY {\it et al.} - utilities to operate on linear structures}

These routines perform service operations
on linear structures.
The parameter \Lit{LLS} is the address of the first bank
of the linear structure.


\Shubr{ZTOPSY}{(IXSTOR,*!LLS*)}

reverses the order of the banks in the linear structure,
ie. the first bank becomes the last, and the last the first,
for walking through the structure with \Lit{L=LQ(L)}.
Starting with Zebra version 3.67, \Lit{LLS} is updated to point to
the first bank of the inverted structure on return.

\Shubr{ZPRESS}{(IXSTOR,!LLS)}

removes by bridging dead banks still present
in the linear structure pointed to by \Lit{LLS}.

\Filename{H2-LZFIND}
\section{LZFIND {\it et al.} - utilities to interrogate linear structures}

These routines perform service functions for linear structures.
The parameter \Rarg{LLS} is the address of the first bank
of the linear structure.

\Sfunc{LZLAST}{!LF = LZLAST (IXSTOR,!LLS)}

searches the linear structure pointed to by \Rarg{LLS} for its end.
It returns in \Rarg{LF} the address of the last bank in the structure.
\Lit{LF = 0} is returned if the structure is empty.

\Sfunc{LZFIND}{!LF = LZFIND (IXSTOR,!LLS,IT,JW)}

searches the linear structure pointed to by \Rarg{LLS}
for the first bank containing \Rarg{IT} in word \Rarg{JW};
it returns its address in \Rarg{LF}.
If none:  \Lit{LF=0}.

\Sfunc{LZLONG}{!LF = LZLONG (IXSTOR,!LLS,NW,ITV,JW)}

has the same function as \Rind{LZFIND},
but \Rarg{ITV} is a vector of \Rarg{NW} words expected
in words \Rarg{JW} to \Lit{JW+N-1} of the bank.

\Sfunc{LZBYT}{!LF = LZBYT  (IXSTOR,!LLS,IT,JBIT,NBITS)}

has the same function as \Rind{LZFIND},
but it looks for a bank having \Rarg{IT} in byte \Lit{(JBIT,NBITS)}
of the status word.

\Sfunc{LZFVAL}{!LF = LZFVAL (IXSTOR,!LLS,VAL,TOL,JW)}

has the same function as \Rind{LZFIND},
but it looks for a bank having in word \Rarg{JW} a floating point number
which is equal to \Rarg{VAL} within the tolerance \Rarg{TOL}.

\Sfunc{NZBANK}{N  = NZBANK (IXSTOR,!LLS)}

counts the number of banks in the linear
structure pointed to by \Rarg{LLS}.

\Sfunc{NZFIND}{N  = NZFIND (IXSTOR,!LLS,IT,JW)}

searches like \Rind{LZFIND}, but for all banks.
It returns the number of such banks in \Lit{N}
and stores the addresses of the first 100 such banks into \IQUEST,
starting at \Lit{IQUEST(1)}.

\Sfunc{NZLONG}{N  = NZLONG (IXSTOR,!LLS,NW,ITV,JW)}

searches like \Rind{LZLONG}, but for all banks.
It returns the number of such banks in \Lit{N}
and stores the addresses of the first 100 such banks into \IQUEST,
starting at \Lit{IQUEST(1)}.

\Filename{H2-LZFID}
\section{LZFID {\it et al.} - utilities to find a bank by sequential scan}

Unlike the routines of the previous paragraphs which access
banks by following the links of the structure,
the routines of this paragraph perform a scan over the memory,
looking at each bank in turn in the order in which they happen
to be in the dynamic store,
to find the bank wanted.
For large memories with many banks this is likely to be an expensive
operation and should not be used unless there is no other way.


\Sfunc{LZFID}{!LF = LZFID (IXDIV, IDH,IDN, !LGO)}

searches the division indicated by \Rarg{IXDIV}, either starting
at its beginning if \Lit{LGO=0} or with the first bank after the bank
at \Rarg{LGO}, for the first bank with has the identifiers \Rarg{IDH} and \Rarg{IDN}.


\Sfunc{LZFIDH}{!LF = LZFIDH (IXDIV, IDH, !LGO)}

searches the division indicated by \Rarg{IXDIV}, either starting
at its beginning if \Lit{LGO=0} or with the first bank after the bank
at \Rarg{LGO}, for the first bank with has the Hollerith identifier \Rarg{IDH}.


\Sfunc{LZSCAN}{!LF = LZSCAN (IXDIV, !LGO)}

searches the division indicated by \Rarg{IXDIV}, either starting
at its beginning if \Lit{LGO=0} or with the first bank after the bank
at \Rarg{LGO}, for the first bank.

\Rind{LZSCAN} returns \Lit{\IQUEST(1)} containing zero 
if the bank at \Lit{LF} is live, or one if the bank is dead.

