%%%%%%%%%%%%%%%%%%%%%%%%%%%%%%%%%%%%%%%%%%%%%%%%%%%%%%%%%%%%%%%%%%%
%                                                                 %
%   ZEBRA User Guide -- LaTeX Source                              %
%                                                                 %
%   Chapter RZ (Direct access input/output)                       %
%                                                                 %
%   The following external EPS files are referenced:              %
%   none                                                          %
%                                                                 %
%   Editor: Michel Goossens / AS-MI                               %
%   Last Mod.:  9 Dec. 1990   mg                                  %
%                                                                 %
%%%%%%%%%%%%%%%%%%%%%%%%%%%%%%%%%%%%%%%%%%%%%%%%%%%%%%%%%%%%%%%%%%%
\chapter{Direct access input-output}
\section{Main goals}
\subsection{General}
\par 
The ZEBRA RZ package permits the storage and retrieval of 
ZEBRA data structures or FORTRAN vectors 
in random access files. Files may reside on standard
direct access devices such as magnetic disk, or be
mapped to virtual memory. 
RZ files can be accessed by several users simultaneously,
even across networks.
Remote file access and transfer is provided for RZ files
using standard tools, such as NFS and ftp. In the heterogeneous
environment, the tools provided in the CSPACK package may be used.
\par 
The RZ package is not a relational database management system,
but organises data in a hierarchical manner which is suitable
for many applications in High Energy Physics, and probably outside.
\subsection{Pathnames}
\par 
The basic unit of information addressed in an RZ file
is a ZEBRA data structure, in the simplest case a single ZEBRA bank.
We call this an RZ
{\bf data object}.
Each data object is referred to by a unique object name.
Object names are composed of a
{\bf pathname}, and one or more identifiers known as {\bf keys}.
\par
\index{UNIX}
The pathnames used by the RZ package were inspired by
the Unix file naming syntax and hence they typically 
carry mnemonic meanings and show the relationships
between different objects.
Unlike UNIX, however, RZ pathnames are {\bf not} case sensitive, i.e.
upper and lower case are both treated as upper case.
\par 
As in the Unix file system, one may have directories and subdirectories
seperated by slash characters {\tt /}.
An interelated group of objects may be grouped together in a directory.
\par
When an RZ file is opened, a user specified name is associated with it.
This name is known as the {\bf top directory} and is not
part of the file itself. This allows the user to have simultaneous
access to multiple files with the same RZ directory structure.
\par
At the very highest level in the RZ tree is the root, referred
to by a double slash, {\tt //}.
\par
The directory above a given subdirectory is known as the
{\bf parent directory} and may be referred to by a backslash
character \bs .
\par
Two other concepts are also provided, namely the {\bf current working directory},
or {\bf CWD} and the {\bf naming directory}. Objects are retrieved
and stored relative to the current working directory. The naming directory
is a mechanism for referring to a frequently used directory. 
It is initially set to the top directory, but may be reset at any time.
The naming directory may be referred to by the symbol {\tt\~{}}.
\par
The following FORTRAN program provides examples of the above
terms. For simplicity, the code to initialise the ZEBRA system
and open the RZ files (via the routine \Rind{RZOPEN}) has
been omitted.
\begin{XMPt}{Example of RZ pathnames and terms}
*
*     Initialise RZ files on FORTRAN units LUN1, LUN2
*     with top directory names TOP1 and TOP2
*
      CALL RZFILE(LUN1,'TOP1',' ')
      CALL RZFILE(LUN2,'TOP2',' ')
*
*     Print the current naming directory
*     (It will have been set to TOP2 by RZFILE)
*
      CALL RZNDIR(' ','P')
*
*     Set the current working directory
*
      CALL RZCDIR('//TOP1/SUB1/SUB2/SUB3',' ')
*
*     Change directory relative to current working directory
*     (to parent directory in this case)
*
      CALL RZCDIR('\bs ',' ')
*
*     Change directory to naming directory
*
      CALL RZCDIR('\~{}',' ')
\end{XMPt}
\index{RZ!object}
\index{RZ!naming tree}
\subsection{Keys and Cycles}
\par Data objects are identified beyond the pathname by {\bf keys},
which may be a single word of information
(integer, bit string or Hollerith)
or a vector of such words. The keys are not part of the pathname itself.
\par
For example, in the case of HBOOK histograms a single integer
key, the histogram ID, may be used. Histograms relating to different
information could be stored in separate subdirectories and referred
to in a unique and clear manner by the associated pathname and
key, e.g. //HISTOS/CUT1, keys (or IDs) 1-10.
\par Successive versions of objects with identical
pathname/key combination may exist simultaneously.
They are distinguished by a {\bf cycle number},
which is incremented automatically upon creation of successive data
objects. Cycles may be referred to explicitly,
the usual default is the highest cycle number.
The concept of cycles for successive versions of data objects with
identical names was taken from the VAX/VMS file system.
\index{VAX}
\index{VMS}
\section{Pratical examples of usage of the RZ package}
\subsection{HBOOK}
\par
The RZ package is probably most widely used to store HBOOK 
histograms and ntuples, e.g. for subsequent analysis
with PAW. 
In such cases, shared write access is not normally
required. The file is typically created by a single user
or job and subsequently read a small number of times.
\begin{XMPt}{Example of storing HBOOK histograms in an RZ file}
PAW > ldir



 ************** Directory ===> //LUN1 <===

                  Created 911030/1215  Modified 911030/1215


 ===> List of objects 
     HBOOK-ID  CYCLE   DATE/TIME   NDATA   OFFSET    REC1    REC2     
          1       1   911030/1215    103       1       3    
          2       1   911030/1215    104     104       3    
          3       1   911030/1215    107     208       3    
          4       1   911030/1215    106     315       3    
          5       1   911030/1215    106     421       3    
          6       1   911030/1215     56     527       3    

  Number of records =    2  Number of megawords =  0 +  1606 words
  Per cent of directory quota used =    .050
  Per cent of file used            =    .050
  Blocking factor                  =  28.418
 PAW >
\end{XMPt}
\par
The above output from the PAW command LDIR shows the contents
of an RZ file which has no subdirectories and a few histograms.
The objects are accessed using the top directory name {\tt //LUN1}
and the histogram ID. 
\par
One could of course have used a more complex directory structure,
but the number of directories and objects such a file is typically
rather small.
\subsection{FATMEN}
\par
The FATMEN system uses the ZEBRA RZ package in a more complex manner.
In this case the RZ files are read by many jobs simultaneously,
often over the network. Much more complex object names are used,
with pathnames such as the following example from the DELPHI 
collaboration. 
\begin{XMPt}{Example of an RZ pathname in FATMEN}
FM> pwd
Current Working Directory = //CERN/DELPHI/P01_ALLD/CDST/PHYS/Y90V03/E093.3/L0312
FM>
\end{XMPt}
A single RZ file that is used by FATMEN may well
contain in excess of one hundred thousand 
entries in several thousand directories.
In addition, these RZ files are constantly updated and must
retain consistancy to long running batch jobs.
\par
These goals are met by ensuring that only a single process ever
has write access to a FATMEN RZ file. All updates are performed
by dedicated servers.
\chapter{Description of user callable RZ routines}
\section{Open a direct access file}
\Subr{CALL RZOPEN (LUN,CHDIR*,CHNAME,CHOPT,*LRECL*,ISTAT*)}
\Idesc
\begin{DLtt}{1234567}
\item[LUN]Logical unit number associated with the RZ file.
The RZOPEN routine issues a FORTRAN OPEN statement for the
specified logical unit.
\item[CHDIR]Character variable in which the top directory
name is returned (option W). The name has the form
'LUNn', e.g. 'LUN1' or 'LUN99'
\item[CHNAME]Character variable specifying the name of the 
file to be opened.
\item[CHOPT]Character variable specifying the options required.
\begin{DLtt}{12}
\item[' ']  default, open file in readonly mode
\item['L']  create file with relative organization (VAX only)
\item['N']  open a new file
\item['S']  open file in shared readonly mode
\item['U']  open file in update mode
\item['SU'] open file in shared update mode
\item['1']  open file read/write assume single user
\item['W']  return in CHDIR directory name include
\item['Y']  suppress LRECL consistency check
\item['C']  Use C I/O instead of FORTRAN I/O
\end{DLtt}
\item[LRECL]Integer variable specifying the record length
of the file in machine words.
If a value of zero (0) is specified, the RZOPEN routine 
will attempt to obtain the correct record length from
the file itself. A value of zero must not be specified for new files.
\item[ISTAT]Integer variable in which the status code
is returned. 
\end{DLtt}
\par
The RZOPEN routine opens a new or existing RZ file
on the specified logical unit. A call to RZFILE, for
existing files, or RZMAKE, for new files, must follow
a successful call to RZOPEN.
\par
On MVS systems, the prefix for the current userid
will be automatically prepended to the filename
unless the filename begins with a dot ('.').
\begin{XMPt}{Example of valid MVS filenames}
*
*     RZOPEN will open the file R01JDS.RZFILE.DATA
*     for both of the following filenames, assuming
*     that R01JDS is the current userid prefix.
*
      CHFILE = 'RZTEST.DATA'
      CHFILE = '.R01JDS.RZTEST.DATA'
\end{XMPt}
\section{Create a new RZ file}
\index{RZ!initialization}
\Subr{CALL RZMAKE (LUN,CHDIR,NWKEY,CHFORM,CHTAG,NREC,CHOPT)}
\Idesc
\begin{DLtt}{1234567}
\item[LUN]Logical unit number associated with the RZ file.
A FORTRAN {\tt OPEN} statement or call to the
routine \Rind{RZOPEN} must precede the call to \Rind{RZMAKE}.
\newline Starting address of the memory area which will contain the
RZ information ('M' option)
\item[CHDIR]Character variable specifying the name of the top directory to be
associated with unit {\tt LUN} (up to 16 characters).
\item[NWKEY]Number of words associated to a key {\bf (maximum 9)}
\item[CHFORM]Character variable describing each element of the key vector
\begin{DLtt}{12}
\item['B']Bit string but not zero
\item['H']Hollerith (4 characters)
\item['A']Same as H except for RZLDIR (see RZLDIR)
\item['I']Integer (nonzero)
\end{DLtt}
\item[CHTAG]Character array defined as {\tt CHARACTER*8 CHTAG(NWKEY)}.
\newline
Each element of the array allows the description of the corresponding
element in the key vector with a tag of up to 8 characters.
\item[NREC]Number of physical records for the primary allocation
\item[CHOPT]Character variable specifying the selected options.
\begin{DLtt}{1234567}
\item[medium]
\begin{DLtt}{12}
\item[' ']Disk (default)
\item['M']Memory - The user must allocate at least {\tt NREC*LUN} words
of memory starting at address {\tt LUN} if he uses this option.
\end{DLtt}
\item[mode]
\begin{DLtt}{12}
\item[' ']Native mode (default)
\item['X']Exchange mode (32 bit machines only)
\end{DLtt}
\item[other]
\begin{DLtt}{12}
\item['F']Format NREC records, unless option M.
\end{DLtt}
\end{DLtt}
\end{DLtt}
\par 
Subroutine \Rind{RZMAKE} creates a new RZ file on the specified
logical unit. Should the file already exist, the routine
\Rind{RZFILE} should be used.
\par
On return from RZMAKE, IQUEST(1) will be set to 0
if the routine was successful. A non-zero value for
IQUEST(1) indicates an error.
\par
The following example opens and creates a new RZ file,
whose top directory contains
three words per key, the first one being an integer (the year) and the
two others being Hollerith (the month and the day).
A total of 5000 records of length 4096 bytes are requested.
\begin{XMPt}{Example of using the routine RZMAKE}
\begin{verbatim}
      CHARACTER*16 CHDIR
      CHARACTER    CHTAG(3)*8
      DATA CHTAG/'Year','Month','Day'/
 
      LRECL = 1024
      CALL RZOPEN(LUN,CHDIR,'RZTEST.DAT','N',LRECL,ISTAT)
      IF(ISTAT.NE.0) GOTO 999
      CALL RZMAKE(LUN,'Top_Dir',3,'IHH',CHTAG,5000,' ')
 
  999 PRINT *,' ERROR issuing RZOPEN'
\end{XMPt}
\par
Option F is particularly important for RZ files on
VM/CMS systems, when shared access is required. Further
details are given in Appendix A.
\section{Access an existing RZ file}
\index{RZ!access}
\Subr{CALL RZFILE (LUN,CHDIR,CHOPT)}
\Idesc
\begin{DLtt}{1234567}
\item[LUN]Logical unit number associated with the RZ file.
A FORTRAN {\tt OPEN} statement must precede the call to \Rind{RZFILE}.
\item[CHDIR]Character variable specifying the name of the top directory to be
associated with unit {\tt LUN}.
\item[CHOPT]Character variable specifying the selected options.
\begin{DLtt}{1234567}
\item[medium]
\begin{DLtt}{12}
\item[' ']Disk (default)
\end{DLtt}
\item[mode]
\begin{DLtt}{12}
\item[' ']Read mode (default)
\item['S']Shared mode
\item['U']Update mode
\item['1']Update mode and only one user (no LOCKs necessary)
\item['L']List current LOCK identifiers
\item['D']Reset "locking" word of the file (after program crash !)
\end{DLtt}
\end{DLtt}
\end{DLtt}
\index{RZ!file-mode shared}
\index{RZ!file-mode update}
\par 
Subroutine \Rind{RZFILE} accesses an existing RZ file on the specified
logical unit. Should the file not yet exist, the routine
\Rind{RZMAKE} should be used.
\par
On return from RZFILE, IQUEST(1) will be set to 0
if the routine was successful. A non-zero value for
IQUEST(1) indicates an error.
\section{Set the logging level}
\Subr{CALL RZLOGL (LUN,LOGLEV)}
\index{RZ!logging level}
\index{logging level}
\Idesc
\begin{DLtt}{1234567}
\item[LUN]Logical unit number for which the logging level has to be set
\item[LOGLEV]Logging level
\begin{DLtt}{12}
\item[-3]Suppress all messages
\item[-2]Error messages only
\item[-1]Terse logging
\item[ 0]Normal logging: \Rind{RZFILE}, \Rind{RZMAKE}, \Rind{RZEND}
\item[ 1]Log to watch rare events
\item[ 2]Log to monitor calls
\item[ 3]Short dumps to debug user-written output routines
\item[ 4]Full dumps to debug user-written output routines
\end{DLtt}
\end{DLtt}
\par 
The logging level
(i.e. the verboseness of the messages of the ZEBRA system) can be
controlled for a given RZ unit number by a call to \Rind{RZLOGL}.
\par 
Each declaration of an RZ file via \Rind{RZMAKE} or \Rind{RZFILE}
associates a default logging level of 0 to the file.
At any point in a program the logging level can be reset to a new
value by calling \Rind{RZLOGL} with the appropriate parameters.
\section{Close a direct access file}
\par A direct access file, identified by a top directory name,
is closed by a call to \Rind{RZEND}.
The directories present in memory,
when they have been changed,
are copied to the file and then deleted from memory, else
the directories in memory are simply deleted.
Note that a FORTRAN close statment must be provided by the
user for the associated file.
\index{RZ!file close}
\Subr{CALL RZCLOS (CHDIR)}
\Idesc
\begin{DLtt}{1234567}
\item[CHDIR]Character variable specifying the name of the top directory of the
file to be closed.
\end{DLtt}
\par
This routine terminates RZ access to the file referenced
by the specified top directory CHDIR, and issues a FORTRAN
or C close for the associated file. For this reason,
it should be used in preference to the routine RZEND.
\index{RZ!file deaccess}
\Subr{CALL RZEND (CHDIR)}
\Idesc
\begin{DLtt}{1234567}
\item[CHDIR]Character variable specifying the name of the top directory of the
file to be closed.
\end{DLtt}
\section{Save the modified directories}
\index{RZ!save modified directory}
\Subr{CALL RZSAVE}
\par 
All directories which have been modified in memory
and the current output buffer are written to the output file by a call
to \Rind{RZSAVE}. This routine is called
automatically by the system when using \Rind{RZCDIR}, 
\Rind{RCLOS}, \Rind{RZEND} or \Rind{RZFREE}.
In an interactive environment it may save to call \Rind{RZSAVE} from
time to time.
\section{Operations on RZ directories}
\subsection{Define the naming directory}
\index{RZ!naming directory}
\Subr{CALL RZNDIR (*CHPATH*,CHOPT)}
\Idesc
\begin{DLtt}{1234567}
\item[*CHPATH*]Character variable specifying the complete pathname of the
naming directory ('S' option)
\item[CHOPT]Character variable specifying the option
\begin{DLtt}{12}
\item[' ']Set the naming directory to the path specified in
{\tt CHPATH} (default)
\item['P']Print the naming directory
\item['R']Read the naming directory pathname into {\tt CHPATH}
\end{DLtt}
\end{DLtt}
\Odesc
\begin{DLtt}{1234567}
\item[*CHPATH*]Character variable containing the complete pathname of the
naming directory ('R' option).
\end{DLtt}
\par 
When one is working with many different directories, and has to
refer frequently the same directory, then the latter can be defined
as the {\bf naming directory}, designated by the symbol
'\verb!~!'
in pathnames.
A typical example would be an application where subdirectories have
to be created in user routines in which the complete pathname of the
naming directory is unknown.
To set the naming directory a call to \Rind{RZNDIR} must be made.
\subsection{RZNDIR return codes}
\begin{DLtt}{12}
\item[0]Normal completion
\item[1]'S' (default) option and the pathname {\tt CHPATH} is invalid
\end{DLtt}
\subsection{Define the current working directory}
\index{RZ!current working directory}
\index{RZ!CWD}
\Subr{CALL RZCDIR (*CHPATH*,CHOPT)}
\Idesc
\begin{DLtt}{1234567}
\item[*CHPATH*]Character variable specifying the pathname of the {\tt CWD}
(default).
\newline {\tt CHPATH = ' '} means the {\tt CWD} (useful with the 'U' option)
\newline
Unless several RZ files are open at the same time, the path name can
be specified either as a path starting with the character '/', in
which case an absolute pathname is intended for the given top directory.
When several RZ files are open, an absolute pathname must start with a
double slash '//' and the top directory.
When the pathname does not start with a '/', the pathname is prefixed
with the path of the {\tt CWD}.
\item[CHOPT]Character variable specifying the option
\begin{DLtt}{12}
\item[' ']Set the {\tt CWD} (default)
\item['P']Print the {\tt CWD}
\item['R']Read the {\tt CWD} pathname into {\tt CHPATH}
\item['U']The same as the default but the time stamp in the
directory in memory is checked against the one on the file and if
needed the directory in memory is brought up to date.
This option should be used when the user expects that directories can be
changed concurrently by another user and he wants to use the latest
version.
\item['K']Keep the Current Directory in memory. By default, space occupied
by the Current Directory may be released in case there is not enough
space to accomodate the new directory.
\end{DLtt}
\end{DLtt}
\Odesc
\begin{DLtt}{1234567}
\item[*CHPATH*]Character variable containing the complete pathname of the
{\tt CWD} ('R' option)
\end{DLtt}
\par 
The {\tt CWD} is set to the top directory after a call to \Rind{RZMAKE}.
The {\tt CWD} can be changed, displayed or obtained by a call to \Rind{RZCDIR}.
\par 
All operations of RZ routines manipulating keys
(i.e. \Rind{RZIN}, \Rind{RZOUT}, \Rind{RZRDIR}, \Rind{RZKEYS},
\Rind{RZPURG}, \Rind{RZDELK}, \Rind{RZDELT}, \Rind{RZQUOT},
\Rind{RZPASS}) refer to objects in
the "Current Working Directory" or {\tt CWD} for short.
\subsection{RZCDIR return codes}
\index{QUEST!IQUEST}
\begin{DLtt}{1234567}
\item[IQUEST(1)]Error status
\begin{DLtt}{1}
\item[0]Normal completion
\item[1]The pathname {\tt CHPATH} is invalid (default option)
\end{DLtt}
\par
\item[IQUEST(7)]NKEYS, number of keys in the directory
\item[IQUEST(8)]NWKEY, number of words in a key
\item[IQUEST(9)]Number of directories below {\tt CWD}.
\item[IQUEST(10)]NQUOTA, the record quota for the {\tt CWD} tree.
\end{DLtt}
\subsection{Examples:}
\par It is not necessary to specify {\tt //Top\_dir} in a pathname
unless several RZ files are open simultaneously.
If only one RZ file is declared, the following two calls
are equivalent:
\begin{verbatim}
      CALL RZCDIR('//top_dir/dira/dirb/dirc',' ')
and
      CALL RZCDIR('/dira/dirb/dirc',' ')
\end{verbatim}
\par If the {\tt CWD} was already set to {\tt /dira/dirb}
one can further abbreviate
the calling sequence to
\begin{verbatim}
      CALL RZCDIR('dirc',' ')
\end{verbatim}
\par To go one level up in the directory tree one can use '$\backslash$', e.g.
if the {\tt CWD} is {\tt /dira/dirb/dirc}
then the two following calls are equivalent:
\begin{verbatim}
      CALL RZCDIR('\',' ')
and
      CALL RZCDIR('/dira/dirb',' ')
\end{verbatim}
To set the {\tt CWD} to the Naming directory one uses:
\begin{verbatim}
      CALL RZCDIR('~',' ')
\end{verbatim}
\subsection{Creation of a directory}
\index{RZ!directory creation}
\Subr{CALL RZMDIR (CHDIR,NWKEY,CHFORM,CHTAG)}
\Idesc
\begin{DLtt}{1234567}
\item[CHDIR]Character variable with a maximum of 16 characters (for the given
level), specifying the name of the directory to be
created. All characters, but \verb!/, \ ,* ,~ or ?!
are allowed in a directory name.
\item[NWKEY]Number of words associated to a key {\bf (maximum 9)}
\item[CHFORM]Character variable describing each element of the key vector
(a blank is equivalent to 'I').
\begin{DLtt}{12}
\item['B']Bit string but not zero
\item['H']Hollerith (4 characters)
\item['A']same as 'H' (see RZLDIR)
\item['I']Integer (nonzero)
\end{DLtt}
\item[CHTAG]Character array defined as {\tt CHARACTER*8 CHTAG(NWKEY)}.
\newline
Each element of the array allows the description of the corresponding
element in the key vector with a tag of up to 8 characters.
\end{DLtt}
\par 
A directory below the current "working directory" (see \Rind{RZCDIR})
can be created by a call to \Rind{RZMDIR}.
\subsection{Example: Creating the geometry file of a LEP experiment}
\par 
To create a geometry file for the OPAL detector
the data base for the experiment has as top directory called
{\tt //OPAL}.
A directory called {\tt Geometry} is created, which will contain
the names of the 12 main detectors of OPAL.
\begin{verbatim}
      CHARACTER TAGS(2)*8
      INTEGER   KEY(2)
 
      CALL RZMDIR('Geometry',1,'H','Detector')
      CALL RZCDIR('Geometry',' ')
      TAGS(1)='Volume'
      TAGS(2)='Number'
      CALL RZMDIR('CDET',2,'HI',TAGS)
      CALL RZMDIR('ECAL',2,'HI',TAGS)
      CALL RZMDIR('HCAL',2,'HI',TAGS)
      CALL RZMDIR('FDET',2,'HI',TAGS)
      CALL RZMDIR('MUON',2,'HI',TAGS)
              .......
\end{verbatim}
As we now want to introduce information into the {\tt CDIR} directory,
we put our working directory equal to the latter by a call to \Rind{RZCDIR}:
\begin{verbatim}
      CALL RZCDIR('CDET',' ')
\end{verbatim}
which is equivalent to
\begin{verbatim}
      CALL RZCDIR('//OPAL/Geometry/CDET'),' ')
\end{verbatim}
\subsection{Example 3: Using the geometry file of a LEP experiment}
\par Logical records can then be entered corresponding to the parameters
of each of the 24 sectors of the Jet chamber of the Central detector,
of the vertex detector and of the Z chambers
(routine \Rind{RZOUT} is described below).
\begin{verbatim}
C--     Write the information for the 24 Jet chamber sectors
      CALL UCTOH('SECT',KEY,4,4)
      DO 10 ISECT=1,24
          KEY(2)=ISECT
          CALL RZOUT(IXSTOR,LQ(LCDET-ISECT),KEY,ICYCLE,' ')
   10 CONTINUE
C--     Write the information for the vertex chamber
      CALL UCTOH('VERT',KEY,4,4)
      KEY(2)=1
      CALL RZOUT(IXSTOR,LVERT,KEY,ICYCLE,' ')
C--     Write the information for the Z chambers
      CALL UCTOH('ZCHA',KEY,4,4)
      CALL RZOUT(IXSTOR,LZCHA,KEY,ICYCLE,' ')
\end{verbatim}
Update records for the geometry of each detector can be foreseen, e.g.
by creating a directory {\tt'Updates'} below {\tt'CDET'}
\begin{verbatim}
      CALL RZMDIR('Updates',1,'I','RUN')
\end{verbatim}
The Logical records in the {\tt'Updates'} directory will contain the
detector's identification as well as update parameters. {\tt KEY(1)} could be
the {\tt RUN} number from which the given corrections should be applied.
The procedure to build the geometry data structure could be the following:
\begin{UL}
\item Read the standard parameters in directory {\tt'CDET'}
\item Set the {\tt CWD} to 'Updates' and check if there are corrections
to be applied for that run,etc.
\end{UL}
\subsection{Get the key definitions for the current working directory}
\index{RZ!CWD key definition}
\Subr{CALL RZKEYD (NWKEY*,CHFORM*,CHTAG*)}
\Odesc
\begin{DLtt}{1234567}
\item[NWKEY*]Number of words associated to a key in the {\tt CWD}
\item[CHFORM*]Character variable describing each element of the key vector
(see \Rind{RZMDIR})
\item[CHTAG*]Character array defined as {\tt CHARACTER*8 CHTAG(NWKEY)}.
\newline Each element of the array describes the corresponding
element in the key vector.
\end{DLtt}
\par 
Information about the key definitions, as declared by \Rind{RZMDIR},
for the {\tt CWD} can be obtained be a call to \Rind{RZKEYD}:
\subsection{Lock and unlock a directory}
\index{RZ!file-mode shared}
\index{RZ!file-mode update}
\Subr{CALL RZLOCK (CHLOCK)}
\Idesc
\begin{DLtt}{1234567}
\item[CHLOCK]Character variable (up to 8 characters) identifying the owner
of the lock (e.g.
specifying the name of the user, his computer identifier,...)
This parameter is used to avoid two users, who have both the
write password for a directory, trying to change it at the same time.
{\tt CHLOCK} is also useful in the case of a system crash while a directory
was locked.
\end{DLtt}
\par 
When an RZ random access file is declared mode 'SU' (shared/update)
with \Rind{RZFILE} , then care must be taken to propagate the changes made
to the file to other processes, which are accessing the file
concurrently. Therefore, whenever the
directory structure or the data part of the CWD has to be changed by
calling one of the following routines:
\Rind{RZMDIR}, \Rind{RZCOPY}, \Rind{RZDELT}, \Rind{RZDELK},
\Rind{RZFRFZ}, \Rind{RZOUT}, \Rind{RZPURG}, \Rind{RZQUOT}, \Rind{RZRENK},
\index{RZ!directory locking}
\index{RZ!directory unlocking}
then, before using the first time any of these routines,
the {\tt CWD} must be locked by a calling routine \Rind{RZLOCK}.
To use this routine the write
password must have been specified if one has been defined.
Once a directory is locked, all
subdirectories become unavailable for locking. Hence when the top
directory is locked, the complete file is locked.
\par 
Note that two or more branches of a directory can be modified
concurrently
by different users (each one making a call to \Rind{RZLOCK}), as long as
for any given directory to be locked there is no higher level
directory already in a locked state.
\Subr{CALL RZFREE (CHLOCK)}
\Idesc
\begin{DLtt}{1234567}
\item[CHLOCK]Character variable identifying the owner of the lock.
\end{DLtt}
\par 
Once all modifications to a directory are performed, it must
be unlocked by a call to \Rind{RZFREE}. This routine outputs the updated
directories and provides them with a time stamp, so that other users
can determine whether they want to update the copy of the directories
they are working with.
\subsection{Set the space quota for the current working directory}
\index{RZ!directory quota}
\Subr{CALL RZQUOT (NQUOTA)}
\Idesc
\begin{DLtt}{1234567}
\item[NQUOTA]The maximum number of records which can be used by the CWD and its
subdirectories
\newline By default {\tt NQUOTA} is equal to the minimum of the total number of
records allowed for the complete file (parameter {\tt NREC}
in \Rind{RZMAKE}) and the quota of the parent directory.
\end{DLtt}
\par 
Routine \Rind{RZQUOT} allows the user to define a
space quota for the {\tt CWD} and all its subdirectories.
\subsection{List the contents of a directory}
\index{RZ!list directory}
\Subr{CALL RZLDIR (CHPATH,CHOPT)}
\Idesc
\begin{DLtt}{1234567}
\item[CHPATH]Character variable specifying the directory pathname.
\begin{DLtt}{12}
\item[' ']List information for the {\tt CWD} (default).
\item['//']List all the RZ files.
\end{DLtt}
\item[CHOPT]Character variable specifying the options
\begin{DLtt}{12}
\item['A']List all keys created with option 'A' by \Rind{RZOUT}
or \Rind{RZVOUT}.
\item[' ']By default such keys are not listed.
\end{DLtt}
\end{DLtt}
\par 
The keys and the subdirectory names belonging to a given pathname can
be listed by a call to \Rind{RZLDIR}.
\par 
If the keys have been defined by \Rind{RZMAKE} or \Rind{RZMDIR}
with format 'H',
they are listed each with 4 characters. If keys have been defined
with format 'A', they are listed without separators.
\subsection{Retrieve the contents of a directory}
\index{RZ!directory retrieve}
\Subr{CALL RZRDIR (MAXDIR,CHDIR*,NDIR*)}
\Idesc
\begin{DLtt}{1234567}
\item[MAXDIR]Length of the character array {\tt CHDIR}
\end{DLtt}
\Odesc
\begin{DLtt}{1234567}
\item[CHDIR*]Character array which will contain the directory names attached to
the {\tt CWD}. If the length of the directory name is greater then the length
of one element of {\tt CHDIR} (as obtained by the {\tt LEN} function), only
as many characters as will fit in the array element are returned, and
an error code will be set in {\tt IQUEST(1)}.
\item[NDIR*]Actual number of subdirectories attached to the {\tt CWD}
\newline If this number is greater than {\tt MAXDIR}, only the first
{\tt MAXDIR} directory names will be returned in {\tt CHDIR}
(see {\tt IQUEST(11)})
\end{DLtt}
\par 
The list of {\tt NDIR} directories attached to the {\tt CWD} is 
retrieved and stored into the character array CHDIR.
\par
\subsection{RZRDIR return codes}
\index{QUEST!IQUEST}
\begin{DLtt}{1234567}
\item[IQUEST(1)]Error status
\begin{DLtt}{12}
\item[0]Normal completion
\item[1]More entries present in the directory than returned in {\tt CHDIR}
(see {\tt NDIR} and {\tt IQUEST(11)}).
\end{DLtt}
\par
\item[IQUEST(11)]Actual number of subdirectories
\end{DLtt}
\subsection{Set the password of the current working directory}
\Subr{CALL RZPASS (CHPASS,CHOPT)}
\index{RZ!current password}
\Idesc
\begin{DLtt}{1234567}
\item[CHPASS]Character string specifying the password.
\item[CHOPT]Character string specifying the options desired:
\begin{DLtt}{12}
\item[' ']Specify a password (default),
\item['S']Set or change a password (to change a password a previous call to
\Rind{RZPASS} specifying the old password must have been made).
\end{DLtt}
\end{DLtt}
\par 
Each directory of an RZ file can have its own write password.
When an RZ file is first initialized with \Rind{RZMAKE} there is
no write password set.
Routine \Rind{RZPASS} can be used to specify
or change the password of the {\tt CWD}.
\index{RZ!CWD}
\par By default, when a directory is created (\Rind{RZMDIR}), the write
password is set equal to the one of the parent directory.
If a password is set, a call to \Rind{RZPASS} is necessary to be able
to write a new key, create a new directory or delete a key or directory.
The password specified using \Rind{RZPASS} is
checked against the one encrypted in the RZ directory referenced.
\subsubsection{Examples}
\begin{verbatim}
      CALL RZPASS('password',' ')          -- specifies a write password
 
      CALL RZPASS('New_password,'S')       -- changes or sets a password
\end{verbatim}
\section{Write a bank or data structure}
\Subr{CALL RZOUT (IXDIV,LSUP,KEY,*ICYCLE*,CHOPT)}
\index{RZ!output data structure}
\Idesc
\begin{DLtt}{1234567}
\item[IXDIV]Index of the division(s)
\newline May be zero if the 'D' option is not selected
\newline May be a compound index
(see Rind{MZIXCO} on page~\pageref{SR_MZIXCO})
if the 'D' option is selected
\item[LSUP]Supporting address
of the data structure (may be zero if the 'D' option is selected)
\item[KEY]Keyword vector of length {\tt NWKEY} as specified by \Rind{RZMDIR}.
\item[ICYCLE]Cycle number ('A' option only)
\item[CHOPT]Character variable specifying the selected options.
\begin{DLtt}{1234567}
\item[data structure]
\begin{DLtt}{12}
\item[' ']The data structure supported by the bank at
{\tt LSUP} is written out (the next link is not followed)
\index{link!next}
\item['D']Complete division(s)
\newline default: Dropped banks are squeezed out
\newline\phantom{default: }(slower but maybe more economic than 'DI')
\item['DI']Immediate dump of divisions with dropped banks included
\item['L']Write the data structure supported by the linear structure
at {\tt LSUP}
(the next link is followed)
\index{link!next}
\item['S']Single bank at {\tt LSUP}
\end{DLtt}
\item[mode]
\begin{DLtt}{12}
\item[' ']Keep banks available after output (default)
\item['N']No links, i.e. linkless handling
\item['W']Drop data structure or wipe division(s) after output
\item['A']Key will not be visible by RZLDIR
\end{DLtt}
\end{DLtt}
\end{DLtt}
\Odesc
\begin{DLtt}{1234567}
\item[ICYCLE]Cycle number associated to the key entered
\newline {\tt ICYCLE=1} if {\tt KEY}
was not already present in the directory,
and one larger than the previous cycle associated to the key otherwise.
\end{DLtt}
\par 
To write a bank, data structure or a complete division to an RZ file and enter the
associated key
into the current working directory, a call to \Rind{RZOUT} should be made.
If the key is not yet present in the directory, a cycle number of
one is returned, while in any other case the cycle number is the old
one present on the file increased by one.
\subsection{RZOUT return codes}
\index{QUEST!IQUEST}
\begin{DLtt}{1234567}
\item[IQUEST(1)]Error status
\begin{DLtt}{12}
\item[0]Normal completion
\item[1]The directory quota is exhausted, no more space
-- nothing has been written
\end{DLtt}
\item[IQUEST(2)]Number of physical records written
\item[IQUEST(3)]Record number of the first record written
\item[IQUEST(4)]Offset of the information inside the first record
\item[IQUEST(5)]Record number of the continuation record
\item[IQUEST(6)]Cycle number of the data structure written
\item[IQUEST(7)]Number of keys in the directory
\item[IQUEST(8)]NWKEY, the number of words per key
\item[IQUEST(9)]Number of records still available in the current subdirectory
\item[ ]
\item[IQUEST(11)]NWBK, number of words of bank material
\end{DLtt}
\section{Output an array}
\Subr{CALL RZVOUT (VECT,NOUT,KEY,*ICYCLE*,CHOPT)}
\index{RZ!output an array}
\Idesc
\begin{DLtt}{1234567}
\item[VECT]Array to be output onto the RZ file
\newline {\tt VECT} should be dimensioned at least to {\tt NOUT}
\item[NOUT]number of words of array {\tt VECT} to be output
\item[KEY]Keyword vector of length {\tt NWKEY} as specified by \Rind{RZMDIR}.
\item[ICYCLE]Cycle number ('A' option only)
\item[CHOPT]Character variable specifying the selected options.
\begin{DLtt}{1234567}
\item[format]
\begin{DLtt}{12}
\item[' ']The array contains floating point data (default)
\item['B']The array contains bitted data
\item['H']The array contains Hollerith data
\item['I']The array contains integer data
\item['A']Key will not be visible by \Rind{RZLDIR}
\end{DLtt}
\end{DLtt}
\end{DLtt}
\Odesc
\begin{DLtt}{1234567}
\item[ICYCLE]Cycle number associated to the key entered
\newline {\tt ICYCLE=1} if {\tt KEY} was not already present in the directory,
and one larger than the previous cycle associated to the key otherwise.
\end{DLtt}
\par 
The contents of a FORTRAN array can be written
into an RZ file and associated with a key in the {\tt CWD}
by a call to a call to \Rind{RZVOUT}.
The convention for the cycle number is the same as for \Rind{RZOUT}.
\section{Read a bank or data structure}
\Subr{CALL RZIN (IXDIV,*LSUP*,JBIAS,KEY,ICYCLE,CHOPT)}
\index{RZ!input data structure}
\Idesc
\begin{DLtt}{1234567}
\item[IXDIV]Index of the division to receive the data structure
\newline {\tt IXDIV = 0} means division 2 of the primary store
\item[*LSUP*]
\item[JBIAS]{\tt JBIAS < 1:} LSUP is the supporting bank
and JBIAS is the link bias specifying where the data structure has to be
introduced into this bank, i.e. the data structure will be connected
to {\tt LQ(LSUP+JBIAS)}.
\newline {\tt JBIAS = 1:}
{\tt LSUP} is the supporting link, i.e. the data structure
is connected to {\tt LSUP} (top level data structure)
\newline {\tt JBIAS = 2:} Stand alone data structure, no connection.
\item[KEY]Keyword vector of the information to be read (default)
\newline sequential number of the key vector in the directory if 'S' option
\item[ICYCLE]Cycle number of the key to be read
\newline {\tt ICYCLE > 0} highest cycle number means read the highest cycle
\newline {\tt ICYCLE = 0} means read the lowest cycle
\item[CHOPT]Character variable specifying the options selected.
\begin{DLtt}{1234567}
\item[data structure]
\begin{DLtt}{12}
\item[' ']Default - Same as 'D' below
\item['C']Provide information about the cycle numbers associated with {\tt KEY}.
\newline The total number of cycles and the cycle number identifiers
of the 19 highest cycles are returned in {\tt IQUEST(50)} and
{\tt IQUEST(51..89)} respectively
\item['D']Read the data structure with the {\tt (KEY,ICYCLE)} pair specified.
\item['N']Read the neighbouring
\footnote{
Directory entries are stored in ``historical'' order so that it
makes sense to talk of neighbouring records.
This can be used, e.g. to update records of calibration
constants or to scan files with events, where the keys correspond to
event or run numbers.}
keys (i.e. those preceding and following {\tt KEY}).
\newline The key-vectors of the previous and next key are available
respectively as {\tt IQUEST(31..35)} and {\tt IQUEST(41..45)}, see below.
\item['R']Read data into existing bank at {\tt LSUP, JBIAS}. Note that the bank
must have the same size as the one stored in the file.
\item['S']KEY(1) contains the sequential number of the key vector
in the current directory (No search required).
\end{DLtt}
\end{DLtt}
\end{DLtt}
\Odesc
\begin{DLtt}{1234567}
\item[*LSUP*]For {\tt JBIAS = 1} or {\tt 2, LSUP} contains
the entry address to the data structure
\newline In any case {\tt IQUEST(11)} returns the entry address
\end{DLtt}
\par 
When one wants to read a bank, data structure or division from
a direct access file into memory one calls \Rind{RZIN} or \Rind{RZINPA}.
The information identified by a given {\tt KEY} and cycle in the {\tt CWD} are
input. If the cycle specified is not present on the file, the information
associated with the highest cycle of the given key will be used.
\Subr{CALL RZINPA (CHPATH,IXDIV,*LSUP*,JBIAS,KEY,ICYCLE*,CHOPT)}
\index{RZ!input data structure}
\Idesc
\begin{DLtt}{1234567}
\item[CHPATH] Character variable specifying the name of the
directory containing the objects to be retrieved.
\item[others] Remaining arguments as for \Rind{RZIN}.
\end{DLtt}
\par
When one wants to read information from a key associated to a directory
which is not the {\tt CWD}, then a call to \Rind{RZINPA} can be made.
This routine has a supplementary character type argument {\tt CHPATH}, which
specifies the pathname
of the directory where the information has to read.
\subsection{RZINPA return codes}
\par RZIN returns the read status, either normal or error completion,
in the {\tt QUEST} vector as follows:
\index{QUEST!IQUEST}
\par {\bf Normal read status returns are:}
\begin{DLtt}{1234567}
\item[IQUEST(1)]Operation status code
\begin{DLtt}{12}
\item[1]key/cycle pair not present in the {\tt CWD}
\item[0]normal completion
\end{DLtt}
\item[IQUEST(2)]number of physical records read
\item[IQUEST(3)]Record number of the first record read
\item[IQUEST(4)]Offset of the start of the information in the first record.
\item[IQUEST(5)]Record number of the continuation record (0 if not 'A' option).
\item[IQUEST(6)]{\tt ICYCLE:} cycle number of information returned.
\item[IQUEST(7)]Number of keys in the directory
\item[IQUEST(8)]{\tt NWKEY}, the number of words per key\\[3mm]
\item[IQUEST(11)]{\tt LSUP}, the entry address into the data structure\\
zero means: empty data structure
\item[IQUEST(12)]{\tt NWBK}, the number of words occupied
by the data structure in memory                \\
zero means: empty data structure
\item[IQUEST(14)]Time stamp of the information(compressed).
In order to get
the unpacked date and time (integers), one can use the RZ internal
routine \Rind{RZDATE} as follows
\begin{verbatim}
   CALL RZDATE(IQUEST(14),IDATE,ITIME,1)
\end{verbatim}
\item[IQUEST(20)]Key serial number in the directory
\item[IQUEST(21..20+NWKEY)]{\tt KEY(1)...KEY(NWKEY)} if 'S' option given\\[3mm]
\item[IQUEST(30)]{\tt NWKEY} or zero if no previous key is present ('N' option)
\item[IQUEST(31..35)]The key vector for the element preceding {\tt KEY}
(if {\tt IQUEST(30) > 0}) \\
Only {\tt IQUEST(31..30+NWKEY)} are significant\\[3mm]
\item[IQUEST(40)]{\tt NWKEY} or zero if no previous key is present ('N' option)
\item[IQUEST(41..45)]The key vector
for the element following {\tt KEY} (if {\tt IQUEST(40) > 0}) \\
Only {\tt IQUEST(41..40+NWKEY)} are significant \\[3mm]
\item[IQUEST(50)]Number of cycles present for {\tt KEY}
('C' option)
\item[IQUEST(51..69)]The cycle number identifiers associated with {\tt KEY}\\
If {\tt IQUEST(50) $\leq$ 19} then only {\tt IQUEST(51..50+IQUEST(50)})
are meaningful
\newline If {\tt IQUEST(50) > 19} then {\tt IQUEST(51..69)} contain the 19
highest cycle number identifiers for {\tt KEY}
\item[IQUEST(71..89)]The time stamp
information corresponding to each of the
initialized cycle numbers in {\tt IQUEST(51..69)}
\end{DLtt}
\par If the pair {\tt (KEY,ICYCLE)} is not present in the
{\tt CWD (IQUEST(1) = 1)}
and the 'N' option is given, then {\tt IQUEST(30...)} and {\tt IQUEST(40...)}
will contain, respectively, the "lowest" and "highest" key vectors present.
\section{Input an array from an RZ file}
\Subr{CALL RZVIN (VECT*,NDIM,NFILE*,KEY,ICYCLE,CHOPT)}
\index{RZ!input array}
\Idesc
\begin{DLtt}{1234567}
\item[NDIM]Number of words available in array {\tt VECT} (e.g. declared
dimension)
item[KEY]Keyword vector of the information to be read
\item[ICYCLE]Cycle number of the key to be read
\newline {\tt ICYCLE > 0} highest cycle number means read the highest cycle
\newline {\tt ICYCLE = 0} means read the lowest cycle
\item[CHOPT]Character variable specifying the options selected(see \Rind{RZIN}).
\end{DLtt}
\Odesc
\begin{DLtt}{1234567}
\item[VECT*]FORTRAN array to contain the information input
\newline The array {\tt VECT} should be at least dimensioned to {\tt NDIM} words
\item[NFILE*]Actual length of the array on the file
\end{DLtt}
\par 
The information associated with a (key,cycle) pair on an RZ file can
be read into an array by a call to \Rind{RZVIN}.
The same conventions used by RZIN for KEY and cycle in the CWD are used.
\subsection{RZVIN return codes}
\par 
\Rind{RZVIN} returns the read status, either normal or error completion,
in the QUEST vector in a way similar to \Rind{RZIN}.
\index{QUEST!IQUEST}
\section{Operations on keys and cycles}
\subsection{Purge old cycles}
\index{VAX}
\index{VMS}
\index{RZ!key purge}
(cf. the {\tt PURGE} command on the VAX/VMS system)
\Subr{CALL RZPURG (NKEEP)}
\Idesc
\begin{DLtt}{1234567}
\item[NKEEP]Number of cycles which must be kept for the given key
\newline If {\tt NKEEP < 1} then {\tt NKEEP} is taken to be 1 and only the
highest cycle is kept
\end{DLtt}
\par 
All but the last {\tt NKEEP} cycles of all
key are deleted from the {\tt CWD} by a call to \Rind{RZPURG}.
\subsection{RZPURG return codes}
\index{QUEST!IQUEST}
\begin{DLtt}{1234567}
\item[IQUEST(9)]Number of records still available in the current subdirectory
\par
\item[IQUEST(11)]Maximum number of cycles purged
\item[IQUEST(12)]Number of words freed
\item[IQUEST(13)]Number of records freed
\end{DLtt}
\subsection{Delete a subtree from the current working directory}
\Subr{CALL RZDELT (CHDIR)}
\Idesc
\begin{DLtt}{1234567}
\item[CHDIR]Character variable specifying the directory name of the subtree of
the CWD.
\end{DLtt}
\subsection{RZDELT return codes}
\index{QUEST!IQUEST}
\begin{DLtt}{1234567}
\item[IQUEST(1)]Error status
\begin{DLtt}{12}
\item[0]Normal completion
\item[1]Invalid directory subtree name
\end{DLtt}
\end{DLtt}
\par 
A subtree of the CWD can be deleted by a call to RZDELT
\index{RZ!directory delete subtree}
\subsection{Delete a key from the current working directory}
\Subr{CALL RZDELK (KEY,ICYCLE,CHOPT)}
\Idesc
\begin{DLtt}{1234567}
\item[KEY]Key array of dimension {\tt NWKEY} (see \Rind{RZMDIR})
\item[ICYCLE]Cycle number of the key to be deleted
\newline {\tt ICYCLE > 0} highest cycle number means delete the highest cycle
\newline {\tt ICYCLE = 0} means delete the lowest cycle
\newline {\tt ICYCLE = -1, -2,...} means delete the highest cycle
{\tt -1, -2,...}
\item[CHOPT]Character variable specifying the options selected.
\begin{DLtt}{12}
\item[' ']Delete the explicitly specified cycle {\tt ICYCLE} only (default).
\newline If cycle {\tt ICYCLE} does not exist, no action is taken.
\item['C']Delete {\bf all} cycles corresponding to key ({\tt ICYCLE} not used)
\item['K']Delete all keys in the {\tt CWD} ({\tt ICYCLE} and {\tt KEY} not used)
\item['S']Delete all cycles smaller than cycle {\tt ICYCLE} for the given
key-vector
\end{DLtt}
\end{DLtt}
\par 
When a key-cycle pair has to be deleted from the
{\tt CWD} a call to \Rind{RZDELK} must be made
\index{RZ!key deletion}
\subsection{RZDELK return codes}
\index{QUEST!IQUEST}
\begin{DLtt}{1234567}
\item[IQUEST(1)]Operation status code
\begin{DLtt}{12}
\item[1]No entry for key/cycle pair specified
\item[0]normal completion
\end{DLtt}
\vspace*{3mm}
\item[IQUEST(11)]Maximum number of cycles deleted
\item[IQUEST(12)]Number of words freed
\item[IQUEST(13)]Number of records freed
\end{DLtt}
\subsection{Examples}
\begin{verbatim}
      CALL RZDELK(KEY,2,' ')
\end{verbatim}
deletes the information associated with key {\tt KEY} and cycle number 2 in
the {\tt CWD}.
\begin{verbatim}
      CALL RZDELK(KEY,4,'S')
\end{verbatim}
deletes all information associated with key {\tt KEY}
and a cycle number smaller than 4 in the {\tt CWD}.
\begin{verbatim}
      CALL RZDELK(0,0,'K')
\end{verbatim}
deletes all cycles of all keys in the {\tt CWD}.
\subsection{Rename a key in the current working directory}
\Subr{CALL RZRENK (KEYOLD,KEYNEW)}
\Idesc
\begin{DLtt}{1234567}
\item[KEYOLD]Key array of dimension NWKEY containing the old key vector
\item[KEYNEW]Key array of dimension NWKEY containing the new key vector
\end{DLtt}
\subsection{Return codes}
\index{QUEST!IQUEST}
\par
\begin{DLtt}{1234567}
\item[IQUEST(1)]Operation status code
\begin{DLtt}{12}
\item[1]No entry for {\tt KEYOLD} in the {\tt CWD}
\item[0]normal completion
\end{DLtt}
\end{DLtt}
\par 
A key in the {\tt CWD} can be renamed by a call to \Rind{RZRENK}
\index{RZ!key rename}
\subsection{Retrieve the keys associated to the current working directory}
\Subr{CALL RZKEYS (MAXDIM,MAXKEY,KEYS*,NKEYS*)}
\Idesc
\begin{DLtt}{1234567}
\item[MAXDIM]The actual first dimension of output array {\tt KEYS}.
It should in principle be at least equal to the number of key elements
{\tt NWKEY} as declared to \Rind{RZMDIR}.
\item[MAXKEY]The actual second dimension of output array {\tt KEYS}.
\end{DLtt}
\Odesc
\begin{DLtt}{1234567}
\item[KEYS*]A 2-dimensional array dimensioned {\tt KEYS(MAXDIM,MAXKEY)}.
It will contain the key vectors associated with the {\tt CWD}.
\newline Its first index runs over the key elements for a given key, while
its second index runs over the different keys.
\item[NKEYS*]Number of keys returned in array {\tt KEYS}.
\end{DLtt}
\subsection{Return codes}
\index{QUEST!IQUEST}
\begin{DLtt}{1234567}
\item[IQUEST(1)]Error status
\begin{DLtt}{12}
\item[0]Normal completion
\item[1]The keys have a length {\tt NWKEY > MAXKEY}
\newline  or more entries present in the directory than returned in {\tt KEYS}
(see {\tt IQUEST(11)}).
\end{DLtt}
\par
\item[IQUEST(11)]Actual number of keys in the {\tt CWD}.
\item[IQUEST(12)]{\tt NWKEY}, number of words characterizing a key vector
element for the {\tt CWD} (as defined on Page~\pageref{SR_RZMDIR} for RZMDIR).
\end{DLtt}
\par 
Subroutine \Rind{RZKEYS} returns the list of keys created in the
{\tt CWD}. The keys are returned in historical order.
\index{RZ!key retrieve}
\subsection{Examples}
\par 
For the lead glass blocks file in the example in
section 1, we could write:
\begin{verbatim}
      INTEGER KEYS(5000)
 
      CALL RZKEYS(1,5000,KEYS,NKEYS)
\end{verbatim}
For the events to be scanned we could have:
\begin{verbatim}
      INTEGER KEYS(2,500)                      ! Up to 500 keys vectors
 
      CALL RZKEYS(2,500,KEYS,NKEYS)
\end{verbatim}
\section{Copy a data structure from one directory to the \tt CWD}
\index{RZ!copy directory}
\Subr{CALL RZCOPY (CHPATH,KEYIN,ICYCIN,KEYOUT,CHOPT)}
\Idesc
\begin{DLtt}{1234567}
\item[CHPATH]The pathname of the directory tree which has to be copied
to the {\tt CWD}
\item[KEYIN]Key-vector of the object to be copied from {\tt CHPATH}.
\item[ICYCIN]Cycle number of the key to be copied
\item[KEYOUT]Key array of the object in the {\tt CWD} after the copy
\item[CHOPT]Character variable specifying the options selected.
\begin{DLtt}{12}
\item[' ']Copy the object from {\tt (KEYIN,ICYCIN)} from
{\tt CHPATH} to the {\tt CWD} (default).
\newline If {\tt KEYOUT} already exists, a new cycle is created.
\item['C']Copy all cycles for the specified key ({\tt ICYCIN} not used)
\item['K']Copy all keys in the {\tt CWD} ({\tt ICYCIN} and {\tt KEYIN} not used)
\newline
Given together with the 'C' option it copies all cycles of all keys.
\item['T']Not yet implemented. Copy the complete tree {\tt CHPATH}.
By default only the highest cycles are copied.
\newline Given together with the 'C option all cycles are copied.
\end{DLtt}
\end{DLtt}
\par Note that the input and output keys {\tt KEYIN} and {\tt KEYOUT} may be
identical. In this case, if {\tt KEYOUT} already exists in the {\tt CWD}, a new
cycle (or several) is created.
\par 
A directory tree identified by its
pathname {\tt CHPATH} can be copied
to the {\tt CWD} with the help of subroutine \Rind{RZCOPY}.
Routine {\tt RZCOPY} can also be used to merge two RZ files.
\subsection{Return codes}
\index{QUEST!IQUEST}
\begin{DLtt}{1234567}
\item[IQUEST(1)]Error status
\begin{DLtt}{12}
\item[1]Invalid pathname
\item[0]Normal completion
\end{DLtt}
\end{DLtt}
\section{Copy information from a directory from/to a sequential file}
\Subr{CALL RZTOFZ (LUNFZ,CHOPT)}
\Idesc
\begin{DLtt}{1234567}
\item[LUNFZ]Logical unit number of the FZ sequential access file
\item[CHOPT]Character variable specifying the options selected.
\begin{DLtt}{12}
\item[' ']Write the highest cycle of the keys in the {\tt CWD}
to the FZ file (default).
\item['C']Write all cycles of the keys in the {\tt CWD} to the FZ file
\end{DLtt}
\end{DLtt}
\par 
In order to provide
easy transportability of data between different computer
systems information stored in an RZ directory tree can
be written to or read from a sequential file.
All keys in the tree associated with the {\tt CWD} can be copied
to an FZ sequental file by using \Rind{RZTOFZ}.
The sequential file must be opened with \Rind{FZFILE} prior to
the call to \Rind{RZTOFZ} and thus the transport mode (native or exchange)
is determined by the mode declared to \Rind{FZFILE}.
\index{FZ!FZFILE}
\index{FZ!mode exchange}
\index{FZ!mode native}
The data structures are read into the system division of the
primary store before their transfer to the output file.
\index{RZ!Random to sequential}
\index{FZ!Random to sequential}
\index{RZ!Random from sequential}
\index{FZ!Random from sequential}
\Subr{CALL RZFRFZ (LUNFZ,CHOPT)}
\Idesc
\begin{DLtt}{1234567}
\item[LUNFZ]Logical unit number of the FZ sequential access file
\item[CHOPT]Character variable specifying the options selected.
\begin{DLtt}{12}
\item[' ']Read all cycles of the keys present on the FZ file into the
{\tt CWD} (default).
\item['H']Read the highest cycle of the keys present on the FZ file into
the {\tt CWD}.
\end{DLtt}
\end{DLtt}
\par 
A directory tree can be read
from an FZ sequential file
into the {\tt CWD} using the routine \Rind{RZFRFZ}.
If a sub-directory with the same name as the one read in is already
present in the {\tt CWD}, then a new cycle is created for the introduced keys.
The sequential file must be opened with \Rind{FZFILE}
prior to the call to \Rind{RZFRFZ}
and hence the transport format (native or exchange)
is determined by the mode declared to \Rind{FZFILE}.
\section{Retrieve statistics about a given RZ directory}
\Subr{CALL RZSTAT (CHPATH,NLEVELS,CHOPT)}
\Idesc
\begin{DLtt}{1234567}
\item[CHPATH]The pathname of the directory about which information
has to be provided.
\item[NLEVELS]Number of levels below {\tt CHPATH} about which space information
has to be accumulated.
\item[CHOPT]Character variable specifying the options desired
\begin{DLtt}{12}
\item[' ']Print the statistics (default)
\item['Q']Return the statistics in the user communication vector {\tt IQUEST}\\
\end{DLtt}
\end{DLtt}
\par 
Routine \Rind{RZSTAT} provides information about the usage statistics
of an RZ direct access file associated with a given directory,
as specified by its pathname.
The routine can be used in two ways,
namely to print the global statistics at the end of a run, or
to retrieve, at any given moment, useful data about the space usage
(e.g. to verify whether there is
still enough space left to add another record).
If option Q is specified, the IQUEST vector contains on return:
\begin{XMP}
IQUEST(11) = number of records used
IQUEST(12) = number of words   used
\end{XMP}
\index{QUEST!IQUEST}
\index{RZ!statistics}
