\Filename{H1-TZ-principle}
\chapter{The Title Package---Principles}

\Filename{H2-TZ-general-principles}
\section{General information}

The TZ linear data-structure of easy-creation and easy-access
banks has been provided primarily to satisfy the need of
programs for run-dependent control-information
(called ``titles'').

There may be one title structure for each Zebra store.
The title stucture normally resides in some
long-range division of the user.

To give a means of creating into the TZ structure banks which
can easily be changed from run to run,
the routine \Rind{TZINIT} is provided.
This reads a text file provided by the user,
containing the instructions of which title banks to create
and also the data to be placed into these banks.

Access by the user program to a given bank in the TZ structure
in store designated by IXSTOR is obtained with:
\begin{XMP}
      CALL TZFIND (IXSTOR, L, IDH, IDN, IFLAG)
\end{XMP}
This finds the title bank with the Hollerith bank identifier \Lit{IDH}
and the numeric identifier \Lit{IDN} (if non-zero);
it returns in L the pointer to the title bank found.
The last parameter controls the action to be taken
in case of missing title.

Some applications need different versions of a given title bank
as a function of the data they are processing.
\Rind{TZVERS} has been provided for this problem; it receives a selection
integer as one of its parameters, and it will scan the different
versions of the title bank until it finds one whose validity range
matches the selector.
The validity range must be specified on the first two words of
all the versions of all title banks to be handled with \Rind{TZVERS}.

\section*{Acknowledgements}

This package is derived from the HYDRA package TQ

V.Frammery, U.Herbst-Berthon, J.Zoll,\\
HYDRA Topical Manual, book TQ, CERN Program Library

\Filename{H1-TZ-pachage}
\chapter{Using the TZ package}

\Filename{H2-TZ-access-title-banks}
\section{TZFIND - access to title banks}

The banks in the linear TZ structure are identified by their
normal Hollerith and numeric bank identifiers.
To find a particular title bank, one calls:

\Shubr{TZFIND}{(IXSTOR, !L*, IDH, IDN, IFLAG)}

\begin{DLtt}{123456}
\item[IXSTOR]  index of the store holding the title-structure
               (or index of any division in this store);\\
               = zero for the primary store
\item[!L*]     output parameter to contain the link pointing
               to the title bank found; = zero if not found
\item[IDH]     the Hollerith identifier,  either
               an integer variable of the form \Lit{4Hxxxx}, or
               a literal character string '\Lit{xxxx}' of 4 characters exactly
\item[IDN]     the numeric identifier,
               if zero the first bank matching \Lit{IDH} is returned,
               otherwise the search continues to find the bank \Lit{IDH}
               which has the numeric identifier \Lit{IDN};
\item[IFLAG]   indicating the action to be taken for a missing title:
            \begin{DLtt}{12}
            \item[=0]:  return with \Lit{L=0}
            \item[>0]:  \Rind{TZFIND} shall take the error-exit
                        \Lit{CALL \Rind{ZTELL} (n,1)}  with \\
                        \Lit{n = 61}    for \Lit{IFLAG=1},\\
                        \Lit{n = IFLAG} for \Lit{IFLAG>99}.
            \end{DLtt}
\end{DLtt}

\Rind{TZFIND} returns the first bank with \Lit{IDH}/\Lit{IDN} 
in the linear structure.
If there are other banks with the same \Lit{IDH}/\Lit{IDN} further down,
they cannot be reached with \Rind{TZFIND};
but they could be reached with \Rind{TZVERS}
or with \Rind{LZFIND}.

\Rind{TZVERS} is similar to \Rind{TZFIND}, but it allows one also to select
a particular version from a set of title banks all having the
same \Lit{IDH} (and \Lit{IDN}, maybe).

The use of \Rind{TZVERS} requires that the first two words of the title banks
to be selected contain two integers specifying the validity range
of the bank.

\Shubr{TZVERS}{(IXSTOR, !LR*, IDH, IDN, ISELECT, IFLAG)}

\begin{DLtt}{123456}
\item[IXSTOR]  index of the store holding the title-structure
               (or index of any division in this store);\\
               = zero for the primary store
\item[!LR*]    output parameter to contain the link pointing
               to the title bank found; = zero if not found
\item[IDH]     the Hollerith identifier,  either
               an integer variable of the form \Lit{4Hxxxx}, or
               a literal character string '\Lit{xxxx}' of 4 characters exactly
\item[IDN]     the numeric identifier,
               if zero the first bank matching \Lit{IDH} is returned,
               otherwise the search continues to find the bank \Lit{IDH}
               which has the numeric identifier \Lit{IDN};
\item[ISELECT] selector to find the bank whose validity range
               matches this integer inclusively;
\item[IFLAG]   indicating the action to be taken for a missing title:
            \begin{DLtt}{12}
            \item[=0]:  return with \Lit{LR=0}
            \item[>0]:  \Rind{TZFIND} shall take the error-exit
                        \Lit{CALL \Rind{ZTELL} (n,1)}  with \\
                        \Lit{n = 62}    for \Lit{IFLAG=1},\\
                        \Lit{n = IFLAG} for \Lit{IFLAG>99}.
            \end{DLtt}
\end{DLtt}

\Filename{H2-TZINIT-create-title-banks}
\section{TZINIT - creating title banks from a text file}

The data to be read by \Rind{TZINIT} are prepared on a formatted file,
where they can easily be changed.
Typically normal free-field format titles on this file
look like this:

\begin{verbatim}
   *DO MEDI
       3                    #. this is a trailing comment
       1.      217.5993     #. MEDIUM 1 : AIR PATH
       1.5259    7.496      #.        2 : VAC TANK WINDOW
       1.5324   17.0278     #.        3 : MAIN WINDOW
       1.1005   50.54       #. HYDROGEN

   *DO FZO 21 -ACW
       4  0  :TLO3  :/dev/mt12    #. FZ option and file name
\end{verbatim} 

The data for the title banks are given one after the other.
Each title starts with the 'title header line',
marked by \Lit{*DO} in column 1,
giving the Hollerith identifier (like \Lit{MEDI} or \Lit{FZO}) and
possibly the numeric identifier (like 21),
possibly followed by options valid for this title (like \Lit{-ACW}),
selecting the way in which the data are to read.

Full specifications for the formats are given in chapter~\ref{sec:TZFORMATS}.

With these data \Rind{TZINIT} will create in the title structure a bank
with a Hollerith \Lit{ID} \Lit{'MEDI'} and with 9 data words,
and also a bank \Lit{FZO / 21}.
If this resides in the primary store the program can 
access the contents of bank \Lit{MEDI} by 
issueing \Lit{CALL TZFIND (0,L,'MEDI',0,1)} 
with has the following contents:
\begin{verbatim}
         IQ(L+1)     3
          Q(L+2)     1.
          Q(L+3)     217.5993
          Q(L+4)     1.5259
            ...
          Q(L+9)     50.54
\end{verbatim} 

If, after digesting the data, the program no longer needed
the bank it could remove it with  \Lit{CALL MZDROP (0,L,'.')}

To read the title data from a file connected to the logical
unit number \Lit{LUN} one calls:

\Shubr{TZINIT}{(LUN, IXDIV)}

\begin{DLtt}{123456}
\item[LUN] logical unit number,\\
           \Lit{ 0}:  Zebra ``card reader'' \Lit{IQREAD}\\
           \Lit{-1}:  Zebra ``terminal input'' \Lit{IQTTIN}
\item[IXDIV] division into which the title structure is to be stored;
         \begin{DLtt}{12345678}
            \item[IXSTOR]    division 2 of store \Lit{IXSTOR},
            \item[0]         division 2 of the primary store,
            \item[IXSTOR+24] system division of store \Lit{IXSTOR},
            \item[24]        system division of the primary store.
         \end{DLtt}
\end{DLtt}

\Rind{TZINIT} may be called several times in succession for different files,
thus building up the title structure from different sources;
on the second or later calls for the same store the division
part of \Lit{IXDIV} is ignored.

New title banks are always linked by \Rind{TZINIT} to the end
of the title structure;
thus this linear structure reflects the chronological order
in which the banks have been read.
This allows to over-ride a title bank from one file
by a title bank from a file read earlier.
Take this common example:
\begin{verbatim}
      CALL TZINIT (0,IXTITL)
      CALL TZINIT (7,IXTITL)
\end{verbatim} 
Giving the same title with different contents on the ``card reader''
over-rides the one from \Lit{LUN=7}.

Note the following problem:
as explained when discussion routine \Rind{MZXREF},
Zebra assumes by default that banks in any division do not have
links pointing into the system division.
Thus, if one puts the title structure into the system division
one must not also make links in banks point to title banks,
because such a link would not be updated for a garbage collection
in the system division.
The same is true if the user sends the titles into
a ``packag\'e' division (see the discussion of routine \Rind{MZDIV}).
This problem does not exist for links in link areas.

\Filename{H2-TZUSER-editing-title-banks}
\section{TZUSER - editing title banks during input}

Sometimes it is inconvenient to have to prepare
a given title on the input file in just the way in which
it is expected by the program.
Therefore \Rind{TZINIT} allows for a transformation of the data read,
to be done by the user routine \Rind{TZUSER}
at the level of individual titles.

This transformation by \Rind{TZUSER} is requested either globally
for all titles by the control-line  \Lit{*USER}  (see section~\ref{sec:tzcontrollines})
or individually for particular titles by giving
option \Lit{-U} or \Lit{-Un} on the header line for the title bank.
This causes \Rind{TZINIT} to transfer control to \Lit{TZUSER} as soon
as the input of the data for the particular title is complete,
and before starting to handle the next title.

Communication between \Rind{TZINIT} and \Lit{TZUSER} is as follows:

\Shubr{TZUSER}{(!LOLD)}

\begin{DLtt}{123456}
\item[!LOLD] address of the original bank filled with the input data
\end{DLtt}

\begin{verbatim}
      COMMON /TZUC/ JSTOR, IXTITL, NPARA, LNEW, NWOCC, NAME(20)
\end{verbatim}

\begin{DLtt}{123456}
\item[JSTOR] the number of the store = 0, 1, 2, ... currently being
             used, in case \Rind{TZUSER} has to cope with several structures,
\item[IXTITL]  the index of the title division for convenience,
\item[NPARA] the value \Lit{"n"} from option \Lit{-Un},zero  if \Lit{n} not given,
\item[LNEW*]  adr of the replacement bank lifted by the user, if any,\\
              initialized to zero by \Rind{TZINIT};\\
              \Lit{ -1}:  skip this title bank\\
              \Lit{-99}:  kill this run
\item[*NWOCC*] number of useful words available in the bank at \Lit{LOLD};
              this may be reduced or increased by \Rind{TZUSER}, but not
              above \Lit{NAME(4)}.  (\Rind{TZINIT} lifted the bank with
              \Lit{NAME(4)} words and has read \Lit{NWOCC} words into it.)
\item[NAME]  the ``nam\'e' used by \Rind{TZINIT} to lift the bank at \Lit{LOLD}:
         \begin{DLtt}{12345678}
            \item[NAME(1)]    Hollerith identifier
            \item[NAME(2/3)]  \Lit{NL} / \Lit{NS} - number of links = 0
            \item[NAME(4)]    \Lit{ND} - maximum number of data words
            \item[NAME(5->)]  I/O characteristic
         \end{DLtt}
\end{DLtt}

By programming \Rind{TZUSER}, the user may do one of four things:
\begin{OL}
\item change the information and the size of the original bank;
\item create a new bank to replace the old one;
\item simply suppress this title altogether by setting \Lit{LNEW=-1};
\item kill the run by setting \Lit{LNEW=-99}.
\end{OL}

\subsection*{Change the information}

Modify the data as necessary, set \Lit{NWOCC} to the number of useful
words in the bank if this has changed (without exceeding \Lit{NAME(4)}!),
and return.

When control returns to \Rind{TZINIT} it will take the appropriate action
according to the content of \Lit{LNEW} and maybe \Lit{NWOCC}.

\subsection*{Lift a replacement bank}

To avoid problems in case of garbage collection,
get the address of the original bank as follows:
\begin{verbatim}
      SUBROUTINE TZUSER (LPARA)
      DIMENSION    LPARA(9)
      LOLD = LPARA(1)
\end{verbatim} 
Set \Lit{NAME(4)} to the wanted size of the replacement bank,
and maybe also\Lit{ NAME(5-)} for the I/O characteristic,
and call:
\begin{verbatim}
      CALL MZLIFT (IXTITL,LNEW,LPARA,0,NAME,-1)
      LOLD = LPARA(1)
\end{verbatim} 
This second statement re-loads the local variable \Lit{LOLD} with
the address of the old bank, which might have changed.
Copy the information, possibly modified, from the bank at \Lit{LOLD}
into the bank at \Lit{LNEW}, and return.

\Rind{TZINIT} relies on the replacement bank having been
linked as the ``next'' bank to the original.

\subsection*{Kill execution}

If \Rind{TZUSER} discovers problems it can signal to \Rind{TZINIT}
that the job should be killed by setting \Lit{-99} into \Lit{LNEW}.
\Rind{TZINIT} will still read to the end of the file to find other
problems, maybe, and then call \Rind{ZFATAL}.

\subsection*{Loading TZUSER}

As always with user routines called from a general library
routine, there is the problem of getting the right \Rind{TZUSER} loaded.
If it is sent through the compiler to the linker there is no
problem. But if it is compiled and then put onto a user
library one must force its loading from this library,
either by loader directives, if available, or more simply
by having a \Lit{CALL TZUSER} in one's \Lit{MAIN} program.
In this latter case this must be a conditional
call which will never be executed.

For the applications which do not need \Rind{TZUSER} there is a dummy
version on the Zebra library. It will cause the job to fail
in case it is reached by accident.

\Filename{H2-TZSHUN-insert-banks}
\section{TZSHUN - insert banks into a title structure}

Although most commonly the banks in the TZ structure
are created by \Rind{TZINIT},
it may sometimes be handier
to set-up some titles directly in the program,
rather than taking them from an external text file.
The main advantage of introducing the titles via \Rind{TZINIT} is
that they are easily changed without re-compilation of the program.
But for a title which never changes one can lift a bank
in the right division and hand it to \Rind{TZSHUN} to re-link
it into the TZ structure:

\Shubr{TZSHUN}{(IXSTOR, !L, IFLAG)}

\begin{DLtt}{123456}
\item[IXSTOR] index of the store holding the title structure
              (or index of any division in this store);
\item[!L]     address of the (linear structure of) new bank(s)
\item[IFLAG]  \Lit{1}:  insert at the start,\\
              \Lit{0}:  insert at the end of the structure.
\end{DLtt}

Example: create a default title bank \Lit{TRAN} in the system division
of store \Lit{IXSTOR} if it does not already exist:

\begin{verbatim}
      DIMENSION    NAME(5)
      DATA NAME /4HTRAN, 0, 0, nd, 3/

      CALL TZFIND (IXSTOR,L,NAME,0,0)
      IF (L.EQ.0)  THEN
          CALL MZLIFT (IXSTOR+24, L, 0,2, NAME)
          Q(L+1) =  title word 1
          Q(L+2) =  title word 2
           ...       ...
          Q(L+nd)=  title word nd
          CALL TZSHUN (IXSTOR,L,0)
        ENDIF
\end{verbatim} 

Note: the system division of store \Lit{IXSTOR} is specified
symbolically as \Lit{IXSTOR+24}.

Note: if \Lit{L} points in fact to a linear structure  all banks are taken.

\Filename{H2-TZINQ-inquire}
\section{TZINQ - inquire about the title d/s}

The title data structure is supported by a system link not directly accessible
to the user.
To get access to this structure as such, not to a particular bank,
one can use this routine.

\Shubr{TZINQ}{(IXSTOR, IXTITL*, !L*, IFLAG)}

\begin{DLtt}{123456}
\item[IXSTOR]  index of the store holding the title structure
               (or index of any division in this store);
\item[IXTITL*] returns the index of the division holding
               the title structure
\item[!L*]      returns the address of the first or the last bank\\
                = zero if there are no titles
\item[IFLAG] 1: get the adr of the first bank\\
             0: get the adr of the  last bank\\
             other values are reserved for future extensions
\end{DLtt}

\subsection*{Example}

Suppose for a particular program one has a considerable volume of titles,
such that one spends too much time in \Rind{TZINIT} for every individual
execution of this program.
One could have a separate job to translate most of the text titles
into a binary FZ file whenever they have been updated.
The job which needs these titles reads them from the FZ file
and appends them to any titles which it has already read with \Rind{TZINIT}.

\subsection*{Job to translate text titles to binary}

\begin{verbatim}
      PROGRAM TTXTFZ

      CHARACTER    NAME*(*)
      PARAMETER   (NAME = 'cba1990')

      CHARACTER    FIN*(*), FOUT*(*)
      PARAMETER   (FIN = NAME // '.ttx',  FOUT = NAME // '.tfz')

      PARAMETER   (LIM2Q=60000, NNQ=80000)
      DIMENSION    LQ(NNQ), IQ(NNQ), Q(NNQ)
      EQUIVALENCE (Q(1),IQ(1),LQ(9))
      COMMON //    FENCE(4), LQ, LASTQ

      PRINT 9001, FIN, FOUT
 9001 FORMAT (/' TTX to TFZ executing'
     F/' with  input file = ',A /'      output file = ',A)

      CALL MZEBRA (-1)
      CALL MZVERS
      CALL MZSTOR (IXST,'/DYN/','.',FENCE,LQ,LQ,LQ,LQ(LIM2Q),LASTQ)

      OPEN (11, FILE=FIN, FORM='FORMATTED', STATUS='OLD')
      CALL TZINIT (11, 1)          { read into forward division 1
      CLOSE (11)

      OPEN (11, FILE=FOUT, FORM='UNFORMATTED', STATUS='UNKNOWN')
      CALL FZFILE (11, 0, 'O')

      CALL TZINQ  (0, IXTITL, LGO, 1)
      CALL FZOUT  (11, 1, LGO, 0, 'LDI', 0,0,0)
      CALL FZENDO (0, 'T')
      CALL MZEND
      END
\end{verbatim}

\subsection*{Piece of code loading the binary titles}

\begin{verbatim}
      CHARACTER    FIN*(*)
      PARAMETER   (FIN  = 'cba1990.tfz')

C--       first read the variable titles from the 'card reader'

      IXTITL = set the title division
      CALL TZINIT (0,IXTITL)

C--       then append the binary titles

      OPEN (49, FILE=FIN, FORM='UNFORMATTED',STATUS='OLD')
      CALL FZFILE (49, 0, '.')

      CALL TZINQ  (IXSTOR, IXTT, LTT, 0)
      IF (LTT.EQ.0)  THEN
          L     = 0
          JBIAS = 1
        ELSE
          L     = LTT
          JBIAS = 0
        ENDIF

      CALL FZIN   (49, IXTITL,L,JBIAS, '.', 0,0)
      IF (IQUEST(1).NE.0)          GO TO trouble
      IF (LTT.EQ.0)  CALL TZSHUN (IXTITL,L,0)
      CALL FZENDI (49, 'TXQ')
\end{verbatim}

\Filename{H1-TZ-formats-text-input-tzinit}
\chapter{Formats for the text input to TZINIT}
\label{sec:TZFORMATS}

\Filename{H2-TZ-outline}
\section{Out-line}

The input to \Rind{TZINIT} is presented on a formatted text file
of 80-column card-images,
giving the titles one after the other,
as shown by the example on the next page.

Each title starts with the 'title header line'
\Lit{*DO ...} (see section~\ref{sec:TZ-titleheaderlines})
which selects mode and options for the reading of its
associated ``title data''.
These have to come immediately and completely after the
title header line.

Normal mode is ``free-field format'' where the data are read line-by-line
and the text is interpreted.
The mode of each data word is taken from the way it is written:
integer data have neither the decimal point nor an exponent,
floating data have either or both,
Hollerith data start with : or ' or ",
octal data are pre-fixed with \Lit{#O};
see section~\ref{sec:TZfreefieldinput} for full details.

With ``Fortran format'' the data are read by a single
Fortran \Lit{READ} statement with the FORMAT taken from the \Lit{-F} option on
the title header line.
This is useful for computer generated titles.

``Control lines'' to \Rind{TZINIT} may be given interspersed
with the titles.
Global options are selected with control lines;
other examples are the comment line with  \Lit{*--}  in column 1,
and the input terminator \Lit{*FINISH};
see section~\ref{sec:tzcontrollines} for details.

Control lines to \Rind{TZINIT}, including title-header lines starting
with \Lit{*DO}, are handled case-insensitive, ie. they are converted
to upper case before processing.

\begin{verbatim}
*-------            TITLE VERSION A3
*PRINT
*.-------1---------2---------3---------4---------5---------6---------7
*DO  RTBC  -iF                   #. 810127     CERN  ROLL 103
13.       1100.
            .99999  +.00042  +.00302
           -.00042   .99993  -.0023
           -.00302  +.0023    .99993     0.     +.015     1.963

*DO  CAM1  -e -n27 -c11 -iF      #. 810121 17.27  ROLL 103
 1 CAM       3.  12.    9.067    11.8516   75.6561     25.    -75.6561
 1 MED       2.        1.51      1.5       1.458     2.382     1.57
             4.06      1.0884
 1 DIST       -.00169     -.00004      .015        .0011      -.0001
               .00394     -.00245      .00378      .051        .0023
               .033        .09

*DO  FID1  -c11 -iF              #. 810121 17.27  ROLL 103  ERASME
            12.
 1 4   1     1.  -17.2475  -18.6369
 1 4   2     2.  -17.2595   -4.2708
 1 4   3     3.    -.7761   -4.3177
 1 1   1    11.  -13.8081  -15.9118
 1 1   2    12.  -13.7343   -7.5253
 1 1   3    13.   -4.8669   -7.5898
 1 1   4    14.   -4.8931  -15.9473
 1 2   1    21.  -13.0353  -15.6000
 1 2   2    22.  -12.8979   -7.3464
 1 2   4    24.   -3.7895  -15.6767
 1 3   3    33.   -2.6674  -17.2822
 1 3   4    34.   -3.1636  -17.8522

*DO  CAM2  -e -n27 -c11 -iF      #. 810121 17.27  ROLL 103
 2 CAM       3.  12.    9.1488  -10.3208   75.673      25.    -75.673
 2 MED       2.        1.51      1.5       1.458     2.382     1.57
             4.06      1.0884
 2 DIST       -.0017       .00005      .015       -.0011      -.0001
               .00043     -.00443     -.00278      .0065      -.0024
               .033        .05

*DO  FID2  -c11/40 -iF           #. 810121 17.27  ROLL 103  ERASME
            13.
 2 4   1     1.  -17.4350    3.9107     examples of other representations
 2 4   2     2.  -17.4786   18.2503     integer    15  -24  0
 2 4   3     3.    -.8612   18.2306     floating   -.123E-7  1E-12
 2 1   1    11.  -13.8891    7.4810     octal      #0731244600
 2 1   2    12.  -13.8296   15.8810     hex        #xffff
 2 1   3    13.   -4.9569   15.7810     Hollerith  :PISA :BARI
 2 1   4    14.   -4.9724    7.4133                :K*(1430)
 2 2   1    21.  -13.1122    7.3163                'text with blanks'
 2 2   3    23.   -3.7338   15.3468                "text with ' and blank"
 2 2   4    24.   -3.8710    7.1994
 2 3   1    31.  -15.4499    4.5935
 2 3   2    32.  -15.8947    5.0865
 2 3   4    34.   -3.2437    4.4470
*.-------1---------2---------3---------4---------5---------6---------7
\end{verbatim} 

\newpage
\Filename{H2-TZ-controllines}
\section{Control-lines}
\label{sec:tzcontrollines}

Control-lines carry special instructions to \Rind{TZINIT};
they may only be given between titles,
and not within the data-body of a particular title.
An option selected by a control line is switched on or off
at the moment the line is read, and stays in effect until
changed by the inverse selection,
but the effect does not carry from one call to \Rind{TZINIT} to the next.

\begin{verbatim}
   *---        normal comment line
   *.--        non-printing comment line
               blank lines are allowed and ignored

   *LOG        logging information to be printed by TZINIT;
   *LOG  OFF   switch off  (default : according to the log level of
                                      the store, normally 'on')

                  If LOG is 'on' TZINIT will echo on the log-file
                  each control line and each title header line read.

   *PRInt      echo data lines for free-field format titles;
   *PRInt OFF  switch off  (default : 'off')

                  *PRINT implies *LOG

   *USer       call TZUSER for every title bank;
   *USer OFF   switch off  (default : 'off')

   *KIll       kill the run also for slight trouble;
   *KIll OFF   switch off  (default : 'off')

   *ANYWAY     TZINIT is to return to the caller even for fatal errors,
               IQUEST(1) = number of errors, =0 normally

   *FINish     end of input data
\end{verbatim} 

One gives \Lit{*FINISH} only to terminate titles on
the ``standard input file'' (the ``card reader'') in order to
separate from further input data.
For stand-alone files this is not necessary, as the EOF will
terminate correctly.
On the IBM it could even be harmful with concatenated title files.

\newpage
\Filename{H2-TZ-title-header-lines}
\section{Title header lines}
\label{sec:TZ-titleheaderlines}

A title-header line signals the start of data for a new title bank;
it specifies the ID of the title bank,
the way the data are to be read, and further options.
Its format is:

\begin{verbatim}
   *DO  <idh> [idn]  [-<opt><option text>]   [#. comment ]
\end{verbatim}

\begin{DLtt}{1234}
\item[*DO]   in cols 1/4 marks the header line;
\item[<idh>] is the 4 character Hollerith bank \Lit{ID};
\item[<idn>] is the numeric bank \Lit{ID}, integer, if any;
\item[<opt>] is a single option letter,
\item[<otx>] is the option text, if needed.
\end{DLtt}

The following options are available:

\begin{DLtt}{12345678}
\item[-F(format)]  read the data with a single formatted Fortran
               \Lit{READ} statement, using the \Lit{FORMAT} given, which must
               not contain blanks, the brackets must be present.
\item[-Itext]  set the I/O characteristic for the bank, with \Lit{text}:\\
               \Lit{B}, \Lit{I}, \Lit{F} or \Lit{D} if the bank is all bits, integer,
               floating, or double precision;\\
               \Lit{(string)} where \Lit{string} is the I/O characteristic,
               (see routine \Rind{MZFORM});\\
               the default is: type undefined
\item[-Nn] expect \Lit{n} data words, for \Lit{-F} this must be exact; for free-field
               format this may be omitted if less than 2000 words are to be
               read and if the size is not to be changed by \Rind{TZUSER}.
               A bank will initially be lifted to accomodate \Lit{<n>} data words to
               be read and will be shortened when input processing is complete.
\item[-U\lsb{}n\rsb] call \Rind{TZUSER} for this bank, passing \Lit{n} as \Lit{NPARA}, 
               zero if absent.
\end{DLtt}

   For free-field format only:

\begin{DLtt}{12345678}
\item[-C\lsb{}a\rsb\lsb/e\rsb]  use columns \Lit{a/e} only on each line,
                defaults:  1 for a,   80 for e\\
                Ex.:  \Lit{-C11/72}   use only the text on columns 11 to 72
                      on each line, except: \Lit{'*'} in col. 1 is forbidden
\item[-E\lsb{}n\rsb]  expect exactly \Lit{<n>} words (as given by \Lit{-Nn}).
                If less or more data words are found, \Rind{TZINIT} will
                print a message and go to \Rind{ZFATAL} on end of input.
\item[-S\lsb{}n\rsb]  true size of the bank is \Lit{<n>} words (as given by \Lit{-Nn}).
                The bank is not reduced to the number of data words
                actually read.
\item[-A\lsb{}n\rsb\lsb{}C\rsb\lsb{}W\rsb]  
                initialize Hollerith handling as though the control-item
                \Lit{#A[n][C][W]}  had been read; see next paragraph.
\end{DLtt} 

\newpage
\Filename{H2-TZ-free-field-input}
\section{Free-field input}
\label{sec:TZfreefieldinput}

In free-field format the data-items are given one after the other,
separated by blanks.
The mode of each data-item is recognized from the way it
has been written by the user;
thus 12 is an integer,
\Lit{#012} is an octal integer (of value 10),
\Lit{12.} is a floating-point number,
and \Lit{:ABC} is Hollerith.
The rules for writing the data items try to strike a balance
between freedom and catching mistakes; they are the following:

\textbf{Integer}:  a string of digits, maybe preceded by + or --
\begin{verbatim}
      Examples:    123  +123  -123  1  0
\end{verbatim}
\textbf{Floating}: a number containing a decimal dot and/or an exponent
\begin{verbatim}
      Examples:       12.     +12.34  -.34   1.  .1  -12.34
                      12E     +12.34E -.34E  1E  .1E -12.34E
                     -12.34E5  12.E5  12E5  12E+5  12E-5
                     -12.34+5  12.+5  12+5  12+5   12-5
\end{verbatim}
\textbf{Double precision}: a number containing an exponent letter \Lit{D}
\begin{verbatim}
      Examples:       12D     +12.34D -.34D  1D  .1D -12.34D
                     -12.34D5  12.D5  12D5  12D+5  12D-5
\end{verbatim} 
\textbf{Binary -- Octal -- Hexadecimal}:  a number preceded
by \Lit{#B}, \Lit{#b}, \Lit{#O}, \Lit{#o}, \Lit{#0}, \Lit{#X} or \Lit{#x},
which will be stored right-justified with zero-fill;
bits beyond the capacity of a computer word are truncated on the left.
\begin{verbatim}
      Examples :  #B10001  #b11101  #b1111111000111000011111111
                  #017777  #O17777  #o33211234567 (32 bits!)
                  #X177AB  #x17cde  #xffffFFFF    (32 bits!)
\end{verbatim} 
\textbf{BCD}:  must start with  \Lit{'} or \Lit{"} or \Lit{:} thereby
selecting the delimiting character, followed by the Hollerith string,
terminated with the delimiter (which is ``\Lit{blank}'' for ``:'').
\begin{verbatim}
      Examples:    :ABCD    :PI+   :K*   :K(1430)   :RATHERLONGSTRING
                    'AB CD'  :it's        "'ab' 'cd'"
\end{verbatim} 
\textbf{Repeat count}:  indicates the number of times which the data item which
follows immediately is to be stored.
It is an unsigned integer larger than 1 followed by ``\Lit{*}''.
\begin{verbatim}
      Examples:  100*0   4* -.0379   3*:PI-   2* :K(1430)  2*:LONGSTRING
\end{verbatim} 
\textbf{Comment}:    can be either trailing or interspersed.
A comment starts with  \Lit{#}.  and stops with the next  \Lit{#}  or the end-of-line.
\begin{verbatim}
      Example:   12  #. any text, but not a number-sign  #  13
\end{verbatim} 

\newpage
\subsection*{Hollerith handling}

Hollerith text is by default stored in A4 format,
thus the data-item \Lit{":ABCDEF"} gives on all machines 2 words
containing \Lit{"ABCD"} and \verb'"EF  "' with blank-fill.
If this is not what is needed it can be changed either with
the ``control item'' \Lit{#A} or with the title header line option \Lit{-A},
which have the same syntax.

Variable length strings could be awkward to handle,
therefore one can ask \Rind{TZINIT} to store the Hollerith string preceded
by a word count as an extra integer word,
optionally preceded in turn by a character count:
\begin{verbatim}
Header-line option:   -A[n][C][W]
      Control item:   #A[n][C][W]

   with   n:  if given it selects An representation,
               if absent or if n is larger than the maximum number
               of characters per word, the data are stored continuous
               without trailing blanks in each word.

          C:  if given an extra word is stored specifying the
               number of characters in the string, not counting
               trailing blanks in any word.

          W:  if given an extra word is stored specifying the
               number of words occupied by the BCD string,
               excluding itself and the character count, if present.

Example:  #A4CW  :LONGSTRING

          would give 5 words:
          IQ(L)   = 10       character count
          IQ(L+1) = 3        word count
          IQ(L+2) = 4HLONG
          IQ(L+3) = 4HSTRI
          IQ(L+4) = 4HNG
\end{verbatim} 
\newpage

\subsubsection{Other control items}

\textbf{\#Double } instructs \Rind{TZINIT} to read and store all floating-point
numbers that follow as double-precision numbers.

\textbf{\#Normal } reverts to normal, cancelling the effect
of a previous \Lit{#D} for the numbers that follow the \Lit{#N}.

Control-items must be blank-terminated like data items
and may be freely mixed with data items.
They must not occur between a multiplier and its data item.

The option selected by a control item applies to all following
data until changed again by a new control item.

As an example of usage, suppose we have this title bank:

\begin{verbatim}
      *DO FZO 21  -i(3I *H 1I *H)
      #. mode nwrec  opt    name #   #ACW
            4     0  :TLO3  :/dev/mt12
\end{verbatim}

giving the parameters for a file to be used by FZ of Zebra.
(The I/O characteristic is only useful if one wanted Zebra
 to print this bank, which is unlikely in this case.)

The program could digest this as follows:
\begin{verbatim}
      CHARACTER    CHOPT*8, CHNAM*80

      LUN = 21
      CALL TZFIND (0, LT, 'FZO ', LUN, 0)
      IF (LT.EQ.0)                 GO TO no output
      MODE   = IQ(LT+1)
      NWREC  = IQ(LT+2)
      LOPT   = LT+3
      LNAM   = LOPT + IQ(LOPT+1) + 2
      NCHOPT = IQ(LOPT)
      NCHNAM = IQ(LNAM)
      CALL UHTOC (IQ(LOPT+2),99, CHOPT, NCHOPT)
      CALL UHTOC (IQ(LNAM+2),99, CHNAM, NCHNAM)
      CALL MZDROP (0,LT,'.')

      IF (MODE ...
               ...
        ELSEIF (MODE.EQ.4)  THEN
          IF (NWREC.EQ.0)  NWREC = 5760
          CALL CFOPEN (IQUEST(1),1,NWREC,'w',0,CHNAME(1:NCHNAM),ISTAT)
          CALL FZFILE (LUN,NWREC,CHOPT(1:NCHOPT))
        ELSEIF (MODE ...
                     ...
        ENDIF
\end{verbatim} 

\newpage
\Filename{H2-TZ-fortran-formatted-input}
\section{Fortran formatted input}

With Fortran formatted input all the data for the complete title bank
are read with a single \Lit{READ} statement using the format given
on the header line.
For this to be possible the exact number of words to be read
must be specified with the \Lit{-N} option.

\subsection*{Example}

\begin{verbatim}
       *DO  FIDU 1 -IF -N37 -F(10X,F10.0/(10X,3F10.0))
                   12.
        1 4   1     1.     -17.2475  -18.6369
        1 4   2     2.     -17.2595   -4.2708
        1 4   3     3.       -.7761   -4.3177
        1 1   1    11.     -13.8081  -15.9118
        1 1   2    12.     -13.7343   -7.5253
        1 1   3    13.      -4.8669   -7.5898
        1 1   4    14.      -4.8931  -15.9473
        1 2   1    21.     -13.0353  -15.6000
        1 2   2    22.     -12.8979   -7.3464
        1 2   4    24.      -3.7895  -15.6767
        1 3   3    33.      -2.6674  -17.2822
        1 3   4    34.      -3.1636  -17.8522
       *.-------1---------2---------3---------4---------5
\end{verbatim}

The option \Lit{-IF} specifying the I/O characteristic
is important in this case.
It causes the execution of a Fortran \Lit{READ} statement with an I/O list
of type \Lit{REAL};
on some machines Fortran is not able to read acting solely under
\Lit{FORMAT} control, ignoring the type of the I/O list.
