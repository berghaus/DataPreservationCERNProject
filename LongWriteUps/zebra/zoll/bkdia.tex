\Filename{H1-DIA-Error-diagnostics}
\chapter{Error Diagnostics}

\Filename{H2-DIA-general-principles}
\section{General information}

Zebra has been programmed to be as helpful as reasonably
possible in the task of detecting and diagnosing errors.
Depending on the kind of the error,
three different approaches are used :
for the convenience of the user we have made it a general rule
that he does not have to test on the success of a request to Zebra,
the return from the CALL to Zebra implies successful completion.
For example : if the user calls \Rind{MZLIFT} he can be sure that
the bank has been created if he receives control on the next
Fortran statement;
if \Rind{MZLIFT} is in trouble, either because the parameters supplied
by the user are faulty,
or because there is no free memory left,
Zebra will not return to the calling routine,
but take an escape road,
by calling \Rind{ZFATAL} in the first case to stop the program with
diagnostics,
or by calling \Rind{ZTELL} in the second case to allow the user
to re-gain control at the top-level to handle the problem.

The exceptions to this general rule are again dictated by the
convenience of usage,
since there are cases where it is necessary to be able to
handle errors in-line to the code as a matter of routine.
This is realized with the first approach:

\subsection*{Return-status codes}

This approach is used in all instances where "errors" are part of
the normal processing.
The obvious example is the routine \Rind{FZIN} which must allow the
user to handle read errors in particular, exceptions in general.

The status-return code
is always in the word \Lit{\IQUEST(1)} of \Lit{/QUEST/IQUEST(100)}.
This word is zero if the request has been completed successfully;
a non-zero value indicates an exception.
The value is positive for a normal exception (such as end-of-file),
the value is negative for errors (such as read errors);
the particular value indicates which exception has occurred.
The significance of the status-return code is part of the
specifications of the routine.

For an error status, other words in \FCdef{/QUEST/} carry more
information to identify the problem,
the exact details are found in the present paper
under the name of the routine in question.

\subsection*{Exit to \Rind{ZTELL}}

    This approach is used in the instances where Zebra cannot
satisfy a request, without the program being really at fault.
The obvious example is a request for memory, such as with \Rind{MZLIFT},
which cannot be satisfied even after garbage collection.
The exit to \Rind{ZTELL} allows the user at the high level to take
control away from the failing low level of his program.
The first parameter to \Rind{ZTELL} is an integer identifying the
cause of the problem; this \Lit{ID} is 99 for 'no memory'.
On page~\pageref{sec:diaztell} a list of
the Zebra routines which call \Rind{ZTELL} is given,
together with their identifiers;
further details are found under the name of the calling routine.

\subsection*{Exit to \Rind{ZFATAL}}

    This approach is used to catch programming mistakes,
such as faulty parameters in a Zebra call.
Also, some Zebra routine may detect that the user has overwritten
some system information in a bank,
thereby destroying the sequential chaining from one bank in
memory to the physically next bank
('bank chaining clobbered').

    Any error of this kind is trapped as soon as it is detected,
information to localize the problem is loaded into
\FCind{/QUEST/} and control is transferred to \Rind{ZFATAL}.
\Rind{ZFATAL} will print the name of the routine which has called it
(if \Rind{ZFATAL} is reached by in an internal transfer within Zebra)
and the diagnostic information,
whose significance is found in the present paper under
the name of the routine which catches the error.

    The routine which detects the problem is not necessarily the
routine actually called by the user.
For example : the user may call \Rind{MZLIFT} which in turn may need to
call the garbage collector;
one of its routines may discover that the memory is invalid,
and it will transfer control to \Rind{ZFATAL}.
\Rind{ZFATAL} will print its name, which the user may never have
heard of.
In order to be able to tell the user which routine he
actually called,
Zebra keeps an internal trace-back stack which is printed by \Rind{ZFATAL}.
If the Fortran trace-back is available on a particular
machine, this will also be printed as it contains information
very useful to localize the problem.

This chapter gives the details on error information
under the name of the routine in question, sorted alphabetically.

\Filename{H2-DIA-FZFILE}
\section{Diagnostics for FZFILE}
\Shubrz{FZFILE}{(LUN,LREC,options)}

\subsection*{\Rind{ZFATAL} termination}

Messages printed via \Rind{ZFATAM}:

\begin{verbatim}
       FZFILE - no dynamic store initialized.

          if MZSTOR has not been called at least once before
          the first call to FZFILE.

       FZFILE - File already open.

          if FZFILE has already been called for this unit without
          it having been closed by one of the FZEND routines
          with the T option.

       FZFILE - LUN invalid.

          if LUN does not contain a small positive integer.

       FZFILE - Requested format not available.

          if the format is not available, either not at all or
          not on the particular Zebra library used.

       FZFILE - L mode and IQUEST(1)=0 on call.

          normally this means that CFOPEN failed to open
          the file for reading in C library mode.
\end{verbatim}

\Filename{H2-DIA-FZDIA-FZIN-FZCOPY}
\section{Diagnostics for FZDIA: FZIN - FZCOPY}

\begin{flushleft}
\Shubrz{FZIN}(LUN,IXDIV,LSUP,JBIAS,CHOPT,NWUH,IUHEAD)}\\
\Shubrz{FZCOPY}{(LUNIN,LUNOUT,IEVT,CHOPT,NIO,NWUH,IUHEAD)}
\end{flushleft}

If an exception is found by one of the routines which handle
an \Rind{FZIN} request, or an input request for \Rind{FZCOPY},
it stores information into \IQUEST{} which is then handled
by \Rind{FZDIA}, passing the relevant information back to the caller.

The primary status information is :

\begin{verbatim}
 IQUEST(1)= -1  BAD CALLING : options T/A/D given, but no pending d/s
            -2  NOT ENOUGH SPACE in the store to receive the data
                                 and the table
            -3  BAD DATA, data corrupted or file written by user
                          output routine not yet fully debugged
            -4  BAD CONSTRUCTION, maybe not a ZEBRA file
            -5  READ ERROR, error return from the Fortran READ
                            or maybe some other system read routine;
                            the error code, if any, is in IQUEST(14)
            -6  for 2 consecutive errors
            -7  for 3 consecutive errors
            -8  -- etc --
 IQUEST(2)= number of logical records read so far
 IQUEST(3)= number of physical records (exchange file format only)
\end{verbatim}

The detailed error information is in :

\begin{verbatim}
 IQUEST(11)= -1, ..., -5  as above
       (12)= error number JERROR as shown below
       (13)= LRTYP, type of the last record read :
                    0 : not known
                    1 : start or end of run
                    2 : pilot record for start event
                    3 : pilot record
                    4 : pilot continuation record (native file format)
                    4 : bank material continuation record (exchange fifo)
               5 or 6 : dummy padding record
                    7 : bank material record (native file format)
                    8 : last bank material record in the d/s (native fifo)
   (14->17)= further information related to the error
\end{verbatim}

\Rind{FZDIA} prints an error message, showing \Lit{LUN} 
and the contents of \IQUEST,
and returns to the caller, expecting he will test on \Lit{IQUEST(1)}.

In the following the error numbers \Lit{JERROR} are shown, giving the name
of the lower level routine which detected the error, and the meaning
of \Lit{IQUEST(14...)} if they contain useful further information.

\subsection*{\Rind{FZIN}}

\begin{verbatim}
   JERROR = 10  FZIN called for Channel or Memory mode,
                user routine or user memory not yet connected
   JERROR = 11  multiple options T A D not allowed
   JERROR = 12  no pending d/s for entry with T A D options
   JERROR = 13  no segment table for entry with D option :
                either FZIN had not been called with the T option,
                or the table has meanwhile been overwritten by
                some other call to ZEBRA
   JERROR = 14  options T A D not allowed together with R 2 3 4
\end{verbatim}

\subsubsection*{\Lit{FZIMTB} -- table builder}

\begin{verbatim}
   JERROR = 14  invalid division index in segment table
                IQUEST(14)= JS, the segment number
                IQUEST(16)= IQSEGD(JS)

   JERROR = 15  NQSEG has been changed by the user
                IQUEST(14)= NQSEG on entry
                IQUEST(15)= number of words in the segment table
                    = 3 times the number of segments

   JERROR = 21  not enough space
\end{verbatim}

\subsubsection*{\Lit{FZIREL} -- relocator}

\begin{verbatim}
   JERROR = 31  segment table tries to overshoot rel. table
   JERROR = 32  segment limit does not match a rel. table entry
   JERROR = 33  ends of segment and rel. tables do not match
   JERROR = 34  bank chaining clobbered in the input data
\end{verbatim}

\subsubsection*{\Lit{FZIFFN} -- reading file format native for \Rind{FZIN}}

\begin{verbatim}
     131 -> 133  record of unexpected record type read
   JERROR = 131  expect pilot continuation for text vector
   JERROR = 132  expect pilot continuation for table
   JERROR = 133  expect bank material

   JERROR = 134  end of segm skipped does not coincide with LR
                IQUEST(14)= segment number
                IQUEST(15)= size of the segment
                IQUEST(16)= number of words mis-match

   JERROR = 135  end of segm read does not coincide with LR
                IQUEST(14)= segment number
                IQUEST(15)= size of the segment
                IQUEST(16)= number of words mis-match

   JERROR = 136  last bank material record needed is not type 8
   JERROR = 137  emergency-stop record seen

   JERROR = 141  LRTYP invalid when hunting for pilot

   JERROR = 142  check-word in word 1 of the pilot is wrong
                IQUEST(16)= check-word read from the file
                IQUEST(17)= check-word expected (12345.0 floating)

   JERROR = 143  control wd of I/O char. for user header invalid
                IQUEST(14)= NWUHCI (size I/O char. + header vector)
                IQUEST(16)= first word of the characteristic

   JERROR = 144  NWSEG control-word in the pilot is wrong
                IQUEST(14)= NWSEG, the length of the segment table

   JERROR = 145  text vector NWTXI words longer than its record
                IQUEST(14)= NWTX control-word in the pilot
                IQUEST(15)= NWR - NWDONE, words available
                IQUEST(16)= NWDONE words in the record before the text

   JERROR = 146  number of early table words wrong
                IQUEST(14)= NWTAB control-word in the pilot
                IQUEST(15)= NTBE, number of early words inferred

   JERROR = 147  table end does not coincide with LR
                IQUEST(14)= NWTAB control-word in the pilot
                IQUEST(15)= LIN - LQTE words of mis-match

   JERROR = 151  read error, the IOSTAT error code is returned :
\end{verbatim}
                IQUEST(14)= error code, if any

\subsubsection*{\Lit{FZIPHR} -- read physical records, exchange format sequential}

\begin{verbatim}
   JERROR = 201  Steering block expected, but control-words invalid;
                 if the log level is -1 or more they are printed

   JERROR = 202  Block size (physical record size) wrong
                IQUEST(14) = expected size, as specified to FZFILE
                IQUEST(15) = size seen in word 5 of the steering block

   JERROR = 204  Break in block sequence number
                IQUEST(14) = expected sequence number
                IQUEST(15) = sequence number seen

     205 -> 208  Fast burst stopped by ...
   JERROR = 205  ... by unusable start/end-of-run
   JERROR = 206  ... by unusable steering block
   JERROR = 207  ... by usable start/end-of-run in unusable block
   JERROR = 208  ... by usable steering block
                IQUEST(14/15/16) = word 6/9/10 of the steering block
                 (a block is 'unusable' if only the first few words
                 have been read, but not the whole block; this can
                 happen if FZIN is instructed to skip the current d/s)

   JERROR = 209  emergency stop record seen

   JERROR = 215  read error, the IOSTAT error code is returned :
                IQUEST(14)= error code, if any
\end{verbatim}

\subsubsection*{\Lit{FZIREC} - handle logical records, exchange format}

\begin{verbatim}
   JERROR = 221  LR overshoots steering block control
                IQUEST(14) = # of words consumed in current block
                IQUEST(15) = # of words missing to complete LR

   JERROR = 222  LR undershoots steering block control
                IQUEST(14) = # of words consumed in current block
                IQUEST(15) = # of words before start of next LR

   JERROR = 224  LR type 1,2,3 seen when 4 expected
                IQUEST(14) = # of words before start of next LR

   JERROR = 225  Faulty LR type
   JERROR = 226  Faulty LR length
   JERROR = 227  LR of type 1,2,3 must start on steering block
   JERROR = 228  Padding record longer than one physical record
   JERROR = 229  More than 4 padding records in a row
                IQUEST(14) = # of words before start of next LR
                IQUEST(15) = # of words missing to complete LR
                IQUEST(16) = LR length
\end{verbatim}

\subsubsection*{\Lit{FZIFFX} -- reading file format exchange for \Rind{FZIN}}

\begin{verbatim}
   JERROR = 240  FZIN called for Channel or Memory mode,
                 user routine or user memory not yet connected

   JERROR = 241  check-word in word 1 of the pilot is wrong
                IQUEST(16)= check-word read from the file
                IQUEST(17)= check-word expected (12345.0 floating)

   JERROR = 242  control wd of I/O char. for user header invalid
                IQUEST(14)= NWUHCI (size I/O char. + header vector)
                IQUEST(16)= first word of the characteristic

   JERROR = 243  NWSEG control-word in the pilot is wrong
                IQUEST(14)= NWSEG, the length of the segment table

   JERROR = 251  inconsistent bank parameters
   JERROR = 252  link part of bank overshoots segment end
   JERROR = 253  data part of bank overshoots segment end
   JERROR = 254  short dead region overshoots segment end
   JERROR = 255  bank material does not end exactly with LR
\end{verbatim}

\subsubsection{\Lit{FZIPHA} -- read physical records, exchange alfa}

\begin{verbatim}
   JERROR = 301  Control-words of steering block invalid;
                 if the log level is -1 or more they are printed

   JERROR = 302  Block size (physical record size) wrong
                IQUEST(14) = expected size, as set by FZFILE
                IQUEST(15) = size seen in word 5 of the steering block

   JERROR = 303  Unexpected fast record
   JERROR = 307  Unexpected and faulty steering block

   JERROR = 308  Fast burst stopped by unexpected steering block
                IQUEST(14) = word 6
                IQUEST(15) = word 9
                IQUEST(16) = word 10 of the steering block
\end{verbatim}

\subsubsection{\Lit{FZIASC} -- crack one physical alfa record for \Lit{FZIPHA}}

\begin{verbatim}
   JERROR = 309  read error, the IOSTAT error code is returned :
                IQUEST(14)= error code, if any
\end{verbatim}

   The following error codes simply indicate that the text on the file
   has been corrupted, which particular code will appear is a matter
   of chance and is of no interest to the user :

\begin{verbatim}
   JERROR = 310  Invalid character in column 1
   JERROR = 311  Record shorter than expected
   JERROR = 312  Faulty type code
   JERROR = 313  Faulty numeric value, > 31
   JERROR = 314  Record longer than expected
   JERROR = 315  Repetition count overshoots record end
   JERROR = 316  Repetition count negative
   JERROR = 317  Double open square bracket
   JERROR = 318  Record shorter than expected
   JERROR = 319  Check-sum error
   JERROR = 320  Illegal combination =< or <=
\end{verbatim}

subsection*{\Rind{FZCOPY}}

\begin{verbatim}
   JERROR = 401  input/output different data format
   JERROR = 402  native input record length too long
   JERROR = 403  input/output both channel mode
   JERROR = 404  Alfa mode not allowed
   JERROR = 409  no 'pending' d/s on input
\end{verbatim}

\subsubsection*{\Lit{FZCFFN} -- reading file format native for \Rind{FZCOPY}}

\begin{verbatim}
   JERROR = 433  record of unexpected record type read
   JERROR = 434  table or d/s data longer than expected
   JERROR = 435  premature LR type 8
   JERROR = 436  last bank material record needed is not type 8
   JERROR = 437  emergency stop record seen
   JERROR = 451  (Fortran) read error
               IQUEST(14) = IOSTAT error code
\end{verbatim}

\subsubsection*{\Lit{FZCFFX} -- reading format exchange for \Rind{FZCOPY}}

\begin{verbatim}
   JERROR = 455  bank material does not end exactly with LR
\end{verbatim}

\subsubsection*{\Lit{FZIPHD} -- read physical records, exchange direct-access}

\begin{verbatim}
   JERROR = 501  Steering block expected, but control-words invalid;
                 if the log level is -1 or more they are printed

   JERROR = 502  Block size (physical record size) wrong
                IQUEST(14) = expected size, as specified to FZFILE
                IQUEST(15) = size seen in word 5 of the steering block

   JERROR = 504  Break in block sequence number
                IQUEST(14) = expected sequence number
                IQUEST(15) = sequence number seen

   JERROR = 508  Fast burst stopped by usable steering block
                IQUEST(14/15/16) = word 6/9/10 of the steering block

   JERROR = 509  emergency stop record seen

   JERROR = 514  seek error, the IOSTAT error code is returned :
                IQUEST(14)= error code, if any

   JERROR = 515  read error, the IOSTAT error code is returned :
                IQUEST(14)= error code, if any
\end{verbatim}

\subsubsection*{\Lit{FZIPHM} -- read physical records, exchange memory}

\begin{verbatim}
   JERROR = 521  Steering block expected, but control-words invalid;
                 if the log level is -1 or more they are printed

   JERROR = 522  Block size (physical record size) wrong
                IQUEST(14) = expected size, as specified to FZFILE
                IQUEST(15) = size seen in word 5 of the steering block

   JERROR = 523  Block size larger than buffer
                IQUEST(14) = expected size, as specified to FZFILE
                IQUEST(15) = size seen in word 5 of the steering block
\end{verbatim}

\Filename{H2-DIA-FZITRX-FZOTRX}
\section{Diagnostics for FZITRX + FZOTRX}

\Shubrz{FZITRX}{}

This routine is called from \Rind{FZIN} etc. to convert arrays of data
from tranport to internal format.
If it has problems it prints a message for the last bad number
in each array and bumps the corresponding counter in the
statistics for the file. Since this very low level routine
has no information  on the context of the data it is converting
it cannot help the user on localizing the problem exactly.

The message is:

\begin{verbatim}
      FZITRX.  LUN= lun, Conversion problem:  n1 n2 n3

with the 3 numbers n1, n2, n3, printed in hexadecimal:

          n1 = -1  faulty sector control word
               -2  odd number of words for double precision
                2  conversion for integer
                3  conversion for floating single
                4  conversion for double precision

          n2   address local to the array of
          n3   the word in trouble before conversion
\end{verbatim}

\Shubrz{FZOTRX}{}

Same as above but for \Rind{FZOUT}.

\Filename{H2-DIA-FZLOC}
\section{Diagnostics for FZLOC}

\Shubrz{FZLOC}{(LUN,IOMODE)}

This routine is called from the FZ routines to access the control
information for the needed stream.
\Rind{FZLOC} checks whether the access is legal.

\subsection*{\Rind{ZFATAL} termination}

\begin{verbatim}
  case 1 :  WRITE after READ without switching by FZENDI
  case 2 :  READ after WRITE without switching by FZENDO
  case 3 :  access permission fault, for example : trying to
            write to a file which is open for input only
  case 4 :  access to a file not declared with FZFILE

the following numbers are given in /QUEST/IQUEST(100)

for cases 1 to 4 :

 IQUEST(11) = LUN, the logical unit number
 IQUEST(12) = I/O mode :   1 : needed for input
                           2 : needed for output

for cases 1 to 2 :

 IQUEST(13) = last activity on this file
              (cf. FZ control-bank decription in book FZ)

      =  0  file unused             10  switch input to output
         1  read start-of-run       11  write start-of-run
         2  read d/s                12  write d/s
         3  read end-of-run         13  write d/s, buffer flushed
         4  read Zebra EoF          14  write end-of-run
         5  read system EoF         15  write end-of-file
         6  read end-of-data        16  write end-of-data
         7  attempted read          17  attempted write beyond EoD
         8  rewind after read       18  rewind after write

for case 3 :

 IQUEST(13) = ready for  input :  1 or 3
                        output :  2 or 3
\end{verbatim}

\Filename{H2-DIA-FZOUT}
\section{Diagnostics for FZOUT}

\Shubrz{FZOUT}{(LUN,IXDIV,LENTRY,IEVENT,options,IOCH,NUH,IUHEAD)}

\subsection*{\Rind{ZTELL} exits}

\begin{verbatim}
      CALL ZTELL (11,1)   LENTRY invalid
                            IQUEST(11) = LENTRY

      CALL ZTELL (12,1)   bank chaining clobbered
                            IQUEST(11) = store number
                            IQUEST(12) = division number

      CALL ZTELL (13,1)   not enough space to construct rel. table

      CALL ZTELL (14,1)   medium 'memory' only : user memory too small
                          Starting from Zebra version 3.62 the following
                          will be provided :
                            IQUEST(8) = size of the user memory
                            IQUEST(9) = number of words needed
\end{verbatim}

\subsection*{\Rind{ZFATAL} termination}

messages printed via \Rind{ZFATAM}:

\begin{verbatim}
    FZOUT - IOCH invalid.
                   faulty I/O characteristic for user header

    FZOUT - Faulty parameter IEVENT.
                   only 0 or 1 are allowed
                   IQUEST(11) = IEVENT

    FZOUT - Going beyond Eod.
                   second attempt to read beyond EoD

    FZOUT - User routine / buffer not connected

    FZOPHR - error from IORITE.
                   IQUEST(11) = IOPACK error code
\end{verbatim}

\Filename{H2-DIA-LZFID-LZSCAN}
\section{Diagnostics for LZFID - LZSCAN}

\begin{flushleft}
\Shubrz{LZFIDH}{(IXDIV,IDH,LGO)}\\
\Shubrz{LZFID}{(IXDIV,IDH,IDN,LGO)}\\
\Sfuncz{LNEXT}{{LNEXT = LZSCAN (IXDIV,LGO)}
\end{flushleft}

\subsection*{\Rind{ZFATAL} termination}

\begin{verbatim}
  case 1 :  LGO (if non-zero) does not point into the division IXDIV
  case 2 :  LGO is not a valid bank address
  case 3 :  bank chaining clobbered at address LN
\end{verbatim}

the following numbers are given in \FCind{/QUEST/}

for cases 1 to 3 :

\begin{verbatim}
 IQUEST(11) = IXDIV
 IQUEST(12) = IDH
 IQUEST(13) = IDN   (zero in case of LZFIDH)
 IQUEST(14) = LGO
 IQUEST(15) = LSTA, start adr of the division
 IQUEST(16) = LEND,   end adr of the division

for case 3 :

 IQUEST(17) = LN
\end{verbatim}

\Shubrz{MZBOOK}{(IXDIV,L*,LSUP,JB, IDH,NL,NS,ND,IOD,NZERO)}

See \Rind{MZLIFT}

\Filename{H2-DIA-MZCHNB}
\section{Diagnostics for MZCHNB}

\Shubrz{MZCHNB}{(L)}

This routine is called from Zebra routines which take as input
a link which may be updated for the user, like for example \Rind{MZLIFT}.
Such a link must not be a link in a bank, thus

\begin{verbatim}
      CALL MZLIFT (IXDIV, L, ...)         is allowed,
      CALL MZLIFT (IXDIV, LQ(L-4), ...)   is not.
\end{verbatim}

\Rind{MZCHNB} takes the \Lit{LOCF} of the link passed and checks that that this
does not fall into the bank space of the current store.

\subsection*{\Rind{ZFATAL} termination}

\begin{verbatim}
   case 1 :  LOCF(L) is in the bank space
\end{verbatim}

the following numbers are given in \FCind{/QUEST/}

\begin{verbatim}
   IQUEST(11) = K, the adr of the link relative to the
                   start of store, ie. LQ(K) contains L
   IQUEST(12) = L, the content of the link
\end{verbatim}

\Filename{H2-DIA-MZCOPY}
\section{Diagnostics for MZCOPY}

\Shubrz{MZCOPY}{(IXDVFR,LENTRY,options,IXDVTO,LSUP,JBIAS)}

      If an error exit occurs the space allocated for the copy,
      if any, is released.

\subsection*{\Rind{ZTELL} termination}

\begin{verbatim}
      CALL ZTELL (15,1)   with  IQUEST(2) = JERROR,

where JERROR :

        = 1 :  LENTRY invalid
                 IQUEST(11) = LENTRY

        = 2 :  bank chaining clobbered found in the input d/s
                 IQUEST(11) = store number
                 IQUEST(12) = division number

        = 3 :  not enough space to construct the relocation table

        = 4 :  d/s larger than the target space
                 IQUEST(11) = NW, size of the d/s
                 IQUEST(12) = NWMAX, size ot the target space

        = 5 :  d/s to be copied is empty

        = 6 :  bank chaining clobbered found in the copied d/s

        = 7 :  target division not specified
\end{verbatim}

\subsubsection*{Error return}

\begin{verbatim}
      If the P option was selected by the user to permit error return
      IQUEST(1) in COMMON /QUEST/IQUEST(100) is set non-zero :

                 IQUEST(1) = JERROR
\end{verbatim}

\Filename{H2-DIA-MZDIV}
\section{Diagnostics for MZDIV}

\Shubrz{MZDIV}{(IXSTOR,IXDIV*,CHNAME,NW,NWMAX,CHOPT)}

\subsection*{\Rind{ZFATAL} termination}

\begin{verbatim}
  case 1 :  invalid input parameters
  case 2 :  maximum number of divisions possible exceeded
  case 3 :  not enough space to create the division
\end{verbatim}

the following numbers are given in \FCind{/QUEST/}

\begin{verbatim}
for cases 1 to 3 :

 IQUEST(11) = char 1:4
       (12) =      5:8 of the printing name of the new division
       (13) = parameter NW, the primary allocation
       (14) = parameter NWMAX
       (15) = mode option :  0/1 = forward/reverse  (R)
       (16) = kind option
                   1 : user short-range
                   2 : user long-range  (L)
                   3 : package          (P)
       (17) = C option :  0/1 = no/yes

for case 3 :

 IQUEST(18) = number of words missing
       (19) = number of words required
       (20) = no. of words available in the reserve area
       (21) = max. shift of division 2 allowed
\end{verbatim}

\Filename{H2-DIA-MZDROP}
\section{Diagnostics for MZDROP}

\Shubrz{MZDROP}{(IXSTOR,L,CHOPT)}

\subsection*{\Rind{ZFATAL} termination}

\begin{verbatim}
  case 1 :  invalid input parameter L
  case 2 :  LN = LQ(L) is invalid
\end{verbatim}

The following numbers are given in \FCind{/QUEST/}

\begin{verbatim}
for cases 1 and 2 :

 IQUEST(11) = L

for case 2 :

 IQUEST(12) = LN
\end{verbatim}

In either case the parameters of the 'bank' pointed to by the
invalid link \Lit{L} or \Lit{LN} are found in the 
post-mortem dump of \FCind{/MZCN/}.

\Filename{H2-DIA-MZFLAG-MZMARK-MZVOLM}
\section{Diagnostics for MZFLAG - MZMARK - MZVOLM}

\begin{flushleft}
\Shubrz{MZFLAG}{(IXSTOR,L,IBIT,CHOPT)}\\
\Shubrz{MZMARK}{(IXSTOR,L,CHOPT,NID,IDLIST)}\\
\Shubrz{MZVOLM}{ (IXSTOR,L,CHOPT)}
\end{flushleft}

\subsection*{\Rind{ZFATAL} termination}

\begin{verbatim}
  case 1 :  maximum depth exceeded
  case 2 :  invalid input parameter L
  case 3 :  invalid next-link LNEW contained in bank LCUR
  case 4 :  invalid link LNEW contained in link JL of bank LCUR
  case 5 :  the bank pointed to by the next-link LNEW in bank LCUR
            does not point back to LCUR with its origin-link
\end{verbatim}

The following numbers are given in \FCind{/QUEST/}

\begin{verbatim}
for cases 1 to 5 :

 IQUEST(11) = L, the input parameter

for cases 3 to 5 :

       (12) = LNEW = LQ(LCUR+JL)
       (13) = LCUR, the adr of the bank containing the link LNEW

for case 4 :

       (14) = JL, at this link number

for case 5 :

       (14) = LQ(LNEW+2)
\end{verbatim}

The post-mortem dump of \FCind{/MZCN/} 
shows the parameters of the bank
pointed to by \Lit{L} or \Lit{LNEW}.

\Filename{H2-DIA-MZGARB-MZGAR1}
\section{Diagnostics for MZGARB - MZGAR1}

\Shubrz{MZGARB}{(IXGARB,IXWIPE)}

\subsection*{\Rind{ZFATAL} termination}

\begin{verbatim}
  case 1 :  the store numbers of IXGARB and IXWIPE do not agree
\end{verbatim}

The following numbers are given in \FCind{/QUEST/}

\begin{verbatim}
for case 1 :

 IQUEST(11) = store number of IXGARB
 IQUEST(12) = store number of IXWIPE
\end{verbatim}

\Shubrz{MZGAR1}{}

This is called from various parts of the system to force
an automatic garbage collection for lack of space;
\Lit{NQRESV} contains the amount of space available,
a negative value indicates how much space has still to be found.

\subsection*{\Rind{ZFATAL} termination}

\begin{verbatim}
  case 1 :  not enough space can be found for a long range division
\end{verbatim}

The following numbers are given in \FCind{/QUEST/}

\begin{verbatim}
for case 1 :

 IQUEST(11) = NQRESV, the amount of free space available,
                      a negative number in this context
 IQUEST(12) = current store number
 IQUEST(13) = current division number (which may not be meaningful)
\end{verbatim}

\subsection*{\Rind{ZTELL} exit}

\begin{verbatim}
      CALL ZTELL (99,1)   not enough space for a short-range division
                          IQUEST(11:13) as above
\end{verbatim}

\Filename{H2-DIA-MZIOCH-MZIOBK-MZFORM}
\section{Diagnostics for MZIOCH - MZFORM}

\begin{flushleft}
\Shubrz{MZIOCH}{(IOWDS,NWIOMX,CHFORM)}\\
\Shubrz{MZIOBK}{(MMBK,NWBK,CHFORM)}\\
\Shubrz{MZFORM}{(CHIDH,CHFORM,IXIO)}
\end{flushleft}

\subsection*{\Rind{ZFATAL} termination}

\begin{verbatim}
  case 0 :  CHFORM too long with NCH > 120 characters
  case 1 :  CHFORM contains NCHU significant recognized and also
            NINV non-recognized characters
  case 2 :  bad syntax in CHFORM detected at character number JCH
  case 3 :  -t not allowed after /
  case 4 :  no characters allowed after -t
  case 5 :  character / must not occur twice
  case 6 :  more than 15 leading sectors
  case 7 :  NWIO > NWIOMX, to many I/O words needed
            (one word for the control-byte counted)
\end{verbatim}

The following numbers are given in \FCind{/QUEST/}

\begin{verbatim}
for all cases :

 IQUEST(11) = IDH, the Hollerith bank ID, if called from MZFORM/MZIOBK

 IQUEST(14) onwards : the characters of CHFORM with non-significant
                      or invalid characters removed

for case 0 :  (12) = NCH

for case 1 :  (12) = NCHU
              (13) = NINV

for cases 2 to 5 :

              (12) = JCH, as counted in the string of IQUEST(14)

for case 6 :  (12) = number of sectors in all
              (13) = number of leading sectords

for case 7 :  (12) = NWIOMAX, maybe derived from NWBK,
                              in any case not more than 16
              (13) = NWIO words would be needed (at least)
\end{verbatim}

\Filename{H2-DIA-MZIXCO}
\section{Diagnostics for MZIXCO}

\Sfuncz{MZIXCO}{IXCO = MZIXCO (IX1,IX2,IX3,IX4)}

\subsection*{\Rind{ZFATAL} termination}

\begin{verbatim}
  case 1 :  the    store byte of parameter IXn is invalid
  case 2 :  the division byte of parameter IXn is invalid
  case 3 :  parameter IXn does not specify the same store as IX1
\end{verbatim}

The following numbers are given in \FCind{/QUEST/}

\begin{verbatim}
for cases 1 to 3 :

 IQUEST(11) = input parameter IX1
 IQUEST(12) = input parameter IX2
 IQUEST(13) = input parameter IX3
 IQUEST(14) = input parameter IX4
 IQUEST(15) = n for parameter IXn being faulty
 IQUEST(16) = the    store byte of parameter IXn, bits 27 to 32
 IQUEST(17) = the division byte of parameter IXn, bits  1 to 26
\end{verbatim}

\Filename{H2-DIA-MZLIFT-MZBOOK}
\section{Diagnostics for MZLIFT - MZBOOK}

\begin{flushleft}
\Shubrz{MZLIFT}{(IXDIV,L*,LSUP,JB, NAME, NZERO)}\\
\Shubrz{MZBOOK}{(IXDIV,L*,LSUP,JB, IDH,NL,NS,ND,IOD,NZERO)}
\end{flushleft}

\subsection*{\Rind{ZFATAL} termination}

\begin{verbatim}
  case 1 :  faulty NAME parameters
  case 2 :  JB < 1 : LSUP does not point to a live bank
            JB = 1 : LSUP contains an invalid link
  case 3 :  JB < 0 : the bank at LSUP has less than -JB structural links
  case 4 :  LNEXT is invalid
  case 5 :  I/O characteristic for IDH does not exist
  case 6 :  I/O parameter NAME(5) = IOD(1) is invalid
  case 7 :  trying to lift the bank into a wrong division,
            ie. the predecessor or the successor bank, address LSAME,
            is not in the division selected by IXDIV.
  case 8 :  with JB=1 : trying to connect the new bank to the link
                        at LQ(LP) in bank-space
\end{verbatim}

The following numbers are given in \FCind{/QUEST/}

\begin{verbatim}
for all cases :

 IQUEST(11) = input value of LSUP
       (12) = JB
       (13) = IDH  -  NAME(1)
       (14) = NL   -  NAME(2)
       (15) = NS   -  NAME(3)
       (16) = ND   -  NAME(4)
       (17) = IOD  -  NAME(5)

for case 4 :

 IQUEST(18) = LNEXT          if JB < 1 :  LNEXT=LQ(LSUP+JB)
                                   = 1 :  LNEXT=LSUP

for case 7 :

 IQUEST(18) = LSAME

for case 8 :

 IQUEST(18) = LP
\end{verbatim}

The parameters of the bank at \Lit{LCHK} are found in the post-mortem
dump of \FCind{/MZCN/}, where \Lit{LCHK} is \Lit{LSUP} for cases 2 and 3;
for case 4 LCHK is \Lit{LNEXT}.

\Filename{H2-DIA-MZLINK-MZLINT}
\section{Diagnostics for MZLINK - MZLINT}

\begin{flushleft}
\Shubrz{MZLINK}{(IXSTOR,CHNAME,LAREA,LREF,LREFL)}\\
\Shubrz{MZLINT}{(IXSTOR,CHNAME,LAREA,LREF,LREFL)}
\end{flushleft}

\subsection*{\Rind{ZFATAL} termination}

\begin{verbatim}
  case 1 :  the parameters are inconsistent
  case 2 :  the new link-area overlaps with the table of some store
  case 3 :  the new link-area overlaps with some store
  case 4 :  the new link-area overlaps with a previously defined
            link-area for some store
\end{verbatim}

The following numbers are given in \FCind{/QUEST/}

\begin{verbatim}
for cases 1 to 4 :

 IQUEST(11) = char 1:4
       (12) =      5:8 of the printing name of the new link-area
       (13) = absolute adr of parameter LAREA
       (14) = absolute adr of parameter LREF
       (15) = absolute adr of parameter LREFL
       (16) = number of structural links derived
       (17) = total number of links derived

for cases 2 to 4 :

 IQUEST(18) = serial number of the clashing store;
                = 0 for the primary store,
                = 1 for the first secondary store, etc.
       (19) = char 1:4
       (20) =      5:8 of the printing name of the clashing store

for case 4 :

 IQUEST(21) = char 1:4
       (22) =      5:8 of the printing name of the clashing link-area
       (23) = absolute adr of the start of the clashing link-area
       (24)   only for MZLINT : = initial content of LAREA(2)
\end{verbatim}

Case 4 will happen with \Rind{MZLINT} if the user over-writes
the system word in \Lit{LAREA(2)},
in which case the link-area will appear to clash with itself.

\Shubrz{MZMARK}{(IXSTOR,L,CHOPT,NID,IDLIST)}

See \RindMZFLAG}

\Filename{H2-DIA-MZPUSH}
\section{Diagnostics for MZPUSH}

\Shubrz{MZPUSH}{(IXSTOR,*L*,INCNL,INCND,CHOPT)}

\subsection*{\Rind{ZFATAL} termination}

\begin{verbatim}
  case 1 :  invalid input parameter L
  case 2 :  L designates a dead bank
  case 3 :  invalid input parameters INCNL or INCND
  case 4 :  attempt to give up non-zero structural link N
  case 5 :  invalid origin-link in bank at L
\end{verbatim}

The following numbers are given in \FCind{/QUEST/}

\begin{verbatim}
for cases 1 to 5 :

 IQUEST(11) = L

for cases 2 to 5 :

 IQUEST(12) = ID  original parameters of the bank
       (13) = NS
       (14) = NL
       (15) = ND
       (16) = NIO  (number of extra I/O words)
       (17) = INCNL  the input parameters
       (18) = INCND

for case 4 :

 IQUEST(19) = N, the link number
       (20) = the link content

for case 5 :

 IQUEST(19) = K = the origin-link of the bank at L
\end{verbatim}

\Filename{H2-DIA-MZRELB-MZRELL}
\section{Diagnostics for MZRELB - MZRELL}

\Shubrz{MZRELB}{}

This is the relocator for links in banks.

\subsection*{\Rind{ZFATAL} termination}

\begin{verbatim}
  case 1 :  bank chaining clobbered
  case 2 :  a structural link pointing into a dead area, and to be
            bridged, does not contain a valid bank address
\end{verbatim}

The following numbers are given in \FCind{/QUEST/}

\begin{verbatim}

the following numbers are given in /QUEST/

for case 1 :

 IQUEST(11) = LN, the start of the clobbered region

for case 2 :

 IQUEST(11) = LS, the adr of the bank containing the invalid link
 IQUEST(12) = LW, the adr of the word containing the invalid link
 IQUEST(13) = the content of this link

\end{verbatim}
\Shubrz{MZRELL}{}

This is the relocator for links in Link Areas.

\subsection*{\Rind{ZFATAL} termination}

\begin{verbatim}
  case 1 :  a structural link pointing into a dead area, and to be
            bridged, does not contain a valid bank address
\end{verbatim}

The following numbers are given in \FCind{/QUEST/}

\begin{verbatim}
for case 1 :

 IQUEST(11) = absolute adr of the first word of the link area
 IQUEST(12) = relative position within the link area
                       of the link to be bridged
 IQUEST(13) = the content of this link
 IQUEST(14) = char 1:4
       (15) =      5:8 of the printing name of the link area
\end{verbatim}

\Filename{H2-DIA-MZREPL}
\section{Diagnostics for MZREPL}

\Shubrz{MZREPL}{(IXDIV, LIX, CHOPT)}

\subsection*{\Rind{ZFATAL} termination}

\begin{verbatim}
   case 1 :  division not defined
   case 2 :  not enough table space
   case 3 :  LIX is not a valid (index) bank adr
   case 4 :  bank at LIX has less than 2 links or zero data words
   case 5 :  bank LOLD or LNEW of index bank LIX is not in division JDIVI
   case 6 :  LOLD is not a valid bank adr
   case 7 :  LNEW is not a valid bank adr
\end{verbatim}

The following numbers are given in \FCind{/QUEST/}

\begin{verbatim}
for cases 1 to 7 :

   IQUEST(11) = JDIVI, the division number for the old/new banks

for cases 3 to 7 :

   IQUEST(12) =  start adr of division JDIVI
         (13) =    end adr of division JDIVI
         (14) =  LIX, the adr of the current index bank

for cases 4 to 7 :

   IQUEST(15:17) = NL, NS, ND of the current index bank
         (18:19) = LOLD, LNEW of the current index bank
\end{verbatim}

\Filename{H2-DIA-MZSDIV}
\section{Diagnostics for MZSDIV}

\Shubrz{MZSDIV}{(IXDIV,IFLAG)}

This routine is called from everywhere within the system to switch
the current store and/or division according to the parameter \Lit{IXDIV}.

The parameter \Lit{IFLAG} requests :

\begin{verbatim}
      IFLAG = -1  :  set only the store
               0  :  set store and also division,
                     provided a specific division is specified
              +1  :  set store and specific division
\end{verbatim}

\subsection*{\Rind{ZFATAL} termination}

\begin{verbatim}
  case 1 :  the    store byte of parameter IXDIV is invalid
  case 2 :  the division byte of parameter IXDIV is invalid
  case 3 :  a specific division is required,
            but IXDIV is a generic or compound division index
\end{verbatim}

The following numbers are given in \FCind{/QUEST/}

\begin{verbatim}
for cases 1 to 3 :

 IQUEST(11) = IXDIV
 IQUEST(12) = IFLAG
 IQUEST(13) = the store byte of IXDIV

for cases 2 to 3 :

 IQUEST(14) = the division byte of IXDIV
\end{verbatim}

\Filename{H2-DIA-MZSTOR}
\section{Diagnostics for MZSTOR}

\Shubrz{MZSTOR (IXSTOR*, CHNAME, CHOPT, FENCE}
        LQ(1), LQ(LR), LQ(LW), LQ(LIM2), LQ(LAST))}

\subsection*{\Rind{ZFATAL} termination}

\begin{verbatim}
  case 1 :  the sizes of the various regions of the store
            are either inconsistent or too small
  case 2 :  the size of the fence is less than 1 or more than 1000
  case 3 :  the maximum number of stores which can be handled
            by ZEBRA is exceeded
  case 4 :  the new store overlaps with the table of a previous store
  case 5 :  the new store overlaps with a previous store
  case 6 :  the primary store cannot be a split store
\end{verbatim}

The following numbers are given in \FCind{/QUEST/}

\begin{verbatim}
for all cases :

 IQUEST(11) = char 1:4
       (12) =      5:8 of the printing name of the new store
       (13) = size of the fence
       (14) = number of permanent structural links
       (15) = number of permanent links
       (16) = number of low words
       (17) = total size of the store
       (18) = minimum size of the reserve area required
       (19) = minimum size of the data part of the store required

for cases 4 to 5 :

 IQUEST(20) = serial number of the clashing previous store;
                 = 0 for the primary store,
                 = 1 for the first secondary store, etc.
       (21) = char 1:4
       (22) =      5:8 of the printing name of the clashing store
\end{verbatim}

\Filename{H2-DIA-MZTABC}
\section{Diagnostics for MZTABC}
\Shubrz{MZTABC}{}

This is part of the table builder;
it scans all banks of a given memory region (division usually)
to identify the wanted/unwanted banks.
It is called for garbage collection, but also for output.

\subsection*{\Rind{ZFATAL} termination}

\begin{verbatim}
  case 1 :  bank chaining clobbered
\end{verbatim}

The following numbers are given in \FCind{/QUEST/}

\begin{verbatim}
for case 1 :

 IQUEST(11) = LN, the start of the clobbered region
 IQUEST(12) = LIM1, the start of
 IQUEST(13) = LIM2, the end   of the region to be scanned
\end{verbatim}

\Shubrz{MZVOLM}{(IXSTOR,L,CHOPT)}

See \Rind{MZFLAG}

\Filename{H2-DIA-MZWORK}
\section{Diagnostics for MZWORK}

\Shubrz{MZWORK}{(IXSTOR,DFIRST,DLAST,IFLAG)}

\subsection*{\Rind{ZFATAL} termination}

\begin{verbatim}
Let   NQREF  be the number of permanent links in the store
      NL     be the number of wsp links derived from DFIRST
      ND     be the number of data words derived from DLAST

      NEWL = NQREF + NL
      NEWD = NQREF + NL + ND

  case 1 :  the value of parameter IFLAG is illegal
  case 2 :  NEWL is less than NQREF, ie. NL is negative
  case 3 :  NEWD is less than NEWL,  ie. ND is negative
  case 4 :  NEWD reaches beyond the end of division 2
\end{verbatim}

The following numbers are given in \FCind{/QUEST/}

\begin{verbatim}
 IQUEST(11) = NQREF
       (12) = NEWL
       (13) = NEWD
       (14) = IFLAG
\end{verbatim}

\Filename{H2-DIA-MZXREF}
\section{Diagnostics for MZXREF}

\Shubrz{MZXREF}{(IXFROM,IXTO,CHOPT)}

\subsection*{\Rind{ZFATAL} termination}

\begin{verbatim}
  case 1 :  parameter IXFROM is invalid
  case 2 :  IXFROM and IXTO do not specify the same store
  case 3 :  parameter IXTO is invalid
\end{verbatim}

The following numbers are given in \FCind{/QUEST/}

\begin{verbatim}
for cases 1 to 3 :

 IQUEST(11) = IXFROM
 IQUEST(12) = IXTO
 IQUEST(13) = option :  -2/-1/0/1 =  C/A/normal/R
\end{verbatim}

\Filename{H2-DIA-ZSHUNT}
\section{Diagnostics for ZSHUNT}

\ShubrzZSHUNT}{(IXSTOR,LSH,LSUP,JB,IFLAG)}

\subsection*{\Rind{ZFATAL} termination}

\begin{verbatim}
  case 1 :  LSH does not contain a valid bank address
                (see /MZCN/ in the post-mortem dump)
  case 2 :  LSUP does not contain a valid address (see /MZCN/)
  case 3 :  the bank at LSUP has less than -JB str. links (see /MZCN/)
  case 4 :  LSH is not in the same division as the bank at L of the
            target linear stucture
  case 5 :  LNEX, the link 0 of the bank at L, is an invalid address
\end{verbatim}

The following numbers are given in \FCind{/QUEST/}

\begin{verbatim}
for cases 1 to 5 :

 IQUEST(11) = LSH  - the input parameters
       (12) = LSUP
       (13) = JB
       (14) = IFLAG

for cases 4 and 5 :

 IQUEST(15) = L

for case 5 :

 IQUEST(16) = LNEX
\end{verbatim}

\Filename{H2-DIA-ZTELL}
\section{Diagnostics for ZTELL}

\Shubrz{ZTELL}{(ID,IFLAG)}
\label{sec:diaztell}

\begin{verbatim}
   with      ID  identifier, 1 to 99 reserved for system
          IFLAG  severity :
                 0  ZTELL may return to the calling routine
                 1  ZTELL may not return
                 2  fatal error, the run must stop

     --- List of ID's used by the system ---

 11-14  FZOUT, RZOUT via FZOTAB

    15  MZCOPY

    99  MZGAR1, called for automatic garbage collection
\end{verbatim}

\Filename{H2-DIA-ZVERIF}
\section{Diagnostics for ZVERIF}

\Shubrz{ZVERIF}{(IXDIV,IFRETN,CHIDENT)}

This routine will go to \Rind{ZFATAL} when it finds corruption of
the selected areas of the Zebra stores, if the users has
so requested with \Lit{IFRETN}, or if automatic verification
is running.

\Rind{ZVERIF} will print details about all inconsistencies which it finds.

Two separate routines, \Rind{ZVDO1} 
and \Rind{ZVDO2}, first check that the global
Zebra parameters in \FCind{/MZCA/} and the store table for each store
corresponding to \FCind{/MZCC/} are intact.
If this is not the case there is serious trouble,
and therefore these two routines go to \Rind{ZFATAL} unconditionally.















