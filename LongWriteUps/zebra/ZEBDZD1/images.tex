\batchmode

\documentstyle[11pt,epsfig,longtable,changebar,makeidx,lscape]{cernman}
\makeatletter
\makeindex
\newcommand{\FZfile}{FZ~file\index{FZ!Sequential input/output}\index{input/output!FZ}}
\newcommand{\RZfile}{RZ~file\index{RZ!Random input/output}\index{input/output!RZ}}
\newcommand{\IQUEST}{\Lit{IQUEST}\index{IQUEST@{\tt IQUEST}!user communication vector in common {\tt QUEST}}\index{IQUEST@{\tt IQUEST}!error reporting}\index{error reporting!{\tt IQUEST}}\index{QUEST@{\tt QUEST}!user communication common}}
\newcommand{\QUEST}{\Lit{QUEST}\index{IQUEST@{\tt IQUEST}!user communication vector in common {\tt QUEST}}\index{IQUEST@{\tt IQUEST}!error reporting}\index{error reporting!{\tt IQUEST}}\index{QUEST@{\tt QUEST}!user communication common}}
\renewcommand{\ZEBRA}{\textsc{ZEBRA}}
\renewcommand{\Copt}[1]{\texttt{#1}}
\renewcommand{\Ropt}[1]{\texttt{#1}}
\renewcommand{\Rarg}[1]{\texttt{#1}}
\def\condbreak#1{}
\driver{DVIPS}
\setlongtables
\makeindex
\romanfont{times}
\PScommands\setcounter{secnumdepth}{3}
\setcounter{tocdepth}{2}
\newenvironment{landscapebody}{\begin{landscape}}{\end{landscape}}
\makeatletter
\def\LS@rot{\setbox\@outputbox=\vbox{\@rotr\@outputbox}}
\makeatother
\long\def\NODOC#1{#1}
\renewcommand{\ZEBRA}{\textsc{ZEBRA}}\renewcommand{\Copt}[1]{\texttt{#1}}\renewcommand{\Ropt}[1]{\texttt{#1}}\renewcommand{\Rarg}[1]{\texttt{#1}}\def\condbreak#1{}\def\LS@rot{\setbox\@outputbox=\vbox{\@rotr\@outputbox}}\def\NODOC#1{#1}
\makeatother
\newenvironment{tex2html_wrap}{}{}
\usepackage{screen}
\begin{document}
\pagestyle{empty}
\newpage

{\samepage \clearpage $\@@underline {\hbox {\string\pbf\space A choice with wildcard characters}}\mathsurround =\z@ $
}


\newpage

{\samepage \clearpage $\@@underline {\hbox {\string\pbf\space An integer choice value}}\mathsurround =\z@ $
}


\newpage

{\samepage \clearpage $\@@underline {\hbox {\string\pbf\space A bit mask}}\mathsurround =\z@ $
}


\newpage

{\samepage \clearpage $\@@underline {\hbox {\string\pbf\space Description of any bit portion}}\mathsurround =\z@ $
}


\newpage

{\samepage \clearpage $\@@underline {\hbox {\string\pbf\space ZEBRA exchange formatted Holleriths, dates, binaries}}\mathsurround =\z@ $
}


\newpage

{\samepage \clearpage $\@@underline {\hbox {\string\pbf\space Pointers, labels and repetition counts}}\mathsurround =\z@ $
}


\stepcounter{chapter}
\newpage

{\samepage \clearpage \begin{UL}\item produce documentation of bank trees in graphical and alphanumeric
      form.
\item make the documentation available in an interactive session.
\item generate Fortran code with calls to \Rind{MZBOOK}/\Rind{MZLIFT}
      to build the documented banktrees and \Lit{PARAMETER} statements 
      for link and dataword offsets including the needed type declarations.
\end{UL}
}


\stepcounter{section}
\newpage

{\samepage \clearpage \begin{Note}All cards start by \Lit{"*B"} (all other cards are ignored), explanations
are put inside braces \Lit{\lcb\ \rcb}.
\end{Note}
}


\stepcounter{subsection}
\newpage

{\samepage \clearpage \begin{Notes}% latex2html id marker 79
\item The text on the title card \Lit{*B..} is displayed by DZDISP
      (see figure \ref{fig:DZDOCFIG2})
      in the big box representing the data part of a bank. Sometimes  a
      meaningful title is contained as data in the bank itself.
      To display this one may use the syntax:
<tex2html_verbatim_mark>XMP23
      where N1:N2 describes the range of datawords with hollerith text.
      If \Lit{FORMZ} is specified the text is first converted by ZITOH
      from ZEBRA internal character to hollerith format. 
      
      Similiarly the fields where the hollerith Id, the numerical Id,
      and the number of data words are displayed may be filled with text
      actually contained in the bank itself using the syntax:
<tex2html_verbatim_mark>XMP24
 
A picture showing a Monte Carlo decay tree 
in figure \ref{fig:DZDOCFIG7} was produced using this feature.
 
\item The cards: \Lit{*B..} and \Lit{*B.UP} are mandatory, all others are optional. 
\item \Lit{NDATA}, \Lit{NLINKS} or \Lit{NSLINKS} may also be given as integer numbers. 
\item When generating Fortran code,
      defaults are assumed as described in section~\ref{sec:makecode}.
\end{Notes}
}


\stepcounter{subsection}
\stepcounter{subsection}
\stepcounter{subsection}
\stepcounter{paragraph}
\stepcounter{paragraph}
\stepcounter{paragraph}
\stepcounter{paragraph}
\stepcounter{paragraph}
\stepcounter{paragraph}
\stepcounter{subsection}
\stepcounter{subsection}
\stepcounter{subsection}
\newpage

{\samepage \clearpage \begin{Notes}\item Continuation cards should start with \Lit{*B. } with the 
      sequence number and mnemonic field
      left blank. (Note however the special mnemonics like 
      \Lit{BITVALnn}, \Lit{BITSnnmm} etc. described above.)
\item The program will try to put as many characters on an output line 
      as will fit, embedded multiple blanks are removed unless the last 
      character is a \Lit{"/"}, in
      which case no formatting is done, except for the removal of 
      {\bf leading blanks}.
\item The pair \Lit{BKID} and \Lit{UPID}, i.e. the identifiers of a 
      bank and its up-bank should be unique in one RZ~file\index{RZ!Random input/output}\index{input/output!RZ}. 
      \index{link!up}      The system recognises the structure of the banks by looking for 
      \Lit{BKID}s, \Lit{UPID}s and \Lit{ID}s of the banks described 
      in the links. 
\item The value of \Lit{JB} on the card \Lit{"*B.UP UPID-JB"}
      is not used to recognise the structure but is printed only.
\item Repetitions can be nested (3 levels maximum), 
      each \Lit{"*B.REP"} requiring a \Lit{"*B/REP"}.
\end{Notes}
}


\stepcounter{subsection}
\stepcounter{section}
\stepcounter{section}
\newpage

{\samepage \clearpage \begin{ULc}\item Pure PostScript~\cite{Adobe:red2} output on a file \Lit{xxxx.ps} can be
      directly printed on a PostScript laser printer.
      The \Ropt{P} option should be used with the appropriate commands.
\item Pure \LaTeX~\cite{bib-LATEX} formatted output on a file \Lit{xxxx.tex} 
      is obtained with the \Ropt{L} option.
\end{ULc}
}


\stepcounter{subsection}
\newpage

{\samepage \clearpage $-1$
}


\stepcounter{subsection}
\stepcounter{section}
\stepcounter{subsection}
\newpage

{\samepage \clearpage ${ in front of an INTEGER allows hexadecimal input.
\item[Help]       Print instructions
\item[=>LaTeX]    Print current picture in \LaTeX{} format to \Lit{UNIT} (see below)
\item[CHOPT]      Character options for \Rind{DZSHOW} with the
                  following extensions: 
                  \Lit{'W'} bitwise dump of data words with
                  modifiers \Lit{'1'}, \Lit{'2'}, \Lit{'3'} giving the field width, 
                  \Lit{'0'} forces also the bits with zero value to be shown as \Lit{0}, 
                  the default is blank. 
                  The option \Lit{'C'} forces only the data content to be shown 
                  without any other text like sequence numbers etc., 
                  this is useful if the dump should be read by another
                  program (e.g. \Pind{PAW}).
\item[FIRST]      First word to show (with \Rind{DZSHOW})
\item[LAST]       Last word to show (with \Rind{DZSHOW})
\item[UNIT]       Unit for printed output (\Rind{DZSHOW}, \LaTeX, etc.,  (6 = terminal).
                  If a number $} in front of an INTEGER allows hexadecimal input.
\item[Help]       Print instructions
\item[=>LaTeX]    Print current picture in \LaTeX{} format to \Lit{UNIT} (see below)
\item[CHOPT]      Character options for \Rind{DZSHOW} with the
                  following extensions: 
                  \Lit{'W'} bitwise dump of data words with
                  modifiers \Lit{'1'}, \Lit{'2'}, \Lit{'3'} giving the field width, 
                  \Lit{'0'} forces also the bits with zero value to be shown as \Lit{0}, 
                  the default is blank. 
                  The option \Lit{'C'} forces only the data content to be shown 
                  without any other text like sequence numbers etc., 
                  this is useful if the dump should be read by another
                  program (e.g. \Pind{PAW}).
\item[FIRST]      First word to show (with \Rind{DZSHOW})
\item[LAST]       Last word to show (with \Rind{DZSHOW})
\item[UNIT]       Unit for printed output (\Rind{DZSHOW}, \LaTeX, etc.,  (6 = terminal).
                  If a number $
}


\stepcounter{subsection}
\newpage

{\samepage \clearpage \begin{Notes}\item \Lit{IFIRST} and \Lit{LAST} may be 0, in this case all datawords
      of a bank will be listed. 
\item If no documentation for the bank exists, only the contents of the 
      datawords will be shown (\Rind{DZSHOW}).
\item Only two levels of repetitions are supported, the inner level
      must be given explicitly except for cases where the repetition count
      is a simple expression of the current data word.
\end{Notes}
}


\stepcounter{subsection}
\stepcounter{subsection}
\stepcounter{subsection}
\stepcounter{section}
\stepcounter{subsection}
\stepcounter{subsection}
\stepcounter{subsection}
\stepcounter{section}
\stepcounter{subsection}
\stepcounter{subsection}
\stepcounter{subsection}
\stepcounter{subsection}
\stepcounter{subsection}
\stepcounter{chapter}

\end{document}