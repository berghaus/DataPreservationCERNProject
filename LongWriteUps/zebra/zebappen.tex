%%%%%%%%%%%%%%%%%%%%%%%%%%%%%%%%%%%%%%%%%%%%%%%%%%%%%%%%%%%%%%%%%%%
%                                                                 %
%   ZEBRA User Guide -- LaTeX Source                              %
%                                                                 %
%   Appendix chapter - Diagnostics                                %
%                                                                 %
%   The following external EPS files are referenced:              %
%   none                                                          %
%                                                                 %
%   Editor: Michel Goossens / AS-MI                               %
%   Last Mod.:  8 Dec. 1990   mg                                  %
%                                                                 %
%%%%%%%%%%%%%%%%%%%%%%%%%%%%%%%%%%%%%%%%%%%%%%%%%%%%%%%%%%%%%%%%%%%
\chapter{Diagnostics}
\section{General information}
\par Zebra has been programmed to be as helpful as reasonably
possible in the task of detecting and diagnosing errors.
Depending on the kind of the error,
three different approaches are used:
for the convenience of the user we have made it a general rule
that he does not have to test on the success of a request to ZEBRA,
the return from the call to ZEBRA implies successful completion.
\par For example: if the user calls \Rind{MZLIFT} he can be sure that
the bank has been created if he receives control on the next
FORTRAN statement;
\par \Rind{MZLIFT} is in trouble, either because the parameters supplied
by the user are faulty,
or because there is no free memory left,
Zebra will not return to the calling routine,
but take an escape road,
by calling \Rind{ZFATAL} in the first case to stop the program with
diagnostics,
or by calling \Rind{ZTELL} in the second case to allow the user
to re-gain control at the top-level to handle the problem.
\par The exceptions to this general rule are again dictated by the
convenience of usage,
since there are cases where it is necessary to be able to
handle errors in-line to the code as a matter of routine.
\subsection{Return-status codes}
\par This approach is used in all instances where errors are part of
the normal processing.
The obvious example is the routine \Rind{FZIN} which must allow the
user to handle read errors in particular, exceptions in general.
\par The status-return code is always in the word {\tt IQUEST(1)} of
the common block {\tt/QUEST/IQUEST(100)}.
This word is zero if the request has been completed successfully;
a non-zero value indicates an exception.
The value is positive for a normal exception (such as end-of-file),
the value is negative for errors (such as read errors);
the particular value indicates which exception has occurred.
The significance of the status-return code is part of the
specifications of the routine.
\par For an error status, other words in {\tt/QUEST/} carry more
information to identify the problem,
the exact details are found in the present paper
under the name of the routine in question.
\subsection{Exit to \Rind{ZTELL}}
\par This approach is used in the instances where Zebra cannot
satisfy a request, without the program being really at fault.
The obvious example is a request for memory, such as with \Rind{MZLIFT},
which cannot be satisfied even after garbage collection.
The exit to \Rind{ZTELL} allows the user at the high level to take
control away from the failing low level of his program.
The first parameter to \Rind{ZTELL} is an integer identifying the
cause of the problem; this ID is 99 for 'no memory'.
Further details are found
under the name of the calling routine. 
\subsection{Exit to \Rind{ZFATAL}}
\par This approach is used to catch programming mistakes,
such as faulty parameters in a ZEBRA call.
Also, some ZEBRA routine may detect that the user has overwritten
some system information in a bank,
thereby destroying the sequential chaining from one bank in
memory to the physically next bank
('bank chaining clobbered').
\par Any error of this kind is trapped as soon as it is detected,
information to localize the problem is loaded into
{\tt/QUEST/} and control is transferred to \Rind{ZFATAL}.
\Rind{ZFATAL} will print the name of the routine which has called it
(if \Rind{ZFATAL} is reached by in an internal transfer within ZEBRA)
and the diagnostic information,
whose significance is found in the present paper under
the name of the routine which catches the error.
\par The routine which detects the problem is not necessarily the
routine actually called by the user.
For example: the user may call \Rind{MZLIFT} which in turn may need to
call the garbage collector;
one of its routines may discover that the memory is invalid,
and it will transfer control to \Rind{ZFATAL}.
\Rind{ZFATAL} will print its name, which the user may never have
heard of.
In order to be able to tell the user which routine he
actually called,
ZEBRA keeps an internal trace-back stack which is printed by \Rind{ZFATAL}.
If the FORTRAN trace-back is available on a particular
machine, this will also be printed as it contains information
very useful to localize the problem.
\subsection{Further diagnostics}
\par Further diagnostic tools are provided by the DZ part of ZEBRA,
in particular the user may call during bug hunting the routine
DZVERI to verify the integrity of the data in the dynamic store.
\section{Explicit list of messages (see the ZEBRA DIA reference guide)}
%%%%%%%%%%%%%%%%%%%%%%%%%%%%%%%%%%%%%%%%%%%%%%%%%%%%%%%%%%%%%%%%%%%
%                                                                 %
%   ZEBRA User Guide -- LaTeX Source                              %
%                                                                 %
%   Appendix chapter - Glossary                                   %
%                                                                 %
%   The following external EPS files are referenced:              %
%   none                                                          %
%                                                                 %
%   Editor: Michel Goossens / AS-MI                               %
%   Last Mod.:  8 Dec. 1990   mg                                  %
%                                                                 %
%%%%%%%%%%%%%%%%%%%%%%%%%%%%%%%%%%%%%%%%%%%%%%%%%%%%%%%%%%%%%%%%%%%
\chapter{\protect\label{NUTSHEL}ZEBRA in a nutshell}
\section{ZEBRA Glossary}
\begin{description}
\item[Bank]Unit of logically independent block of data
\item[Division]Stores are divided into disjoint parts (divisions)
for reasons of efficiency or logical separation.
Up to 20 divisions per store are allowed.
\item[Garbage collection]Process whereby dispersed unused space
inside a division is recovered by moving all banks which are still
in use (active) such that they become contiguous in memory.
\item[Link]Part of a ZEBRA bank or link area, where pointers to objects
in ZEBRA stores are kept.
\item[Link area]Vector inside a store or in a {\tt COMMON} block which
contains a series of (structural or reference) links associated with that
store. \item[List]Set of structural links containing
pointers to active banks inside a store.
\item[Pointer]Variable containing an address with respect to the beginning of
a ZEBRA store, which allows to access information inside that store.
\item[Reference link]Link containing a pointer to information {\bf anywhere}
inside a ZEBRA store.
This link, as opposed to a structural link, does not describe
any hierarchical inter-dependence between two (or more) ZEBRA objects.
It merely offers the possibility to store in a convenient way
pointers to information, which has to be referenced often,
without having to recalculate the latter each time they are needed.
\item[Structural link]Link containing a pointer to an object in a ZEBRA store,
describing at the same time the structural inter-dependence
between the ZEBRA objects being referenced.
\item[Structure]Hierarchical tree which expresses the logical relations
between ZEBRA objects (banks or other structures).
\item[Store]Vector inside a {\tt COMMON} block where data items (banks) 
are dynamically stored. Up to 16 different stores are allowed.
\item[Wipe]Process whereby a division is emptied of its bank contents
and all references to addresses inside the division are eliminated (zeroed).
\item[ZEBRA]Data manager which allows the user to build a graph
expressing structural and logical relations between data units (banks).
\end{description}
%%%%%%%%%%%%%%%%%%%%%%%%%%%%%%%%%%%%%%%%%%%%%%%%%%%%%%%%%%%%%%%%%%%
%                                                                 %
%   ZEBRA User Guide -- LaTeX Source                              %
%                                                                 %
%   Appendix chapter - Calling Sequences                          %
%                                                                 %
%   The following external EPS files are referenced:              %
%   none                                                          %
%                                                                 %
%   Editor: Michel Goossens / AS-MI                               %
%   Last Mod.:  8 Dec. 1990   mg                                  %
%                                                                 %
%%%%%%%%%%%%%%%%%%%%%%%%%%%%%%%%%%%%%%%%%%%%%%%%%%%%%%%%%%%%%%%%%%%
\section{ZEBRA subroutine calling sequences}
{\tt \begin{tabular*}{\textwidth}{l@{\extracolsep{\fill}}r}
DZAREA(CHTEXT,IXSTOR,CHLA,LLA,CHOPT)&\pageref{SR_DZAREA}\\
DZCHST(CHTEXT,IXSTOR,LBANK,CHOPT,*ISUM*)&\pageref{SR_DZCHST}\\
DZSHOW(CHTEXT,IXSTOR,LBANK,CHOPT,ILNK1,ILNK2,IDAT1,IDAT2)&\pageref{SR_DZSHOW}\\
DZSNAP(CHTEXT,IXDIV,CHOPT)&\pageref{SR_DZSNAP}\\
DZSTOR(CHTEXT,IXSTOR)&\pageref{SR_DZSTOR}\\
DZSURV(CHTEXT,IXSTOR,LBANK)&\pageref{SR_DZSURV}\\
DZVERI(CHTEXT,IXSTOR,CHOPT)&\pageref{SR_DZVERI}\\[2mm]
FZENDI(LUN,CHOPT)&\pageref{SR_FZENDI}\\
FZENDO(LUN,CHOPT)&\pageref{SR_FZENDO}\\
FZIN(LUN,IXDIV,*LSUP*,JBIAS,CHOPT,*NUH*,IUHEAD*)&\pageref{SR_FZIN}\\
FZLOGL(LUN,LOGLEV)&\pageref{SR_FZLOGL}\\
FZLIMI(LUN,ALIMIT)&\pageref{SR_FZLIMI}\\
FZFILE(LUN,LREC,CHOPT)&\pageref{SR_FZFILE}\\
FZOUT(LUN,IXDIV,LENTRY,IEVENT,CHOPT,IOX,NUH,IUHEAD)&\pageref{SR_FZOUT}\\
FZRUN(LUN,NRUN,NUH,IUHEAD)&\pageref{SR_FZRUN}\\[2mm]
LF*= LZBYT(IXSTOR,LLS,IT,JBIT,NBITS)&\pageref{SR_LZBYT}\\
LF*= LZFID(IXDIV,IDH,IDN,LGO)&\pageref{SR_LZFID}\\
LF*= LZFIDH(IXDIV,IDH,LGO)&\pageref{SR_LZFIDH}\\
LF*= LZFIND(IXSTOR,LLS,IT,JW)&\pageref{SR_LZFIND}\\
LF*= LZFVAL(IXSTOR,LLS,VAL,TOL,JW)&\pageref{SR_LZFVAL}\\
LF*= LZLAST(IXSTOR,LLS)&\pageref{SR_LZLAST}\\
LF*= LZLOC(IXDIV,CHIDH,IDN)&\pageref{SR_LZLOC}\\
LF*= LZLONG(IXSTOR,LLS,N,IT,JW)&\pageref{SR_LZLONG}\\[2mm]
MZBOOK(IXDIV,L*,*LSUP*,JB,CHIDH,NL,NS,ND,NIO,NZERO)&\pageref{SR_MZBOOK}\\
MZCOPY(IXFROM,LFROM,IXDVTO,*LSUP*,JBIAS,CHOPT)&\pageref{SR_MZCOPY}\\
MZDIV(IXSTOR,IXDIV*,NAME,NW,NWMAX,CHOPT)&\pageref{SR_MZDIV}\\
MZDROP(IXSTOR,L,CHOPT)&\pageref{SR_MZDROP}\\
MZEBRA(LIST)&\pageref{SR_MZEBRA}\\
MZEND&\pageref{SR_MZEND}\\
MZFLAG(IXSTOR,L,IBIT,CHOPT)&\pageref{SR_MZFLAG}\\
MZFORM(CHIDH,CHFORM,IXIO*)&\pageref{SR_MZFORM}\\
MZGARB(IXGARB,IXWIPE)&\pageref{SR_MZGARB}\\
MZIOBK(MMBK*,NWMM,CHFORM)&\pageref{SR_MZIOBK}\\
MZIOCH(IOWDS*,NWIO,CHFORM)&\pageref{SR_MZIOCH}\\
IXC*= MZIXCO(IX1,IX2,IX3,IX4)&\pageref{SR_MZIXCO}\\
MZLIFT(IXDIV,L*,*LSUP*,JB,MMBK,NZERO)&\pageref{SR_MZLIFT}\\
MZLINK(IXSTOR,NAME,LAREA,LREF,LREFL)&\pageref{SR_MZLINK}\\
MZLINT(IXSTOR,NAME,LAREA,LREF,LREFL)&\pageref{SR_MZLINT}\\
MZLOGL(IXSTOR,LOGLEV)&\pageref{SR_MZLOGL}\\
MZNEED(IXDIV,NWNEED,CHOPT)&\pageref{SR_MZNEED}\\
MZPUSH(IXSTOR,*L*,INCNL,INCND,CHOPT)&\pageref{SR_MZPUSH}\\
MZSTOR(IXSTOR*,NAME,CHOPT,FENCE,LQ(1),LQ(LR),LQ(LW),LQ(LIM2),LQ(LAST))&\pageref{SR_MZSTOR}\\
MZWIPE(IXWIPE)&\pageref{SR_MZWIPE}\\
MZWORK(IXSTOR,DFIRST,DLAST,IFLAG)&\pageref{SR_MZWORK}\\
MZXREF(IXFROM,IXTO,CHOPT)&\pageref{SR_MZXREF}\\
N*= NZBANK(IXSTOR,LLS)&\pageref{SR_NZBANK}\\
N*= NZFIND(IXSTOR,LLS,IT,JW)&\pageref{SR_NZFIND}\\
N*= NZLONG(IXSTOR,LLS,N,IT,JW)&\pageref{SR_NZLONG}\\
\end{tabular*}\\
\newpage
\begin{tabular*}{\textwidth}{l@{\extracolsep{\fill}}r}
RZCDIR(*CHPATH*,CHOPT)&\pageref{SR_RZCDIR}\\
RZCOPY(CHPATH,KEYIN,ICYCIN,KEYOUT,CHOPT)&\pageref{SR_RZCOPY}\\
RZDELT(CHDIR)&\pageref{SR_RZDELT}\\
RZDELK(KEY,ICYCLE,CHOPT)&\pageref{SR_RZDELK}\\
RZEND(CHDIR)&\pageref{SR_RZEND}\\
RZFILE(LUN,CHDIR,CHOPT)&\pageref{SR_RZFILE}\\
RZFREE(CHLOCK)&\pageref{SR_RZFREE}\\
RZFRFZ(LUNFZ,CHOPT)&\pageref{SR_RZFRFZ}\\
RZIN(IXDIV,*LSUP*,JBIAS,KEY,ICYCLE,CHOPT)&\pageref{SR_RZIN}\\
RZINPA(CHPATH,IXDIV,*LSUP*,JBIAS,KEY,ICYCLE*,CHOPT)&\pageref{SR_RZINPA}\\
RZKEYD(NWKEY*,CHFORM*,CHTAG*)&\pageref{SR_RZKEYD}\\
RZKEYS(MAXDIM,MAXKEY,KEYS*,NKEYS*)&\pageref{SR_RZKEYS}\\
RZLDIR(CHPATH,CHOPT)&\pageref{SR_RZLDIR}\\
RZLOCK(CHLOCK)&\pageref{SR_RZLOCK}\\
RZLOGL(LUN,LOGLEV)&\pageref{SR_RZLOGL}\\
RZMAKE(LUN,CHDIR,NWKEY,CHFORM,CHTAG,NREC,CHOPT)&\pageref{SR_RZMAKE}\\
RZMDIR(CHDIR,NWKEY,CHFORM,CHTAG)&\pageref{SR_RZMDIR}\\
RZNDIR(*CHPATH*,CHOPT)&\pageref{SR_RZNDIR}\\
RZOUT(IXDIV,LSUP,KEY,*ICYCLE*,CHOPT)&\pageref{SR_RZOUT}\\
RZPASS(CHPASS,CHOPT)&\pageref{SR_RZPASS}\\
RZPURG(NKEEP)&\pageref{SR_RZPURG}\\
RZQUOT(NQUOTA)&\pageref{SR_RZQUOT}\\
RZRDIR(MAXDIR,CHDIR*,NDIR*)&\pageref{SR_RZRDIR}\\
RZRENK(KEYOLD,KEYNEW)&\pageref{SR_RZRENK}\\
RZSAVE&\pageref{SR_RZSAVE}\\
RZSTAT(CHPATH,NLEVELS,CHOPT)&\pageref{SR_RZSTAT}\\
RZTOFZ(LUNFZ,CHOPT)&\pageref{SR_RZTOFZ}\\
RZVIN(VECT*,NDIM,NFILE*,KEY,ICYCLE,CHOPT)&\pageref{SR_RZVIN}\\
RZVOUT(VECT,NOUT,KEY,*ICYCLE*,CHOPT)&\pageref{SR_RZVOUT}\\[2mm]
ZABEND&\pageref{SR_ZABEND}\\
ZEND&\pageref{SR_ZEND}\\
ZFATAL&\pageref{SR_ZFATAL}\\
ZFATAM(CHTEXT)&\pageref{SR_ZFATAM}\\
ZPHASE(NPH)&\pageref{SR_ZPHASE}\\
ZPOSTM(CHOPT)&\pageref{SR_ZPOSTM}\\
ZPRESS(IXSTOR,LLS)&\pageref{SR_ZPRESS}\\
ZSHUNT(IXSTOR,LSH,LSUP,JB,IFLAG)&\pageref{SR_ZSHUNT}\\
ZSORT(IXSTOR,LLS,JKEY)&\pageref{SR_ZSORT}\\
ZSORTH(IXSTOR,LLS,JKEY)&\pageref{SR_ZSORTH}\\
ZSORTI(IXSTOR,LLS,JKEY)&\pageref{SR_ZSORTI}\\
ZSORV(IXSTOR,LLS,JKEY,NKEYS)&\pageref{SR_ZSORV}\\
ZSORVH(IXSTOR,LLS,JKEY,NKEYS)&\pageref{SR_ZSORVH}\\
ZSORVI(IXSTOR,LLS,JKEY,NKEYS)&\pageref{SR_ZSORVI}\\
ZTELL(ID,IFLAG)&\pageref{SR_ZTELL}\\
ZTOPSY(IXSTOR,LLS)&\pageref{SR_ZTOPSY}\\
\end{tabular*}} % Close \tt also
 
