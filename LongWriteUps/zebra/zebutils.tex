%%%%%%%%%%%%%%%%%%%%%%%%%%%%%%%%%%%%%%%%%%%%%%%%%%%%%%%%%%%%%%%%%%%
%                                                                 %
%   ZEBRA User Guide -- LaTeX Source                              %
%                                                                 %
%   Chapter Utilities                                             %
%                                                                 %
%   The following external EPS files are referenced:              %
%   fshunt1, fshunt2, fshunt3, fshunt4, fshunt5, fshunt6, fshunt7 %
%                                                                 %
%   Editor: Michel Goossens / AS-MI                               %
%   Last Mod.:  8 Dec. 1990   mg                                  %
%                                                                 %
%%%%%%%%%%%%%%%%%%%%%%%%%%%%%%%%%%%%%%%%%%%%%%%%%%%%%%%%%%%%%%%%%%%
\chapter{Utilities}
\section{Operations on data structures}
\subsection{Dropping a bank and its dependents}
\par \Rind{MZDROP} allows one to drop either the {\bf bank}
or the {\bf linear structure} at L ('L' option).
Dropping a bank implies dropping also the whole partial
data structure which depends on this bank downwards.
\par Dropped banks stay in memory with the drop status bit set,
links pointing to them continue to do so,
except for the immediate structural link indicated via the
origin link of the bank at L, which is bridged or reset to zero,
whichever is appropriate.
\Subr{CALL MZDROP (IXSTOR,L,CHOPT)}
\index{bank!drop}
\Idesc
\begin{DL}{MMMMM}
\item[IXSTOR]Index of the store containing the bank to be dropped
\item[L]Address of the bank or linear structure to be dropped
\item[CHOPT]Character variable specifying the action desired:
\begin{DL}{MM}
\item['']Default - Drop the bank at L and its vertical dependents
i.e. the next link of this bank is not followed.
\item['L']Drop the linear structure pointed to by L
i.e. the next link of the bank at L is followed until
all the banks in the linear structure and its dependents have been
dropped.
\item['V']Drop only the partial data structure
dependent vertically on the bank at L, but not the the bank itself.
\end{DL}
\end{DL}
\subsection{Set one of the status bits in the bank of a data structure}
\par By following the structural links,
\Rind{MZFLAG} sets the selected status bit into the status words
of all the banks of the data structure supported
by the vertical links of the specified start bank.
Optionally it can include in the marking
also the banks of the linear structure supported
by link 0 of the start bank and all their dependents.
The start bank itself may or may not be marked.
\Subr{CALL MZFLAG (IXSTOR,L,IBIT,CHOPT)}
\index{bank!status bit}
\Idesc
\begin{DL}{MMMM}
\item[IXSTOR]Index of the store containing the bank to be flagged
\item[L]Address of the start bank supporting the partial
data structure.
No action if {\tt L=0}.
\item[IBIT]The bit number of the status bit to be set (using the convention that
the least significant bit in a word is identified with the number 1).
\item[CHOPT]Character variable specifying the action desired:
\begin{DL}{MM}
\item['']Default - Flag the bank at L (and its vertical dependents),
but the next link of this bank is not followed.
\item['L']Flag the linear structure pointed to by L,
i.e. the next link of the bank at L is followed.
\item['V']Flag only the the partial data structure dependent vertically
on the bank at L, but not the bank itself.
\item['Z']Set to zero the status bit IBIT in each bank to be marked.
In the default case bit IBIT in the status word is set to one.
\end{DL}
\end{DL}
\subsubsection{Example:}
\begin{verbatim}
      CALL MZFLAG (0,LQMAIN,IQDROP,'L')
\end{verbatim}
drops the complete 'Main' data structure, and is equivalent to
\begin{verbatim}
      CALL MZDROP (0,LQMAIN,'L'),
\end{verbatim}
except that it does not alter the contents of {\tt LQMAIN}.
\subsection{Change structural relations}
\par Because of the presence of the reverse pointers,
the operation of moving a bank by relinking from one data structure
to another one is a non trivial operation.
The routine \Rind{ZSHUNT} is provided to execute such an operation.
\par \Rind{ZSHUNT} may be used to modify the links associated with
either a single bank ({\tt IFLAG=0})
or a whole linear structure ({\tt IFLAG=1}), using information provided
by the parameters
{\tt LSUP} and {\tt JBIAS}, which have the same significance as in
\Rind{MZLIFT}.
\Subr{CALL ZSHUNT (IXSTOR,LSH,LSUP,JBIAS,IFLAG)}
\index{data structure!change}
\Idesc
\begin{DL}{MMMM}
\item[IXSTOR]Index of the store containing the bank to be shunted
{\tt IXDIV}, the index of the division containing
the bank to be shunted, may be given instead
\item[LSH]The address of the data structure to be shunted
\item[LSUP]if {\tt JBIAS < 0:}  the address of the new supporting up bank\\
if {\tt JBIAS = 0:}  the address of the new supporting previous bank\\
if {\tt JBIAS = 1:}  the new supporting link
\item[JBIAS]if {\tt JBIAS < 1:}  the link bias in the new supporting bank\\
if {\tt JBIAS = 1:} the origin link in bank {\tt LSH}
will point to link {\tt LSUP}\\
if {\tt JBIAS = 2:}  detach without insertion
\item[IFLAG]if {\tt IFLAG = 0:}  shunt the one single bank at {\tt LSH}\\
if {\tt IFLAG = 1:}  shunt the whole linear structure pointed to by {\tt LSH}
\end{DL}
\par If the bank or the structure to be relinked is in fact inserted
or added into an existing linear structure,
both must be contained in the same division.
\subsubsection{Examples}
\begin{figure}
\epsffile{fshunt1.ps}
\caption{ZSHUNT - The original data structures}
\label{FSHUNT1}
\end{figure}
\par Originally we have the data structures shown in
Figure~\ref{FSHUNT1}, where any bank may support further
dependent partial data structures since
the corresponding vertical structural links are not changed by \Rind{ZSHUNT}.
\par In what follows the notation {\tt Lxx}is used to designate
a link pointing to bank {\tt xx}.
\par Each example below
refers to the starting situation described in Figure~\ref{FSHUNT1}.
\begin{figure}
\epsffile{fshunt2.ps}
\caption{ZSHUNT - Add bank (and dependents) to front of linear structure}
\label{FSHUNT2}
\end{figure}
\begin{verbatim}
      CALL ZSHUNT (0,LA2,LUN,-7,0)
\end{verbatim}
\par This moves a single bank (with is dependents, if any) out of
a linear structure, and inserts it at the head of the linear
structure supported by link -7 of the bank UN.
\begin{figure}
\epsffile{fshunt3.ps}
\caption{ZSHUNT - Move part of linear structure in front of another one}
\label{FSHUNT3}
\end{figure}
\begin{verbatim}
      CALL ZSHUNT (0,LA2,LUN,-7,1)
\end{verbatim}
\par This is the same as in Figure~\ref{FSHUNT2},
except that the (partial) linear
structure starting with bank A2 is relinked.
\begin{figure}
\epsffile{fshunt4.ps}
\caption{ZSHUNT - Move bank into a linear structure}
\label{FSHUNT4}
\end{figure}
\begin{verbatim}
      CALL ZSHUNT (0,LA2,LN2,0,0)
\end{verbatim}
\par This is again the same as in Figure~\ref{FSHUNT2},
but the bank is inserted inside the linear structure,
rather than at its front.
\begin{figure}
\epsffile{fshunt5.ps}
\caption{ZSHUNT - Move a bank to a top level structure}
\label{FSHUNT5}
\end{figure}
\begin{verbatim}
      CALL ZSHUNT (0,LA2,LQMAIN,1,0)
\end{verbatim}
\par This re-links bank A2 to be the first in the top-level linear
structure supported by {\tt LQMAIN}.
\begin{figure}
\epsffile{fshunt6.ps}
\caption{ZSHUNT - Attach a linear structure to a top level link}
\label{FSHUNT6}
\end{figure}
\begin{verbatim}
      CALL ZSHUNT (0,LA1,LHEAD,1,1)
\end{verbatim}
\par Supposing {\tt LHEAD=0} initially; this connects the linear structure
to the (structural) link {\tt LHEAD}, i.e.
the origin link of the header bank A1
points back to the location of {\tt LHEAD}.
\begin{figure}
\epsffile{fshunt7.ps}
\caption{ZSHUNT - Detach a linear structure}
\label{FSHUNT7}
\end{figure}
\begin{verbatim}
      CALL ZSHUNT (0,LA1,0,2,1)
\end{verbatim}
\par This removes the linear structure from its old position
without inserting it into a new one.
This should only be temporary; one should insert the floating
structure into a new position by a second call to \Rind{ZSHUNT}
soon after.
\section{Copy a data structure}
\par A data structure can be copied from memory to memory by using
routine \Rind{MZCOPY}. The data structure can be in one or more divisions
in one store or in ``stand-alone'' memory and can be copied to a division
in the same store, a different one or to ``stand-alone'' memory.
\par The case of ``stand-alone'' or ``flat''
memory copies is intended for communication
between separate processes running on the same computer through sharable
memory (formally FORTRAN common blocks). The information must belong to
a single data structure
(see the ZEBRA reference manual for more details).
\Subr{CALL MZCOPY (IXDVFR,LENTRY,IXDVTO,*LSUP*,JBIAS,CHOPT)}
\index{data structure!copy}
\Idesc
\begin{DL}{MMMM}
\item[IXDVFR]Index of division(s) containing the data structure to be copied
\item[LENTRY]Address of the data structure to be copied
\item[IXDVTO]Index of the {\bf particular}
division into which the data structure has to be copied.
\item[*LSUP*]
\item[JBIAS]{\tt JBIAS < 1: LSUP} is the supporting bank in the target division
and JBIAS is the link bias specifying where the data structure has to be
introduced into this bank, i.e. the data structure will be connected
to LQ(LSUP+JBIAS).\\
{\tt JBIAS = 1: LSUP} is the supporting link, i.e. the data structure
is connected to {\tt LSUP} (top level data structure)\\
{\tt JBIAS = 2:} Stand alone data structure, no connection.
\item[CHOPT]Character variable specifying the selected options.
\begin{DL}{MMMMMMM}
\item[Data structure]
\begin{DL}{MM}
\item[' ']Copy the data structure at {\tt LENTRY} (the next link is not
followed).
\item['D']Copy complete division(s)\\
default: Dropped banks are squeezed out\\
\phantom{default:} (slower but maybe more economic than 'DI')
\item['DI']Immediate dump of divisions with dropped banks included
\item['L']Copy the data structure supported by the
linear structure at {\tt LENTRY} (the next link of {\tt LENTRY} is followed)
\item['S']Copy the single bank at {\tt LENTRY}.
linear structure at {\tt LENTRY} (the next link of {\tt LENTRY} is followed)
\end{DL}
\item[others]
\begin{DL}{MM}
\item['N']No link, i.e. linkless handling
By default link are significant
\item['P']Permit error returns
By default an error exit with a call to \Rind{ZTELL}{\tt (15,1)} is generated.
\item['Z']Zero all links pointing outside the data structure.
This is implied if origin and target stores are different.
\end{DL}
\end{DL}
\end{DL}
\Odesc
\begin{DL}{MMMM}
\item[*LSUP*]For JBIAS = 1 or 2, LSUP receives
the entry address to the data structure
\end{DL}
\par For a discussion of the cases of ``stand alone' memory, the user is
referred to the ZEBRA reference manual.
\index{stand alone memory}
\subsection{MZCOPY return codes}
\index{QUEST!IQUEST}
\begin{DL}{MMMMM}
\item[IQUEST(1)]Error status
\begin{DL}{M}
\item[0]Normal completion
\item[1]LENTRY invalid
\item[2]Bank chaining clobbered in the input data structure
\item[3]Not enough space for the input data structure
\item[4]The data structure is larger than the target space
\item[5]The data structure to be copied is empty
\item[6]Bank chaining clobbered in the output data structure
\end{DL}
\end{DL}
\subsection{Examples}
\begin{verbatim}
      CALL MZCOPY (0,LENTRY,IXDIV,LTOP,1,' ')
\end{verbatim}
copies the data structure at {\tt LENTRY} in the primary store into division
{\tt IXDIV}. The copied data structure will be addressable via the
top level link {\tt LTOP}.
\begin{verbatim}
      CALL MZCOPY (IXDVFR,LENTRY,IXDVTO,LSUP,-1,'D')
\end{verbatim}
copies the division identified by the identifier {\tt IXDVFR} to the division
specified by {\tt IXDVTO} squeezing out unused space.
The entry address into the
data structure in division {\tt IXDVFR} is {\tt LENTRY}.
In the target division
with index {\tt IXDVTO} this data structure will be attached to link -1
in bank {\tt LSUP} and be addressable as {\tt LCOPY = LQ2(LSUP-1)}
\section{Sort the banks of a linear structure}
\par The routines described below
re-arrange the horizontal linking
within a given linear structure such that the values of the
keywords contained in
each bank increase monotonically when moving through the linear
structure with {\tt L=LQ(L)}.
For equal keyword values the original order is preserved.
\par Key-words may be either floating-point, integer or Hollerith.
For Hollerith sorting a collating sequence
inherent in the representation is used,
thus the results will depend on the machine.
\par Sorting may be done either for a
{\bf single keyword} in every bank
or for a {\bf key vector} in every bank:
\index{linear structure!sort}
\Subr{CALL ZSORT (IXSTOR,LLS,JKEY)}
\par Sorts banks according to a single floating-point keyword
\Subr{CALL ZSORTI (IXSTOR,LLS,JKEY)}
\par Sorts banks according to a single integer keyword
\Subr{CALL ZSORTH (IXSTOR,LLS,JKEY)}
\par Sorts banks according to a single Hollerith keyword
\par
\Subr{CALL ZSORV (IXSTOR,LLS,JKEY,NKEYS)}
\par Sorts banks according to a floating-point key vector
\Subr{CALL ZSORVI (IXSTOR,LLS,JKEY,NKEYS)}
\par Sorts banks according to an integer key vector
\Subr{CALL ZSORVH (IXSTOR,LLS,JKEY,NKEYS)}
\par Sorts banks according to a Hollerith key vector
\Idesc
\begin{DL}{MMMM}
\item[IXSTOR]Store index.
\item[JKEY]{\tt Q(L+JKEY)} - The key word or the first word of the key vector.
\item[NKEYS]Number of words in the key vector.
\item[LLS]Address of the first bank of the linear structure.
\end{DL}
\par The execution time taken by these routines is a function
of the reordering which needs to be done.
For perfect order the operation is a simple verification pass
through the structure.
The maximum time is taken if the banks are arranged with
decreasing keywords.
\par Sorting relinks the banks such that the keywords are in
increasing order.
If one needs them in decreasing order one can use routine \Rind{ZTOPSY}
(see below).
\section{Operations on linear structures}
\par The routines described in this section perform service operations
on linear structures.
The parameter {\tt LLS} is the address of the first bank
of the linear structure.
\index{linear structure!reverse order}
\Subr{CALL ZTOPSY (IXSTOR,LLS)}
\par This routine reverses the order of the banks in the linear structure,
i.e. the first bank becomes the last, and the last the first,
for traversing the structure with {\tt L=LQ(L)}.
\index{linear structure!bridging}
\Subr{CALL ZPRESS (IXSTOR,LLS)}
\par This routine removes, by bridging, dead banks still present
in the linear structure pointed to by {\tt LLS}.
\section{Interrogate a linear structure}
\par These routines perform service functions for linear structures.
The parameter {\tt LLS} is the address of the first bank
of the linear structure.
\index{linear structure!interrogate}
\Func{LFCALL = LZLAST (IXSTOR,LLS)}
\par This function ssearches the linear structure pointed to by {\tt LLS}
for its end.
It returns in {\tt LF} the address of the last bank in the structure.
{\tt LF = 0} is returned if the structure is empty.
\Func{LFCALL = LZFIND (IXSTOR,LLS,IT,JW)}
\par This function searches the linear structure pointed to by {\tt LLS}
for the first bank containing {\tt IT} in word {\tt JW};
it returns its address in {\tt LF}. If none: {\tt LF=0}.
\Func{LFCALL = LZLONG (IXSTOR,LLS,N,IT,JW)}
\par Same functionality as function {\tt LZFIND},
but {\tt IT} is a vector of {\tt N} words expected
in words {\tt JW} to {\tt JW+N-1} of the bank.
\Func{LFCALL = LZBYT (IXSTOR,LLS,IT,JBIT,NBITS)}
\par Similar to function \Rind{LZFIND},
but it looks for a bank having IT in the byte of the status word
starting at {\tt JBIT} and with a width of {\tt NBITS} bits.
\Func{LFCALL = LZFVAL (IXSTOR,LLS,VAL,TOL,JW)}
\par Same functionality as function \Rind{LZFIND},
but it looks for a bank having in word {\tt JW} a floating point number
which is equal to VAL within the tolerance {\tt TOL}.
\Func{NCALL = NZBANK (IXSTOR,LLS)}
\par Function which counts the number of banks in the linear
structure pointed to by {\tt LLS}.
\Func{NCALL = NZFIND (IXSTOR,LLS,IT,JW)}
\par Function similar to \Rind{LZFIND} but for all banks.
It returns the number of such banks in {\tt N}
and stores the addresses of the first 100 such banks into {\tt IQUEST},
starting at {\tt IQUEST(1)} in common {\tt /QUEST/}.
\index{QUEST!IQUEST}
\Func{NCALL = NZLONG (IXSTOR,LLS,N,IT,JW)}
\par Function similar to \Rind{LZLONG} but for all banks.
It returns the number of such banks in {\tt N}
and stores the addresses of the first 100 such banks into {\tt IQUEST},
starting at {\tt IQUEST(1)} in common {\tt /QUEST/}.
\index{QUEST!IQUEST}
\section{Locate a bank in a division}
\par
The routines of this section do not operate on data structures as such,
but they perform a sequential search on the banks in a division.
\subsection{Sequential scan by Hollerith identifier}
\par
Function \Rind{LZFIDH} performs a sequential scan over the banks of a specified
division, starting with the bank following the specified bank,
and returns
the address of the first bank with the specified Hollerith identifier.
\Func{LFCALL = LZFIDH (IXDIV,IDH,LGO)}
\index{division!scan}
\Idesc
\begin{DL}{MMM}
\item[IXDIV]The index of the division to be scanned
\item[IDH]The 4 character {\bf\it Hollerith} identifier (NOT a CHARACTER
variable).
\item[LGO]The address of the bank {\bf after} which the scan has to start.\\
{\tt LGO = 0} means start with the first bank in the division.
\end{DL}
\subsection{Sequential scan by Hollerith and numeric identifier}
\par
Function \Rind{LZFID} performs a sequential scan over the banks of a specified
division, starting with the bank following the specified bank and returns
the address of the first bank with the specified Hollerith/numeric
identifier pair.
\Func{LFCALL = LZFID (IXDIV,IDH,IDN,LGO)}
\index{division!scan}
\Idesc
\begin{DL}{MMM}
\item[IXDIV]The index of the division to be scanned
\item[IDH]The 4 character {\bf\it Hollerith} identifier (NOT a CHARACTER
variable).
\item[IDN]The numeric identifier
\item[LGO]The address of the bank {\bf after} which the scan has to start.
{\tt LGO = 0} means start with the first bank in the division.
\end{DL}
\subsection{Division scan by bank identifier}
\par Starting at the first bank in the specified division,
function \Rind{LZLOC} performs a sequential scan through the
division, and returns
the address of the {\bf first bank} with the specified
Hollerith/numeric bank identifier pair.
\Func{LFCALL = LZLOC (IXDIV,CHIDH,IDN)}
\index{division!scan}
\Idesc
\begin{DL}{MMM}
\item[IXDIV]The index of the division to be scanned
\item[CHIDH]Character variable containing the Hollerith identifier
\item[IDN]The numeric identifier
\end{DL}
\subsection{Number of words available in a division}
\par The number of words available in a division can be obtained
by issueing a call to MZNEED.
If desired (option 'G') a garbage collection can be preformed to
get the needed number of words.
\Subr{CALL MZNEED (IXDIV,NWNEED,CHOPT)}
\Idesc
\begin{DL}{MMMM}
\item[IXDIV]The index of the division where the free space is needed
\item[NWNEED]The number of words needed in the division
\item[CHOPT]Character variable specifying the desired option
\begin{DL}{MM}
\item[' ']Do not garbage collect the division to free space
\item['G']Garbage collect the division to increase the space
\end{DL}
\end{DL}
\subsection{Note}
\index{QUEST!IQUEST}
\par Variable {\tt IQUEST(11)} in common {\tt QUEST} contains the space
available with the words needed already substracted,
i.e. a negative value in {\tt IQUEST(11)} signals that there are
no {\tt NWNEED} words available in the given division.
\section{Global operation aids}
\subsection{Set the program phase}
\par Primarily to avoid recovery to ``next event'' at the wrong moment,
ZEBRA needs to know in which phase the user program is at any
given moment. This is accomplished with the routine \Rind{ZPHASE},
described in the ZEBRA reference manual.
\label{SR_ZPHASE}% Special treatment for not described routine
\index{program!phase}
\subsection{\protect\label{ZEND}Normal program termination}
\Subr{CALL ZEND}
\par The routine
\Rind{ZEND} is defined to be the entry-point for normal run
termination. This routine is normally {\bf provided by the user}
to close files and print accumulated results.
It is important that all termination operations are
done through this routine,
if the user wants them to happen even in abnormal run termination.
\par It would normally look like this:
\index{program!termination!normal}
\begin{verbatim}
       SUBROUTINE ZEND
 
       CALL ZPHASE (-3)           ! start termination
       . . .                      ! any user termination code
       CALL MZEND
       STOP
       END
\end{verbatim}
\subsection{CALL MZEND}
\par \Rind{MZEND} is a ZEBRA routine which prints statistics about
the usage of all divisions.
\par A user routine similar to \Rind{ZEND} is defined for taking over control
of fatal error termination. Its name is \Rind{ZABEND} and it is
described in the next paragraph.
This should perform any extra operations needed
for fatal termination and then it should transfer
to \Rind{ZEND} for termination.
\subsection{Abnormal program termination}
\par To cope with situations where the program ends abnormally,
an entry point \Rind{ZFATAL} is defined.
\index{program!termination!abnormal}
\par A second routine \Rind{ZFATAM} is identical to \Rind{ZFATAL},
except that it prints a message,
given as a character string to the routine.
\par The routines \Rind{ZFATAL} and \Rind{ZFATAM} are {\bf supplied by the
system}. They are protected against recovery loops, and they must
{\bf not} be supplied by the user.
They should be called only when the run cannot usefully continue.
If the application program discovers such a fatal condition,
it too should call \Rind{ZFATAL} or \Rind{ZFATAM},
preceded with some diagnostic printing or
with loading to {\tt IQUEST} in common {\tt /QUEST/} some clue to the trouble.
\label{SR_ZFATAL}% Special treatment for not described routine
\label{SR_ZFATAM}% Special treatment for not described routine
\index{QUEST!IQUEST}
\par The calling sequences are, for example:
\begin{verbatim}
      CALL ZFATAL                         or
      CALL ZFATAM ('MY ERROR MESSAGE')
\end{verbatim}
\par Routine \Rind{ZABEND}
\label{SR_ZABEND}% Special treatment for not described routine
receives control from \Rind{ZFATAL} to handle fatal run termination.
This routine may be supplied by the user.
\par The ZEBRA system contains the standard routine as follows:
\begin{verbatim}
      SUBROUTINE ZABEND
+CDE,ZSTATE                   ! Specifies NQPHAS, the program phase
 
      CALL ZPOSTM('TCWM')     ! Print a DZSNAP in 'TCWM' mode
                              !   for the faulty store
      IF (NQPHAS.LT.0) STOP   ! Immediate stop if 'initialization'
                              !   or 'termination' phase
      NQPHAS = -2             ! Set 'termination' phase
      CALL ZEND
      END
\end{verbatim}
\par This \Rind{ZABEND} routine is not just a dummy; it generates
an optimal post-mortem
dump, including a subroutine trace-back, followed by any normal
user output programmed in \Rind{ZEND}. Transfer to \Rind{ZEND} takes place if
the break-down happened during normal operation, but not if the
program is still in the initialization phase or if it is already under
\Rind{ZEND} control.
\par The parameter to \Rind{ZPOSTM} is passed from there to \Rind{DZSNAP}
\label{SR_ZPOSTM}% Special treatment for not described routine
to select the options for dumping the dynamic store
(see page~\pageref{SR_DZSNAP} for details).
\subsection{Recovery through \Rind{ZTELL}-\Rind{ZTELUS}}
\label{SR_ZTELL}% Special treatment for not described routine
\par During normal operation any request from the user for space
with \Rind{MZWORK}, \Rind{MZBOOK}/\Rind{MZLIFT},
\Rind{MZDIV} and \Rind{MZPUSH} is satisfied,
after garbage collection if that is necessary and possible.
If, however, the request cannot be satisfied,
the normal course of the program must be changed.
To relieve the user of the burden of checking for success
after each space request,
the garbage collector will normally send control to the user at the
entry-point \Rind{QNEXT} (via \Rind{ZTELL} and the KERNLIB routine
\Rind{QNEXTE)}, to skip the current event and to continue
by processing the next one.
\index{KERNLIB}
\par Other ZEBRA packages, apart from MZ, and the user code, may
have similar problems.
Therefore a general trouble-control routine \Rind{ZTELL} has been
included in ZEBRA.
This is a switching routine with several modes of continuation
controlled by the user routine \Rind{ZTELUS},
one of which is to send control to \Rind{QNEXT}.
\index{program!recovery}
\par For more details about the recovery facilities in ZEBRA, the user
is referred to the ZEBRA reference manual.
 
