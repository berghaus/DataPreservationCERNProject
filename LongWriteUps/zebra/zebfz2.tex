%  $Header: /afs/.cern.ch/project/cnas_doc/sources/zebra/RCS/zebfz2.tex,v 1.2 1995/02/02 08:53:08 goossens Exp $
\Filename{H1-FZ-User-Specifications}
\chapter{User specifications for the FZ package}
\label{sec:H1-FZ-userspecs}

\Filename{H2-FZ-D-S-Representation}
\section{Representation of a data-structure}

The unit of information on a ZEBRA file is the
\textbf{data-structure}.
It may consist of zero, one, two, or more
\textbf{data-segments}.
The data-segments reflect the original residence of different
parts of the data-structure in different divisions at the
moment when the d/s was transferred from memory to the file with
\Rind{FZOUT}.
When the d/s is transferred back from the file to memory with \Rind{FZIN}
individual data-segments may be directed to separate divisions,
or may be ignored.

User information which may be associated with each d/s is
the 'user header vector',
specified and received via parameters to \Rind{FZOUT} and \Rind{FZIN},
and the 'text vector' taken from and delivered to the text-buffer
associated to the file with \Rind{FZTXAS} (implementation of this routine
is pending).

On the file the data-structure is represented
by the 'pilot information' followed by the 'bank material'.
The pilot carries all the control and context information,
namely:

\begin{itemize}
\item the amount of memory required to receive the data-structure;
\item the entry address into the data-structure, if any;
\item the user-header vector with its I/O characteristics, if any;
\item the text-vector, if any;
\item the segment table, if any, describing which data segment comes
      from which division;
\item the relocation table, if any, describing the original position
      in memory of each contiguous set of banks, needed to update
      all links in the data-structure for the new positions in
      memory on input.
\end{itemize}

This is followed by the 'bank-material',
which carries the copy of the memory regions originally occupied
by the banks of the data-structure.
The data-structure may in fact be empty,
in which case there is no bank-material.

In Native Data Format the bank material on the file
is a simple dump of the memory;
but in Exchange Data Format the numbers have to be transformed
from the internal to the exchange representation.
To make this possible automatically,
every bank carries its 'I/O characteristic' describing
the integer/floating/Hollerith nature of its contents exactly;
see the descriptions of the routines \Rind{MZLIFT}
and \Rind{MZBOOK}.
Banks of type 'undefined' cannot be transported.

The exact details for the file and data formats
are found in Chapter~\ref{sec:H1-FZ-format-specs}.

\Filename{H2-F2-events-runs-files}
\section{Events, Runs, and Files}
\label{sec:events-runs-files}

The unit of information on a ZEBRA file is the data-structure.

Several data-structures may be (but need not)
be grouped into an \textbf{event}.
On the file events are separated by the 'start-of-event'
flag being present in the first data-structure of each event.
\Rind{FZIN} may be asked to skip forward to and read the next
'start-of-event' data-structure.

Several events (or d/s) may (but need not) be grouped into a
\textbf{run}.
On the file the start and the end of a run are marked by
special \textbf{StoR} and \textbf{EoR} records written by calling \Rind{FZRUN}.
\Rind{FZIN} may be asked to skip forward to and read the next
'start-of-run' record.
(Skipping forward to next run or event should not be used
for the medium Memory or Channel.)

A ZEBRA file has to be terminated.
The writing of End-of-File is a perennial problem,
as the requirements for different kinds of files are
different for different machines and different media.
Thus for example, on the IBM system MVS one should not terminate
a disk file by an ENDFILE statement,
as this inhibits the release of the unused space on the disk.
A tape, on the other hand,
should be terminated by a double EoF which may or may not be
provided by the system, yet on the VAX the program will collapse
if one tries to ENDFILE an unlabelled tape.

In principle, a Zebra file may logically consist of several
files on the same medium.
To implement this rigorously on all machines the special
\textbf{Zebra EoF} record is provided
(an end-of-run record which immediately follows a true end-of-run
is also interpreted as EoF).
It is written by a call to \Rind{FZENDO} with one of the options
T, N, C, or I.

Whether or not the writing of a Zebra EoF signal is followed
by the explicit request to write one or two system file-marks
(for end-of-file or end-of-data)
depends on the circumstances.
Most machines do not support multi-file disk files,
and some machines do not even support multi-file tape files.

In first approximation, a Zebra file is assumed not to
contain imbedded system file-marks.
For output this means that no file-marks are written explicitly,
leaving the file termination to the system;
for input it means that a system file-mark is interpreted
as 'end-of-data'.
A different behaviour can be selected when calling \Rind{FZFILE}
by setting the \Lit{NEOF} parameter associated with the file
to 1, 2, or 3  as explained in section~\ref{sec:FZFILE}.

\Filename{H2-FZ-Outline-Usage}
\section{Outline of usage for medium Disk or Tape}

Before using FZ, the routines \Rind{MZEBRA} and 
\Rind{MZSTOR} must have been called.
FZ uses the system division of the primary store to hold
the control-information about all its files.
One bank per file is used, containing the parameters of the file,
the statistics of usage of the file,
and also the physical record buffer,
if the file format is 'exchange'.

\subsection*{Initialization}

Before using a particular file,
it should be \textbf{opened},
normally with the Fortran OPEN statement
or with the C interface routine \Rind{CFOPEN}
(except for files which are read/written by special
machine-dependent packages, such as IOPACK on IBM).
The ZEBRA handling of this file must be \textbf{initialized}
by calling \Rind{FZFILE}.

Machine-dependent details about opening files are given in 
chapter~\ref{sec:H1FZ-exchange-mode}.

The call to \Rind{FZFILE} specifies the properties of the file
and the processing direction, for example:
\begin{verbatim}
    CALL FZFILE (LUN,0,'.')    native mode, input only, disk file

    CALL FZFILE (LUN,0,'IO')   native mode, disk file,
                               input-output or output-input,

    CALL FZFILE (LUN,0,'XO')   exchange mode, output only, disk file

    CALL FZFILE (LUN,0,'D')    exchange mode, input only, disk file
                               reading with direct-access Fortran

    CALL FZFILE (LUN,0,'TL')   exchange mode, input only,
                               tape file to be read via the C Library

\end{verbatim}
Note that the Fortran systems on some Unix machines,
like on Sun or Silicon Graphics,
are not capable of handling fixed-length records in sequential mode,
ie. \Lit{RECORDTYPE='FIXED'} is not available in their Fortran
OPEN statement.
In this case one has to use the direct-access mode,
or the C library mode, of FZ for exchange format files.

If one is debugging a program,
it can be useful to set the logging level of FZ for this file
to 2 with
\begin{verbatim}
      CALL FZLOGL (LUN,2)
\end{verbatim}
causing FZ to print a log message
whenever it is called for this file.

\subsection*{Input}

To simply read the next data-structure, one calls for example with:
\begin{verbatim}
      PARAMETER   (NUHMAX=100)
      DIMENSION    IUHEAD(NUHMAX)
      COMMON /QUEST/IQUEST(100)

      NUH = NUHMAX
      CALL FZIN (LUN, IXDIV, LSUP,JBIAS, '.', NUH,IUHEAD)
      IF (IQUEST(1).NE.0)    GO TO special
\end{verbatim}
This will read the next d/s into the division indicated by \Lit{IXDIV},
it will transfer the user-header-vector into \Lit{IUHEAD*},
\Lit{NUHMAX} words at most, returning in \Lit{*NUH*} its useful size.
It will connect the d/s read into a higher level d/s (if any)
according to the parameters \Lit{!LSUP} and \Lit{JBIAS},
which have the same significance as with \Rind{MZLIFT} or \Rind{ZSHUNT}.

On normal completion \Rind{FZIN} returns \Lit{IQUEST(1)=0};
a positive value indicates an exception,
like Start-of-run or End-of-data;
a negative value signals trouble.
\Lit{IQUEST(1)} \textbf{must} be tested after every call to \Rind{FZIN}.

Frequently one is interested in processing only a particular
kind of data-structure, wanting to rapidly skip any others
which might be on the file.
To make this possible the data must be organised to contain
all the information relevant to selection in the user header vector,
because one can ask \Rind{FZIN} to start the d/s by reading
the pilot information only, delivering the user header vector to
the caller, leaving the bank-material in suspense,
waiting for a decision.
If the d/s is to be rejected, all the work of bringing
it into memory with adjustment of the links can be saved.

To get the user header vector of the next d/s one specifies
the S option (Select) in a first call to \Rind{FZIN};
a second call with the A option will transfer the d/s to memory,
for example:
\begin{verbatim}
C--       Ready to select next d/s

   11 NUH = NUHMAX
      CALL FZIN (LUN, IXDIV, 0,0, 'S', NUH,IUHEAD)
      IF (IQUEST(1).NE.0)    GO TO special
      IF (not wanted)        GO TO 11

C--       Accept pending D/S

      CALL FZIN (LUN, IXDIV, LSUP,JBIAS, 'A', 0,0)
      IF (IQUEST(1).NE.0)    GO TO special
\end{verbatim}
Whilst accepting is done by an explicit call with the A option,
rejection is done implicitly by asking for the next d/s.

Having reached the end of the input file (or having decided
to stop input for some other reason),
one can get the statistics of file usage printed by
\begin{verbatim}
      CALL FZENDI (LUN,option)
\end{verbatim}
'option' indicates the further action to be taken on this file,
such as REWIND and re-start reading from the beginning,
or start writing on the file positioned by reading it,
or simply terminate.

Beware: for exchange format files one can switch from input
to output only after having read an end-of-run or end-of-file.

If one wants to read several different files on the same
logical unit number (thereby possibly saving I/O buffers
in the system), this can be done as indicated by this sketch,
provided all the files have the same characteristics:
\begin{verbatim}
      OPEN (LUN,FILE=<file 1>,...)
      CALL FZFILE (LUN,0,opt)
         read first file
      CALL FZENDI (LUN,'NX')            new file to be connected

      OPEN (LUN,FILE=<file 2>,...)
         read second file
      CALL FZENDI (LUN,'NX')

      OPEN (LUN,FILE=<file 3>,...)
         read third file
            . . . . .
\end{verbatim}
If the files are \textbf{not} of the same kind,
for example if the first file is in native mode
and the second file is in exchange mode,
\Rind{FZENDI} must be told to forget all about the first file,
so that a new file can be started on the same
logical unit number, for example:
\begin{verbatim}
      OPEN (LUN,FILE=<file 1>,...)
      CALL FZFILE (LUN,0,'.')           native mode
         read first file
      CALL FZENDI (LUN,'TX')            terminate

      OPEN (LUN,FILE=<file 2>,...)
      CALL FZFILE (LUN,0,'X')           exchange mode
         read second file
      CALL FZENDI (LUN,'TX')
\end{verbatim}
(This is necessary because the size and character
of the FZ control bank depends on the nature of the file.)

\subsection*{Output}

It may be desirable to group the output into 'runs',
in which case one would start a new run with, for example:
\begin{verbatim}
      JRUN = run number
         . . .
      CALL FZRUN (LUN,JRUN,0,0)
\end{verbatim}
It is possible to store user information into the 'start-of-run'
record via the last two parameters of the call.
There is however the danger,
if this information is essential for the processing of the
data of the run,
that the start-of-run record may get lost due to read errors.

(An end-of-run record can be requested explicitly,
but normally this is not necessary, since it is triggered
by a new run, or by \Rind{FZENDO}.)

To ouput a d/s from the primary store,
supported by the bank at \Lit{!LHEAD},
together with a user header vector in \Lit{IUHEAD} of \Lit{NUH} integer words,
one may call:
\begin{verbatim}
      CALL FZOUT (LUN,0,LHEAD,0,'L',2,NUH,IUHEAD)
\end{verbatim}
In this case, the material to be output is defined solely by
the entry address \Lit{!LHEAD} into the d/s.
Therefore \Rind{FZOUT} has to do a logical walk through the complete d/s
by following all the structural links, to mark all the banks
belonging to this d/s.
A subsequent sequential scan over the memory constructs
the table of the memory regions to be output.
For a large d/s the time spent on this operation may be
non-negligible;
it can be saved if the user has organized his data such
that the d/s to be output resides in a separate division
(or divisions) of which it has exclusive use.
In this case one can instruct \Rind{FZOUT} to simply
output the complete division \Lit{IXDIV} (or divisions \Lit{IXDIV1 + IXDIV2}),
whithout the need for the logical walk, by calling:
\begin{verbatim}
    [ IXDIV = MZIXCO (IXDIV1,IXDIV2,0,0) ]

      CALL FZOUT (LUN,IXDIV,LHEAD,0,'D',2,NUH,IUHEAD)
\end{verbatim}
The entry address \Lit{!LHEAD} is still needed,
no longer to define the data to be written,
but for the receiver to find his way into the d/s read.

Although option 'D' saves the logical walk,
\Rind{FZOUT} still has to do the sequential scan of the division[s]
to identify the live banks to be written,
and the dead banks to be suppressed.
If the user knows that there are no dead banks,
or that their volume is negligible,
he can indicate this to \Rind{FZOUT} with the \Ropt{DI} option,
causing it to write the complete division[s] as it stands:
\begin{verbatim}
      CALL FZOUT (LUN,IXDIV,LHEAD,0,'DI',2,NUH,IUHEAD)
\end{verbatim}

Occasionally the d/s to be written out is not described
as easily as assumed above,
for example one may want to write a data-structure
minus some of its sub-structures.
In this case (see section~\ref{sec:FZmarking}) the user may pre-mark
the banks to be output and
\begin{verbatim}
      CALL FZOUT (LUN,IXDIV,LHEAD,0,'M',2,NUH,IUHEAD)
\end{verbatim}
Output of a file must be terminated,
to make sure that the last physical record is transfered
from the buffer to the file, for example with:
\begin{verbatim}
      CALL FZENDO (LUN,'I')
\end{verbatim}
to re-read the file just written; or with:
\begin{verbatim}
      CALL FZENDO (LUN,'TX')
\end{verbatim}
if the program no longer needs this file.

The recommended procedure is to have a standard job-termination
routine, called \Rind{ZEND},
normally called from the Main program.
This routine is called also from the ZEBRA recovery system in case
of abnormal job termination.
Into this routine one should include a
\begin{verbatim}
      CALL FZENDO (0,'TX')
\end{verbatim}
to terminate all pending output files.
However, this call pulls in the non-negligible volume of code
for the \Rind{FZOUT} complex, and should hence be present only
for programs really using \Rind{FZOUT}.

Writing several different files to the same logical unit
can be done in complete analogy to the case of reading;
in the examples given above one has to add the 'O' option
for \Rind{FZFILE},
and one has to change the calls to \Rind{FZENDI} into calls to \Rind{FZENDO}.

\Filename{H2-FZ-Initialize-ZEBRA-file}
\section{FZFILE - initialize a ZEBRA file}
\label{sec:FZFILE}

To initialize a Zebra file:
\begin{verbatim}
       Fortran:   OPEN (LUN, FILE=name, ...
       C:         CALL CFOPEN (LUNPTR, ..., name, ...)
                  IQUEST(1) = LUNPTR
\end{verbatim}
\Shubr{FZFILE}{(LUN,LREC,CHOPT)}
\begin{verbatim}
with    LUN:  logical unit number (Fortran)
               or Zebra stream identifier (otherwise),
               this must be a unique small positive integer

       LREC:  record length, in words (ignored if A option)
               native file format - maximum logical record length
                  zero: standard limit: 2440 words
                   +ve: user defined limit, but < 2500

               exchange file format - physical record length
                  zero: standard length: 900 words
                   +ve: user defined length
                         must be a multiple of 30 words

      CHOPT:  character string,
               individual characters select options:

         medium:  *  sequential binary disk file, default
                   T  magnetic tape
                   D  direct access disk file
                   A  alfa: 80 column card-image disk file
                   C  channel mode
                   M  memory mode

          usage:  F  read/write with Fortran, default (except IBM)
                   Y  read/write with special machine specific code
                      (IBM has IOPACK, NORD has MAGTAP)
                   L  read/write with interface to the C Library
                   K  read/write with user supplied code

    file format:     native file format is default
                   X  exchange file format
                      modes  M, C, A, D, L, K  all imply 'X'

    data format:     native   is default for native   file format
                      exchange is default for exchange file format
                   N  native data format
\end{verbatim}

\begin{verbatim}
      direction:     default direction is 'input only'
                   I  input enabled
                   O  output enabled
                  IO  input/output enabled

        various:  S  separate d/ss
                   U  unpacked d/ss, only with modes M or C
                   R  initial rewind
                   Q  quiet, set logging level to -2
                   P  permissive, enable error return,
                      see 'Status returned' just below

           NEOF:     handling of system EoF
                      for output:
                   0  write no file-marks at all
                   1  write file-mark only for End-of-File
                   2  write file-mark only for End-of-Data
                   3  write file-marks both for EoF and EoD

                      for input:
              1 or 3  one file-mark signals 'end-of-file',
                      otherwise: file-mark signals 'end-of-data'
\end{verbatim}

\subsection*{Status returned in \Lit{/QUEST/IQUEST(100)}}

\begin{verbatim}
IQUEST(1) =  0  all is well
             1  file has already been initialized with FZFILE
             2  LUN is invalid
             3  requested format is not available on the particular
                Zebra library (either because of the installation
                options taken, or because the code is not ready
                for the particular machine)
             4  the file pointer is zero for modes L or K
\end{verbatim}

The error returns are enabled only if the P option is selected,
otherwise control goes to ZFATAL. If the P option is selected,
the status must be checked, because the file will not be initialized
if an error exit is taken.

\subsection*{Option compatibility diagram}

\begin{verbatim}
                   M C A D T * K L Y
      channel   C  -                       +  combination useful
      alfa      A  - -                     -  combination not allowed
      direct    D  - + -                   i  option implied
      tape      T  - -   -                 d  option default
      neither   *  - - - - -               ?  depends on the user's
                                                     implementation
      user      K  - - - ? ? +
      lib C     L  - - - + + + -
      special   Y  - - - - + + - -
      Fortran   F  - - i d d d - - -

      exchange  X  i i i i + + i i i
      native    N  + + + + + d + + +

      separate  S  i + + + + + + + +
      unpacked  U  + + - - - - - - -
                   M C A D T * K L Y

\end{verbatim}

\subsection*{Notes}

\subsubsection*{OPENing:}

\Rind{FZFILE} initializes only the Zebra controls for this file;
the opening of the file has to be done by the user in his calling
program, according to the needs of his machine and operating system.

\subsubsection*{Fortran OPEN:}

if the file is to be handled with Fortran READ/WRITE one needs an
OPEN statement; one will find some hints in chapter~\ref{sec:H1FZ-exchange-mode}.

\subsubsection*{C open:}

for modes L or K the file should be opened by calling \Rind{CFOPEN}
(see the specifications at the end of this paragraph)
and the 'file pointer' returned by \Rind{CFOPEN} must be passed
on to \Rind{FZFILE} via \Lit{IQUEST(1)}.

\subsubsection*{LUN:}

this is the Zebra stream identifier which will be used in all subsequent
calls for this file; if the file is to be handled with Fortran this
is at the same time the logical unit number.

\subsubsection*{LREC:}

for exchange file format it is important to choose a good value for
all one's files, and then stick to it.
One has to compromise between conflicting things:
on tapes one would like to make this large,
but this costs memory for the Zebra buffer,
multiplied by the number of files concurrently open,
and it wastes disk space for end-of-run records which occupy
a whole physical record.
Some numbers can be found in chapter~\ref{sec:H1-FZ-format-specs}.

The physical record size for the exchange file format needs to be
specified both to Zebra with \Rind{FZFILE} and to the system with the OPEN
statement and maybe even with some JCL,
in which case the user may need to know this:
the block size is specified to Zebra in words, the default is 900 words.
These words correspond to words in the Zebra dynamic store,
such that a bank of 900 words could just fill one block.
Except for 32-bit machines,
the number of bits written to the file for each word depends
on the data format: for the exchange data format each word
generates 32 bits, for the native data format a full machine word
is transferred.
To the system the block size has to be specified either in bytes
or in native words.
For example, on the CRAY (64-bit words) the record-size of
a standard block will have to be given as 7200 bytes
(900 machine words) for the native data format,
but as 3600 bytes (450 machine words) for the exchange data format.

\subsection*{medium M or C:}

for the media 'memory' and 'channel' the Exchange File Format
is implied, and this cannot be changed.
The Native Data Format can be selected by giving the \Ropt{N} option.
Instructions on the use of these media is given in separate
sections near the end of this chapter.

\subsection*{medium A:}

Alfa mode should only be used to transmit data over a network
connection which cannot handle binary file transfers.
The character representation (ASCII, EBCDIC, etc) used
is that of the originating machine;
the translation is expected to happen in the network station.
Alfa mode must not be used for writing magnetic tapes,
it is at least a factor of ten slower than binary.

\subsection*{medium D:}

this serves two different purposes:
on some machines Fortran is not capable of handling
fixed-length records without system control words in sequential mode,
only in direct-access mode, but this only for disk files.
A side-effect advantage is better error recovery from lost records
on files which have been moved from tape to disk.
No timing studies have yet been made to check whether direct access
is slower than sequential acccess.

The other purpose is random access to the d/ss on the file,
this is described in section~\ref{sec:random-access} 
``Usage for random access''.
Selecting D only gives the possibility, but no obligation for
random access: for input Zebra will read the file sequentially
except at moments when the user interfers with calls to \Rind{FZINXT};
for output Zebra operates strictly sequentially.

\subsection*{medium T:}

at the moment no distinction is made internally in Zebra between
disk and tape files (exception: NORD),
but it may turn out that the C interface will have to have
a separate branch for tape files on some machines.

\subsection*{usage F or Y:}

read/write with Fortran, option F, is the default if none of Y, L, K
are specified.
Exception IBM: up to including Zebra version 3.66 the default
for sequental files is Y, that is handling with IOPACK;
from version 3.67 onwards the default will be F.
Most people give now (version 3.65) option F,
those who really want IOPACK should change their programs to
request Y to be insensitive to the transition to 3.67.
The only other machine sensitive to Y is presently the NORD:
magnetic tapes must be written through the MAGTAP utility,
on this machine TX implies Y.

\subsection*{usage L:}

read/write is with the routines \Rind{CFGET}/\Rind{CFPUT} which are part of the
interface to the C Library for handling files with fixed-length records.
This mode must be used for exchange file format tape files on those
Unix machines where Fortran does not provide the parameter
\Lit{RECORDTYPE='FIXED'} (or equivalent) in the OPEN statement,
like the Sun, or SGI, or DecStation.
On the same machines one might use this also for disk files as
an alternative to option D; no studies have yet been made to
see which is faster.
L can be combined with D for random access using the C interface.

\subsection*{usage K:}

this is a hook to enable a user to write his own handling of physical
records in case that none of the modes provided are satisfactory.
Chapter~\ref{sec:H1-FZ-thecnical-details} gives some hints of how to do this.

\subsection*{file-format:}

default is 'native' if none of 
\Ropt{A}, \Ropt{C}, \Ropt{D}, \Ropt{K}, \Ropt{L},
\Ropt{M} is given,
which necessarily operate with exchange file format.

\subsection*{data-format:}

for native file format this is 'native';
for exchange file format data format 'exchange' is assumed by default,
but native data format can be requested by giving the N option.
In this case LREC native words are written for each physical record,
and no data translation, packing, or byte inversion, is done.

\subsection*{direction:}

the option \Ropt{IO} is needed in two separate cases:
   - if the program first writes a new file which it then reads;
   - if the program positions an existing file by reading
     for further output.
In this case the input or output mode of the file is defined
by the first I/O action on the file;
it can be changed at the end of the first phase only with \Rind{FZENDI}
from input to output, or with \Rind{FZENDO} the other way round.

\subsection*{various S:}

for the exchange file format,
\Rind{FZOUT} normally places the start of a given d/s just after
the end of the previous structure in the same physical record,
to economize file space.
This may be inconvenient if the file is later to be handled
by means other than calling \Rind{FZIN}: giving the \Ropt{S} option
will force each d/s to start on a new physical record.
For the medium 'memory' the \Ropt{S} option is implied.

\subsection*{various U:}

only for media 'memory' and 'channel':
When handling the physical records for the Data Format 'exchange'
it may be more convenient for the user to do himself the
unpacking (\Rind{FZIN}) or packing (\Rind{FZOUT}) operation needed,
because in this case he has immediate access to the control
information in the records.
(Note: on the VAX 'packing/unpacking' is in fact byte inversion.)
The \Ropt{U} option allows this:
if given, \Rind{FZOUT} delivers the data non-packed,
and \Rind{FZIN} expects data which have already been unpacked by the user.

\subsection*{various R:}

if the initial REWIND is selected the file has to be OPENed
before calling \Rind{FZFILE}.

\subsection*{various Q:}

giving this option suppresses message printing for this file.

\subsection*{NEOF:}

this parameter controls for output
the explicit writing of system file-marks;
for input it controls the interpretation of a system file-mark,
which can mean either end-of-file or end-of-data
(two file-marks in succession always act as end-of-data).
On most machines the default value is NEOF=0,
meaning single-file files only.
This can be over-ridden by giving the 1, 2, or 3 option
if multi-file files must be handled.
See also section~\ref{sec:events-runs-files} for more explanations.

\subsection*{Specifications for CFOPEN}

Since this is a new KERNLIB routine not yet documented we print this
here.

\Shubr{CFOPEN}{(LUNPTR*,MEDIUM,NWREC,IOMODE,NBUF,NAME,ISTAT*)}
\begin{verbatim}
      LUNPTR*  is the 'file pointer' returned by the C library
               routine 'fopen', CFOPEN returns it to the caller who
               must hand it on to FZFILE via IQUEST(1).
               This will be zero if the open fails.

      MEDIUM   = 0  for disk file, normal
                 1      tape file, normal
                 2      disk file, user coded I/O
                 3      tape file, user coded I/O

      NWREC    the number of machine words per physical record, this
               is used to calculate the buffer size if NBUF not zero.

      MODE     the 'type' parameter of 'fopen', of type CHARACTER:
                 r   open for reading
                 w   truncate or create for writing
                 a   append: open for writing at end of  file,  or
                     create for writing
                 r+  open for update (reading and writing)
                 w+  truncate or create for update
                 a+  append; open or create for update at EOF

      NBUF     not currently used, always give zero

      NAME     the name of the file, of Fortran type CHARACTER.

      ISTAT*   status returned, zero if all is well,
               otherwise a system error code.
\end{verbatim}

\Filename{H2-FZ-Change-logging-level}
\section{FZLOGL - change the logging level of a file}

To change the logging level for a file:

\Shubr{FZLOGL}{(LUN,LOGLEV)}
\begin{verbatim}
with    LUN:  logical unit number

     LOGLEV:  logging level
                   -3: suppress all messages
                   -2: print error messages only
                    0: normal mode
                    1: normal mode + details of conversion problems
                    2: print to monitor CALLs to FZ
                    3: print short diagnostic dumps to debug
                    4: print full  diagnostic dumps to debug
                                   user-written output routines
\end{verbatim}
A logging level is attached to each FZ file;
by default this is the general system-wide default logging level
set by \Rind{MZEBRA}.
By giving the \Ropt{Q} (quiet) option with \Rind{FZFILE} the level is set to -2.
It can be changed later at any time by calling \Rind{FZLOGL}.

\Filename{H2-FZMEMO}
\section{FZMEMO - connect user memory area for medium Memory}

To connect the memory area for use by a 'file':

\Shubr{FZMEMO}{(LUN,MBUF,NWBUF)}
\begin{verbatim}
with    LUN:  stream number

       MBUF:  user memory of NWBUF machine words
\end{verbatim}
This must be called after the 'file' has been initialized
with \Rind{FZFILE}, and before it is used with \Rind{FZIN} or \Rind{FZOUT}.
Different memory areas may be connected by recalling this
routine any number of times;
see section~\ref{sec:FZMEMO} for explanations.

\Filename{H2-FZHOOK}
\section{FZHOOK - connect user routine for medium Channel}

To connect a particular user routine to be called
by \Rind{FZIN} or \Rind{FZOUT} for this 'file':
\begin{verbatim}
      EXTERNAL  UserSR
\end{verbatim}

\Shubr{FZHOOK}{(LUN, UserSR, 0)}

\begin{verbatim}
with    LUN:  stream number

     UserSR:  name of the user routine

      dummy:  the third parameter is not at the moment used,
               but it must be present in the call
\end{verbatim}

This must be called after the 'file' has been initialized
with \Rind{FZFILE}, and before it is used with \Rind{FZIN} or \Rind{FZOUT}.
Different user routines may be connected by recalling this
routine any number of times;
see section~\ref{sec:FZHOOK} for explanations.

\Filename{H2-FZLIMI}
\section{FZLIMI - limit the size of an output file}

\Shubr{FZLIMI}{(LUN,ALIMIT)}

\begin{verbatim}
with    LUN:  logical unit number

      ALIMIT:  floating point number giving, in Mega-words, the
                  limit of the data to be written to one reel of tape;
               if zero: increase the limit by one more reel of tape
               if  -ve: unlimited (as intialized by FZFILE)

Example:   CALL FZLIMI (21, 12.75)   sets the file-size
                                      to 12.75 Mwords for unit 21

         Re-calling later with:

            CALL FZLIMI (21, 0.)      sets the file-size to be the current
                                      data-volume plus 12.75 Mwords
\end{verbatim}

The reason for this facility is the fact that detecting 'end-of-tape'
is a problem which cannot be solved satisfactorily in full
generality.
To help the user who wants control over tape reel switching,
ZEBRA counts the total number of words written,
and checks after every data-structure written out
(but not for start-of-run, end-of-run, end-of-file)
whether the limit has been reached.
If so, it returns the 'pseudo end-of-tape' condition (cf. \Rind{FZOUT})
for every data-structure output until an increase of the limit
to include one more reel of tape is requested with \Lit{ALIMIT=0}.
Thus the user can switch tape,
call \Lit{\Rind{FZLIMI} (LUN,0.)},
and continue to write another tape,
again waiting for the 'end-of-tape' signal.

\Filename{H2-FZODAT}
\section{FZODAT - storing and recovering the direct access table}

The routines \Rind{FZODAT} and \Rind{FZIDAT} store and retrieve the direct-acces table
onto and from a file.

See section~ref{sec:random-access} 
``Usage for random access'' for explanations.

To store the Direct-access Table bank:

\Shubr{FZODAT}{(LUN,IXDIV,!LDAT)}
\begin{verbatim}
with    LUN:  logical unit number

      IXDIV:  index of division or store having the DaT bank

      !LDAT:  address of the DaT bank, if non-zero
\end{verbatim}
If \Lit{LDAT} is zero the DaT 'forward reference' record is written
to be updated later to contain the address of the DaT;
this is useful only as the very first record on the file.

If \Lit{LDAT} is non-zero the DaT bank is written and the forward reference
record is updated if possible.

To retrieve the DaT bank:

\Shubr{FZIDAT}{(LUN,IXDIV,!LSUP,JBIAS)}
\begin{verbatim}
with    LUN:  logical unit number

      IXDIV:  index of the division to receive the DaT bank

           The d/s read is linked into a pre-existing d/s as directed by
           !LSUP and JBIAS, which have the same significance as for MZLIFT:

      !LSUP:  if JBIAS < 1: !LSUP is the supporting bank,
      JBIAS:                 connection to link LQ(!LSUP-JBIAS)
                              IQUEST(13) returns the entry adr to the d/s

               if JBIAS = 1: *!LSUP is the supporting link,
                              connection to *!LSUP* (top-level d/s)
                              !LSUP* returns the entry adr to the d/s

               if JBIAS = 2: stand-alone d/s, no connection
                              !LSUP* returns the entry adr to the d/s

     Status return:  IQUEST(1) =  0  success
                                  -1  DaT not found
                                  -2  file is empty
\end{verbatim}

\Filename{H2-FZRUN}
\section{FZRUN - write a RUN record}

To write a start-of-run or end-of-run record:

\Shubr{FZRUN}{(LUN,NRUN,NUH,IUHEAD)}
\begin{verbatim}
with    LUN:  logical unit number

       NRUN:  run number,
               if  +ve: new run, run number literal
                  zero: new run, increase current run number by one
                   -ve: end-of-run record

        NUH:  length of the user information, may be zero, < 401

     IUHEAD:  NUH words of user information, integers only
\end{verbatim}

\textbf{Write / Error status} returned: as for \Rind{FZOUT}

A start-of-run record will be preceded by an end-of-run signal
if the last action on the file was the writing of a data-structure.

The request to write an end-of-run will be by-passed if the last
action on the file was the writing of EoR or EoF.

For the media 'memory' or 'channel' the writing of end-of-run,
if needed, should be requested by an explicit call to \Rind{FZRUN}
with \Rarg{NRUN} negative, since an implicit generation will not
get through to the user.

\Filename{H2-FZOUT}
\section{FZOUT - write one data-structure}

To write one data-structure:

\Shubr{FZOUT}{(LUN,IXDIV,!LENTRY,IEVENT,options,IOCH,NUH,IUHEAD)}
\begin{verbatim}
with    LUN:  logical unit number

      IXDIV:  index of division(s)
               may be zero [or IXSTOR] if the D option is not selected
               may be a compound index if the D option is selected

    !LENTRY:  entry address of the d/s
               may be zero if the Z option is selected

     IEVENT:  start-of-event flag
                 = 0  for event continued
                   1  for new event

               the following values are for use by FZRUN and FZENDO
               and are illegal for calls by the user:
                  13  flush the buffer       15  write end-of-file
                         (X mode only)       16  write end-of-data
                  14  write end-of-run       -1  write start-of-run

    options:  character string,
               individual characters select options:

     select d/s:     mutually exclusive options

                     by default the d/s supported by the bank at
                     LENTRY is written out (link 0 not followed)

                   L write the d/s supported by the linear structure
                     at LENTRY (link 0 followed)

                   M write the banks marked by the user
                     see section~\re{sec:FZmarking} for details

                   D  write complete division(s)
                      default: dead banks are squeezed out
                               (slower but maybe more economic than DI)
                   DI immediate dump of division(s),
                      dead banks, if any, are also written out

                   S write the single bank at LENTRY
                   Z zero banks, ie. empty d/s, header only

         others:   N no links, ie. linkless handling (cf 'Principles')
                     default: links are significant

                   P permit error returns
                     default: exit to ZTELL

       IOCH:   the I/O characteristic for the user header vector; as
               for a bank this may be either 'immediate' if the whole
               vector is of the same type, or it may be composite.

               - immediate:  IOCH = 1  all bits
                                    2  all integers
                                    3  all floating
                                    4  all double precision
                                    5  all Hollerith
                                    7  self-describing

               - composite: set up with

                             CALL MZIOCH (IOCH,NW,'format')  where

                             IOCH is now a vector of NW words at most

        NUH:  number of words in the user header vector, < 401,
              may be zero, in which case IOCH is not used

     IUHEAD:  the user header vector
\end{verbatim}

\subsection*{Write status returned in \Lit{/QUEST/IQUEST(100)}}

\begin{verbatim}
   IQUEST(1)  =   0 normal completion
                 +1 'pseudo end-of-tape' condition (cf. FZLIMI)
                 -1 first attempt to write after end-of-data
                 -2 error return
   IQUEST(5)  =  word 1 of the direct access adr of the d/s just written
   IQUEST(6)  =  word 2
   IQUEST(9)  =  # of useful machine words ready in the user's memory
                 only for medium 'memory'
   IQUEST(11) =  NWBK, number of words of bank material
   IQUEST(12) =  NWTB, size of the relocation table
   IQUEST(13) =  number of pilot records written so far
   IQUEST(14) =  number of Mwords written so far
   IQUEST(15) =  number of words (up to 1 M) written so far
                 ie. the total is  IQUEST(15) + IQUEST(14)*10**6
   IQUEST(16) =  number of logical records written so far
   IQUEST(17) =  number of physical records written (exchange mode only)
\end{verbatim}
   Further information about the file can be obtained
   by calling \Rind{FZINFO}, see section~\ref{sec:FZINFO}.

\subsection*{Error status returned in \Lit{/QUEST/IQUEST(100)}}

Normally \Rind{FZOUT} does not return to the caller for (program) errors,
but exits to \Rind{ZTELL}.
Exceptionally, error returns may be enabled by the \Ropt{P} option.
\begin{verbatim}
   IQUEST(1) = -2
   IQUEST(2) = 11: !LENTRY invalid or pointing to a dead bank
             = 12: bank chaining clobbered
             = 13: not enough space for the relocation table
             = 14: medium 'memory': user's memory too small
\end{verbatim}
If the P option is not taken exit is with
\Lit{CALL ZTELL (i,1)} with $i=11,12,13,14$. \\[2mm]

If the actual write operation fails,
for example because the disk is full,
control is handed to \Rind{ZTELL} (which may return) with:
\begin{verbatim}
      CALL ZTELL (19,0)
with
      IQUEST(1) = 19
      IQUEST(2) = who is in trouble ?
                    1 - Fortran sequential
                    2 - Fortran direct access
                   21 - L mode sequential
                   22 - L mode direct-access
                   41 - Alfa mode
      IQUEST(3) = IOSTAT error code return by the 'write'
      IQUEST(4) = LUN (Zebra stream identifier)
      IQUEST(5) = C file descriptor if writing in L mode
\end{verbatim}

\Filename{H2-FZIN}
\section{FZIN - read one data-structure}

To read the next data-structure one calls \Rind{FZIN}.
The return code in IQUEST(1) will tell the caller whether
the READ operation was free of error, and whether the
object read was a d/s, a start-of-run, an end-of-run,
or an end-of-file signal.
\Rind{FZIN} may be asked to skip to and then read the next start-of-event
d/s or the next start-of-run record.

In the simplest case (opt = '.' or blank) \Rind{FZIN} will read the next
data-structure into the division indicated by the parameter \Rarg{IXDIV},
at the same time delivering the user-header vector to \Rarg{IUHEAD}.

The selective read has been provided to rapidly skip unwanted
d/ss without expansion into memory and without relocation
of the links:
calling \Rind{FZIN} with opt='S' causes reading of the next pilot
information only,
returning to the user the header-vector (and the text-vector, if any)
for taking a decision to read or to skip the 'pending d/s'.
Skipping is done by asking for the next d/s;
accepting is done by calling \Rind{FZIN} with opt='A'.
Note that every call to \Rind{FZIN} has to be checked for the success of
the operation by testing on \Lit{IQUEST(1)}.

In the cases described so far the complete data-structure is
read and is deposited into one particular division.
It is however possible to steer individual data segments of
the d/s into particular divisions, or to cause them to be ignored.
This can be done by using the options \Ropt{T} and \Ropt{D},
as described separately in the next paragraph.

\subsection*{Specifications for \Lit{FZIN}}

\Rind{FZIN} returns the read status, either normal or error, in \Lit{IQUEST};
be careful about the meaning of status codes 4 and 5:
'4' means EoF seen on a file which can be a multi-file file;
'5' means 'End-of-Data'.
Reading a file which cannot be multi-file can never
produce status 4, the end will always be indicated by status 5.

\subsubsection*{Error status}

\begin{verbatim}
IQUEST(1) =  -8  . . .
             -7  for 3 consecutive errors
             -6  for 2 consecutive errors
             -5  read error
             -4  bad constructs, maybe not a file written by FZOUT
             -3  bad data
             -2  not enough space to read the d/s and its table
             -1  faulty call: T,D,A option given, but no pending d/s

IQUEST(2) =  number of logical records read so far

IQUEST(3) =  number of physical records read so far (exchange mode)
\end{verbatim}
Details about the error occurred are stored in \Lit{IQUEST(11)} ff.
as described in the diagnostics chapter.

To read the next data-structure:

\Shubr{FZIN}{(LUN,IXDIV,!LSUP,JBIAS,opt,*NUH*,IUHEAD*)}
\begin{verbatim}
with    LUN:  logical unit number

      IXDIV:  index of the default division to receive the d/s
               zero: division 2 of the primary store
               (ignored if S option given)

               The d/s read is linked into a pre-existing d/s as directed by
               !LSUP and JBIAS, which have the same meaning as for MZLIFT:

      !LSUP:  if JBIAS < 1: !LSUP is the supporting bank,
      JBIAS:                 connection to link LQ(!LSUP-JBIAS)
                              IQUEST(13) returns the entry adr to the d/s

               if JBIAS = 1: *!LSUP is the supporting link,
                              connection to *!LSUP* (top-level d/s)
                              !LSUP* returns the entry adr to the d/s

               if JBIAS = 2: stand-alone d/s, no connection
                              !LSUP* returns the entry adr to the d/s
               (ignored if options S or T selected)

    options:  character string,
               individual characters select options:

          event:    default: go for the next d/s
                   E skip to and read the next start-of-event d/s
                   R skip to and read the next start-of-run record
                   2 skip to and read the next   end-of-run record
                   3 skip to and read the next   Zebra end-of-file
                   4 skip to and read the next machine end-of-file

                     any skip operation stops also on machine EoF;
                     option E or 2 skipping stop also on Zebra EoF,
                     option R skipping does not stop on Zebra EoF.

         select:    default: read the next header and its d/s
                     (may mean: skip pending d/s or current event)

                   S select, read next header and text-vector only
                     (may mean: skip pending d/s or current event)
                     (LSUP and JBIAS not used)

                   T table,  load the segment table for the current
                             d/s into /FZCSEG/
                     (LSUP, JBIAS, NUH, and IUHEAD not used)

                   A accept, read the pending d/s
                     (NUH and IUHEAD not used)

                   D divisional accept, read the pending d/s under
                                        control from /FZCSEG/
                     (NUH and IUHEAD not used)

                   F accept also DaT records, which are normally ignored;
                     see section~\re{sec:random-access} 

      *NUH*:  size of the user header vector
                 on  input:  maximum size of IUHEAD
                 on output:  useful size stored in IUHEAD
               (ignored if options T, A, or D  selected)

    IUHEAD*:  user header vector
               (ignored if options T, A, or D  selected)
\end{verbatim}

\subsubsection*{Normal read status returned in \Lit{/QUEST/IQUEST(100)}}

\begin{verbatim}
IQUEST(1) =  -ve  error, see separate list
               0  normal completion
               1  start-of-run record
               2  end-of-run record
               3  Zebra  end-of-file
               4  system end-of-file, continuation possible
               5  system end-of-data, continuation not possible
               6  first attempt to read beyond EoD

IQUEST(2) =  number of logical records read so far

IQUEST(3) =  number of physical records read so far (exchange mode)

IQUEST(5) =  word 1 of the direct access adr of the d/s read
IQUEST(6) =  word 2                     (exchange mode only)

IQUEST(11)   if IQUEST(1)=0:  = 1 or 0 for yes/no start new event
             if IQUEST(1)=1:  = run number for start/end of run

IQUEST(12) = processing bits of pilot, normally zero

IQUEST(13) = LENTRY, the entry address into the data structure
             zero means: empty d/s
             (not yet a valid address if S option return)

IQUEST(14) = NWBK, the number of words occupied by the d/s in memory
             zero means: empty d/s

IQUEST(20) = NWIOCH, size of the I/O characteristic

IQUEST(21) = NWIOCH words of I/O characteristic
      ...           for the user header vector
\end{verbatim}
   Further information about the file can be obtained
   by calling \Rind{FZINFO}.

\Filename{H2-FZIN1}
\section{FZIN - read one data-structure by segments}

It may be convenient to represent an event by several separate
d/ss on the file.
This permits in an easy way to selectively read only a particular
part of every event.
This method has one draw-back:
if there are reference links pointing from one part to an other part
of the event, where both parts are residing simultaneously in
memory, and if the two parts are written out by two separate calls to
\Rind{FZOUT}, the cross links will be lost on read-back.

To amend for this, the following scheme has been implemented:
when a data-structure is transferred from several divisions to the
FZ file, the data are 'segmented', i.e. a table is included
into the pilot information, indicating the divisions from which
the different data segments originated, together with their sizes.
On read-back the user can either skip particular data-segments or
he can direct data-segments into particular divisions individually.

To do so, three calls to \Rind{FZIN} are necessary:
\begin{enumerate}
  \item call with the S option to read the pilot;
  \item call with the T option to ready the segment table in /FZCSEG/;
  \item call with the D option to read the d/s with distribution.
\end{enumerate}

The second call will present the 'segment table' to the caller
in the labelled Common Block
\begin{verbatim}
  COMMON /FZCSEG/ NQSEG,IQSEGH(2,20),IQSEGD(22)    where:

           NQSEG  = number of segments contained in the pending d/s
                             if NQSEG = 0:  d/s is not segmented
      IQSEGH(1,J) = char 1-4
      IQSEGH(2,J) =      5-8  of the Hollerith name of the division
                              from which segment J derives
       IQSEGD(J)    index of division selected for segment J

      Note: IQSEGD(21+22) are working elements of the system and,
             like NQSEG, must not be modified by the user.
\end{verbatim}
To direct segment \Lit{J} into a given division one should set
\Lit{IQSEGD(J)} to the index of that division
(or merely to the division number;
the store is selected by the parameter \Lit{IXDIV} to \Rind{FZIN}).
To cause this segment to be ignored \Lit{IQSEGD(J) = -1} should be set.
\Lit{IQSEGD(J)} containing zero directs this segment into the
'default' division selected by the parameter \Rarg{IXDIV} to \Rind{FZIN}.
(The vector \Lit{IQSEGD} is preset to zero by \Rind{FZIN}.)

Since \Lit{/FZCSEG/} is used for segment handling with all streams,
both input and output, there must not occur some other call
to FZ for any stream between the second and the third call.
Also, having called with the \Ropt{T} option does not oblige the user
to follow it by a call with the \Ropt{D} option;
he may call with the \Ropt{A} or even the \Ropt{S} option,
in which cases the segment table is simply ignored.

\Filename{H2-FZINXT}
\section{FZINXT - reset the read point on a direct access file}

To reset the read point:

\Shubr{FZINXT}{(LUN,MDSA1,MDSA2)}
\begin{verbatim}
with    LUN:  logical unit number or Zebra stream ID

               the 2 word d/s address of the next d/s to be read,
      MDSA1:  word 1:  physical record number
      MDSA2:       2:  off-set within the record,
                         if this is zero the first d/s
                         starting in the record will be used
\end{verbatim}
See section~\ref{sec:random-access} for context information.

\Filename{H2-FZCOPY}
\section{FZCOPY - copy one data-structure from input to output}

\Rind{FZCOPY} will copy a data-structure from the input to the output 'file'
without expansion into memory and without translating the data
representation, thereby saving the time which would otherwise
be spent on these operations.

The file-format and the data-format of the input or the output file
may be 'exchange' or 'native', but the following restrictions are
imposed:
\begin{enumerate}
   \item the data-format of the output file must be the same as that
      of the input file, ie. FZCOPY will not translate between
      native and exchange data-format. (Note however that on some
      machines native and exchange data-format are identical.)
   \item the input and the output cannot both be in 'channel' mode.
   \item Alfa format is not handled.
   \item the locical record length for the input in native file-format
      must not be longer than 2500 words.
\end{enumerate}

To copy a data-structure one has first to start reading it by calling
\Rind{FZIN} with the S option; thereby one obtains the user header,
which will normally be used to decide whether or not the copy is
wanted. This 'pending' data-structure may then be copied by
calling:

\Shubr{FZCOPY}{(LUNIN,LUNOUT,IEVENT,options,IOCH,NUH,IUHEAD)}
\begin{verbatim}
with  LUNIN:  logical unit number of the input file
     LUNOUT:  logical unit number of the output file

     IEVENT:  start-of-event flag
                 = 0  for event continued
                   1  for new event

    options:  character string,
               individual characters select options:

     I/O descr.:    by default the I/O descriptor from the input
                     file is used for IUHEAD

                   I use the new I/O descriptor given in IOCH for
                     the user header vector

               P special 'permit' option not normally given

       IOCH:  the I/O characteristic for the user header vector;
               this is ignored if the I option is not given;  as
               for a bank this may be either 'immediate' if the whole
               vector is of the same type, or it may be composite.

               - immediate: IOCH = 1  all bits
                                    2  all integers
                                    3  all floating
                                    4  all double precision
                                    5  all Hollerith
                                    7  self-describing

               - composite: set up with

                             CALL MZIOCH (IOCH,NW,'format')  where

                             IOCH is now a vector of NW words at most

        NUH:  number of words in the user header vector, < 401,
               may be zero, in which case IOCH is not used

     IUHEAD:  the user header vector
\end{verbatim}

\subsection*{Status returned in \Lit{/QUEST/IQUEST(100)}}
\begin{verbatim}
   IQUEST(1)  =   0 normal completion
                 +1 'pseudo end-of-tape' condition (cf. FZLIMI)
                < 0 input  error return, see below
                > 1 output error return, see below

   If normal completion:

   IQUEST(5)  =  word 1 of the direct access adr of the d/s just written
   IQUEST(6)  =  word 2
   IQUEST(9)  =  # of useful machine words ready in the user's memory
                 only for medium 'memory'
   IQUEST(11) =  NWBK, number of words of bank material
   IQUEST(12) =  NWTB, size of the relocation table
   IQUEST(13) =  number of pilot records written so far
   IQUEST(14) =  number of Mwords written so far
   IQUEST(15) =  number of words (up to 1 M) written so far
                 ie. the total is  IQUEST(15) + IQUEST(14)*10**6
                 careful: if this compound is bigger than 2G it
                          needs more than 32 bits to hold it
   IQUEST(16) =  number of logical records written so far
   IQUEST(17) =  number of physical records written (exchange mode only)
\end{verbatim}

\subsubsection*{Input error status}

\begin{verbatim}
IQUEST(1) =  -8   . . .
             -7  for 3 consecutive errors
             -6  for 2 consecutive errors
             -5  read error
             -4  bad constructs, maybe not a file written by FZCOPY
             -3  bad data
             -2  not enough space to read the d/s and its table
             -1  faulty call: no pending d/s, or: Alfa mode,
                 or: input/output have differen data format,
                 or: both input/output in channel mode,
                 or: native input record length too long;
                 (code -1 causes ZFATAL unless P option given)

IQUEST(2) =  number of logical records read so far

IQUEST(3) =  number of physical records read so far (exchange mode)
\end{verbatim}
Details about the error occurred are stored in IQUEST(11) ff.
as described in the diagnostics chapter for \Rind{FZIN}.

\subsubsection*{Output error status returned in \Lit{/QUEST/IQUEST(100)}}

Normally \Rind{FZCOPY} does not return to the caller for (program) errors,
but exits to \Rind{ZTELL} or to \Rind{ZFATAL}. 
Exceptionally, some such error
returns may be enabled by giving the P option in the call.

\begin{verbatim}
   IQUEST(1) = +2
   IQUEST(2) = 14: medium 'memory': user's memory too small
\end{verbatim}

\Filename{H2-FZENDO}
\section{FZENDO - output file termination}

To terminate one or all \textbf{output} files:

\Shubr{FZENDO}{(LUN,options)}

\begin{verbatim}
with    LUN:  logical unit number
               if zero: all FZ output files

    options:  character string,
               individual characters select options:

         main:  T  terminate:
                    - ensure end-of-data (unless done)
                    - print file statistics (unless done)
                    - drop FZ control-bank

                 N  continue output to a new file to be connected
                    by the user to LUN after this call:
                    - ensure end-of-data (unless done)
                    - print file statistics (unless done)

                 C  continue on the next file of the same stream:
                    - ensure end-of-file (unless done)
                    - print file statistics (unless done)

                 I  switch to input, to read the file just written:
                    - ensure end-of-data (unless done)
                    - print file statistics (unless done)
                    - remove the 'output' permission
                    - rewind and change status to 'input'

                 O  output again, to over-write the file just written:
                    - ensure end-of-data (unless done)
                    - print file statistics (unless done)
                    - rewind

              none  print file statistics only

                    over-ruling:   T -> N -> C -> I -> O

     variants:  R  execute REWIND function, only with T or N
                 U  execute UNLOAD function, only with T or N (no action yet)
                 X  execute CLOSE function, only with T or N
                 O  keep the 'output' permission, only with I
                 Q  quiet, suppress printing of file statistics
        0,1,2 or 3  only with I: change the NEOF parameter
                    of FZFILE for reading
\end{verbatim}
To be sure that all output files are closed correctly,
even on abnormal job termination,
the user should call from \Rind{ZEND}: \qquad\Lit{CALL FZENDO (0, 'TX')}

If necessary this is taken as a final close-down signal
to be passed on to special I/O packages on some machines
(such as \Rind{IOPACK} on IBM).

\Filename{H2-FZENDI}
\section{FZENDI - input file termination}

To terminate one or all \textbf{input} files:

\Shubr{FZENDI}{(LUN,options)}

\begin{verbatim}
with    LUN:  logical unit number
               if zero: all FZ input files

    options:  character string,
               individual characters select options:

         main:  T  terminate:
                    - print file statistics (unless done)
                    - drop FZ control-bank

                 N  continue input from a new file to be connected
                    by the user to LUN after this call:
                    - print file statistics (unless done)

                 C  continue on the next file of the same stream:
                    - print file statistics (unless done)
                    - step over the system EOF as required on some machines

                 I  input again, to re-read the same file:
                    - print file statistics (unless done)
                    - rewind

                 O  switch to output, to permit writing on a file
                    positioned by reading:
                    - print file statistics (unless done)
                    - change status to 'output'

              none  print file statistics only

                    over-ruling:   T -> N -> C -> I -> O

     variants:  R  execute REWIND function, only with T or N
                 U  execute UNLOAD function, only with T or N (no action yet)
                 X  execute CLOSE function, only with T or N
                 Q  quiet, suppress printing of file statistics
        0,1,2 or 3  only with O: change the NEOF parameter
                    of FZFILE for writing
\end{verbatim}

Both \Rind{FZENDI} and \Rind{FZENDO} also load the file statistics
into the Common \Lit{/FZSTAT/} just like the routine \Rind{FZINFO},
to provide the final statistics,
which are not yet available just before \Rind{FZENDO},
and maybe no longer available just after \Rind{FZENDO}.

\Filename{H2-Usage-for-random-access}
\section{Usage for random access}
\label{sec:random-access}

\subsection*{Principles}

For random access the location of a d/s within its file is specified
by its 'data-structure address', the DsA, consisting of 2 words:

\begin{DLtt}{123456}
\item[word 1] number of the physical record in which the d/s starts;
\item[word 2] off-set within this record.
\end{DLtt}

If the file contains less than 2 Gigawords you can pack this into
one word with:

\begin{verbatim}
  JDSAP = (JDSA1-1) * LREC  + JDSA2
\end{verbatim}
%
where \Lit{LREC} is the number of words in each physical record, as specified
to \Rind{FZFILE}.

For every successful call to \Rind{FZOUT} or \Rind{FZIN} Zebra returns the DsA
of the latest d/s in \Lit{IQUEST(5)} and \Lit{IQUEST(6)},
provided the file-format is 'exchange' and even for a tape file.
To prepare a file for reading it randomly the user would construct
a table, the 'direct access table', or DaT, which contains the
relevant properties of the d/ss on the file plus their DsA's.
This can be done either by an extra read pass over the file
to collect the data, or more economically at the time when the
file is created.
In the latter case the DaT can be written to the file as
the last d/s before end-of-file,
and is then ready for use for any future reading of the file.

When reading a file initialized with mode D given to \Rind{FZFILE},
repeated calls to \Rind{FZIN} will read the data sequentially.
To obtain a particular data-structure one has to call \Rind{FZINXT} specifying
the DsA of the wanted d/s; this will reset the 'current read point'
for the file and the next call to \Rind{FZIN} will then deliver the wanted d/s.

\subsection*{File creation}

Do this:
\begin{enumerate}
\item call \Rind{FZFILE} with \Lit{CHOPT = 'DO'} or \Lit{'LDO'} for a disk file, or
      \Lit{CHOPT = 'TO'} or \Lit{'TLO'} for a tape file.
\item lift a bank, generously large enough to hold the DaT.
\item \Lit{CALL FZODAT (LUN,0,0)}  to create the
      'DaT forward-reference record'.
      This is a small record to provide 2 words to hold the DsA of the DaT
      itself; they are initialized to zero and on the final call \Rind{FZODAT} will
      try to update them with the true address if the file is ready for
      random access.
      This should be provided on every file which later one may want
      to access randomly, like a tape file to be copied to disk.
      This is useful only if this record is the very first record on the file.
\item create the file by repeated calls to \Rind{FZOUT},
      compiling at the same the Direct Acces Table,
      including into it \Lit{IQUEST(5)} and \Lit{(6)} returned from \Rind{FZOUT}.
\item call \Rind{MZPUSH} to reduce the bank holding the DaT to the size
      actually needed.
\item \Lit{CALL FZODAT (LUN,IXDIV,LBANK)}  to store the bank with the DaT
      onto the file and to update the forward reference record if possible;
      Zebra will remember the DsA of this d/s in the control-bank for this file.
\item \Lit{CALL FZENDO (LUN,'TX')}  to terminate and close the file.
      This will store the DsA of the DaT bank into the user header of
      the special Zebra EoF record, which is the very last item on the file.
\end{enumerate}

\subsection*{Reading the file}

Do this:

1) call \Rind{FZFILE} with \Lit{CHOPT = 'D'} or \Lit{'LD'} to permit random access.

2) retrieve the direct-access table:
\begin{verbatim}
        CALL FZIDAT (LUN,IXDIV,LDAT,1)
        IF (IQUEST(1).NE.0)             GO TO trouble
\end{verbatim}
   \hspace*{4mm} This will try to find the Direct access Table:
\begin{enumerate}
\item first it will read the first record on the file to see whether
      it is a DaT forward reference record containing a non-zero DsA.
      If so it will use this adr to read the DaT and deliver it to the user.
\item if this is not successful it will try to get from the operating
      system the length of the file and then read its last physical record;
      this should contain the Zebra EoF record with the DsA of the DaT.
\item if this fails it will scan either the last 24 physical records
      or the whole file for the DaT record, depending on whether
      it did or did not get some indication of the file size.
\end{enumerate}

3) to get a wanted d/s: find it in the DaT, get its DsA
and call \Lit{\Rind{FZINXT} (LUN,MDS1,MDS2)},
with \Lit{MDS} the 2-word data-structure address.
This resets the read point, calling \Lit{\Rind{FZIN} (LUN,...)} will then
deliver the wanted d/s.

If acccess to the table was not immediate,
FZIDAT will try to update the forward reference record, if it exists,
to ease future use.
It can do this only if the file hase been opened for read/write,
and this has to be signalled by giving CHOPT='DIO' to \Rind{FZFILE}.

Keeping the DaT on the file to be read is obviously the simplest
way to proceed, but other organisations are possible.
For example, if one has a large disk with many files as the
data-base for the events of an experiment,
one may want to have a separate global DaT
for all the events on the disk,
with a structure reflecting the files and holding their names.

Remember: if one needs a data-base with key-word access
or update capability one should use the RZ package of Zebra.

\Filename{H2-Usage-for-medium-Memory}
\section{Usage for medium Memory}
\label{sec:FZMEMO}

In this mode all the records assembled by \Rind{FZOUT} to represent
one particular data-structure are placed one after the other
contiguously into an area of memory belonging to the user
(rather than being written to a file).
One is then free to move the data around,
normally from a machine which has no I/O facilities
to a machine which has, as in emulator farms.
They may there be written to tape.
Eventually the data will be brought back to be read with \Rind{FZIN},
either directly from a file, or again from user memory.
To 'read' a d/s from user memory,
\Rind{FZIN} expects all the records representing the d/s
to be present in the memory area.

FZ handles the medium 'memory' only with the
\textbf{file format} 'exchange',
ie.~with logical 'records' blocked onto fixed-length physical
'records'.
The \textbf{data format} can be 'exchange' or 'native'.

The details of this format are found in chapter~\ref{sec:H1-FZ-format-specs}.

To use this mode the following is necessary:
\begin{itemize}
\item the 'stream' must be initialized with the \Ropt{M} 
      option in \Rind{FZFILE},
\item the options \Ropt{I} or \Ropt{O} 
      set this stream for input or output
      (it cannot be used for both concurrently),
\item the parameter \Rarg{LUN} to \Rind{FZFILE} identifies the stream,
\item the \Ropt{N} option selects the native data format,
\item the \Ropt{U} option selects unpacked mode, as explained below;
\item the user memory to be used for this stream
      has to be connected by calling \Rind{FZMEMO};
      different streams may use different memory regions.
\end{itemize}

The connection call is

\Shubrz{FZMEMO}{(LUN,MBUF,NWBUF)}

indicating to FZ the where-abouts and the size of one's
memory region for stream LUN.
NWBUF gives the size in terms of machine words,
it must be large enough to hold the largest d/s to be handled;
see the memory-size considerations further down.

\subsection*{Output}

In the case of output one calls \Rind{FZOUT} (also \Rind{FZRUN}),
on return one will find the physical records representing
the d/s in one's memory.
\Rind{FZOUT} starts this string of records with a 'steering block',
ie.~a physical record with 8 control words
(see chapter~\ref{sec:H1-FZ-format-specs} for details),
placed at the start of the user's memory;
the remaining data are placed as 'fast blocks', if any.
The last block is normally used only partially,
the unused part is marked out with a padding record,
ie.~a logical record of type 5,
of which only the record length and the type code are significant,
the data contents are irrelevant.

\Rind{FZOUT} returns in \Lit{IQUEST(9)} the number of useful machine words
stored into the memory.
It includes the (normally) 2 control words of the padding record,
but not its data words.
If one needs it,
one can compute the number of physical records
as  \Lit{1 + (IQUEST(9)-1)/LENREC}  where \Lit{LENREC}
is the physical record size in machine words.
On machines with a word-size greater than 32 bits this needs
a little care,
as explained in the note about \Lit{LREC} in section~\ref{sec:FZFILE}
for \Rind{FZFILE},
and also below.

To obtain a start-of-run record one can call \Rind{FZRUN}
to get it into the memory, the useful size is again found
in \Lit{IQUEST(9)}.
To get an end-of-run record one has to call \Rind{FZRUN} explicitly;
one may call \Rind{FZENDO} to have the file statistic printed,
but this will not put anything into the user's memory.

\subsection*{Input}

To read back a d/s originally produced by \Rind{FZOUT}
and then shipped around,
one has to store it in the user memory,
connected with \Rind{FZMEMO},
and call \Rind{FZIN}.
This will transfer the data to the dynamic store,
converted if the Exchange Data Format is in use,
and relocate the links.
The user's memory is left intact.
\Lit{IQUEST(1)} returns the read status and must be checked after
every call to \Rind{FZIN}.

\subsection*{Lay-out of the memory and size considerations}

Using the information of chapter~\ref{sec:H1-FZ-format-specs},
one obtains the lay-out of the d/s in the memory
produced by \Rind{FZOUT} as shown on the next page.
NWPHR is the physical record length as defined by the call
to \Rind{FZFILE}, it is 900 by default.

If one is using the native data format
(or if one is running on a machine where the native format
is identical to the exchange format, ie.~on IEEE machines
such as Apollo, Alliant, Motorola)
this lay-out applies literally,
the numbers are usable directly,
and \Rind{FZOUT} will return \Lit{IQUEST(9)=NWUSE}.

However, if one is using the exchange data format
the data are converted to their exchange representation,
and they are also 'packed'.
For example, on the Cray two words are packed into one
machine word,
and \Rind{FZOUT} will return \Lit{IQUEST(9)=(NWUSE+1)/2}.
On a VAX 'packing' means byte inversion,
so also in this case the data are not usable directly.

Use of the \Ropt{U} option:
the situation can arise that one needs the exchange format,
because communication is between different machines,
but that one also needs access to the control information
at the beginning of the data.
One could gain this access by local unpacking, but this is messy.
It is more elegant to take charge of the packing oneself,
and instruct \Rind{FZOUT} to deliver the data non-packed
(by selecting the U option in \Rind{FZFILE}).
In this case the lay-out shown on the next page
applies again literally, or almost:
the data are delivered as one word per each machine word
(and on the VAX the bytes are not yet inverted),
but they are converted to the exchange representation,
32 bits right justified with zero-fill.
This conversion affects negative integers, floating point numbers,
Hollerith, but not positive integers and bit patterns.
Since the control information has been deliberately chosen
to be of this invariant kind, it remains usable.
\Rind{FZOUT} returns again \Lit{IQUEST(9)=NWUSE}.
When shipping the data one has to execute the 'packing' operation
applicable on the given machine such that they arrive in
the correct form at the destination.

From these considerations one easily derives the size
requirement on the memory to be connected with \Rind{FZMEMO}.
On the Cray for example, to handle data-strutures of not more
than 80000 words (including the control and context information)
one needs a memory region of 40000 words if one operates
in normal exchange mode,
but 80000 words for native mode or if the \Ropt{U} option is selected.

\subsection*{Lay-out of a data-structure in the user memory}

\begin{verbatim}
 word
    1-4  steering block stamp
         =  hex  0123CDEF  80708070  4321ABCD  80618061
      5  bits 1->24: NWPHR, the physical record length
                 30: set if start-of-run in this block
                 31: set if end-of-run
                 32: set if emergency stop block
      6  zero (physical record counter)
      7  NWTOLR = 8
      8  NFAST, the number of fast blocks to follow

      9  NWLR, logical record length
     10  LRTYP = 2 or 3, the record type
     11  floating point 12345.0 as check word
     12  Zebra version number, integer = 10000. * QVERSIO
     13  zero
     14  zero
     15  NWTX=0, number of words in the text-vector
     16  NWSEG, number of words in the segment table
     17  NWTAB, number of words in the relocation table
     18  NWBK,  number of words of bank material
     19  LENTRY, entry address into the d/s
     20  NWUHIO = NWUH + NWIO, number of words in the user header
           vector plus its I/O characteristic, zero if no header

      NWIO words of the I/O characteristic for the u. h. vector
      NWUH words of the user header vector
     NWSEG words of the segment table
      NWTX words of the text-vector
     NWTAB words of the relocation table
      NWBK words of bank material

 word
     NT  is the last word of the d/s
   NT+1  NWPAD, record length of the padding record
   NT+2  5, record type

          NWPAD-1 untouched padding words
\end{verbatim}
The following numbers are calculated:
\begin{verbatim}
   log. record length   NWLR = 10+NWIO+NWUH+NWSEG+NWTX+NWTAB+NWBK
   incl. log. c/words   NWDS = 2 + NWLR
  incl. phys. c/words     NT = 8 + NWDS
     # of fast blocks  NFAST = (NT-1) / NWPHR
     total # of words   NALL = NWPHR * (NFAST+1)
       padding length  NWPAD = NALL - NT - 1         (can be 0 or -1)
    # of useful words  NWUSE = NT + 1 + MIN(NWPAD,1)
\end{verbatim}

\Filename{H2-Usage-for-medium-Channel}
\section{Usage for medium Channel}
\label{sec:FZHOOK}

In this mode the records assembled by \Rind{FZOUT} are channelled
through a user routine one-by-one to their destination;
(rather than being written to a file or to memory).
Similiarly for \Rind{FZIN} the data are acquired not
from tape or disk directly, but through the same user routine.
The name of this routine is not decided by Zebra.

Channelled mode operates with
\textbf{file format} 'exchange',
ie.~the data are collected into fixed-length records,
and each record is handed to the user routine when complete
(for \Rind{FZOUT}, the inverse for \Rind{FZIN}).

The details of this format are found in chapter~\ref{sec:H1-FZ-format-specs}.

The \textbf{data format} can be 'exchange' or 'native'.

To use this mode the following is necessary:

\begin{itemize}
\item the 'stream' must be initialized with the \Ropt{C} option in \Rind{FZFILE},
\item the options \Ropt{I} or \Ropt{O} set this stream for input or output
      (it cannot be used for both concurrently),
\item the parameter \Rarg{LUN} to \Rind{FZFILE} identifies the stream,
\item the \Ropt{N} option selects the native data format,
\item the \Ropt{S} option forces each d/s to start on a new
      physical record, the \Ropt{U} option selects unpacked mode, as explained
      in the previous paragraph;
\item the user routine \Lit{UserSR} to be used for this stream
      must be hooked up to FZ by calling \Rind{FZHOOK};
      different streams may use different user routines;
\item the user routine UserSR must be provided.
\end{itemize}

The connection call is
\begin{verbatim}
      EXTERNAL  UserSR
\end{verbatim}
\Shubrz{FZHOOK}{(LUN, UserSR, 0)}

passing to FZ the address of the user routine;
the third argument is not used for the time being.
The specifications for the user routine are:

\medskip
\colorbox{boxgray}{\makebox[\linewidth][l]{\quad\ttfamily
                   SUBROUTINE UserSR (IBUF,IOWAY)}}
\par\medskip

\begin{verbatim}
with   IBUF:  the data of the 'record'

      IOWAY:  the I/O direction:
                = 0  if called from FZIN  for input
                  1  if called from FZOUT for output
                     other values are reserved to the user

     IQUEST
        (1):  on entry: LUN, the stream ID
               on exit:  status flag
        (2):  on entry: the number of machine words for transmission
               on exit:  number of machine words delivered
        (3):  kind of record
        (4):  = zero if sequential access
               = ordinal number of the record wanted if direct-access
        (5):  0 / 1 for disk / tape
        (6):  if FZIN: number of words per physical record

\end{verbatim}

\subsection*{UserSR called from \Rind{FZIN}}

In this case \Rarg{IOWAY} is zero on entry, and \Lit{IQUEST(2)} specifies
the maximum number of words which the buffer \Rarg{IBUF} can accept
without the program being destroyed.

\Lit{IQUEST(3)} indicates the kind of record expected,
if this is zero a normal continuation record is wanted;
if it is =1 then \Rind{FZIN} is expecting a physical record
starting a new d/s;
the user routine is supposed to discard trailing records
of the previous d/s if this has been de-selected.
Note that selective reading with \Rind{FZIN} in channel mode
is not yet fully tuned.

UserSR is supposed to fill the buffer \Lit{IBUF},
store into \Lit{IQUEST(2)} the number of words received,
and return zero in \Lit{IQUEST(1)}.
Exeptions may be signalled by setting
\begin{verbatim}
  IQUEST(1) =  -1  end of data
              > 0  error, the value of this status code will be
                   displayed to the caller of FZIN in IQUEST(14)
\end{verbatim}

\subsection*{User routines called from \Rind{FZOUT}}

In this case \Rarg{IOWAY} is 1 on entry, and \Lit{IQUEST(2)} specifies
the number of words in the buffer \Rarg{IBUF} waiting to be transmitted.

UserSR is supposed to dispatch the buffer \Rarg{IBUF},
and return zero in \Lit{IQUEST(1)}.
(At least for the time being, a non-zero status code in \Lit{IQUEST(1)}
is ignored.)

\Filename{H2-User-marking-FZOUT}
\section{User marking of data-structures for \protect\Rind{FZOUT}}
\label{sec:FZmarking}

Normally the identification of the banks belonging to the
data-structure to be written out is left to \Rind{FZOUT} itself.
Naturally this can cover only logically simple cases,
such as the complete d/s supported by the bank at the
entry address specified to \Rind{FZOUT}.

The situation does however arise that one needs a more
complex description.
For this the M option has been provided which tells \Rind{FZOUT}
that the user has already marked the banks to be transfered
by setting system status-bit \Lit{IQMARK (=26)},
and that he has designated the memory interval which contains
his banks by storing the addresses of the lowest and the highest
bank into the \Lit{COMMON/ZLIMIT/LLOW,LHIGH}.
Thus in principle one can set up one's selection in full
generality,
except that one must take care that the banks marked
actually form a connected d/s with the entry address \Rarg{LENTRY}.

It may however be quite tedious to do this job completely
'by hand'.
So one tries to provide some tools for formalizable situations.
At present the only such tool is \Rind{MZMARK},
which scans a d/s for marking,
but at the start of every new linear structure reached during
the scan it checks the Hollerith ID of the start bank
with a list to see whether the new sub-structure should
be included into the marking process.

To take an example, suppose the header bank at \Rarg{LENTRY}
supports 4 primary sub-structures with bank names
\Lit{RAW}, \Lit{GEOM}, \Lit{KIN}, \Lit{DST}, of which the first 3 support in turn
4 sub-structures with bank names \Lit{TEC}, \Lit{BGO}, \Lit{CAL}, \Lit{MUC},
and at the even lower levels there may be any unspecified further
sub-structures.

If now one wants to write out only the data for \Lit{GEOM}, \Lit{KIN}, \Lit{DST},
and for the first 2 only the \Lit{BGO} results,
one can do this with

\begin{verbatim}
      PARAMETER  (NID=4)
      DIMENSION  IDLIST(NID)
      DATA       IDLIST  /  4HRAW , 4HTEC  , 4HCAL , 4HMUC  /

      CALL MZMARK (0,LENTRY,'-',NID,IDLIST)
      CALL FZOUT (LUN,0,LENTRY,IEVENT,'M',IOCH,NUH,IUHEAD)
\end{verbatim}

Note that we have used the anti-selection option (\Lit{-}) of \Rind{MZMARK}
to veto at the high levels,
which permits the low level linear stuctures to be accepted without
one having to specify which exactly they are.

\Filename{H2-Suppress-loading}
\section{Suppress loading of unused parts of FZ}

Because the format for processing an FZ file
(native, exchange, ALFA mode)
is selected by the initializing call to \Rind{FZFILE},
and because the handling of all formats is done
by the one set of routines \Rind{FZIN} and \Rind{FZOUT},
the user's call to \Rind{FZIN} or to \Rind{FZOUT} causes the loading
of all the code to handle all the formats.
This volume is non-negligible,
and for production programs one may want to suppress
loading of the non-used parts of it.
This can be done by adapting this dummy routine,
which as it stands is valid for a program which only
reads files in native mode and which does not call
\Rind{FZOUT}/\Rind{FZENDO}:

\begin{verbatim}
      SUBROUTINE FZDUMY
      CHARACTER  NAME*6

C--       No output native mode
*n    ENTRY   FZOFFN
*n    NAME = 'FZOFFN'
*n    GO TO 17

C--       No output exchange mode, neither binary nor ALFA
*n    ENTRY   FZOFFX
*n    NAME = 'FZOFFX'
*n    GO TO 17

C--       No output ALFA mode, but binary mode
*n    ENTRY   FZOASC
*n    NAME = 'FZOASC'
*n    GO TO 17

C--       No input native mode
*u    ENTRY   FZIFFN
*u    NAME = 'FZIFFN'
*u    GO TO 17

C--       No input exchange mode, neither binary nor ALFA
      ENTRY   FZIFFX
      NAME = 'FZIFFX'
      GO TO 17

C--       No input ALFA mode, but binary mode
*n    ENTRY   FZIPHA
*n    NAME = 'FZIPHA'

   17 CALL ZFATAM (NAME//' in FZDUMMY reached.')
      END
\end{verbatim}

Note that the dummy entry \Lit{FZIFFN} is not active because
the true routine is needed, \Lit{FZIPHA} is not needed because
\Lit{FZIFFX} is stronger, and that the entries \Lit{FZO...} are not
needed because \Rind{FZOUT} is not called.
If the dummy is reached by mistake, the program is stopped
via \Rind{ZFATAL} with a message.

\Filename{H2-FZ-installation-options}
\section{FZ installation options}

The various modes of operation
of FZ have been made optional at the source code level
to allow a tailor-made installation for specific applications.
For example, if on a given machine, like an emulator,
FZ is used exclusively in 'channeled mode',
the code for all other modes can be removed
by giving the command line
\begin{verbatim}
   +USE, FZFFNAT, FZDACC, FZLIBC, FZMEMORY, FZALFA, T=INHIBIT.
\end{verbatim}
in the cradle to the Patchy run which generates the code
to be compiled for the Zebra Library.

These are the options:
\begin{DLtt}{123456789}
\item[FZFFNAT]   file format Native
\item[FZFORTRAN] sequential Fortran I/O for file format exchange
\item[FZLIBC]    read/write through the C library interface
\item[FZDACC]    any direct access mode
\item[FZDACCF]   direct access with Fortran
\item[FZDACCL]   direct access with C Library
\item[FZCHANNEL] channeled mode
\item[FZMEMORY]  memory mode
\item[FZALFA]    Alfa exchange mode
\end{DLtt}

The standard Zebra libraries are usually prepared with all
FZ modes selected, except \Lit{FZLIBC} which is selected only under UNIX.
