\batchmode

\documentstyle[11pt,epsfig,longtable,changebar,makeidx,lscape]{cernman}
\makeatletter
\makeindex
\newcommand{\FZfile}{FZ~file\index{FZ!Sequential input/output}\index{input/output!FZ}}
\newcommand{\RZfile}{RZ~file\index{RZ!Random input/output}\index{input/output!RZ}}
\newcommand{\IQUEST}{\Lit{IQUEST}\index{IQUEST@{\tt IQUEST}!user communication vector in common {\tt QUEST}}\index{IQUEST@{\tt IQUEST}!error reporting}\index{error reporting!{\tt IQUEST}}\index{QUEST@{\tt QUEST}!user communication common}}
\newcommand{\QUEST}{\Lit{QUEST}\index{IQUEST@{\tt IQUEST}!user communication vector in common {\tt QUEST}}\index{IQUEST@{\tt IQUEST}!error reporting}\index{error reporting!{\tt IQUEST}}\index{QUEST@{\tt QUEST}!user communication common}}
\renewcommand{\ZEBRA}{\textsc{ZEBRA}}
\renewcommand{\Copt}[1]{\texttt{#1}}
\renewcommand{\Ropt}[1]{\texttt{#1}}
\renewcommand{\Rarg}[1]{\texttt{#1}}
\def\condbreak#1{}
\driver{DVIPS}
\setlongtables
\makeindex
\romanfont{times}
\PScommands\setcounter{secnumdepth}{3}
\setcounter{tocdepth}{2}
\newenvironment{landscapebody}{\begin{landscape}}{\end{landscape}}
\makeatletter
\def\LS@rot{\setbox\@outputbox=\vbox{\@rotr\@outputbox}}
\makeatother
\long\def\NODOC#1{#1}
\renewcommand{\ZEBRA}{\textsc{ZEBRA}}\renewcommand{\Copt}[1]{\texttt{#1}}\renewcommand{\Ropt}[1]{\texttt{#1}}\renewcommand{\Rarg}[1]{\texttt{#1}}\def\condbreak#1{}\def\LS@rot{\setbox\@outputbox=\vbox{\@rotr\@outputbox}}\def\NODOC#1{#1}
\makeatother
\newenvironment{tex2html_wrap}{}{}
\usepackage{screen}
\begin{document}
\pagestyle{empty}
\stepcounter{chapter}
\stepcounter{section}
\stepcounter{section}
\stepcounter{subsection}
\stepcounter{section}
\stepcounter{section}
\newpage

{\samepage \clearpage \begin{UL}\item IBM systems (Fortran I/O only). 
      The file is first opened for sequential
      access and a Fortran unformatted read is issued,
      using the IBM extension NUM=nbytes, e.g.
<tex2html_verbatim_mark>XMP2
      The file is then closed and reopened for direct-access
      I/O.
\item VAX/VMS and Apollo systems (SR9) (Fortran I/O only). 
      The file is first
      opened for sequential access and a Fortran inquire
      statment is issued, e.g.
<tex2html_verbatim_mark>XMP3
      The file is then closed and reopened for direct-access
      I/O.
\item All other systems plus VAX/VMS systems using C I/O.
      The record length is determined from the data in the file
      itself. For this reason, the RZ package must know if the
      file is in native or exchange format.
      For this reason, the option \Lit{'X'} is recommended
      when processing exchange format files.
\end{UL}
}


\stepcounter{chapter}
\stepcounter{section}
\stepcounter{section}
\stepcounter{section}
\stepcounter{chapter}

\end{document}