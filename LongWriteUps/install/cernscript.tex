\chapter{Description of the Unix scripts}

\section{cernsys and cernsys.csh}

Shell scripts to set various environmental variables used for 
the installation of CERNLIB.
The variables set are as follows:

\begin{DLtt}{1234567890}
\item[PLIUWC]Unix Workstation Code 
\item[PLISYS]Unix system type
\item[PLINAME]Variable used by {\bf yexpand} to select
appropriate installation options. e.g.
\begin{DLtt}{12345678901234567890}
\item[Sun]Sun
\item[Sun running Solaris]Sun,Solaris
\item[RS6000]IBMRT
\end{DLtt}
\end{DLtt}

The current definitions of the Unix workstation codes and system types
are given below.

\begin{DLtt}{1234}
\item[alt]Alliant            
\item[amx]Amiga-UX 3000
\item[apo]Apollo 3000        
\item[a10]Apollo 10000       
\item[cru]Cray/UNICOS        
\item[cvx]Convex             
\item[dec]DECstation 5000    
\item[hpx]HP 9000/7xx        
\item[irs]IBM RISC 6000
\item[ibx]IBM AIX/370        
\item[lnx]i486/Linux
\item[nsx]NEC SX-3
\item[sgi]Silicon Graphics
\item[sol]Sun Solaris
\item[sun]Sun SunOS
\end{DLtt}

\section{getdev}

Shell script to get the device on which the directory
given as first argument resides.

\section{grouplib}

\section{makefile}

\section{makepack}

makepack is the primary script for building the program library. It has
the following options:

\begin{DLtt}{12}
\item[-c]Run {\it compile} step
\item[-l]Run {\it library} step, i.e. build archive library
\item[-p]Run {\it PATCHY} step. This step also runs the {\bf KUIP}
compiler {\bf KUIPC} on any output files with extension {\bf .cdf}.
\item[-s]Run {\it split} step using {\bf FCASPLIT}.
\item[-i]Run {\it inlib} step to created installed libraries.
\item[-C]Run special compilation step for generating installed libraries.
\item[-t]Turn timing on for each step
\item[-D]Development mode - each routine is compared against previous
version and only changed routines are compiled.
\item[-L]To add additional libraries to link step for target module.
\item[-P]Production mode - rebuild all routines.
\end{DLtt}

\section{namefind}

Shell script to extract information from a names file, e.g. {\bf cernlib.names}.

\section{plidd}

\section{plienv.csh and plienv.sh}

Shell scripts for the C shell and Bourne/Korn shells to set the CERNLIB
installation environment.

\section{plilog}

Shell script to view a CERNLIB installation logfile.

\section{plitar}

Shell script to unpack a CERNLIB (compressed) tar file.

\section{shexit}

{\bf shexit} is a shell script to print a warning message in a standard
fashion in case of abnormal termination of one of the installation scripts.

\section{sumlog, sumlog2 and sumlog3}

{\bf sumlog}, {\bf sumlog2} and {\bf sumlog3} are shell scripts to summarize 
logfiles from installation
jobs looking for potential problems. They use various control files, such as
{\bf sumlog.grep}, {\bf sumlog.sed}, {\bf sumlog.sed2}.

\section{testpack}

\section{xdiff}

{\bf xdiff} is a shell script to compare complete directories.

\section{xmv}

{\bf xmv} is a shell script to {\it mv} a set of files.

\section{xvi}

{\bf xvi} is a shell script to run the {\it vi} editor in a separate
window as a subprocess.

\section{yexpand}

{\bf YEXPAND} is a shell script to expand environmental variables in
{\bf PATCHY} cradles. 

