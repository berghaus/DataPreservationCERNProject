\chapter{Accessing the asis server at CERN}
\label{sect-ASIS}

{\bf N.B. use {\it asisftp.cern.ch} when accessing asis
as an ftp server and {\it asisnfs.cern.ch} when accessing
asis as an NFS server.} Originally, both of these services
were offered on asis01 and old documentation may refer to
this address. 
\index{asis}
\index{asis01}
\index{asisnfs}
\index{asisftp}

\section{Accessing the asis server as a file repository}
You may access the {\bf asis} server via anonymous ftp as show below.

\begin{XMPt}{Accessing the {\bf asis} server via anonymous ftp}
 
ftp asisftp.cern.ch      (IP address 128.141.202.89, userid anonymous)
Connected to asisftp.cern.ch.
220 asisftp FTP server (Version 2.0WU(14) Fri Sep 17 15:39:37 MET DST 1993) ready
.
Name (asisftp:jamie): anonymous
331 Guest login ok, send your complete e-mail address as password.
Password:
230-                   ________________________________________
230-                   Application Software Installation Server
230-                   ________________________________________
230-
230-   Welcome to the ASIS ftp server, developed by the CERN Computing and
230-   Networking Division to serve the High Energy Physics research community.
230-
230-   ftp clients may abort due to improper handling of such introductory
230-   messages. A dash (-) as the first character of your pw will suppress it.
230-
230-   The CERNlib software, located in the "cernlib" directory, is covered by
230-   CERN copyright. Before taking any material from this directory, please
230-   read the copyright notice "cernlib/copyright".
230-
230-   Please contact cernlib@cernvm.cern.ch for site registration. General
230-   support questions should be addressed to asis-support@asis01.cern.ch. 
230-
230 Guest login ok, access restrictions apply.
ftp> dir
200 PORT command successful.
150 Opening ASCII mode data connection for /bin/ls.
total 72
-rw-r--r--   1 cernlib  local        5991 Apr  7  1993 README.cernlib
d--x--x--x   2 root     local         512 Sep 23 15:30 bin
drwxr-s---   2 cernlib  local        1024 Nov  2 15:07 cernlib
-rw-r--r--   1 cernlib  local        1490 Nov 13  1992 cernlib.registration
-rw-r--r--   1 asis     software    10107 Jun 15 09:59 cnasdoc.faq
drwxr-xr-x   7 cernlib  local         512 Oct 22 16:39 cnl
drwxr-xr-x   2 root     daemon        512 Sep 17 10:08 dev
d--x--x--x   2 root     local         512 Oct 26  1992 etc
drwxrwxr-x   2 defert   software      512 Jul  2 13:04 hepix
drwxrwxr-x 255 eric     staff        8704 Nov 16 12:22 preprints
drwxr-x---  20 asis     local         512 Sep 17 13:53 pub
dr-xr-xr-x   4 root     local         512 Sep 17 10:07 usr
drwxr-x---   2 asis     local         512 Sep 17 16:57 usr.local
226 Transfer complete.
ftp> cd cernlib
250-Please read the file README
250-  it was last modified on Wed Oct 27 17:32:33 1993 - 21 days ago
250 CWD command successful.
ftp>get README
200 PORT command successful.
150 Opening ASCII mode data connection for README (1288 bytes).
226 Transfer complete.
1307 bytes received in 0.006299 seconds (202.6 Kbytes/s)
ftp> ls
200 PORT command successful.
150 Opening ASCII mode data connection for file list.
next_mach21
next_mach20
hp700_ux807
rs_aix32
pmax_ul4
sun4c_411
pc_dos50
sg_irix40
doc
copyright
rules
dec_ultrix4
README
alpha_osf1
apo_sr10
a10_sr10
hp700_ux90
share
sun4_solaris2
convex_os10
cray_unicos6
pc_linux
pro.src.car
new.src.car
mac_os6
alpha_wnt
sg_irix51
226 Transfer complete.
ftp> 
\end{XMPt}
 
You will only be able to access the {\bf cernlib} directory if you
are registered. If you cannot access the {\bf cernlib} directory,
follow the procedure below.

\section{Registering for the asis server}
If you are not registered you will not be able to access the
cernlib directory. In this case retrieve the {\bf cernlib.registration} file,
fill it and return it via electronic mail. Access will normally be enabled within
one working day.

\begin{XMPt}{Obtaining the {\bf CERNLIB} registration form}
ftp> get cernlib.registration
200 PORT command successful.
150 Opening ASCII mode data connection for cernlib.registration (1490 bytes).
226 Transfer complete.
1532 bytes received in 0.01752 seconds (85.37 Kbytes/s)
ftp> get README.cernlib
200 PORT command successful.
150 Opening ASCII mode data connection for README.cernlib (5991 bytes).
226 Transfer complete.
6146 bytes received in 0.02635 seconds (227.8 Kbytes/s)
ftp> quit
221 Goodbye
\end{XMPt}

\begin{XMPt}{The {\bf cernlib} registration form}

              Access to asisftp.cern.ch (anonymous ftp)
                        Registration Form
------------------------------------------------------------------------
This form is used to update our database on CERNlib users. This allows us
to keep you informed of various CERNlib issues and follow up requests for
software and documentation.  

Please answer ALL of the questions. In particular, it is extremely useful
for us to know which systems and packages are of most importance.

Please send this information back by e-mail to cernlib@cern.ch.
 
Thank you in advance, CERN Program Library Office
------------------------------------------------------------------------
 
 Name ...........:
 Email address ..:
 
 Postal Address .:
 
 
 
 Phone number ...:
 Fax number .....:                      (if available)
 
 Machine(s) to be enabled to access CERNlib software (please justify if
 you indicate more than one)
                          Machine1          Machine2     ....
 Internet name ..:
 IP address .....:
 Workstation type:
 Access by (1) ..:
 
    (1) please indicate the main role of the specified systems, e.g.
        personal workstation, group server, deparmental system etc.)
        registered (personal station, group system, university system,etc)
 Research area ..:
  (CERN experiment, HEP, chemistry, medicine, engineering, etc.)
 
 Which programs (all CERNlib, fragments - which, public domain - which,
 etc) and versions ( Sun, HP, etc) will you be taking regularly?
 
\end{XMPt}

\begin{XMPt}{CERNLIB README file} 
This information file concerns the distribution, use and installation of
the CERN program library with particular emphasis on its availability
via the asis server.
 
Legal and commercial regulations covering the usage of this library are
described in the file "cernlib/copyright", the contents of which you
shall be deemed to have taken note.
 
To get access to the library material, users must be registered. The
procedure to follow is described in the file cernlib.registration,
which is also in this directory. Proceed as follows-
 
 cd /tmp
  ftp asisftp.cern.ch
 (IP address 128.141.202.89, userid anonymous)
 ftp> get cernlib.registration
 ftp> quit
* fill it and send it back to cernlib@cernvm.cern.ch
* when the request has been processed you will be notified by e-mail.
 
 
ORGANISATION
------------
On asis the library organisation is a tree structure as follows:
 
  README.cernlib   cernlib    cernlib.registration   .........
		      |
		      |
      -------------------------------------------
      |            |           |         |  |   |
 dec_ultrix4  hp700_ux807  sun4c_411   .......  transarc names which try
      |                                to indicate the hardware type and
      |                                operating system, e.g dec_ultrix4
      |                                for Decstation with Ultrix 4.x
      |
    ------------------------------------------------------------------
    |    |    |    |     |      |       |    |     |      |      |   |
   pro  new  old  cmz  patchy  mad  ........ distribution levels and
         |                                   products directories
	 |
    --------------------------
    |    |    |    |    |    |
   tar  bin  lib  src  mgr  doc
 
where the directories contain
 
tar    Compressed tar files for efficient storage and transfer
bin    Ready to run modules, eg paw, paw++, kxterm, zftp, etc...
lib    Object libraries (ar format), eg packlib, graflib, etc...
src    Source files in Patchy format (card and cradles)
mgr    Tools and files for installers
doc    Documentation, mostly for Monte-Carlo libraries (mclibs),
       for other documentation of library material, see the /cernlib/doc
       tree.
 
 
DISTRIBUTION BY FTP
-------------------
The fastest and most convenient way to get parts or all of the library
products is by anonymous ftp to asisftp. If you want have the whole
distribution, complete sub-directories or packages, the best method is
to transfer the relevant compressed tar files and run the plitar shell
script to unpack them.  The compression factor is approximately 50%,
giving a substantial reduction in network traffic and transfer time.
 
To transfer a small number of specific products, go to the appropriate
directory to get the files you want. But you must be aware that in many
cases to run one module, you may need other files or modules too.
 
Examples
--------
1) To get the whole current production (pro) version of the Cern Program
   Library for an HP:
 
cd /tmp              (or the directory where you want to temporarily store
                      the compressed tar files)
ftp asisftp.cern.ch       (IP address 128.141.202.89)
    (give userid anonymous and password your e-mail address at the ftp prompts)
ftp> cd cernlib/hp700_ux807/pro/tar
ftp> get plitar
ftp> mget README*
ftp> mget *.contents
ftp> binary
ftp> mget *.Z
ftp> quit
 
Then
for csh do:
     setenv CERN /cern  (or where you want the files to be unpacked)
     setenv PLITMP /tmp (or the directory where you stored the tar files)
for sh/ksh do:
     CERN=/cern;export CERN     (see comments for csh)
     PLITMP=/tmp;export PLITMP  (see comments for csh)
     plitar xvf
will uncompress and untar the files into the specified directory.
More details of plitar are given below.
 
 
2) To get the current production (pro) versions of pawX11, paw++ and kxterm
   for a Decstation running Ultrix;
 
ftp> cd cernlib/dec_ultrix4/pro/bin
ftp> binary
ftp> get paw           (a shell script to invoke pawX11)
ftp> get pawX11        (the X11 version)
ftp> get paw++         (Motif version, needs X11 Release 4  and Motif 1.1)
ftp> get kxterm        (xterm handler for paw++, must be in the search PATH)
ftp> quit
 
 
Notes on the ftp access
-----------------------
For a variety of reasons (including ease of access, security, disk
space) many of the files and directories are reached by symbolic links.
This has the unfortunate side effect that you may lose your way trying
to go back using the ../ method until you get used to the tree. When in
doubt,
  cd /cernlib
will return you to the top of the tree.
 
At present, there are a limited number of ftp connections and at busy
periods you might be refused. We hope to improve this soon.
 
 
Notes on installing the tar files on your machine
-------------------------------------------------
- Put the plitar script in a convenient place ( e.g. $HOME/bin ) and make
  it executable (chmod +x plitar).
- Run plitar with the needed parameters, e.g.:
    plitar tvf           to verify the contents of the downloaded files
    plitar xvf cernlib   to unpack the tar files for the cern
                         library directory for example.
 
The plitar command makes use of two/three environment variables:
 
      Variable    Default   defining
      ----------  -------   ---------------------
  -   CERN        /cern     target directory
  -   PLITMP      /tmp      location of tar files
( -   PLIUWC                obsolete since 93b, was type of machine before. )
 
Any can be redefined using setenv (in C-shell) or export (in sh,ksh)
For example:
 
      setenv PLITMP $HOME/tmp    or      export PLITMP=$HOME/tmp
 
to redefine the area where you installed the tar files (maybe because
/tmp is too small)
 
DISTRIBUTION on TAPE
--------------------
Tapes with the complete distribution in compressed tar format can be
ordered from the CERN Program Library Office, email address
 cernlib@cernvm.cern.ch
who will arrange the formalities.
\end{XMPt}
\section{Accessing the asis server as a file server}
