%%%%%%%%%%%%%%%%%%%%%%%%%%%%%%%%%%%%%%%%%%%%%%%%%%%%%%%%%%%%%%%%%%%
%                                                                 %
%   EPIO User Guide -- LaTeX Source                               %
%                                                                 %
%   Chapter 3                                                     %
%                                                                 %
%   The following external EPS files are referenced:              %
%                                                                 %
%   Editor: Michel Goossens / CN-AS                               %
%   Last Mod.: 18 Nov 1993 22:55 mg                               %
%                                                                 %
%%%%%%%%%%%%%%%%%%%%%%%%%%%%%%%%%%%%%%%%%%%%%%%%%%%%%%%%%%%%%%%%%%%
 
 
\Filename{H1EPIO-User-routines}
\chapter{User Routines}
\label{sec:H1Userroutines}

\Filename{H2EPIO-User-routines-initialisation}
\section{Initialisation}
 
\Shubr{EPINIT}{}
 
Must be called once per job, in the  root. Pulls common block
\Lit{/EPCOMM/} in and initializes the package.
 
More than 10 units can be handled simultaneously as follows:
 
\begin{enumerate}
\item Load a common block \Lit{/EPCOMM/} that is big enough 
      (see \Lit{$dim} as dimension of \Lit{LIST}, 
       table~\ref{tab:statuswords})
\item After the call to \Rind{EPINIT} set \Lit{NMUNIT} to the new maximum.
\end{enumerate}
 
\Filename{H2EPIO-User-routines-reading}
\section{Reading}
 
\Shubr{EPREAD}{(LUNIT,MODE,NW*,*IREC*,IBUF*,IERR*)}
 
\begin{DLtt}{123456}
\item[{\rm\bf Input parameters:}]
\item[LUNIT] logical unit number
\item[MODE] \Lit{=j,  j=1,2,3} : get next logical record data\\
            \Lit{j=1} : data transferred in packed form\\
            \Lit{j=2} : data unpacked as 16-bit bytes/word\\
            \Lit{j=3} : data unpacked as 32-bit bytes/word\\
            \Lit{\ =10+j, j=1,2,3} as above: get data of current logical record
            (only possible after a previous call with \Lit{j=20}).\\
            \Lit{\ =20} : get header of next logical record. The header will
             always be unpacked in units (16 or 32-bit words).\\
            \Lit{\ =30} : get next physical header.\\
            The header will always be unpacked in units.
\item[{\rm\bf Input/Output buffer:}]
\item[IBUF] user provided unit buffer. Must not be touched by user.
            Minimum size = (16*status word 1 - 1)/(bits/word) + 1
\item[{\rm\bf Output parameters:}]
\item[NW] no. of words transferred into \Lit{IREC}.\\
          For \Lit{j=1} no. of units, for \Lit{j=2} or \Lit{j=3} 
          no. of 16 or 32-bit words, i.e.,
          number of machine words occupied in \Lit{IREC}
\item[IREC] user provided area to store the header or data requested
\item[IERR] error flag. See separate list.
\end{DLtt}
\subsection*{Remarks:}
\begin{enumerate}
\item For reading the old EP format status word 27 must be set to zero by a
\begin{XMP}
             CALL EPSETW(LUN,27,0,IERR)
\end{XMP}
and the previous call in the old ``\Lit{EVENT}'' package
\begin{XMP}
             CALL EVENT(IARRAY,ISTAT)
\end{XMP}
has to be replaced by a call to \Rind{EPREAD} with \Lit{j=2}. 
In this case, as for \Lit{EVENT}, the complete record 
(including the header) will be transferred. 
Please note that the  data  now  start  in  \Lit{IREC(1)},
instead of \Lit{IARRAY(2)} as before and the data are unpacked as 16-bit
bytes/word only.
 
Old format ``special records'' are not specifically decoded by
\Rind{EPREAD} but return \Lit{IERR=7}. 
The user is left to do the unpacking himself, e.g.:
\begin{XMP}
    CALL EPREAD(LUN,2,NW,irec,IBUF, ierr)
    :
    :
    IF (IERR.NE.7) GO TO 10
    NW=IWD16(IBUF,1)
    CALL BLO16W(IBUF,1,IREC,1,NW)
   10 CONTINUE
\end{XMP}
\item To read consecutive files of multifile tapes the user has only
to make further calls to \Rind{EPREAD} after each end of file (signified by
\Lit{IERR=1}). 
In the case of labelled tapes the labels will be treated as
data and \Rind{EPREAD} will return a length error on reading (\Lit{IERR=5}). 
The user program should choose to continue reading in this case. 
A combination of
these techniques with \Lit{MODE=30} can provide the same facilities as
\Rind{EVENTIN} in the old package. 
To prevent tapes running off the end of reel
users should stop reading at end of information (\Lit{IERR=3}), signified by 2
consecutive EOF's, or by using the file or record count.
\end{enumerate}

\Shubr{EPFHDR}{(LUNIT,MLUSER,IHEAD*,*IBUF*,IERR*)}
 
Fast logical record header reading routine.
 
\begin{DLtt}{123456}
\item[{\rm\bf Input parameters:}]
\item[LUNIT] user unit number
\item[MLUSER] no. of header words transferred to user
\item[{\rm\bf Input/Output buffer:}]
\item[IBUF] user buffer
\item[{\rm\bf Output parameters:}]
\item[IHEAD] \Lit{MLUSER} words of logical record header (regardless of actual
             length or of status word 26)
\item[IERR] error number
\end{DLtt}

\subsection*{Conditions for use:}
\begin{enumerate}
\item 16-bit units only (i.e. logical record header consists of 16-bit
      words)
\item No spanned headers
\item Always \Lit{MLUSER} words transferred to user (word 3 is header length).
      This may exceptionally lead to a program range error if the input
      buffer \Lit{IBUF} coincides with the end of the user program.
      Remedy: increase size of \Lit{IBUF} by 
      \Lit{16*MLUSER/(no. of bits per word)}
\item No headerless blocks
\item No old EP format
\item No unknown length records
\end{enumerate}

In addition, no checks performed whether input unit, or whether header
cut.
Conditions 2. to 6. are always fulfilled when writing with EPIO,
condition 1. is default when writing with EPIO.

Calls to this routine are entirely compatible with \Rind{EPREAD} calls.

\Shubr{EPFRD}{(LUNIT,MODE,NW,IREC,IBUF,IERR)}

Fast logical record data reading routine.
 
\begin{DLtt}{123456}
\item[{\rm\bf Input parameters:}]
\item[UNIT] user unit number
\item[MODE] one of 11, 12, 13 (see \Rind{EPREAD})
\item[{\rm\bf /Input/Output buffer:}]
\item[IBUF] user buffer
\item[{\rm\bf Output parameters:}]
\item[NW] no. of words in \Lit{IREC}
\item[IREC] record transferred to user
\item[IERR] error number
\end{DLtt}

\subsection*{Conditions for use:}

\begin{enumerate}
\item 16-bit units only (i.e. logical record header consists of 16-bit
       words)
\item no headerless blocks
\item no old EP format
\item no unknown length records
\item modes 11, 12, 13 only (otherwise error 8)
\end{enumerate}
 
In addition, no checks performed whether input unit.
Conditions 2. to 4. are always fulfilled when writing with EPIO,
condition 1. is default when writing with EPIO.
User data will be truncated at the value in status word 2,
but no error will be signalled.
 
The routine is poorly protected against bad user data on input,
and may lead to program aborts in cases where \Rind{EPREAD} would recover.
In particular \Rind{EPFRD} does not test for the condition which leads to
\Lit{IERR=18} from \Rind{EPREAD}, and performs unpredictably in the case of this
condition.
 
Therefore, in case the user wants to be better protected (e.g.,
after a fatal error in \Rind{EPFRD}), he should use \Rind{EPREAD}.

Calls to this routine are entirely compatible with \Rind{EPREAD} calls.

\Filename{H2EPIO-User-routines-writing}
\section{Writing}

\Shubr{EPOUTS}{(LUNIT,MODE,NW,IREC,*IBUF*,IERR*)}

Writes one record with standard logical record header (4 words, see
important remark on logical record headers in 
section~\ref{sec:remarkslogicalrecords}.

\begin{DLtt}{123456}
\item[{\rm\bf Input parameters:}]
\item[LUNIT] logical unit number
\item[MODE] transfer more
  \begin{DLtt}{1}
    \item[1:] transfer data as they are (bit string)
    \item[2:] pack data before writing, given as 16-bit right adjusted
    \item[3:] pack data before writing, given as 32-bit right adjusted
  \end{DLtt}
\item[NW] no. of words to be written.\\
    For \Lit{MODE=1} in units (as defined by physical header word 11),
    for \Lit{MODE=2} or \Lit{=3} in 16 or 32-bit words respectively, i.e. the
    number of machine words occupied
\item[IREC] area containing the user data.
\item[{\rm\bf Input/Output buffer:}]
\item[IBUF] user provided unit buffer. Must not be touched by user.\\
            Minimum size = \Lit{(16*status word 1 - 1)/(bits/word) + 1}
\item[{\rm\bf Output parameters:}]
\item[IERR] error flag. See separate list.
\end{DLtt}

\Shubr{EPOUTL}{(LUNIT,MODE,NH,IH,NW,IREC,*IBUF*,IERR*)}
 
This routine writes a user specified logical record header in front
of the data, instead of the header produced automatically from the
status words. Words 1,3 (and possibly 2 and 4) will be set by the
writing routine in any case (basic protection).
 
The logical record data are given in the same call.

\begin{DLtt}{123456}
\item[{\rm\bf Input parameters:}]
\item[LUNIT] logical unit number
\item[MODE] transfer more
  \begin{DLtt}{1}
    \item[1:] transfer data as they are (bit string)
    \item[2:] pack data before writing, given as 16-bit right adjusted
    \item[3:] pack data before writing, given as 32-bit right adjusted
  \end{DLtt}
\item[NH] header length in words
\item[IH] array containing header (in unpacked form)
\item[NW] no. of words to be written.\\
    For \Lit{MODE=1} in units (as defined by physical header word 11),
    for \Lit{MODE=2} or \Lit{=3} in 16 or 32-bit words respectively, i.e. the
    number of machine words occupied
\item[IREC] area containing the user data.
\item[{\rm\bf Input/Output buffer:}]
\item[IBUF] user provided unit buffer. Must not be touched by user.\\
            Minimum size = \Lit{(16*status word 1 - 1)/(bits/word) + 1}
\item[{\rm\bf Output parameters:}]
\item[IERR] error flag. See separate list.
\end{DLtt}

\Shubr{EPCLOS}{(LUNIT,*IBUF*,IERR*)}
 
Routine to close output units (write existing physical
buffer out - writing may continue afterwards). The physical block count
will continue from its current value at re-opening.
 
Closing an input unit has no effect.
 
\fbox{\textbf{Mandatory at end of job for all output units (or \Rind{EPEND})}}

\begin{DLtt}{123456}
\item[{\rm\bf Input parameters:}]
\item[LUNIT] logical unit number
\item[{\rm\bf Input/Output buffer:}]
\item[IBUF] user provided unit buffer. Must not be touched by user.\\
\item[{\rm\bf Output parameters:}]
\item[IERR] error flag. See separate list.
\end{DLtt}

\Filename{H2EPIO-User-routines-rewind}
\section{Auxiliary Unit Operations---Rewind}

\Shubr{EPRWND}{(LUNIT,*IBUF*,IERR*)}
 
Rewinds to the start of the current file in use on this unit.
 
Rewinding an output unit triggers writing the last buffer out and
writes an end of file. After a rewind operation (and only in that case)
the user may switch from reading to writing or vice versa.
 
The physical header is reset to the standard header after a rewind.

\begin{DLtt}{123456}
\item[{\rm\bf Input parameters:}]
\item[LUNIT] logical unit number
\item[{\rm\bf Input/Output buffer:}]
\item[IBUF] user provided unit buffer. Must not be touched by user.\\
\item[{\rm\bf Output parameters:}]
\item[IERR] error flag. See separate list.
\end{DLtt}

\subsection*{Note:}

On the Apollo rewinding a magnetic tape file is done by
explicitly closing the stream.

\Filename{H2EPIO-User-routines-drop}
\section{Drop}

\Shubr{EPDROP}{(LUNIT,IERR*)}
 
Drops an existing unit. User should call \Rind{EPCLOS} beforehand when writing.
This concerns only this package, i.e. the space in common block
\Lit{/EPCOMM/} becomes free for another unit. For the system, the unit
still exists, and may therefore be ``opened'' again later.
The physical block count will be reset at re-opening.

\begin{DLtt}{123456}
\item[{\rm\bf Input parameters:}]
\item[LUNIT] logical unit number
\item[{\rm\bf Output parameters:}]
\item[IERR] error flag. See separate list.
\end{DLtt}

\Filename{H2EPIO-User-routines-writeEOF}
\section{Write EOF mark}

\Shubr{EPEND}{(LUNIT,*IBUF*,IERR*)}

\fbox{\textbf{Mandatory at end of job for all output units (or \Rind{EPCLOS})}}
 
Acts like \Rind{EPCLOS}, but in addition writes an end of file mark on
\Lit{LUNIT} if output unit. On IBM and UNIVAC it must be called
to properly terminate output units.
 
Note that on IBM this has the same effect as \Rind{EPCLOS} - the filemark
written is backspaced over so a new write to the same logical unit
will not result in a new physical file but will merely continue the
existing file. Users who really want to write a multi-file tape on
IBM within the same job step must call \Rind{EPEND} and then 
\Lit{IODD}~\cite{bib-IOPACK}.

\begin{DLtt}{123456}
\item[{\rm\bf Input parameters:}]
\item[LUNIT] logical unit number
\item[{\rm\bf Input/Output buffer:}]
\item[IBUF] user provided unit buffer. Must not be touched by user.\\
\item[{\rm\bf Output parameters:}]
\item[IERR] error flag. See separate list.
\end{DLtt}

\Filename{H2EPIO-User-routines-reading}
\section{Utility Routines}

\Shubr{EPUREF}{(LUNIT)}
 
Returns internal reference number for logical unit in \Lit{LREF} in the
common block \Lit{/EPCOMM/}. 
If \Lit{LUNIT} is invalid or there are no units
open, then \Lit{LREF=0}.

\begin{DLtt}{123456}
\item[{\rm\bf Input parameters:}]
\item[LUNIT] logical unit number
\end{DLtt}

\Shubr{EPDEFU}{(LUNIT,CHNAME,IERR*)}

Allows to specify a file name other than the EPIO default name.

\begin{DLtt}{123456}
\item[{\rm\bf Input parameters:}]
\item[LUNIT] logical unit number
\item[CHNAME] Character variable, specifying the file name.
\item[{\rm\bf Output parameters:}]
\item[IERR] error flag. See separate list.
\end{DLtt}
 
\subsection{Get Status Words}
\Shubr{EPGETW}{(LUNIT,NUMBER,IW*,IERR*)}

\begin{DLtt}{123456}
\item[{\rm\bf Input parameters:}]
\item[LUNIT] logical unit number
\item[NUMBER] Number of the desired status word.
\item[{\rm\bf Output parameters:}]
\item[IW] value of the status word.
\item[IERR] error flag. See separate list.
\end{DLtt}

\Shubr{EPGETA}{(LUNIT,NW,IWS*,IERR*)}

\begin{DLtt}{123456}
\item[{\rm\bf Input parameters:}]
\item[LUNIT] logical unit number
\item[NW]    Number of desired status words (always start at word 1).
\item[{\rm\bf Output parameters:}]
\item[IWS] value of status words \Lit{1..NW}.
\item[IERR] error flag. See separate list.
\end{DLtt}

\Shubr{EPGETC}{(NWCONT*,IWCONT*)}

\begin{DLtt}{123456}
\item[{\rm\bf Output parameters:}]
\item[NWCONT] no. of overall control words (word 3 of common \Lit{/EPCOMM/}).
\item[IWCONT] control words (words \Lit{1..NWCONT} 
              of list of common \Lit{/EPCOMM/})
\end{DLtt}

\subsection{Set Status Words}
 
These routines enable the user to change defaults kept in the  status
words, such as the maximum length of the physical block he wants to
read, the maximum length of the logical record, and so on.
 
Normally, control words should be changed before any writing on  that
unit takes place. Otherwise, it is strongly recommended to make a
close call to the corresponding unit (\Rind{EPCLOS}) beforehand in order to
avoid problems. The user should be fully aware of what he is doing,
and foresee the possible effects of any change.
 
For the use of \Rind{EPSETA}, and \Rind{EPSETC} 
(setting more than one word), it is
recommended to get hold of the existing status words by calls to
\Rind{EPGETA} or \Rind{EPGETC}, modify where desired, and restore.

\Shubr{EPSETW}{(LUNIT,NUMBER,IWORD,IERR*)}

\begin{DLtt}{123456}
\item[{\rm\bf Input parameters:}]
\item[LUNIT] logical unit number
\item[NUMBER] Number of the status word to change.
\item[IWORD] new value of the status word.
\item[{\rm\bf Output parameter:}]
\item[IERR] error flag. See separate list.
\end{DLtt}

\Shubr{EPSETA}{(LUNIT,NW,IWORDS,IERR*)}

\begin{DLtt}{123456}
\item[{\rm\bf Input parameters:}]
\item[LUNIT] logical unit number
\item[NW] Number of status words to set (always start at word 1).
\item[IWORDS] new value for the status words \Lit{1..NW}.
\item[{\rm\bf Output parameter:}]
\item[IERR] error flag. See separate list.
\end{DLtt}

\Shubr{EPSETC}{(NW,IW)}

\begin{DLtt}{123456}
\item[{\rm\bf Input parameters:}]
\item[NW] no. of overall control words to set (from 1 up to \Lit{NCONT})
\item[IW] new overall control words.
\end{DLtt}

\fbox{\textbf{Handle with care!}}

\Shubr{EPADDH}{(LUNIT,NH,IH,*IBUF*,IERR*)}
 
Adds a user portion to the standard physical header. The current block
is written out beforehand if not empty.
 
The physical header is reset to the standard header after a rewind.

\begin{DLtt}{123456}
\item[{\rm\bf Input parameters:}]
\item[LUNIT] logical unit number
\item[NH]  number of 16-bit words to add (0 resets the standard physical header).
\item[IH]  user physical header words to add (16-bit, right adjusted, 0 filled)
\item[{\rm\bf Input/Output buffer:}]
      User provided unit buffer. Must not be touched by the user.
\item[{\rm\bf Output parameter:}]
\item[IERR] error flag. See separate list.
\end{DLtt}

\Shubr{EPUPDH}{(LUNIT,NH,IH,IBUF,IERR)}
 
Updates the (optional) user portion of the physical block header.
The necessary space for this optional portion must have been reserved
by the user through a previous call to \Rind{EPADDH} (typically once at the
start of the job). Whereas a call to \Rind{EPADDH} causes the existing (if
any) block to be written out, \Rind{EPUPDH} does not have this effect.

\begin{DLtt}{123456}
\item[{\rm\bf Input parameters:}]
\item[LUNIT] logical unit number
\item[NH] no. of user words in header to be updated.
          It is cut at the actual no. of extra user words available in the
          physical block header.
\item[IH] Array containing the new values.
          Words \Lit{1..NH} of \Lit{IH} will replace the current words 
          \Lit{1..NH} of the user portion in the header.
\item[{\rm\bf Input/Output buffer:}]
      User provided unit buffer. Must not be touched by the user.
\item[{\rm\bf Output parameter:}]
\item[IERR] error flag. See separate list.
\end{DLtt}

\subsection{Print Overall Status}

\Shubr{EPSTAT}{}

Prints the overall control words, and all unit status words in their

actual state. Useful for debugging.

\Shubr{EPIOT}{(LUN)}

Print PAM title on \Lit{LUN}.
