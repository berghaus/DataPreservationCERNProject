%%%%%%%%%%%%%%%%%%%%%%%%%%%%%%%%%%%%%%%%%%%%%%%%%%%%%%%%%%%%%%%%%%%
%                                                                 %
%   EPIO User Guide -- LaTeX Source                               %
%                                                                 %
%   Chapter 1                                                     %
%                                                                 %
%   The following external EPS files are referenced:              %
%                                                                 %
%   Editor: Michel Goossens / CN-AS                               %
%   Last Mod.: 25 Nov 1993 23:00 mg                               %
%                                                                 %
%%%%%%%%%%%%%%%%%%%%%%%%%%%%%%%%%%%%%%%%%%%%%%%%%%%%%%%%%%%%%%%%%%%
 
\Filename{H1EPIO-Introduction}
\chapter{Introduction}
\label{sec:H1introduction}

\Filename{H2EPIO-Introduction-Usage}
\section{Usage}
 
The package is designed as to make almost all features of the very
flexible EP format (see section~\ref{sec:EPIOformat}) 
available to the user. One can and
should however, use it in as simple a way as possible. In the following,
we give a series of applications ranging from the simplest use to the
most complex one. It should be stressed that for the user wanting to use
the package as a black box,

\begin{UL}
\item no knowledge of the actual format is required
\item the only routines he has to use and understand are
\end{UL}

\begin{XMP}
          EPINIT
          EPREAD(EPFHDR,EPFRD)
          EPOUTS
          EPRWND
          EPCLOS
          EPEND
\end{XMP}
 
Obviously, the more advanced user will have to learn more.
 
After the examples, we give a description of all routines in the
package, and of the utility routines going with it (which are,
however, independent of the package and are intended to fulfil the most
common user needs).
 
While it should not concern most users, the following overview of
the format might help to clarify some problems. The basic "units"
of information are 16-bit words. However, as of version 1.56, all
information including the physical header, maybe given in 32-bit words.
The header of each
physical record or block is usually 12 words. The user's logical
data (or events) can be specified by the user to be in ``units'' of 16 or
32-bit words. Each logical record has a header (normally 4 units long)
which specifies the record length and header length in ``units'' and other
general identifiers. However, irrespective of the value of "unit" the user
can treat the logical data part of the record as 16-bit, 32-bit
or packed quantities, by specifying different ``MODES'' to \Rind{EPREAD} and
\Rind{EPOUTS}, etc. 
Packed quantities are applicable to other types of data,
for example CDC 60-bit words or 24-bit Camac, but the user has to
look after the unpacking himself. The complete description of the
format can be found in section~\ref{sec:EPIOformat}.
 
It should be stressed that the package does not perform automatic
number format conversion, but only packs and unpacks (16 and 32-bit
words). However, there are conversion routines from and to IBM floating
and integer formats and from and to ASCII. To simplify the portability
of data between different computer types, 32-bit headers are written
in IBM format.
 
Users are invited to study the examples carefully concerning
initialisation, termination, and definition of buffer sizes.

\Filename{H2EPIO-Introduction-Usage-UNIX}
\section{Usage on UNIX}
\label{sec:usageunix}
\index{UNIX|(}

The Unix version is restricted to fixed length blocks with full
padding which is in any case the EPIO default.
If a different blocksize than the EPIO default (1800 16-bit words)
is required, it must be specified correctly in status word 1 by a
call to \Rind{EPSETW}, e.g. for a blocksize of 1800 bytes on unit 11
\begin{XMP}
 CALL EPSETW(11,1,900,IERR).
\end{XMP}
Unfortunately, this means that you can not give an arbitrarily large
blocksize to allow your program to read files with different block
sizes.
Otherwise, existing applications should be portable to Unix
workstations.
 
The choice of C or Fortran for the basic I/O
\index{Input/Output}%
\index{Fortran}%
\index{C}%
can be selected using the status word 33,
which is set by the user as detailed in the table below.
\begin{flushleft}
\begin{tabular}{ccc}
\bf Value & Type of I/O          & Default filename (\Lit{lun=nn})\\
       0  & Fortran sequential   & \Lit{for0nn}                   \\          
       1  & Fortran direct access& \Lit{epionn}                   \\
       2  & C using CFIO         & \Lit{epionn}
\end{tabular}
\end{flushleft}
By default, status word 33 is set to 2.
Although it is the old default, Value 0 is not portable and seems
to give more problems.
Values 1 and 2 give identical formats on most systems and are
portable.
 
The association with real filenames can be made in several ways:

\begin{enumerate}
\item Code the filename in the EPIO routine \Rind{EPDEFU} e.g.,
\begin{XMP}
 CALL EPDEFU(LUN,'epdata/test1',IERR)
\end{XMP} 
\item Use the UNIX \Lit{ln} command before execution, e.g.
\begin{XMP}
\Ucom{ln epdata/test epio01}
\end{XMP}
and follow it by:
\begin{XMP}
 CALL EPREAD(1,MODE,NW,IREC,IBUF,IERR)
\end{XMP}
\item Rename (\Lit{mv}) or copy (\Lit{cp}) the file to an EPIO name \Lit{epionn}.
\end{enumerate}
 
\subsection{Transferring files}
\index{FTP}
 
EPIO files for use on or created on mainframes 
should have fixed length blocks.
On VAX/VMS you can use the default settings.
On CERNVM the user should issue a filedef specifying \Lit{RECFM F} 
or \Lit{RECFM U}
and the correct blocksize, which most applications are doing already.
If not, a correct filedef is
\begin{XMP}
  FILEDEF IOFILE11 DISK MYEPIO TESTDATA A1 ( RECFM F BLOCK 3600
\end{XMP}
The files can then be transferred using FTP.
The most convenient way is to use the ZFTP utility and a good
\index{ZFTP}%
description of a session is given in Computer Newsletter 200.
This avoids the need to convert the data file or to modify the
EPIO application.
The actual transfer is performed by the command \Lit{putb}, e.g.-

\begin{XMP}
\Ucom{putb myepio.testdata test1 3600}
\end{XMP}
 
If ZFTP is not available on your system then you must use ftp
directly specifying the qualifier ``binary''.

For files created on a workstation for transfer to a mainframe
it is essential to create them as a byte stream.
Although it is clearly inefficient, conversion utilities between
the two formats will be provided.
 
A typical CERNVM ftp session would be
\begin{XMP}
\Ucom{tcpipibm ftp hostnm}
 > \Ucom{hostname}
 > \Ucom{password}
 > \Ucom{cd wdir}
 > \Ucom{binary fixed 3600}               for a blocksize of 3600 bytes
 > \Ucom{put file.iofile epio11}          or > \Ucom{get epio11 file.iofile}
 > \Ucom{quit}
\end{XMP} 
 
On VAX, the session is similar except that the EPIO file names
are of the form \Lit{FOR0nn.DAT}.
\index{INDEX|)}

\Filename{H2EPIO-Introduction-Usage-IBMVM}
\section{Usage on VM/CMS}
\index{VM/CMS|(}
 
The VM version uses \Rind{IOPACK}, and both packages are available
from the standard CERN program library PACKLIB, by using the command

\begin{XMP}
\Ucom{CERNLIB}
\end{XMP}
 
The \Lit{IOFILEnn} data sets read and written by EPIO must have filemode 4
(i.e. OS simulation) with options \Lit{"RECFM U"} and an appropriate blocksize.
 
For example, to access a disk data set \Lit{"RAWDATA DATA"} on logical
unit 10 for reading or writing an appropriate FILEDEF might be:
\begin{XMP} 
 FILEDEF IOFILE10 DISK RAWDATA DATA A4 (RECFM U BLKSIZE 3600
\end{XMP} 
At present, concatenation of data sets is not supported under VM,
but at CERN the FILEDEF option \Lit{"APPEND"} is available for disk files.
\index{VM/CMS|)}

\Filename{H2EPIO-Introduction-Usage-VAX}
\section{Usage on VAX}
\index{VAX|(}

\subsection{Disk files}

The usage of disk files does not require any particular action,
they are connected to logical units in the standard VMS way. If the
logical name \Lit{FORnnn} is defined then its translation will be used as
file name to connect, otherwise the file \Lit{FORnnn.DAT} will be used.
However, the file name may also be changed at run-time via a call
to the routine \Rind{EPDEFU}.
 
\subsection{Tape files}

The current version of EPIO will recognize when it reads or
writes a tape instead of disk, provided the device name is
of the standard form \Lit{MTAn:}
 
For example: Suppose you want to mount a tape with blocks of length
23040 bytes on unit 0, for the logical unit no. 1:

\begin{XMP}
MOUNT /FOR/BLOC=23040/REC=23040 MTAn: LABEL FOR001
\end{XMP}

where \Lit{LABEL} is just a place holder. 
For unlabelled tapes, on the CENTRAL 8600 at CERN the command should read

\begin{XMP}
SETUP/BLOC=23040/REC=23040/NOLABEL XUTnnn XUTnnn FOR001
\end{XMP}

for a non labelled tape, where the second X\Lit{UTnnn} is again a place
holder. Should the tape be labelled the qualifier
\Lit{/LABEL=ASCII} or \Lit{/LABEL=EBCDIC} is required.
If the tape is to be written the qualifier \Lit{/WRITE} must be
inserted before the \Lit{/[NO]LABEL} qualifier.
Byte swapping to/from tape may be done automatically when using EPIO,
if this is allowed by the drivers' firmware/hardware.
Tapes with EBCDIC labels should be positioned correctly using e.g.:

\begin{XMP}
SET MAGTAPE/SKIP=FILES:1 FOR001
\end{XMP}

Alternately label groups must be treated by the user by recovering
from EPIO error 5. The file name may be changed at run-time via a call
to the routine \Rind{EPDEFU}.
\index{VAX|)}

\Filename{H2EPIO-Introduction-Usage-Apollo}
\section{Usage on the Apollo/Domain}
\index{Apollo|(}

The Apollo version uses the stream I/O routines available in
the system library. The files written by EPIO are of type \Lit{REC} (variable
record length) but even \Lit{UASC} files can be handled provided the
logical block length is fixed and the first status word is properly set.
 
The default file name is \Lit{FORnnn.DAT} like on the VAX where nnn
is the logical unit number. 
This can be changed by calling \Rind{EPDEFU}.
 
Tapes are accessed via a magnetic tape descriptor file. The EPIO
routines address the file which in turn redirect the I/O operation to
the physical device. A magnetic tape descriptor file is created by
the command:

\begin{XMP}
   edmtdesc for011.dat -c -s lab no reo no civ yes spos yes
   edmtdesc for011.dat    -s f 1 rf u bl 3600 rl 3600 ascni no
\end{XMP}
 
This is the case for a standard EPIO unlabelled tape recorded with
a blocksize of 3600 bytes. In case of label or of different blocksize
the parameters should be changed accordingly. 
Type \Ucom{Help EDMTDESC} for
more details. The name of the Magnetic tape descriptor file can
be different from \Lit{FORnnn.DAT} provided the user calls \Rind{EPDEFU}.
\index{Apollo|)}
 
\Filename{H2EPIO-Introduction-Usage-IBMMVS}
\section{Usage on IBM (Wylbur MVS)}
\index{IBM MVS|(}

The IBM version of EPIO uses the input/output package \Rind{IOPACK}
(Z300 in the CERN Program Library).
DD names for the EPIO units are of the form
\begin{XMP} 
      IOFILEnn    nn = 1 to 99
\end{XMP} 
with  \Lit{DCB=(RECFM=U,BLKSIZE=iiii)}
 
The default version of EPIO does not allow concatenation of data sets
but this can easily be changed by setting status word 25 to non-zero;
\begin{XMP}
     CALL EPSETW(LUN,25,1,IERR)
\end{XMP}
\index{IBM}
\index{Usage on IBM}
\index{IFORTCLG}
\index{JFORTCLG}
\index{Concatenation of files on IBM}
\index{IOFILE}
\index{IOPACK}
\index{UNIT=AFF}
 
To access multiple logical units on the same physical unit
using \Lit{"UNIT=AFF"} requires a \Lit{"CALL \Rind{EPRWND}"}
in addition to \Rind{EPDROP}, so
that the unit is properly closed. The details for reading multifile
labelled tapes are in the description of \Rind{EPREAD}.
\index{IBM MVS|)}

\Filename{H2EPIO-Introduction-Usage-CDC}
\section{Usage on CDC}
\index{CDC|(}

The EPIO routines are available from the standard CERN Program
libraries, and use FORTRAN BUFFER IN/OUT to perform the I/O.
 
Magnetic tape files should be accessed specifying record type \Lit{RT=U}.
EPIO disk file default format on NOSBE:
\begin{XMP}
    MFA,MFB      RT=S,BT=C
\end{XMP}
\index{CDC|)}

\Filename{H2EPIO-Introduction-other-considerations}
\section{Other Considerations}

\subsection{Recommended Block Sizes}

For reasons of compatibility with different word lengths, the block size
should be a multiple of 480 bits = 60 bytes. The default block size
is 3600 bytes.
 
In addition, there are restrictions on the recommended physical
record sizes for certain devices. In adition,
the ``standard Fortran'' version, on which the UNIX version
is based, is restricted to block sizes in multiples of machine words.

\subsection{Byte swapping}
\index{byte swapping}

With the status word 27 set to 1 fully automatic byte swapping 
is the default.
This works of course only with the EP format as it requires
two special control words in the physical header.
 
To read EP files created in the \Lit{"OLD"} format (with a 6 word
physical header) the user must call \Rind{EPSETW} to set status
word 27 to 0.
The user can force EPIO to read his data as new
format by setting status word 27 to 2, avoiding peculiar side
effects when there are parity errors for example.
