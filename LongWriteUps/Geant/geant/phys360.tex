%%%%%%%%%%%%%%%%%%%%%%%%%%%%%%%%%%%%%%%%%%%%%%%%%%%%%%%%%%%%%%%%%%%
%                                                                 %
%  GEANT manual in LaTeX form                                     %
%                                                                 %
%  Michel Goossens (for translation into LaTeX)                   %
%  Version 1.00                                                   %
%  Last Mod. Jan 24 1991  1300   MG + IB                          %
%                                                                 %
%%%%%%%%%%%%%%%%%%%%%%%%%%%%%%%%%%%%%%%%%%%%%%%%%%%%%%%%%%%%%%%%%%%
\Origin{I.Gavrilenko}
\Documentation{K.Lassila-Perini, I.Gavrilenko}
\Submitted{17.11.93} \Revised{16.12.93}
\Version{Geant 3.16}\Routid{PHYS360}
\Makehead{Synchrotron radiation}
\section{Subroutines}

\Shubr{GSYNC}{}
\Rind{GSYNC} is called for electron and positrons
when the flag {\tt ISYNC} is set $>$ 0. It 
computes the magnetic field and calls
\Rind{GSYNGE} for the generation of the synchrotron radiation
photons. It reads them in the {\tt GEANT} stack.
\Rind{GSYNC} is called from \Rind{GTELEC}.

\Shubr{GSYNGE}{(GAMMA,AMASS,FLDT,STEP,NTR,ETR,STR)}
\Rind{GSYNGE} generates the synchrotron radiation 
for an e$^-$ or e$^+$ with the Lorentz factor {\tt GAMMA}
in the magnetic field {\tt FLDT} and when the step
length is {\tt STEP}. The number of the secondary
photons is {\tt NTR}, their energy is in the array {\tt ETR} and
their position along the step is in {\tt STR}.
\Rind{GSYNGE} is called by \Rind{GSYNC}.


\section{Method}
A charged particle in a magnetic field emits radiation.
The number of photons emitted in a second is
\begin{equation}
\frac{d^2 \! N}{d\varepsilon \, dt} = \frac{15 \sqrt{3}}{8} \frac{P}{\varepsilon_c}
\end{equation}

where $\varepsilon_c$ is the critical photon energy
(the median of the energy spectrum) and $P$ the
total radiated power:

\begin{eqnarray}
\varepsilon_c & = & \frac{3}{2} \hbar c \frac{\gamma^3}{\rho} \\
P             & = & \frac{2}{3} \frac{ e^2 c}{\rho^2}\beta^4 \gamma^4
\end{eqnarray}
$\gamma$ is the e$^-$/e$^+$ Lorentz factor $E_{tot}/m_e$ and
$\beta = 1 - 1/\gamma^2$. $\rho$ is the
curvature caused by the magnetic field.
For more detailed derivation of these equations,
see \cite{bib-MAIE}.
The velocity of the particle being $\beta c$, the number of
photons per meter is
\begin{eqnarray}
\frac{d^2 \! N}{d\varepsilon \, dx}  & = & 
         \frac{d^2 \! N}{d\varepsilon \, dt} \, \frac{1}{\beta c} \nonumber \\
  & = & \frac{5 \sqrt{3}}{6} \frac{e^2}{\hbar c} 
        \beta^3 \gamma \frac{1}{\rho} \nonumber \\
  & = & \frac{5}{2 \sqrt{3}} \, \alpha \, \frac{\gamma}{\rho} 
        \; \approx  \; 0.01053 \, \frac{\gamma}{\rho}
\end{eqnarray}
The fine structure constant $\alpha = e^2/(\hbar c)$ and it is
assumed that $\beta^3 \approx 1$.

The curvature in a magnetic field $B$ which has a component
$B_{\perp}$ transversal to the particle velocity can
be computed \cite{bib-PDGB}
\begin{equation}
\rho = \frac{p} {0.3 B_{\perp}}
\end{equation} 
where $p$ is the momentum of the particle in GeV. $B$ is in tesla
and $\rho$ is in meters.

Knowing the step length, the energy of the electron
and the curvature of the particle track in the magnetic field,
the number of photons in a step $\delta x$ can be sampled from
a Poissonian distribution around the mean value
\begin{equation}
\overline{n_{\delta x}} = 0.01053 \, \frac{\gamma}{\rho} \delta x
\end{equation}

Now, the energies of $n_{\delta x}$ photons have to be determined.
The energy distribution in a step follows the distribution  \cite{bib-MAIE}
\begin{equation}
f(\varepsilon) \, = \, \frac{dN}{d\varepsilon} \, \propto \, 
        \int_{\varepsilon / \varepsilon_c}^{\infty} K_{5/3}(x) \, dx
\end{equation}
The energy can be sampled from this by inverse transform method:
\begin{eqnarray}
N \, & = & \, F(\varepsilon) = \, \int f(\varepsilon) \, d\varepsilon \\
\varepsilon \, & = & \, F^{-1}(N)
\end{eqnarray}
The double integral is not analytically solved, and the sampling
is done from tabulated values of numerically computes $F(\varepsilon)$. 


Two methods have been implemented. If the flag {\tt ISYNC}
is set to 1, the photons are emitted at the end of the step along the current
direction. If {\tt ISYNC} is set to 3, the photons are
emitted randomly along the tangent the real trajectory of the particle.
In the case when {\tt ISYNC = 3}, the magnetic field tracking routines are 
called for each photon, and therefore this option is
considrably slower than {\tt ISYNC = 1}.

\begin{figure}
\begin{picture}(300,100)(-60,0)

\put(40,80){{\tt ISYNC = 1}}
\put(0,0){\line(2,1){80}}
\put(80,40){\vector(1,0){20}}
\multiput(80,40)(16,0){4}{\vector(1,0){12}}
\put(45,15){{\tt STEP}}
\put(40,5){Current step}
\put(85,57){{\tt VECT}}
\put(85,47){New direction of e}
\put(90,25){Photons}

\put(240,80){{\tt ISYNC = 3}}
\put(200,0){\line(2,1){80}}
\put(280,40){\vector(1,0){20}}
\put(205,5){\circle*{.25}}
\put(210,9){\circle*{.25}}
\put(215,13){\circle*{.25}}
\put(220,16){\circle*{.25}}
\put(225,19){\circle*{.25}}
\put(230,22){\circle*{.25}}
\put(235,25){\circle*{.25}}
\put(240,27){\circle*{.25}}
\put(245,29){\circle*{.25}}
\put(250,31.5){\circle*{.25}}
\put(255,33.5){\circle*{.25}}
\put(260,35.){\circle*{.25}}
\put(265,36.5){\circle*{.25}}
\put(270,38){\circle*{.25}}
\put(275,39){\circle*{.25}}

\put(202,4){\vector(1,1){12}}
\put(218,16){\vector(4,3){12}}
\put(242,29){\vector(2,1){12}}
\put(263,37){\vector(3,1){12}}
\put(245,15){{\tt STEP}}
\put(240,5){Current step}
\put(285,57){{\tt VECT}}
\put(285,47){New direction of e}
\put(190,30){Photons}
\end{picture}
\caption{The point where the synchrotron radiation photon
         is generated. The figure on the left describes
         the situation when {\tt ISYNC = 1}, and the one
         on the right when {\tt ISYNC = 3}. The little
         arrows are the photons and {\tt STEP} is the
         step taken by the e$^-$ or e$^+$. {\tt VECT}
         is the new direction computed in {\tt GTELEC}
         before entering in {\tt GSYNC}.}
\end{figure}
