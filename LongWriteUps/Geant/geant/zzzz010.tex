%%%%%%%%%%%%%%%%%%%%%%%%%%%%%%%%%%%%%%%%%%%%%%%%%%%%%%%%%%%%%%%%%%%
%                                                                 %
%  GEANT manual in LaTeX form                                     %
%                                                                 %
%  Version 1.00                                                   %
%                                                                 %
%  Last Mod.  9 June 1993 19:30  MG                               %
%                                                                 %
%%%%%%%%%%%%%%%%%%%%%%%%%%%%%%%%%%%%%%%%%%%%%%%%%%%%%%%%%%%%%%%%%%%
\Authors{S.Ravndal}      \Origin{GEANT}
\Version{Geant 3.16}\Routid{ZZZZ010}
\Submitted{01.10.84}  \Revised{08.11.93}

\Makehead{List of COMMON Blocks}
\section{Introduction}
 
Here, the
main features of the common blocks used in {\tt GEANT} are summarised,
with special mention of the variables initialised in \Rind{GINIT}
and of the possibility of overriding them through data records
{\tt [BASE040]} or interactive commands {\tt [XINT]}.
In most of the cases there is a correspondence between a
given data structure and a given common block where the current contents of
the banks are stored.
The labelled common blocks are accessible through Patchy/CMZ sequences
identified by the name of the {\tt COMMON}. They are defined in the Patch
\Rind {GCDES}.
 
{\bf Note:}
 
Unless otherwise specified, the long range variables are
initialised in \Rind{GINIT}. When non-zero, default values are
quoted between brackets. If the value may be modified
the keyword for the data record and for the interactive
command is also given in bold characters between brackets.
 
\subsection{Dynamic memory}
 
The {\tt GEANT} data structures are stored in the
common \FCind{/GCBANK/} accessible through the following Patchy sequence:

The \FCind{/GCLINK/} variables are pointers to the {\tt GEANT} data structures 
in the \FCind{/GCBANK/} common. They belong to a permanent area declared in 
\Rind{GZINIT}.
\FComm{GCBANK}{Dynamic core for the GEANT data structures}
\begin{verbatim}
      PARAMETER (KWBANK=69000,KWWORK=5200)
      COMMON/GCBANK/NZEBRA,GVERSN,ZVERSN,IXSTOR,IXDIV,IXCONS,FENDQ(16)
     +             ,LMAIN,LR1,WS(KWBANK)
      DIMENSION IQ(2),Q(2),LQ(8000),IWS(2)
      EQUIVALENCE (Q(1),IQ(1),LQ(9)),(LQ(1),LMAIN),(IWS(1),WS(1))
      EQUIVALENCE (JCG,JGSTAT)
      COMMON/GCLINK/JDIGI ,JDRAW ,JHEAD ,JHITS ,JKINE ,JMATE ,JPART
     +      ,JROTM ,JRUNG ,JSET  ,JSTAK ,JGSTAT,JTMED ,JTRACK,JVERTX
     +      ,JVOLUM,JXYZ  ,JGPAR ,JGPAR2,JSKLT
C
\end{verbatim}
\subsection{Other user accessed common blocks}
\FComm{GCCUTS}{Tracking thresholds}
\begin{verbatim}
      COMMON/GCCUTS/CUTGAM,CUTELE,CUTNEU,CUTHAD,CUTMUO,BCUTE,BCUTM
     +             ,DCUTE ,DCUTM ,PPCUTM,TOFMAX,GCUTS(5)
C
\end{verbatim}
This common contains the threshold for various processes and particles.
The energy values are the kinetic energy in GeV:
\begin{DLtt}{MMMMMMMM}
\item[CUTGAM]    threshold for gamma transport
({\tt 0.001, CUTS});
\item[CUTELE]    threshold for electron and positron transport
({\tt 0.001, CUTS});
\item[CUTNEU]    threshold for neutral hadron transport
({\tt 0.01, CUTS});
\item[CUTHAD]    threshold for charged hadron and ion transport
({\tt 0.01, CUTS});
\item[CUTMUO]    threshold for muon transport
({\tt 0.01, CUTS});
\item[BCUTE]     threshold for photons produced by electron
                 bremsstrahlung ({\tt CUTGAM, CUTS});
\item[BCUTM]    threshold for photons produced by muon bremsstrahlung
({\tt CUTGAM, CUTS});
\item[DCUTE]   threshold for electrons produced by electron $\delta$-rays
({\tt CUTELE, CUTS});
\item[DCUTM]  threshold for electrons produced by muon or hadron $\delta$-rays
({\tt CUTELE, CUTS});
\item[PPCUTM] threshold for \Pep\Pem direct pair production by
              muon ({\tt 0.002, CUTS});
\item[TOFMAX]  threshold on time of flight counted
from primary interaction time ({\tt $10^{10}$, CUTS});
\item[GCUTS]   free for user applications   ({\tt CUTS}).
\end{DLtt}
{\bf Note:}
The cuts {\tt BCUTE, BCUTM} and {\tt DCUTE, DCUTM} are given
the respective default values {\tt CUTGAM} and {\tt CUTELE}.
Experienced
users can make use of the facility offered (command {\tt CUTS})
to change {\tt BCUTE, DCUTE, BCUTM} and {\tt DCUTM}.
 
\FComm{GCDRAW}{Variables used by the drawing package}
\begin{verbatim}
      COMMON/GCDRAW/NUMNOD,MAXNOD,NUMND1,LEVVER,LEVHOR,MAXV,IPICK,
     + MLEVV,MLEVH,NWCUT,JNAM,JMOT,JXON,JBRO,JDUP,JSCA,JDVM,JPSM,
     + JNAM1,JMOT1,JXON1,JBRO1,JDUP1,JSCA1,JULEV,JVLEV,
     + LOOKTB(16),
     + GRMAT0(10),GTRAN0(3),IDRNUM,GSIN(41),GCOS(41),SINPSI,COSPSI,
     + GTHETA,GPHI,GPSI,GU0,GV0,GSCU,GSCV,NGVIEW,
     + ICUTFL,ICUT,CTHETA,CPHI,DCUT,NSURF,ISURF,
     + GZUA,GZVA,GZUB,GZVB,GZUC,GZVC,PLTRNX,PLTRNY,
     + LINATT,LINATP,ITXATT,ITHRZ,IPRJ,DPERS,ITR3D,IPKHIT,IOBJ,LINBUF,
     + MAXGU,MORGU,MAXGS,MORGS,MAXTU,MORTU,MAXTS,MORTS,
     + IGU,IGS,ITU,ITS,NKVIEW,IDVIEW,
     + NOPEN,IGMR,IPIONS,ITRKOP,IHIDEN,
     + ZZFU,ZZFV,MYISEL,
     + DDUMMY(15)
C
\end{verbatim}
\begin{DLtt}{MMMMMMMM}
\item[NUMNOD] number of nodes in non-optimized tree;
\item[MAXNOD] max. number of nodes of non-optimized tree
({\tt MIN(NLEFT, 16,000)});
\item[NUMND1] number of nodes in optimized tree;
\item[LEVVER] vertical level in the tree currently scanned by tree routines;
\item[LEVHOR] horizontal node in the tree currently scanned by
tree routines;
\item[MAXV] max vertical levels in the tree to be scanned by tree routines;
\item[IPICK] node selected by \Rind{GDTREE};
\item[MLEVV] number of vertical levels in the last tree scanned;
\item[MLEVH] number of horizontal nodes in the last tree scanned;
\item[NWCUT] max. workspace allocated by cut routines, ({\tt 5000});
\item[JNAM-JVLEV]  pointers used by the tree routines;
\item[LOOKTB] colour look-up table, ({\tt LOOKTB(I)=I,I=1,16});
\item[GRMAT0] rotation matrix saved by \Rind{GDRVOL}, ({\tt unitary matrix});
\item[GTRAN0] translation vector saved by \Rind{GDRVOL}, ({\tt 0.,0.,0.});
\item[IDRNUM] flag for \Rind{GDRAW}, set to 1 when called by \Rind{GDRVOL}
({\bf 0});
\item[GSIN] sine table (at $9^{\circ}$ steps);
\item[GCOS] cosine table (at $9^{\circ}$ steps);
\item[SINPSI] {\tt SIN(GPSI*DEGRAD)};
\item[COSPSI] {\tt COS(GPSI DEGRAD)};
\item[GTHETA] $\theta$ angle of the parallel projection of 3-dimensional
images on the screen (${\tt 45^{\circ}}$);
\item[GPHI]  $\phi$ angle of the parallel projection of 3-dimensional
images on the screen (${\tt 135^{\circ}}$);
\item[GPSI]  $\psi$ angle of rotation of the image on the screen
(${\tt 0^{\circ}}$);
\item[GU0]  U position (X in screen coordinates) of the origin of the drawing
in screen units ({\tt 10.});
\item[GV0]  V position (Y in screen coordinates) of the origin of the drawing
in screen units ({\tt 10.});
\item[GSCU]   scale factor for the U screen coordinate  ({\tt 0.015});
\item[GSCV]   scale factor for the V screen coordinate ({\tt 0.015});
\item[NGVIEW] flag informing \Rind{GDFR3D} and \Rind{GD3D3D} if
the view point has changed  ({\tt 0});
\item[ICUTFL] flag informing \Rind{GDRAW} if it was called by
{\it cut} drawing routines;
\item[ICUT] axis along which the cut is performed (1, 2 or 3, 0 if no cut);
\item[CTHETA] $\theta$ angle of cut supplied to \Rind{GDRAWX} (used by
\Rind{GDCUT});
\item[CPHI] $\phi$ angle of cut supplied to \Rind{GDRAWX} (used by
\Rind{GDCUT});
\item[DCUT] coordinate value (along axis {\tt ICUT)} at which the cut is
performed;
\item[NSURF] number of surfaces stored in {\tt SURF} to be cut;
\item[ISURF] pointer for array {\tt SURF};
\item[GZUA] zoom parameter (horizontal scale factor)  ({\tt 1.});
\item[GZVA] zoom parameter (vertical scale factor)  ({\tt 1.});
\item[GZUB] zoom parameter  ({\tt 0.});
\item[GZVB] zoom parameter  ({\tt 0.});
\item[GZUC] zoom parameter  ({\tt 0.});
\item[GZVC] zoom parameter  ({\tt 0.});
\item[PLTRNX] drawing X range in cm ({\tt 20.});
\item[PLTRNY] drawing Y range in cm ({\tt 20.});
\item[LINATT] current line attributes ({\tt colour=1, width=1, style=1,
fill=1});
\item[LINATP] permanent line attributes  ({\tt LINATT});
\item[ITXATT] current text attributes  ({\tt colour = 1, width = 1});
\item[ITHRZ] string containing the status of {\tt THRZ} option of
            \Rind{GDOPT}  ({\tt 'OFF '});
\item[IPRJ] string containing the status of {\tt PROJ} option of
            \Rind{GDOPT}  ({\tt 'PARA'});
\item[DPERS] distance of the view point from
the origin for perspective drawing ({\tt 1000.});
\item[ITR3D]track being scanned (used together with {\tt THRZ} option);
\item[IPKHIT]flag for \Rind{GPHITS}, if
$>0$ then print only hit number, ({\tt 0});
\item[IOBJ]type of the object being drawn (detector, track, hit, etc.)
({\tt 0});
\item[LINBUF]flag informing \Rind{GDRAWV} if line buffering is wanted or
not ({\tt 0});
\item[MAXGU]current number of words of graphic unit banks;
\item[MORGU]number of words to extend graphic unit banks;
\item[MAXGS]current number of words of graphic segment banks;
\item[MORGS]number of words to extend graphic segment banks;
\item[MAXTU]current number of words of text unit banks;
\item[MORTU]number of words to extend text unit banks;
\item[MAXTS]current number of words of text segment banks;
\item[MORTS]number of words to extend text segment banks;
\item[IGU]pointer to current graphic unit bank;
\item[IGS]pointer to current graphic segment bank;
\item[ITU]pointer to current text unit bank;
\item[ITS]pointer to current text segment bank;
\item[NKVIEW]number of view data banks ({\tt 0});
\item[IGVIEW]current view bank number or 0 if none active ({\tt 0});
\item[NOPEN]unused ({\tt 0});
\item[IGMR]flag informing if {\tt APOLLO-GMR} is being used ({\tt 0});
\item[IPIONS]unused ({\tt 0});
\item[ITRKOP]string containing the status of {\tt TRAK} option of
\Rind{GDOPT} ({\tt 'LINE'});
\item[ZZFU]
\item[ZZFV]
\item[MYISEL]
\item[DDUMMY]array of dummy words;
\end{DLtt}
\FComm{GCFLAG}{Flags and variables to control the run}
\begin{verbatim}
      COMMON/GCFLAG/IDEBUG,IDEMIN,IDEMAX,ITEST,IDRUN,IDEVT,IEORUN
     +        ,IEOTRI,IEVENT,ISWIT(10),IFINIT(20),NEVENT,NRNDM(2)
      COMMON/GCFLAX/BATCH, NOLOG
      LOGICAL BATCH, NOLOG
C
\end{verbatim}
\begin{DLtt}{MMMMMMMM}
\item[IDEBUG]flag set internally to 1 to activate debug
output if {\tt IDEMIN} $\leq$ {\tt IEVENT} $\leq$ {\tt IDEMAX}
and {\tt IEVENT} is a multiple of {\tt ITEST};
\item[IDEMIN]  first event to debug ({\tt DEBU});
\item[IDEMAX]  last event to debug ({\tt DEBU});
\item[ITEST] number of events between two activations
of the debug printing;
\item[IDRUN]current user run number   ({\tt 1, RUNG});
\item[IDEVT]current user event number  ({\tt 1, RUNG});
\item[IEORUN]flag to terminate run if non-zero;
\item[IEOTRI]flag to abort current event if non-zero;
\item[IEVENT]current event sequence number ({\tt 1});
\item[ISWIT]user flags, the first three are used by \Rind{GDEBUG}
to select the debug output ({\tt 0, SWIT});
\item[IFINIT]internal initialisation flags;
\item[NEVENT]number of events to be processed  ({\tt 10000000, TRIG});
\item[NRNDM]initial seeds for the random number generator. If
{\tt NRNDM(2)=0} the sequence number {\tt NRNDM(1)} is taken from a
predefined set of 215 independent sequences. Otherwise the random
number generator is initialised with the two seeds {\tt NRNDM(1), NRNDM(2)}
({\tt 9876, 54321});
\item[BATCH] true if the job is running in batch; set by the {\tt GXINT}
interactive program;
\item[NOLOG] true if no login kumac file is requested; set by the
{\tt GXINT} interactive program;
\end{DLtt}
\FComm{GCGOBJ}{CG package variables}
\begin{verbatim}
      PARAMETER (NTRCG=1)
      PARAMETER (NWB=207,NWREV=100,NWS=1500)
      PARAMETER (C2TOC1=7.7, C3TOC1=2.,TVLIM=1296.)
      COMMON /GCGOBJ/IST,IFCG,ILCG,NTCUR,NFILT,NTNEX,KCGST
     +             ,NCGVOL,IVFUN,IVCLOS,IFACST,NCLAS1,NCLAS2,NCLAS3
      COMMON /CGBLIM/IHOLE,CGXMIN,CGXMAX,CGYMIN,CGYMAX,CGZMIN,CGZMAX
C
C
\end{verbatim}
\begin{DLtt}{MMMMMMMM}
\item[NTRCG]
\item[NWB]
\item[NWREV]
\item[NWS]
\item[C2TOC1]
\item[C3TOC1]
\item[TVLIM]
\item[IST]
\item[IFCG]
\item[ILCG]
\item[NTCUR]
\item[NFILT]
\item[NTNEX]
\item[KCGST]
\item[NCGVOL]
\item[IVFUN]
\item[IVCLOS]
\item[IFACST]
\item[NCLAS1]
\item[NCLAS2]
\item[NCLAS3]
\item[IHOLE]
\item[CGXMIN]
\item[CGXMAX]
\item[CGYMIN]
\item[CGYMAX]
\item[CGZMIN]
\item[CGZMAX]
\end{DLtt}
\FComm{GCHILN}{Temporary link area for the CG package}
\begin{verbatim}
      COMMON/GCHILN/LARECG(2), JCGOBJ, JCGCOL, JCOUNT, JCLIPS,
     +              IMPOIN, IMCOUN, JSIX, JSIY, JSIZ,
     +              JPXC, JPYC, JPZC, ICLIP1, ICLIP2
*
\end{verbatim}
\begin{DLtt}{MMMMMMMM}
\item[LARECG]
\item[JCGOBJ]
\item[JCGCOL]
\item[JCOUNT]
\item[JCLIPS]
\item[IMPOIN]
\item[IMCOUN]
\item[JSIX]
\item[JSIY]
\item[JSIZ]
\item[JPXC]
\item[JPYC]
\item[JPZC]
\item[ICLIP1]
\item[ICLIP2]
\end{DLtt}
\FComm{GCJLOC}{JMATE substructure pointers for current material}

This common block contains the pointers to various {\tt ZEBRA} data
structures which refer to the current material during tracking.
\begin{verbatim}
      COMMON/GCJLOC/NJLOC(2),JTM,JMA,JLOSS,JPROB,JMIXT,JPHOT,JANNI
     +                  ,JCOMP,JBREM,JPAIR,JDRAY,JPFIS,JMUNU,JRAYL
     +                  ,JMULOF,JCOEF,JRANG
C
      COMMON/GCJLCK/NJLCK(2),JTCKOV,JABSCO,JEFFIC,JINDEX,JCURIN
     +                      ,JPOLAR,JTSTRA,JTSTCO,JTSTEN,JTASHO
C
      EQUIVALENCE (JLASTV,JTSTEN)
C
\end{verbatim}
\begin{DLtt}{MMMMMMMM}
\item[NJLOC]  {\tt ZEBRA} link area control variables;
\item[JTM] tracking medium;
\item[JMA] material;
\item[JLOSS] energy loss table;
\item[JPROB] bank containing some physic constants of 
the material;
\item[JMIXT] mixture parameters;
\item[JPHOT] photoelectric effect cross-section;
\item[JANNI] positron annihilation cross-section;
\item[JCOMP] Compton effect cross-section;
\item[JBREM] bremsstrahlung cross-section;
\item[JPAIR] photon pair production and muon direct pair
production cross-section;
\item[JDRAY] $\delta$-ray production cross-section;
\item[JPFIS] photo-fission cross-section;
\item[JMUNU] muon-nucleus interaction cross-section;
\item[JRAYL] Rayleigh effect cross-section;
\item[JMULOF] {\tt STMIN}, see {\tt [PHYS010]};
\item[JCOEF] bank containing the coefficients of the parabolic 
range-energy fit;
\item[JRANG] range;
\item[NJLCK] {\tt ZEBRA} link area control variables;
\item[JTCKOV] \v{C}erenkov photons energy binning;
\item[JABSCO] absorption coefficient;
\item[JEFFIC] quantum efficiency;
\item[JINDEX] refraction index;
\item[JCURIN] \v{C}erenkov angle integral;
\item[JPOLAR] polarisation information;
\item[JTSTRA] top level bank for PAI energy loss fluctuations model;
\item[JTSTCO] coefficients for PAI energy loss fluctuations model;
\item[JTSTEN] energy binning for PAI energy loss fluctuations model;
\item[JTASHO] coefficients for ASHO energy loss fluctuations model;
\end{DLtt}
 
For more information see {\tt [CONS199]}.
\FComm{GCJUMP}{Pointers for the jump package}
Variable {\tt JU$\dots$} contains the address of the routine {\tt GU$\dots$}.
\begin{verbatim}
      PARAMETER    (MAXJMP=30)
      COMMON/GCJUMP/JUDCAY, JUDIGI, JUDTIM, JUFLD , JUHADR, JUIGET,
     +              JUINME, JUINTI, JUKINE, JUNEAR, JUOUT , JUPHAD,
     +              JUSKIP, JUSTEP, JUSWIM, JUTRAK, JUTREV, JUVIEW,
     +              JUPARA
      DIMENSION     JMPADR(MAXJMP)
      EQUIVALENCE  (JMPADR(1), JUDCAY)
*
\end{verbatim}
\FComm{GCKINE}{Kinematics of current track}
 
\begin{verbatim}
      COMMON/GCKINE/IKINE,PKINE(10),ITRA,ISTAK,IVERT,IPART,ITRTYP
     +      ,NAPART(5),AMASS,CHARGE,TLIFE,VERT(3),PVERT(4),IPAOLD
C
\end{verbatim}
\begin{DLtt}{MMMMMMMM}
\item[IKINE]  user integer word  ({\tt 0, KINE});
\item[PKINE]  user array of real   ({\tt 0, KINE});
\item[ITRA]   track number;
\item[ISTAK] stack track number;
\item[IVERT] vertex number;
\item[IPART] particle number;
\item[ITRTYP] particle tracking type;
\item[NAPART] name of current particle (ASCII codes stored in an integer
array, 4 characthers per word);
\item[AMASS]  mass of current particle in GeV c$^{-2}$;
\item[CHARGE] charge of current particle in electron charge unit;
\item[TLIFE] average life time of current particle in seconds;
\item[VERT] coordinates of origin vertex for current track;
\item[PVERT] track kinematics at origin vertex ({\tt PVERT(4)} not used);
\item[IPAOLD] particle number of the previous track.
\end{DLtt}
\FComm{GCKMAX}{Size of the \FCind{/GCKING/} stack}
\begin{verbatim}
      INTEGER MXGKIN
      PARAMETER (MXGKIN=100)
\end{verbatim}
\FComm{GCMUTR}{Auxiliary variables for the CG package}
\begin{verbatim}
*
      PARAMETER (MULTRA=50)
      CHARACTER*4 GNASH, GNNVV, GNVNV
      COMMON/GCMUTR/NCVOLS,KSHIFT,NSHIFT,ICUBE,NAIN,JJJ,
     +              NIET,IOLDSU,IVOOLD,IWPOIN,IHPOIN,IVECVO(100),
     +              PORGX,PORGY,PORGZ,POX(15),POY(15),POZ(15),GBOOM,
     +              PORMIR(18),PORMAR(18),IPORNT,
     +              ICGP,CLIPMI(6),CLIPMA(6),
     +              ABCD(4),BMIN(6),BMAX(6),CGB(16000),CGB1(16000),
     +              GXMIN(MULTRA),GXMAX(MULTRA),GYMIN(MULTRA),
     +              GYMAX(MULTRA),GZMIN(MULTRA),GZMAX(MULTRA),
     +              GXXXX(MULTRA),GYYYY(MULTRA),GZZZZ(MULTRA)
*
      COMMON/GCMUTC/   GNASH(MULTRA),GNNVV(MULTRA),GNVNV(MULTRA)
*
\end{verbatim}
\begin{DLtt}{MMMMMMMM}
\item[NCVOLS]
\item[KSHIFT]
\item[NSHIFT]
\item[ICUBE]
\item[NAIN]
\item[JJJ]
\item[NIET]
\item[IOLDSU]
\item[IVOOLD]
\item[IWPOIN]
\item[IHPOIN]
\item[IVECVO]
\item[PORGX]
\item[PORGY]
\item[PORGZ]
\item[POX]
\item[POY]
\item[POZ]
\item[GBOOM]
\item[PORMIR]
\item[PORMAR]
\item[IPORNT]
\item[ICGP]
\item[CLIPMI]
\item[CLIPMA]
\item[ABCD]
\item[BMIN]
\item[BMAX]
\item[CGB]
\item[CGB1]
\item[GXMIN]
\item[GXMAX]
\item[GYMIN]
\item[GYMAX]
\item[GZMIN]
\item[GZMAX]
\item[GXXXX]
\item[GYYYY]
\item[GZZZZ]
\item[GNASH]
\item[GNNVV]
\item[GNVNV]
\end{DLtt}
\FComm{GCKING}{Kinematics of generated secondaries}
\begin{verbatim}
+SEQ, GCKMAX
      COMMON/GCKING/KCASE,NGKINE,GKIN(5,MXGKIN),
     +                           TOFD(MXGKIN),IFLGK(MXGKIN)
      INTEGER       KCASE,NGKINE ,IFLGK,MXPHOT,NGPHOT
      REAL          GKIN,TOFD,XPHOT
C
      PARAMETER (MXPHOT=800)
      COMMON/GCKIN2/NGPHOT,XPHOT(11,MXPHOT)
C
      COMMON/GCKIN3/GPOS(3,MXGKIN)
      REAL          GPOS
C
\end{verbatim}
\begin{DLtt}{MMMMMMMMMMMMMM}
\item[KCASE] Mechanism which has generated the secondary particles;
\item[NGKINE]Number of generated secondaries;
\item[GKIN(1,I)]x component of momentum of I$^{th}$ particle;
\item[GKIN(2,I)]y component of momentum;
\item[GKIN(3,I)]z component of momentum;
\item[GKIN(4,I)]total energy;
\item[GKIN(5,I)]particle code (see {\tt [CONS300]});
\item[TOFD(I)]time offset with respect to current time of flight;
\item[IFLGK(I)]Flag controlling the handling of track by 
\Rind{GSKING}, \Rind{GSSTAK};
\begin{DLtt}{MMMM}
\item[$<$0]particle is discarded;
\item[~0]({\bf D}) particle is stored in the temporary stack {\tt JSTAK}
for further tracking;
\item[~1] like {\tt 0} but
particle is stored in {\tt JVERTX/JKINE} structure as well;
\item[$>$1] particle is attached to vertex {\tt IFLGK(I)}.
\end{DLtt}
\item[GPOS(1,I)] x position of I$^{th}$ particle;
\item[GPOS(2,I)] y position;
\item[GPOS(3,I)] z position;
\item[NGPHOT] number of \v{C}erenkov photons generated in the current
step;
\item[XPHOT(1,I)] x position of the I$^{th}$ photon;
\item[XPHOT(2,I)] y position;
\item[XPHOT(3,I)] z position;
\item[XPHOT(4,I)] x component of momentum;
\item[XPHOT(5,I)] y component of momentum;
\item[XPHOT(6,I)] z component of momentum;
\item[XPHOT(7,I)] momentum of the photon;
\item[XPHOT(8,I)] x component of the polarisation vector;
\item[XPHOT(9,I)] y component of the polarisation vector;
\item[XPHOT(10,I)] z component of the polarisation vector;
\item[XPHOT(11,I)] time of flight in seconds of the photon.
\end{DLtt}
\FComm{GCLINK}{See \FCind{/GCBANK/} above}
\FComm{GCLIST}{Various system and user lists}
\begin{verbatim}
      COMMON/GCLIST/NHSTA,NGET ,NSAVE,NSETS,NPRIN,NGEOM,NVIEW,NPLOT
     +       ,NSTAT,LHSTA(20),LGET (20),LSAVE(20),LSETS(20),LPRIN(20)
     +             ,LGEOM(20),LVIEW(20),LPLOT(20),LSTAT(20)
C
\end{verbatim}
\begin{DLtt}{MMMMMMMMMMMMMMMMMM}
\item[NHSTA] number of histograms on data record {\tt HSTA};
\item[NGET] number of data structures on data record {\tt GET};
\item[NSAVE]number of data structures on data record {\tt SAVE};
\item[NSETS]number of items on data record {\tt SETS};
\item[NPRIN]number of items on data record {\tt PRIN};
\item[NGEOM]number of items on data record {\tt GEOM};
\item[NVIEW]number of items on data record {\tt VIEW};
\item[NPLOT]number of items on data record {\tt PLOT};
\item[NSTAT]number of items on data record {\tt STAT} (obsolete);
\item[LHSTA \ldots LSTAT] lists of items set via the input records
({\tt HSTA \ldots,STAT}).
\end{DLtt}
{\tt LSTAT(1)} is reserved by the system for volume statistics.
\FComm{GCMATE}{Parameters of current material}
\begin{verbatim}
      COMMON/GCMATE/NMAT,NAMATE(5),A,Z,DENS,RADL,ABSL
C
\end{verbatim}
\begin{DLtt}{MMMMMMMM}
\item[NMAT]  current material number;
\item[NAMATE]name of current material (ASCII codes stored in an integer
array, 4 characthers per word);
\item[A]atomic weight of current material;
\item[Z]atomic number of current material;
\item[DENS]density of current material in g cm$^{-3}$;
\item[RADL]radiation length of current material;
\item[ABSL]absorption length of current material.
\end{DLtt}
\FComm{GCMULO}{Energy binning and multiple scattering}
 
Precomputed quantities for multiple scattering and energy binning for
{\tt JMATE} banks. See also {\tt [CONS199]} for the energy binning and
{\tt [PHYS325]} for a description of the variables {\tt OMCMOL} and
{\tt CHCMOL}.
\begin{verbatim}
      COMMON/GCMULO/SINMUL(101),COSMUL(101),SQRMUL(101),OMCMOL,CHCMOL
     +  ,EKMIN,EKMAX,NEKBIN,NEK1,EKINV,GEKA,GEKB,EKBIN(200),ELOW(200)
C
\end{verbatim}
\begin{DLtt}{MMMMMMMM}
\item[SINMUL]  not used any more;
\item[COSMUL]  not used any more;
\item[SQRMUL]  not used any more;
\item[OMCMOL]  constant $\Omega_0$ of the Moli\'ere theory;
\item[CHCMOL]  $\chi_{cc}$ constant of the Moli\'ere theory;
\item[EKMIN]   lower edge of the energy range of the tabulated cross
sections ({\tt $10^{-5}$, ERAN});
\item[EKMAX]   upper edge of the energy range of the tabulated cross
sections ({\tt $10^{4}$, ERAN});
\item[NEKBIN]    number of energy bins to be used ({\tt 90, ERAN});
\item[NEK1]    {\tt NEKBIN+1};
\item[EKINV]   $1/ \left ( \log_{10}(\mbox{\tt EKMAX})-
\log_{10}(\mbox{\tt EKMIN}) \right )$;
\item[GEKA]    {\tt NEKBIN*EKINV};
\item[GEKB]    {\tt 1-GEKA*EKBIN(1)};
\item[EKBIN]   $\log \left ( \mbox{\tt ELOW} \right ) $;
\item[ELOW]    low edges of the energy bins.
\end{DLtt}
\FComm{GCMZFO}{I/O descriptors of GEANT banks}
\begin{verbatim}
      COMMON/GCMZFO/IOMATE,IOPART,IOTMED,IOSEJD,IOSJDD,IOSJDH,IOSTAK
     +             ,IOMZFO(13)
C
\end{verbatim}
\begin{DLtt}{MMMMMMMM}
\item[IOMATE] I/O descriptor for the {\tt JMATE} bank;
\item[IOPART] I/O descriptor for the {\tt JPART} bank;
\item[IOTMED] I/O descriptor for the {\tt JTMED} bank;
\item[IOSEJD] I/O descriptor for the detector banks;
\item[IOSJDD] I/O descriptor for the second dependent bank of the
detector banks;
\item[IOSJDH] I/O descriptor for the first dependent bank of the
detector banks;
\item[IOSTAK] I/O descriptor for the {\tt JSTAK} bank;
\item[IOMZFO] free I/O descriptors.
\end{DLtt}
\FComm{GCNUM}{Current number for various items}
\begin{verbatim}
      COMMON/GCNUM/NMATE ,NVOLUM,NROTM,NTMED,NTMULT,NTRACK,NPART
     +            ,NSTMAX,NVERTX,NHEAD,NBIT
      COMMON /GCNUMX/ NALIVE,NTMSTO
C
\end{verbatim}
\begin{DLtt}{MMMMMMMM}
\item[NMATE]   number of material banks;
\item[NVOLUM]  number of volume banks;
\item[NROTM]   number of rotation matrix banks;
\item[NTMED]   number of tracking media banks;
\item[NTMULT]  total number of tracks processed in current event
(including secondaries);
\item[NTRACK]  number of tracks in the {\tt JKINE} bank for current event;
\item[NPART]   maximum particle code;
\item[NSTMAX]  maximum number of tracks ({\it high-water mark})
in stack {\tt JSTAK} for current event;
\item[NVERTX]  number of vertices in {\tt JVERTX} bank for current event;
\item[NHEAD]   number of data words in the {\tt JHEAD} bank ({\tt 10});
\item[NBIT]    number of bits per word (initialised in \Rind{GINIT}
via {\tt ZEBRA});
\item[NALIVE]number of particles to be tracked in the parallel tracking stack
(this mode of tracking is disabled in the current {\tt GEANT} version);
\item[NTMSTO]total number of tracks tracked in the current event so far. Same
as {\tt NTMULT} in \FCind{/GCTRAK/};
\end{DLtt}
\FComm{GCOMIS}{Variables for the {\tt COMIS} package}
Variable {\tt JU\ldots} contains the {\tt COMIS} address of routine
{\tt GU\ldots}.
\begin{verbatim}
      COMMON/GCOMIS/ICOMIS,JUINIT,JUGEOM,JUKINE,JUSTEP,JUOUT,JULAST
*
\end{verbatim}
\begin{DLtt}{MMMMMMMM}
\item[ICOMIS] flag to avoid a double initialisation of {\tt COMIS};
\end{DLtt}

\FComm{GCONST}{Basic constants}
See next section for the value of these parameters.
\begin{verbatim}
      COMMON/GCONST/PI,TWOPI,PIBY2,DEGRAD,RADDEG,CLIGHT,BIG,EMASS
      COMMON/GCONSX/EMMU,PMASS,AVO
C
\end{verbatim}
\begin{DLtt}{MMMMMMMM}
\item[PI]     $\pi$;
\item[TWOPI]  $2\pi$;
\item[PIBY2]  $\pi/2$;
\item[DEGRAD] degrees to radiants conversion factor ($\pi/180$);
\item[RADDEG] radiants to degrees conversion factor ($180/\pi$);
\item[CLIGHT] light velocity in cm s$^{-1}$;
\item[BIG]    arbitrary large number;
\item[EMASS]  electron mass in GeV c$^{-2}$;
\item[EMMU]   muon mass in GeV c$^{-2}$;
\item[PMASS]  proton mass in GeV c$^{-2}$;
\item[AVO]    Avogadro's number $\times 10^{-24}$.
\end{DLtt}
 
\FComm{GCONSP}{Basic constants}
These parameters are in {\tt SINGLE PRECISION} on 64 bits machines.
\begin{verbatim}
      DOUBLE PRECISION PI,TWOPI,PIBY2,DEGRAD,RADDEG,CLIGHT,BIG,EMASS
      DOUBLE PRECISION EMMU,PMASS,AVO
*
      PARAMETER (PI=3.14159265358979324D0)
      PARAMETER (TWOPI=6.28318530717958648D0)
      PARAMETER (PIBY2=1.57079632679489662D0)
      PARAMETER (DEGRAD=0.0174532925199432958D0)
      PARAMETER (RADDEG=57.2957795130823209D0)
      PARAMETER (CLIGHT=29979245800.D0)
      PARAMETER (BIG=10000000000.D0)
      PARAMETER (EMASS=0.0005109990615D0)
      PARAMETER (EMMU=0.105658387D0)
      PARAMETER (PMASS=0.9382723128D0)
      PARAMETER (AVO=0.60221367D0)
*
\end{verbatim}
\FComm{GCOPTI}{Control of geometry optimisation}
\begin{verbatim}
      COMMON/GCOPTI/ IOPTIM
C
\end{verbatim}
\begin{DLtt}{MMMMMMMM}
\item[IOPTIM]Optimisation flag
\begin{DLtt}{MMM}
\item[-1] no optimisation at all; \Rind{GSORD} calls disabled;
\item[~0] no optimisation; only user calls to \Rind{GSORD} kept;
\item[~1] all non-\Rind{GSORD}ered volumes are ordered along the best axis;
\item[~2] all volumes are ordered along the best axis.
\end{DLtt}
\end{DLtt}
\FComm{GCPARA}{Control of parametrized energy deposition}
\begin{verbatim}
      INTEGER    BITPHI, BITTET, BITPOT
      LOGICAL    SYMPHI, SYMTEU, SYMTED
      PARAMETER (LSTACK = 5000)
C     BITPOT is for Phi.Or.Tet
C
C ---------------------------------------------------------
      COMMON    /GCPARA/
     +                   EPSIX0 (LSTACK)       ,
     +                   IDRPHI (LSTACK     )  , IDRTET (LSTACK     ),
     +                   IDROUT (LSTACK     )  , JPLOST (LSTACK     ),
     +                   IPHTMP (LSTACK     )  ,
     +                   BITPHI (LSTACK     )  , BITTET (LSTACK     ),
     +                   BITPOT (LSTACK     )  , JJLOST, JJFILL,
     +                                           JENTRY, JEMPTY,
     +                                           EPSMAX,
     +                   JJTEMP, JJWORK        , JJSTK1,
     +                   J1TEMP,                 J1STK1,
     +                   IFOUNP, IFOUNT        , IFNPOT,
     +                                           SYMPHI,
     +                   SYMTEU, SYMTED
C
\end{verbatim}
\begin{DLtt}{MMMMMMMM}
\item[LSTACK] dimension of the energy ray stack;
\item[JJLOST] number of energy rays lost in each tracking step;
\item[EPSMAX] maximum number of radiation lengths that an energy ray 
can travel;
\item[JJTEMP] temporary pointer;
\item[JJWORK] actual size of the energy ray stack;
\item[JJSTK1]
\item[J1TEMP]
\item[J1STK1]
\item[IFOUNP] Number of energy rays that change cell in $\phi$
direction;
\item[IFOUNT] Number of energy rays that change cell in $\theta$
direction;
\item[IFNPOT] Number of energy rays that change cell either in $\phi$
or in $\theta$;
\item[SYMPHI] {\tt .TRUE.} if {\tt PHIMAX-PHIMIN = }$360^{\circ}$;
\item[SYMTEU] {\tt .TRUE.} if {\tt TETMIN = }$0^{\circ}$;
\item[SYMTED] {\tt .TRUE.} if {\tt TETMAX = }$180^{\circ}$.
\end{DLtt}
\FComm{GCPARM}{Control of parameterisation}
\begin{verbatim}
      COMMON/GCPARM/IPARAM,PCUTGA,PCUTEL,PCUTNE,PCUTHA,PCUTMU
     +             ,NSPARA,MPSTAK,NPGENE
      REAL PACUTS(5)
      EQUIVALENCE (PACUTS(1),PCUTGA)
C
\end{verbatim}
\begin{DLtt}{MMMMMMMM}
\item[IPARAM]Parameterisation flag ({\tt 0, PCUT});
\begin{DLtt}{MMMMMMMM}
\item[0 =]parameterisation is not in effect, normal tracking will be used;
\item[1 =]parameterisation is in effect;
\end{DLtt}
\item[PCUTGA]parameterisation threshold for photons ({\tt 0.,  PCUT})
\item[PCUTEL]parameterisation threshold for electrons and positrons
({\tt 0.,  PCUT});
\item[PCUTNE]parameterisation threshold for neutral hadrons
({\tt 0., PCUT});
\item[PCUTHA]parameterisation threshold for charged hadrons
({\tt 0., PCUT});
\item[PCUTMU]parameterisation threshold for muons
({\tt 0.,  PCUT});
\item[NSPARA] not used;
\item[MPSTAK] optimum size of the Energy ray stack ({\tt 2000});
\item[NPGENE] number of Energy rays generated per primary particle
({\tt 20});
\end{DLtt}
\FComm{GCPHYS}{Control of physics processes}
\begin{verbatim}
      COMMON/GCPHYS/IPAIR,SPAIR,SLPAIR,ZINTPA,STEPPA
     +             ,ICOMP,SCOMP,SLCOMP,ZINTCO,STEPCO
     +             ,IPHOT,SPHOT,SLPHOT,ZINTPH,STEPPH
     +             ,IPFIS,SPFIS,SLPFIS,ZINTPF,STEPPF
     +             ,IDRAY,SDRAY,SLDRAY,ZINTDR,STEPDR
     +             ,IANNI,SANNI,SLANNI,ZINTAN,STEPAN
     +             ,IBREM,SBREM,SLBREM,ZINTBR,STEPBR
     +             ,IHADR,SHADR,SLHADR,ZINTHA,STEPHA
     +             ,IMUNU,SMUNU,SLMUNU,ZINTMU,STEPMU
     +             ,IDCAY,SDCAY,SLIFE ,SUMLIF,DPHYS1
     +             ,ILOSS,SLOSS,SOLOSS,STLOSS,DPHYS2
     +             ,IMULS,SMULS,SOMULS,STMULS,DPHYS3
     +             ,IRAYL,SRAYL,SLRAYL,ZINTRA,STEPRA
      COMMON/GCPHLT/ILABS,SLABS,SLLABS,ZINTLA,STEPLA
     +             ,ISYNC
     +             ,ISTRA
*
\end{verbatim}
\begin{DLtt}{MMMMMMMM}
\item[IPAIR] control variable for the \Pem/\Pep pair production process;
\item[SPAIR] distance to the next pair production in the current material;
\item[SLPAIR] total distance travelled by the $\gamma$ when pair production occurs;
\item[ZINTPA] number of interaction lengths to the next pair production;
\item[STEPPA] interaction length for pair production for the current material
and energy;
\item[ICOMP] control variable for the Compton scattering process;
\item[SCOMP] distance to the next Compton scattering in the current material;
\item[SLCOMP] total distance travelled by the $\gamma$ when Compton scattering occurs;
\item[ZINTCO] number of interaction lengths to the next Compton scattering;
\item[STEPCO] interaction length for Compton scattering for the current material
and energy;
\item[IPHOT] control variable for the photoelectric effect process;
\item[SPHOT] distance to the next photoelectric effect in the current material;
\item[SLPHOT] total distance travelled by the $\gamma$ when photoelectric effect occurs;
\item[ZINTPH] number of interaction lengths to the next photoelectric effect;
\item[STEPPH] interaction length for photoelectric effect for the current material
and energy;
\item[IPFIS] control variable for the $\gamma$-induced nuclear fission process;
\item[SPFIS] distance to the next $\gamma$-induced nuclear fission in the current 
material;
\item[SLPFIS] total distance travelled by the $\gamma$ when $\gamma$-induced nuclear 
fission occurs;
\item[ZINTPF] number of interaction lengths to the next $\gamma$-induced nuclear 
fission;
\item[STEPPF] interaction length for $\gamma$-induced nuclear fission for the 
current material and energy;
\item[IDRAY] control variable for the $\delta$-ray production process;
\item[SDRAY] distance to the next $\delta$-ray production in the current material;
\item[SLDRAY] total distance travelled by the particle when $\delta$-ray production 
occurs;
\item[ZINTDR] number of interaction lengths to the next $\delta$-ray production;
\item[STEPDR] interaction length for $\delta$-ray production for the current 
material and energy;
\item[IANNI] control variable for the positron annichilation process;
\item[SANNI] distance to the next positron annichilation in the current material;
\item[SLANNI] total distance travelled by the positron when positron annichilation 
occurs;
\item[ZINTAN] number of interaction lengths to the next positron annichilation;
\item[STEPAN] interaction length for positron annichilation for the current 
material and energy;
\item[IBREM] control variable for the bremsstrahlung process;
\item[SBREM] distance to the next bremsstrahlung in the current material;
\item[SLBREM] total distance travelled by the particle when bremsstrahlung occurs;
\item[ZINTBR] number of interaction lengths to the next bremsstrahlung;
\item[STEPBR] interaction length for bremsstrahlung for the current material
and energy;
\item[IHADR] control variable for the hadronic interaction process;
\item[SHADR] distance to the next hadronic interaction in the current material;
\item[SLHADR] total distance travelled by the particle when hadronic interaction occurs;
\item[ZINTHA] number of interaction lengths to the next hadronic interaction;
\item[STEPHA] interaction length for hadronic interaction for the current material
and energy;
\item[IMUNU] control variable for the $\mu$ nuclear interaction process;
\item[SMUNU] distance to the next $\mu$ nuclear interaction in the current 
material;
\item[SLMUNU] total distance travelled by the $\mu$ when $\mu$ nuclear interaction 
occurs;
\item[ZINTMU] number of interaction lengths to the next $\mu$ nuclear interaction;
\item[STEPMU] interaction length for $\mu$ nuclear interaction for the current 
material and energy;
\item[IDCAY] control variable for the decay in flight process;
\item[SDCAY] distance to the next decay in flight in the current material;
\item[SLIFE] total distance travelled by the particle when decay in flight occurs;
\item[SUMLIF] time to the next interaction point in $ct$ units;
\item[DPHYS1] not used;
\item[ILOSS] control variable for the energy loss process;
\item[SLOSS] step limitation due to continuous processes: energy loss,
bending in magnetic field, \v{C}erenkov photon generation and multiple 
scattering;
\item[SOLOSS] not used;
\item[STLOSS] not used; set equal to {\tt STEP} for backward compatibility;
\item[DPHYS2] not used;
\item[IMULS] control variable for the energy loss process;
\item[SMULS] maximum step allowed by the multiple scattering simulation;
\item[SOMULS] not used;
\item[STMULS] not used; set equal to step for backward compatibility;
\item[DPHYS3] not used.
\item[ILABS] control variable for the \v{C}erenkov photon absorption process;
\item[SLABS] distance to the next \v{C}erenkov photon absorption process;
\item[SLLABS] not used;
\item[ZINTLA] number of interaction lengths to the next \v{C}erenkov 
photon absorption process;
\item[STEPLA] interaction length for \v{C}erenkov 
photon absorption process;
\item[ISYNC] control variable for synchrotron radiation production;
\item[ISTRA] control variable for energy loss fluctuation simulation;
\end{DLtt}
For more details on {\tt IDRAY} and {\tt ILOSS} see {\tt [BASE040]}.
For all other variables see {\tt [PHYS010]}.
\FComm{GCPOLY}{Internal flags for polygon and polycone shapes}
\begin{verbatim}
      COMMON/GCPOLY/IZSEC,IPSEC
C
\end{verbatim}
\begin{DLtt}{MMMMMMMM}
\item[IZSEC]    Z  section number;
\item[IPSEC]    $\phi$ sector number.
\end{DLtt}
\FComm{GCPUSH}{Initial and incremental size of some mother banks}
\begin{verbatim}
      COMMON/GCPUSH/NCVERT,NCKINE,NCJXYZ,NPVERT,NPKINE,NPJXYZ
C
\end{verbatim}
\begin{DLtt}{MMMMMMMM}
\item[NCVERT] initial size of bank {\tt JVERTX } ({\tt 5});
\item[NCKINE] initial size of bank {\tt JKINE}  ({\tt 50});
\item[NCJXYZ] initial size of bank {\tt JXYZ}  ({\tt 50});
\item[NPVERT] increment for size of bank {\tt JVERTX}  ({\tt 5});
\item[NPKINE] increment for size of bank {\tt JKINE}  ({\tt 10});
\item[NPJXYZ] increment for size of bank {\tt JXYZ}  ({\tt 10}).
\end{DLtt}
\FComm{GCRZ}{Direct access files control variables}
\begin{verbatim}
      COMMON/GCRZ1/NRECRZ,NRGET,NRSAVE,LRGET(20),LRSAVE(20)
      INTEGER      NRECRZ,NRGET,NRSAVE,LRGET    ,LRSAVE
      COMMON/GCRZ2/RZTAGS
      CHARACTER*8 RZTAGS(4)
C
\end{verbatim}
\begin{DLtt}{MMMMMMMMMMMMM}
\item[NRECRZ] record size (argument of {\tt RZMAKE});
\item[NRGET]  number of data structures declared on data record {\tt RGET};
\item[NRSAVE] number of data structures declared on data record {\tt RSAV};
\item[LRGET,LRSAVE] corresponding user lists of items;
\item[RZTAGS]key names (argument of {\tt RZMAKE}).
\end{DLtt}
\FComm{GCSCAL}{Scan geometry {\tt ZEBRA} pointers}
\begin{verbatim}
      PARAMETER(MXSLNK=100)
      COMMON/GCSCAL/ ISLINK(MXSLNK)
      EQUIVALENCE (LSLAST,ISLINK(MXSLNK))
      EQUIVALENCE (LSCAN ,ISLINK(1)),(LSTEMP,ISLINK(2))
      EQUIVALENCE (LSPARA,ISLINK(3)),(LSERAY,ISLINK(4))
*
\end{verbatim}
\begin{DLtt}{MMMMMMMM}
\item[LSCAN]
\item[LSTEMP]
\item[LSPARA]
\item[LSERAY]
\item[LSLAST]
\end{DLtt}
\FComm{GCSCAN}{Scan geometry control parameters}
\begin{verbatim}
      PARAMETER (MSLIST=32,MAXMDT=3)
      COMMON/GCSCAN/SCANFL,NPHI,PHIMIN,PHIMAX,NTETA,TETMIN,TETMAX,
     +              MODTET,IPHIMI,IPHIMA,IPHI1,IPHIL,NSLMAX,
     +              NSLIST,ISLIST(MSLIST),VSCAN(3),FACTX0,FACTL,
     +              FACTR,IPHI,ITETA,ISCUR,SX0,SABS,TETMID(MAXMDT),
     +              TETMAD(MAXMDT)
     +             ,SX0S,SX0T,SABSS,SABST,FACTSF
     +             ,DLTPHI,DLTETA,DPHIM1,DTETM1
     +             ,FCX0M1,FCLLM1,FCRRM1
      LOGICAL SCANFL
      COMMON/GCSCAC/SFIN,SFOUT
      CHARACTER*80 SFIN,SFOUT
*
\end{verbatim}
\begin{DLtt}{MMMMMMMM}
\item[MSLIST] dimension of {\tt ISLIST} array ({\tt 32});
\item[MAXMDT] number of $\theta$ division types ({\tt 3});
\item[SCANFL] SCAN flag ({\tt .FALSE., SCAN, STURN});
\begin{DLtt}{MMMMMMMM}
\item[.TRUE.]creation of {\tt SCAN} geometry, geantinos will be tracked;
\item[.FALSE.]normal tracking;
\end{DLtt}
\item[NPHI] number of $\phi$ divisions ({\tt 90, SCAN}, {\tt PHI});
\item[PHIMIN] minimum $\phi$ in degrees (${\tt 0^{\circ}}$,
{\tt SCAN}, {\tt PHI});
\item[PHIMAX] maximum $\phi$ in degrees (${\tt 360^{\circ}}$,
{\tt SCAN}, {\tt PHI});
\item[NTETA] number of $\theta$ divisions ({\tt 90}, {\tt SCAN},
{\tt TETA});
\item[TETMIN] minimum value of $\theta$
({\tt 0}$^{\circ}$, {\tt SCAN}, {\tt TETA});
\item[TETMAX] maximum value of $\theta$ (180, {\tt SCAN}, {\tt $\theta$});
\item[MODTET] type of $\theta$ division (1, {\tt SCAN}, {\tt $\theta$});
\begin{DLtt}{MMM}
\item[1] $\theta$ is expressed in terms of degrees;
\item[2] $\theta$ is expressed in terms of pseudorapidity;
\item[3] $\theta$ is expressed in terms of $\cos(\theta)$;
\end{DLtt}
\item[IPHIMI] not used;
\item[IPHIMA] not used;
\item[IPHI1] internal index ({\tt PHIMIN});
\item[IPHIL] internal index ({\tt PHIMAX});
\item[NSLMAX] not used;
\item[NSLIST] number of volumes to be scanned ({\tt 1}, {\tt SCAL});
\item[ISLIST] list of volumes to be scanned ({\tt SCAL}, {\tt SLIST});
\item[VSCAN] scan vertex origin ({\tt SCAP}, {\tt VERTEX});
\item[FACTX0] scale factor for {\tt SX0} ({\tt 100.}, {\tt SCAP},
{\tt SFACTORS});
\item[FACTL] scale factor for {\tt SABS} ({\tt 10.}, {\tt SCAP},
{\tt SFACTORS});
\item[FACTR] scale factor for {\tt R} ({\tt 100.},
{\tt SCAP}, {\tt SFACTORS});
\item[IPHI]  $\phi$ bin of the current cell;
\item[ITETA] $\theta$ bin of the current cell;
\item[ISCUR] pointer in {\tt LPHI} to first triplet of words for a
given {\tt ITETA} cell;
\item[SX0] sum of radiation lengths up to current {\tt R} boundary;
\item[SABS] sum of absorption lengths up to current {\tt R} boundary;
\item[TETMID] bound value for {\tt TETMIN} ({\tt 0., -10., -1.} if
{\tt MODTET} is 1, 2 or 3 respectively);
\item[TETMAD] bound value for {\tt TETMAX} ({\tt 180., 10., 1.} if
{\tt MODTET} is 1, 2 or 3 respectively);
\item[SX0S] sum of radiation lengths for the sensitive mediums in the
current cell;
\item[SX0T] sum of radiation lengths in the current cell;
\item[SABSS] sum of absorption lengths for the sensitive mediums in
the current cell;
\item[SABST] sum of absorption lengths in the current cell;
\item[FACTSF] scale factor for the sampling fractions ({\tt 1000.});
\item[DLTPHI] bin in $\phi$, {\tt (PHIMAX-PHIMIN)/NPHI};
\item[DLTETA] bin in $\theta$, {\tt (TETMAX-TETMIN)/NTETA};
\item[DPHIM1] {\tt DLTPHI}$^{-1}$;
\item[DTETM1] {\tt DLTETA}$^{-1}$;
\item[FCX0M1] {\tt FACTX0}$^{-1}$;
\item[FCLLM1] {\tt FACTL}$^{-1}$;
\item[FCRRM1] {\tt FACTR}$^{-1}$;
\item[SFIN] not used;
\item[SFOUT] not used.
\end{DLtt}
\FComm{GSECTI}{Hadronic partial cross-sections for {\tt GHEISHA}}
\begin{verbatim}
      COMMON/GSECTI/ AIEL(20),AIIN(20),AIFI(20),AICA(20),ALAM,K0FLAG
C
\end{verbatim}
\begin{DLtt}{MMMMMMMM}
\item[AIEL]elastic cross-sections. {\tt AIEL(I)} is the elastic cross-section
for the {\tt I}$^{th}$ element composing the current material;
\item[AIIN]   inelastic cross-sections;
\item[AIFI]   fission cross-sections;
\item[AICA]   nuclear capture cross-sections;
\item[ALAM]   total cross-section;
\item[K0FLAG] obsolete.
\end{DLtt}
\FComm{GCSETS}{Identification of current sensitive detector}
\begin{verbatim}
      COMMON/GCSETS/IHSET,IHDET,ISET,IDET,IDTYPE,NVNAME,NUMBV(20)
C
\end{verbatim}
\begin{DLtt}{MMMMMMMM}
\item[IHSET]  set identifier, ASCII equivalent of 4 characters;
\item[IHDET]  detector identifier, ASCII equivalent of 4 characters;
\item[ISET]   position of set in bank {\tt JSET};
\item[IDET]   position of detector in bank {\tt JS=LQ(JSET-ISET)};
\item[IDTYPE] user defined detector type;
\item[NVNAME] number of elements in {\tt NUMBV};
\item[NUMBV]  list of volume copy numbers to identify the detector.
\end{DLtt}
\FComm{GCSHNO}{Symbolic codes for system shapes}
\begin{verbatim}
      PARAMETER ( NSBOX=1,  NSTRD1=2, NSTRD2=3, NSTRAP=4, NSTUBE=5,
     +  NSTUBS=6, NSCONE=7, NSCONS=8, NSSPHE=9, NSPARA=10,NSPGON=11,
     +  NSPCON=12,NSELTU=13,NSHYPE=14,NSGTRA=28, NSCTUB=29 )
\end{verbatim}
\FComm{GCSPEE}{Auxiliary variables for the CG package}
\begin{verbatim}
      COMMON/GCSPEE/S1,S2,S3,SS1,SS2,SS3,LEP,IPORLI,ISUBLI,
     +              SRAGMX,SRAGMN,RAINT1,RAINT2,RMIN1,RMIN2,
     +              RMAX1,RMAX2,JPORJJ,ITSTCU,IOLDCU,ISCOP,
     +              NTIM,NTFLAG,LPASS,JSC
*
\end{verbatim}
\begin{DLtt}{MMMMMMMM}
\item[S1]
\item[S2]
\item[S3]
\item[SS1]
\item[SS2]
\item[SS3]
\item[LEP]
\item[IPORLI]
\item[ISUBLI]
\item[SRAGMX]
\item[SRAGMN]
\item[RAINT1]
\item[RAINT2]
\item[RMIN1]
\item[RMIN2]
\item[RMAX1]
\item[RMAX2]
\item[JPORJJ]
\item[ITSTCU]
\item[IOLDCU]
\item[ISCOP]
\item[NTIM]
\item[NTFLAG]
\item[LPASS]
\item[JSC]
\end{DLtt}
\FComm{GCSTAK}{Control variables for parallel tracking}
\begin{verbatim}
      PARAMETER (NWSTAK=12,NWINT=11,NWREAL=12,NWTRAC=NWINT+NWREAL+5)
      COMMON /GCSTAK/ NJTMAX, NJTMIN, NTSTKP, NTSTKS, NDBOOK, NDPUSH,
     +                NJFREE, NJGARB, NJINVO, LINSAV(15), LMXSAV(15)
C
\end{verbatim}
\begin{DLtt}{MMMMMMMM}
\item[NWSTAK]
\item[NWINT]
\item[NWREAL]
\item[NWTRAC]
\item[NJTMAX]
\item[NJTMIN]
\item[NTSTKP]
\item[NTSTKS]
\item[NDBOOK]
\item[NDPUSH]
\item[NJFREE]
\item[NJGARB]
\item[NJINVO]
\item[LINSAV]
\item[LMXSAV]
\end{DLtt}
\FComm{GCTIME}{Execution time control}
\begin{verbatim}
      COMMON/GCTIME/TIMINT,TIMEND,ITIME,IGDATE,IGTIME
C
\end{verbatim}
\begin{DLtt}{MMMMMMMM}
\item[TIMINT] time reqeusted for the run phase, 
after initialisation  ({\tt TIME}, not used);
\item[TIMEND] time requested
for program termination phase ({\tt 1, TIME});
\item[ITIME] number of events between two tests of time left
({\tt 1, TIME});
\item[IGDATE]current date in integer format {\tt YYMMDD};
\item[IGTIME] current time in integer format {\tt HHMM};
\end{DLtt}
\FComm{GCTMED}{Array of current tracking medium parameters}
\begin{verbatim}
      COMMON/GCTMED/NUMED,NATMED(5),ISVOL,IFIELD,FIELDM,TMAXFD,STEMAX
     +      ,DEEMAX,EPSIL,STMIN,CFIELD,PREC,IUPD,ISTPAR,NUMOLD
      COMMON/GCTLIT/THRIND,PMIN,DP,DNDL,JMIN,ITCKOV,IMCKOV,NPCKOV
C
\end{verbatim}
\begin{DLtt}{MMMMMMMM}
\item[NUMED]  current tracking medium number;
\item[NATMED] name of current tracking medium (ASCII codes stored in an integer
array, 4 characthers per word);
\item[ISVOL]
\begin{DLtt}{MMM}
\item[-1] non-sensitive volume with sensitive volume tracking parameters;
\item[~0] non-sensitive volume;
\item[~1] sensitive volume;
\end{DLtt}
\item[IFIELD]
\begin{DLtt}{MMM}
\item[0] no field;
\item[1] user defined field (\Rind{GUFLD});
\item[2] user defined field (\Rind{GUFLD}) along z;
\item[3] uniform field ({\tt FIELDM}) along z;
\end{DLtt}
\item[FIELDM] maximum field;
\item[TMAXFD] maximum turning angle in one step due to the magnetic
field;
\item[STEMAX] maximum step allowed;
\item[DEEMAX] maximum fraction of energy loss in one step due to 
continuous processes;
\item[EPSIL] boundary crossing accuracy;
\item[STMIN] minimum step size limitation due to: energy loss, multiple 
scattering, magnetic field bending or, if active, \v{C}erenkov photons 
production;
\item[CFIELD]constant for field step evaluation;
\item[PREC]effective step for boundary crossing ($0.1 \times$ {\tt EPSIL});
\item[IUPD]
\begin{DLtt}{MMM}
\item[0] new particle or new medium in current step;
\item[1] no change of medium or particle;
\end{DLtt}
\item[ISTPAR]
\begin{DLtt}{MMM}
\item[0] global tracking parameters are used;
\item[1] special tracking parameters are used for this medium;
\end{DLtt}
\item[NUMOLD] number of the previous tracking medium;
\item[THRIND] $\beta^{-1}$ of the current particle;
\item[PMIN] minimum momentum in GeV c$^{-1}$ for the photon transport;
\item[DP] momentum window to generate the photons;
\item[DNDL] number of photons generated per centimeter;
\item[JMIN] pointer to the photon threshold energy bin;
\item[ITCKOV] flag for the \v{C}erenkov photon generation:
\begin{DLtt}{MMM}
\item[0] disactivated;
\item[1] activated;
\end{DLtt}
\item[IMCKOV] flag for the \v{C}erenkov photon generation in current
material, same meaning than above;
\item[NPCKOV] number of energy bins for the \v{C}erenkov photons;
\end{DLtt}
\FComm{GCTRAK}{Track parameters at the end of the current step}
\begin{verbatim}
      PARAMETER (MAXMEC=30)
      COMMON/GCTRAK/VECT(7),GETOT,GEKIN,VOUT(7),NMEC,LMEC(MAXMEC)
     + ,NAMEC(MAXMEC),NSTEP ,MAXNST,DESTEP,DESTEL,SAFETY,SLENG
     + ,STEP  ,SNEXT ,SFIELD,TOFG  ,GEKRAT,UPWGHT,IGNEXT,INWVOL
     + ,ISTOP ,IGAUTO,IEKBIN, ILOSL, IMULL,INGOTO,NLDOWN,NLEVIN
     + ,NLVSAV,ISTORY
      PARAMETER (MAXME1=30)
      COMMON/GCTPOL/POLAR(3), NAMEC1(MAXME1)
C
\end{verbatim}
\begin{DLtt}{MMMMMMMM}
\item[VECT] track parameters ($x,y,z,p_x/p,p_y/p,p_z/p,p$);
\item[GETOT]particle total energy;
\item[GEKIN]particle kinetic energy;
\item[VOUT]track parameters at the end of the step, used internally by
{\tt GEANT};
\item[NMEC]number of mechanisms active for current step;
\item[LMEC]list of mechanism numbers for current step;
\item[NAMEC]list of mechanism names for current step
(ASCII codes stored in an integer, 4 characthers per word);
\item[NSTEP]number of steps for current track;
\item[MAXNST]maximum number of steps allowed (10000);
\item[DESTEP]total energy lost in current step;
\item[DESTEL]same as {\tt DESTEP}, kept for backward compatibility;
\item[SAFETY]underestimated distance to closest medium boundary;
\item[SLENG]current track length;
\item[STEP] size of current tracking step;
\item[SNEXT]distance to current medium boundary along the direction of
the particle;
\item[SFIELD]obsolete;
\item[TOFG]current time of flight in seconds;
\item[GEKRAT]interpolation coefficient in the energy table {\tt ELOW};
\item[UPWGHT]user word for current particle;
\item[IGNEXT] indicates whether the particles is reaching a medium
boundary in the current step:
\begin{DLtt}{MMM}
\item[0]{\tt SNEXT} has not been computed in current step;
\item[1]{\tt SNEXT} has been computed in current step: particle is
reaching a boundary;
\end{DLtt}
\item[INWVOL]
\begin{DLtt}{MMM}
\item[0]track is inside a volume;
\item[1]track has entered a new volume or it is a new track;
\item[2]track is exiting current volume;
\item[3]track is exiting the setup;
\end{DLtt}
\item[ISTOP]
\begin{DLtt}{MMM}
\item[0]particle will continue to be tracked;
\item[1]particle has disappeared (decay, inelastic interaction \dots);
\item[2]particle has fallen below the cutoff energy or has interacted but
no secondaries have been generated;
\end{DLtt}
\item[IGAUTO]
\begin{DLtt}{MMM}
\item[0]tracking parameters are given by the user;
\item[1]tracking parameters are calculated by {\tt GEANT};
\end{DLtt}
\item[IEKBIN]current kinetic energy bin in table {\tt ELOW};
\item[ILOSL]local energy loss flag (see \FCind{/GCPHYS/});
\item[IMULL]local multiple scattering flag (see \FCind{/GCPHYS/});
\item[INGOTO] daughter number, in the current mother,
which the particle will enter if continuing along
a straight line for {\tt SNEXT} centimeters;
\item[NLDOWN]lowest level reached down the tree (parallel tracking only);
\item[NLEVIN]number of levels currently filled and valid in \FCind{/GCVOLU/};
\item[NLVSAV]current level (parallel tracking only);
\item[ISTORY]User flag for current track history (reset to $0$ in
\Rind{GLTRAC});
\item[POLAR]polarisation vector for current \v{C}erenkov photon;
\item[NAMEC1]additional list of mechanism names for current step
(ASCII codes stored in an integer, 4 characthers per word);
\end{DLtt}
List of mechanisms active in the current step.
\begin{verbatim}
      CHARACTER*4 MEC(MAXMEC),MEC1(MAXME1),DFLT(2)
      PARAMETER (LEFTM1=MAXME1-9)
      DATA MEC/'NEXT','MULS','LOSS','FIEL','DCAY','PAIR','COMP','PHOT'
     +        ,'BREM','DRAY','ANNI','HADR','ECOH','EVAP','FISS','ABSO'
     +        ,'ANNH','CAPT','EINC','INHE','MUNU','TOFM','PFIS','SCUT'
     +        ,'RAYL','PARA','PRED','LOOP','NULL','STOP'/
      DATA MEC1/'LABS','LREF','SMAX','SCOR','CKOV','REFL','REFR',
     +          'SYNC','STRA',LEFTM1*'    '/
\end{verbatim}
\begin{DLtt}{MMMMMMMMM}
\item[NEXT ~~1] particle has reached the boundary of current volume;
\item[MULS ~~2] multiple scattering;
\item[LOSS ~~3] continuous energy loss;
\item[FIEL ~~4] bending in magnetic field;
\item[DCAY ~~5] particle decay;
\item[PAIR ~~6] photon pair-production or muon direct pair production;
\item[COMP ~~7] Compton scattering;
\item[PHOT ~~8] photoelectric effect;
\item[BREM ~~9] bremsstrahlung;
\item[DRAY ~10] $\delta$-ray production;
\item[ANNI ~11] positron annihilation;
\item[HADR ~12] hadronic interaction;
\item[ECOH ~13] hadronic elastic coherent scattering;
\item[EVAP ~14] nuclear evaporation;
\item[FISS ~15] nuclear fission;
\item[ABSO ~16] nuclear absorption;
\item[ANNH ~17] anti-proton annihilation;
\item[CAPT ~18] neutron capture;
\item[EINC ~19] hadronic elastic incoherent scattering;
\item[INHE ~20] hadronic inelastic scattering;
\item[MUNU ~21] muon-nuclear interaction;
\item[TOFM ~22] exceeded time of flight cut;
\item[PFIS ~23] nuclear photo-fission;
\item[SCUT ~24] the particle due to bending in magnetic field was 
unexpectedly crossing volume boundaries and the step has been halved to
avoid this;
\item[RAYL ~25] Rayleigh effect;
\item[PARA ~26] parametrisation activated;
\item[PRED ~27] error matrix computed ({\tt GEANE} tracking);
\item[LOOP ~28] not used;
\item[NULL ~29] no mechanism is active, usually at the entrance of
a new volume;
\item[STOP ~30] particle has fallen below energy threshold and tracking
stops;
\item[LABS 101] \v{C}erenkov photon absorption;
\item[LREF 102] \v{C}erenkov photon reflection/refraction;
\item[SMAX 103] step limited by {\tt STEMAX};
\item[SCOR 104] correction against loss of precision in boundary crossing;
\item[CKOV 105] \v{C}erenkov photon generation;
\item[REFL 106] \v{C}erenkov photon reflection;
\item[REFR 107] \v{C}erenkov photon refraction;
\item[SYNC 108] synchrotron radiation generation;
\item[STRA 109] PAI or ASHO model used for energy loss fluctuations.
\end{DLtt}
\FComm{GCUNIT}{Description of logical units }
\begin{verbatim}
      COMMON/GCUNIT/LIN,LOUT,NUNITS,LUNITS(5)
      INTEGER LIN,LOUT,NUNITS,LUNITS
      COMMON/GCMAIL/CHMAIL
      CHARACTER*132 CHMAIL
C
\end{verbatim}
\begin{DLtt}{MMMMMMMM}
\item[LIN]input unit to read data records;
\item[LOUT]output unit;
\item[NUNITS]number of additional units;
\item[LUNITS]list of additional units;
\item[CHMAIL]character string containing the message to be printed by
             \Rind{GMAIL}.
\end{DLtt}
 
{\tt LIN} and {\tt LOUT} are defined in \Rind{GINIT} through {\tt ZEBRA}.
{\tt NUNITS} and {\tt LUNITS} are reserved
for user {\tt ZEBRA} files.
\FComm{GCVOLU}{Current geometrical information}
\begin{verbatim}
      COMMON/GCVOLU/NLEVEL,NAMES(15),NUMBER(15),
     +LVOLUM(15),LINDEX(15),INFROM,NLEVMX,NLDEV(15),LINMX(15),
     +GTRAN(3,15),GRMAT(10,15),GONLY(15),GLX(3)
C
\end{verbatim}
\begin{DLtt}{MMMMMMMM}
\item[NLEVEL] level in the geometrical tree reached by the last 
successful search;
\item[NAMES]volume names at each level in the current tree
(ASCII codes stored in an integer, 4 characters per word);
\item[NUMBER]volume copy or division numbers at each level in the tree;
\item[LVOLUM]volume numbers in the {\tt JVOLU} bank at each level in the tree;
\item[LINDEX]number of the daughter where the current track is at each 
level in the tree;
\item[INFROM] daughter of the current volume from which the particle
exited;
\item[NLEVMX] maximum number of levels in the geometry tree;
\item[NLDEV] number of the volumes at each level whose structure has
been {\it developed};
\item[LINMX] number of positioned contents or cells from division at
each level;
\item[GTRAN]x,y,z offsets of the cumulative coordinate
transformation from the master system to the system at each level;
\item[GRMAT]rotation matrix elements for the cumulative
transformation from the master system to the system at each level;
${\tt GRMAT(10,LEVEL)}=0$ indicates the null rotation;
\item[GONLY] flag indicating if the volume is {\tt ONLY} (1) or
{\tt MANY} (0) at each level in the tree;
\item[GLX]current point in local coordinates system (local use only!).
\end{DLtt}
\FComm{GCVOL2}{Back-up for \FCind{/GCVOLU/}}
The variables have the same meaning of the variables in common \FCind{/GCVOLU/}
with similar names. 
\begin{verbatim}
      COMMON/GCVOL2/NLEVE2,NAMES2(15),NUMB2(15),
     +LVOL2(15),LIND2(15),INFRO2,NLDEV2(15),LINMX2(15),
     +GTRAN2(3,15),GRMAT2(10,15),GONLY2(15),GLX2(15)
      INTEGER NLEVE2,NAMES2,NUMB2,LVOL2,LIND2,INFRO2,NLDEV2,LINMX2
C
\end{verbatim}
\FComm{GCXLUN}{Logical units number for the interactive version}
\begin{verbatim}
      COMMON/GCXLUN/LUNIT(128)
*
\end{verbatim}
\begin{DLtt}{MMMMMMMM}
\item[LUNIT]Logical units numbers.
\end{DLtt}

\FComm{GCCURS}{Cursor position information for interactive graphics}
\begin{verbatim}
      COMMON/GCCURS/INTFLA,SIZD2,FACHV,HALF,SAVPLX,SAVPLY,YPLT,XPLT
*
\end{verbatim}
\begin{DLtt}{MMMMMMMM}
\item[INTFLA]
\item[SIZD2]
\item[FACHV]
\item[HALF]
\item[SAVPLX]
\item[SAVPLY]
\item[YPLT]
\item[XPLT]
\end{DLtt}

\FComm{GCURSB}{}
\begin{verbatim}
      COMMON/GCURSB/NUMNDS,IADDI,NUMND2,NNPAR,IISELT
      COMMON/GCURSC/MOMO
      CHARACTER*4 MOMO
*
\end{verbatim}
\begin{DLtt}{MMMMMMMM}
\item[NUMNDS]
\item[IADDI]
\item[NUMND2]
\item[NNPAR]
\item[IISELT]
\item[MOMO]
\end{DLtt}

\FComm{GCSTRA}{Variables for the PAI energy loss model}
\begin{verbatim}
      PARAMETER (ILTAB=200)
      COMMON /GCSTR2 / EMAX,EM(200),SFINT,EPSR(ILTAB),EPSI(ILTAB),
     +  FINT(ILTAB),EMIN,EPPS,BETA2,GAMMA2,WP2,S2,MEEV,EMM(200),
     +  GAMLOG(21),NP,NTAB,IE,NFACT,NICOLL
*
\end{verbatim}
\begin{DLtt}{MMMMMMMM}
\item[EMAX]
\item[EM]
\item[SFINT]
\item[EPSR]
\item[EPSI]
\item[FINT]
\item[EMIN]
\item[EPPS]
\item[BETA2]
\item[GAMMA2]
\item[WP2]
\item[S2]
\item[MEEV]
\item[EMM]
\item[GAMLOG]
\item[NP]
\item[NTAB]
\item[IE]
\item[NFACT]
\item[NICOLL]
\end{DLtt}

\FComm{GCASHO}{Variables for the ASHO energy loss model}
\begin{verbatim}
      COMMON/GCASHO/ZMED,AMED,DMED,E0MED,ZSMED(50),ESMED(50),ALFA,
     *             STEP,PLIN,PLOG,BE2,PLASM,TRNSMA,
     *             BOSC(50),AOSC(50),EOSC(50),ZOSC(50),EMEAN,
     *             CMGO(2000),EMGO,EMGOMI,
     *             NSMED,IOSC(50),NOSC,NMGO,NMGOMA
C
\end{verbatim}
\begin{DLtt}{MMMMMMMM}
\item[ZMED]
\item[AMED]
\item[DMED]
\item[E0MED]
\item[ZSMED]
\item[ESMED]
\item[ALFA]
\item[STEP]
\item[PLIN]
\item[PLOG]
\item[BE2]
\item[PLASM]
\item[TRNSMA]
\item[BOSC]
\item[AOSC]
\item[EOSC]
\item[ZOSC]
\item[EMEAN]
\item[CMGO]
\item[EMGO]
\item[EMGOMI]
\item[NSMED]
\item[IOSC]
\item[NOSC]
\item[NMGO]
\item[NMGOMA]
\end{DLtt}

\FComm{GCHIL2}{Temporary {\tt ZEBRA} link area for the drawing of the 
geometrical tree}
\begin{verbatim}
      COMMON/GCHIL2/LARETT(2),JTICK,JMYLL,JFIMOT,JFISCA,JFINAM,
     +              JAASS1,JAASS2,
     +              JAASS3,JAASS4,JTICKS,JMYLLS,JMYMOT
*
\end{verbatim}
\begin{DLtt}{MMMMMMMM}
\item[LARETT] {\tt ZEBRA} control variables for the link area;
\item[JTICK]
\item[JMYLL]
\item[JFIMOT]
\item[JFISCA]
\item[JFINAM]
\item[JAASS1]
\item[JAASS2]
\item[JAASS3]
\item[JAASS4]
\item[JTICKS]
\item[JMYLLS]
\item[JMYMOT]
\end{DLtt}

\FComm{GCVOL1}{Push-pop stack of the volume tree for \v{C}erenkov tracking}
These variables are used to save and restore the variables with the
similar name in the \FCind{/GCVOLU/} common block.
\begin{verbatim}
      COMMON/GCVOL1/NLEVL1,NAMES1(15),NUMBR1(15),LVOLU1(15)
C
\end{verbatim}
For more information on the meaning of these variables see the
{\tt JETSET} documentation~\cite{bib-JETS}.

\FComm{GCLUND}{Control variables for the interface with {\tt JETSET}}
\begin{verbatim}
      COMMON/GCLUND/IFLUND,ECLUND
C
\end{verbatim}
\begin{DLtt}{MMMMMMMM}
\item[IFLUND] flavour of the quarks to be generated, first input variable
to \Rind{LUEEVT};
\item[ECLUND] energy in GeV of the \Pem\Pep collision, second input 
variable to \Rind{LUEEVT}.
\end{DLtt}

\FComm{GCPMXZ}{Number of elements with photoelectric cross-section}
Number of elements for which the Sandia parametrisation is used for
the photoelectric cross-sections.
\begin{verbatim}
      PARAMETER (MAXELZ=100)
C
\end{verbatim}

\FComm{GC10EV}{Lower limit for Sandia parametrisation}
\begin{verbatim}
      PARAMETER (G10EV=1.0E-8)
      PARAMETER (TENEV=1.E-2)
C
\end{verbatim}
\begin{DLtt}{MMMMMMMM}
\item[G10EV] lower limit in GeV;
\item[TENEV] lower limit in keV;
\end{DLtt}

\FComm{GCSHPT}{Shell potentials}
The meaning of the variables is explained in the comments.
\begin{verbatim}
C  Shells are numbered from 1 to 24.
C  Shells used:
C               K,L1,L2,L3,M1,M2,M3,M4,M5
C               N1,N2,N3,N4,N5,N6,N7,
C               O1,O2,O3,O4,O5,P1,P2,P3
C   VARIABLES:
C     NSHLST - value of Z for which the shells starts to be present
C     N1ST   - pointer to K shell of a given Z (in ESHELL array)
C     NSHLLS - Number of used shells for a given Z
C     ESHELL - Shells potentials in eV !!!
      INTEGER LENGTH,MAXSHL
      PARAMETER (LENGTH=  1409)
      PARAMETER (MAXSHL=24)
      INTEGER NSHLST,N1ST,NSHLLS
      REAL ESHELL
      DIMENSION NSHLST(MAXSHL),N1ST(MAXELZ),NSHLLS(MAXELZ)
      DIMENSION ESHELL(LENGTH)
      COMMON /GCSHPT/NSHLST,N1ST,NSHLLS,ESHELL
C
\end{verbatim}

\FComm{GCPHPR}{Probability of radiative decay mode}
\begin{verbatim}
C  Probability of radiative decay mode.
      COMMON /GCPHPR/ GFLUPR(4,MAXELZ)
C
\end{verbatim}

\FComm{GCPHNR}{Nonradiative decay mode for photoelectric effect}
\begin{verbatim}
C  INRFIN - nonradiative decay mode
      COMMON /GCPHNR/ IGNRFN(8,MAXELZ)
C
\end{verbatim}

\FComm{GCPHRD}{Radiative rates for photoelectric effect}
\begin{verbatim}
C  GRATE - radiative modes' rates
      PARAMETER (KSHLS=6)
      PARAMETER (L1SHLS=8)
      PARAMETER (L2SHLS=7)
      PARAMETER (L3SHLS=8)
      PARAMETER (ISHLS=29)
      COMMON / GCPHRD / GPHRAT(ISHLS,MAXELZ),ISHLUS(24,4),ISHLTR(ISHLS)
C
\end{verbatim}

\FComm{GCPHXS}{Sandia parametrisation coefficients}
\begin{verbatim}
+KEEP,GCPHXS.
      PARAMETER (MAXPOW=4)
      PARAMETER (MAXINT=13)
      CHARACTER*6 CRNGUP
      COMMON /GCPXRN/ CRNGUP(MAXINT,MAXELZ)
      COMMON /GCPXCF/ COFS(MAXPOW,MAXINT,MAXELZ),GPOMIN(MAXELZ)
C
\end{verbatim}
\begin{DLtt}{MMMMMMMM}
\item[MAXPOW] maximum power of the variable $E^{-1}$ in the parametrisation;
\item[MAXINT] maximum number of parametrisation intervals;
\item[MAXELZ] maximum number of elements included in the parametrisation;
\item[CRNGUP] limits of the energy intervals for the parametrisation;
\item[COFS] coefficients of the parametrisation;
\item[GPOMIN] minimum value of the parametrisation;
\end{DLtt}
