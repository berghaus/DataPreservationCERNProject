%%%%%%%%%%%%%%%%%%%%%%%%%%%%%%%%%%%%%%%%%%%%%%%%%%%%%%%%%%%%%%%%%%%
%                                                                 %
%  GEANT manual in LaTeX form                                     %
%                                                                 %
%  Version 1.00                                                   %
%                                                                 %
%  Last Mod. 12 June 1993 1800  MG                                %
%                                                                 %
%%%%%%%%%%%%%%%%%%%%%%%%%%%%%%%%%%%%%%%%%%%%%%%%%%%%%%%%%%%%%%%%%%%
\Origin{L.Urb\'{a}n}
\Version{Geant 3.10}\Routid{PHYS451}
\Submitted{26.10.84}   \Revised{16.12.93}
\Makehead{Simulation of  e+e- pair  production by  muons}
\section{Subroutines}
\Shubr{GPAIRM}{}
\Rind{GPAIRM} generates the \Pep\Pem-pair radiated by a high energetic muon.
It uses the following input and output:
\begin{DLtt}{MMMMMMM}
\item[input:]  via common block \FCind{/GCTRAK/}
\item[output:] via common block \FCind{/GCKING/}
\end{DLtt}
\Rind{GPAIRM} is called automatically from the tracking routine
\Rind{GTMUON} if, and when,
the parent muon reaches its radiation point during the tracking.
\section{Method }
The double differential cross-section for the
process can be written \cite{bib-LOHM}:
 \begin{eqnarray}
\frac {d^2 \sigma}{d\nu d \rho}&=& \alpha^4\frac
     {2} {3 \pi} (Z \lambda)^2 \cdot  \frac {1-\nu}{\nu}\left [\phi_{\rm e} +
     (m/M)^2
     \phi_{\mu} \right ]
\end{eqnarray}
All the quantities in the expression above are defined in {\tt [PHYS450]}.
 By computing this cross-section for different
($\nu,\rho$) points, it can be seen that:
\begin{enumerate}
\item
the {\bf shape} of the functions
$\frac{\Tstm d^2 \sigma}{\Tstm d\nu d\rho}$
and
$\frac{\Tstm d \sigma}{\Tstm d\nu}\int d \rho \frac {\Tstm d^2 \sigma}
{\Tstm d\nu d\rho}$
practically does not depend on $Z$
\item
the dominant contribution comes from {\bf the low $\nu$ region:}
\begin{equation}
 \nu_{\rm min} = (4m/E)  \leq \nu  \leq  100*\nu_{\rm min}
\end{equation}
\item
in this low region ($d^2\sigma/d\nu d\rho$)
{\bf is flat} as a function of $\rho$
\end{enumerate}
Therefore, we propose the following sampling method as a rough approximation:
\begin{enumerate}
\item
In the low $\nu$ region the differential cross-section
\[
\frac{d \sigma}{d\nu}=\int d \rho \frac{d^2 \sigma K} {d\nu d\sigma}
\]
can be approximated as:
\begin{equation}
\frac{d \sigma}{d\nu} \approx  \left[1-\frac{\nu_{\rm min}}{\nu}\right]^{1/2}
      \frac{1}{v^a} \mbox{\qquad with:\quad}
a = 2-\frac{\ln E}{10} \mbox{\qquad($E$ in GeV)}
\end{equation}
We can write:
\begin{equation}
 \frac{d \sigma}{d\nu}\approx f(\nu) g(\nu)
\end{equation}
where,
\begin{equation}
f(\nu) = \frac{(a-1)}{\frac{\Tstm 1}{\Tstm \nu_c^{a- 1}}  -
        \left(\frac{\Tstm 1}{\Tstm \nu_{\rm max}}\right)^{a-1}}
        \frac{1}{\nu^a}
\end{equation}
is the normalised distribution in the interval $[\nu_c,\nu_{\rm max}]$ and
\begin{equation}
g(\nu) = \left[1-\frac{\nu_{\rm min}}{\nu}\right]^{1/2}
\end{equation}
is the rejection function.
\item
$r_1$ and $r_2$ being two uniformly distributed random numbers in the
interval $[0,1]$:
\begin{eqnarray}
& - & \mbox{Sample $\nu$ from the distribution $f(\nu)$ as:}\nonumber      \\
&   & \nu = \left(\frac{1-r_1}{\nu_c^{a-1}} \frac{r_1}{\nu_{\rm max}^{a-1}}
      \right)^\frac{\Sstm 1}{\Sstm 1-a}                                    \\
& - & \mbox{Accept $\nu$ if $r_2 \leq g(\nu)$}
\end{eqnarray}
\item Then compute
\begin{equation}
\rho_{\rm max}(\nu) = \left[1 -\frac{6M^2}
      {E^2(1-\nu)}\right]\left[1-\frac{4m}{\nu E}\right]^{1/2} \\
\end{equation}
and generate $\rho $ uniformly in the range $[-\rho_{\rm max},
+\rho_{\rm max}]$.
\end{enumerate}
After the successful sampling of ($ \nu,\rho $), \Rind{GPAIRM} generates the
polar
angles of the radiated \Pep\Pem-pair with respect to
an axis defined along the parent
muon's momentum. $\Theta$ is assigned the approximate average value:
\begin{equation}
 \Theta =\frac{M}{E}
\end{equation}
$\phi^+$ is generated isotropically and $\phi^- = \phi^+ + \pi$
