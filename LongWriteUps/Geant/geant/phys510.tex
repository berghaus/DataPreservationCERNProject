%%%%%%%%%%%%%%%%%%%%%%%%%%%%%%%%%%%%%%%%%%%%%%%%%%%%%%%%%%%%%%%%%%%
%                                                                 %
%  GEANT manual in LaTeX form                              %
%                                                                 %
%  Michel Goossens (for translation into LaTeX)                   %
%  Version 1.00                                                   %
%  Last Mod. Jan 24 1991  1300   MG + IB                          %
%                                                                 %
%%%%%%%%%%%%%%%%%%%%%%%%%%%%%%%%%%%%%%%%%%%%%%%%%%%%%%%%%%%%%%%%%%%
\Origin{H.C. Fesefeldt, F.Carminati}
\Documentation{F. Carminati}
\Version{Geant 3.16}\Routid{PHYS510}
\Submitted{24.02.86}       \Revised{11.10.93}
\Makehead{The GEANT/GHEISHA Interface}
\section{Introduction }
 
There are two packages to describe the hadronic showers in matter
which are available to the users of {\tt GEANT}: {\tt GHEISHA}~\cite{bib-GHEI}
and {\tt FLUKA} (see {\tt [PHYS520]}).
 
The {\tt GHEISHA} code generates hadronic interactions with the nuclei of the
current tracking medium, evaluating cross-sections and sampling the final 
state multiplicity and kinematics, while the {\tt GEANT} philosophy is 
preserved for the tracking.
The {\tt GHEISHA} code is stored in the {\tt GEANH} car file.

The {\tt GHEISHA} printing flags can be set via the {\tt GEANT} {\tt SWIT} 
data record:
each switch greater than 100 but smaller that 111 sets the
corresponding printing flag of {\tt GHEISHA} modulo 100, so that {\tt
SWIT 105} will set the printing flag 5 of {\tt GHEISHA}. Setting the
printing flags of {\tt GHEISHA} the following information will be displayed:
\begin{DLtt}{MMMMMMMM}
\item[NPRT(1)]    one header for each track in the shower;
\item[NPRT(2)]    all tracking information;
\item[NPRT(3)]    kinematic of decays (not effective);
\item[NPRT(4)]    kinematic of nuclear interactions;
\item[NPRT(5)]    kinematic of electromagnetic interactions (not effective);
\item[NPRT(6)]    material constants, dE/dX and absorbed energies
                  (not effective);
\item[NPRT(7)]    event summary;
\item[NPRT(8)]    history of all interactions/decays;
\item[NPRT(9)]    free;
\item[NPRT(10)]   tables of the geometry, cross-sections, etc.;
\end{DLtt}
{\tt NPRT(1), NPRT(2)} and {\tt NPRT(6)} should be used only in case  of
suspected errors. {\tt NPRT(8)} produces the most illustrative output. Those
flags work in conjunction with the {\tt DEBUG} data record 
of {\tt GEANT}.
\subsection{Description of the routines }
\Shubr{GHESIG}{(P,EK,AVER,A,Z,W,NLM,DENS,CORR,IPART)}
\begin{DLtt}{MMMMMMMMMM}
\item[P] ({\tt REAL}) momentum of the particle (in GeV/c);
\item[EK] ({\tt REAL}) kinetic energy of the particle (in GeV);
\item[AVER] ({\tt REAL}) average mass number of the material;
\item[A(NLM)] ({\tt REAL}) vector containing the mass numbers 
of the components of the mixture, the same as {\tt AVER} in case of an
element;
\item[Z(NLM)] ({\tt REAL}) vector containing the atomic numbers of the
components of the mixture, or the atomic number in case of an element;
\item[W(NLM)] ({\tt REAL}) vector of length NLM containing the relative weights
of  the  components of the mixture (normalised to one), one in case of an
element;
\item[NLM]  ({\tt INTEGER}) number of components of the mixture, $1$ in case 
of an element;
\item[DENS]  ({\tt REAL}) density of the material;
\item[CORR]  ({\tt REAL}) if this parameters is $>0$, then corrections are
applied to the cross-section; should be used in case of inorganic 
scintillators;
\item[IPART] ({\tt INTEGER}) {\tt GEANT} particle code.
\end{DLtt}
This routine returns the total macroscopic cross-section in cm$^{-1}$. 
The correction flag is taken from the $26^{th}$ word of the {\it next}
bank in the tracking medium linear structure pointed at by 
{\tt LQ(JTMED-NUMED)}.
 
The cross-sections on nucleus are known only for pions and protons. The
general law:
\[
\sigma(A) = 1.25*\sigma_{tot}^{Proton} \; A^\alpha
\]
is used but it is valid only for momenta $>$ 2 GeV. The parametrisation done
gives only a behaviour averaged over momenta and particle types. 
For H, Al, Cu and Pb the measured cross-sections are
stored in {\tt DATA} statements.
 
The stored cross-sections are the pre-calculated {\tt GHEISHA} ones.
As a starting point the measured cross-sections of pions, kaons,
protons, antiprotons and neutrons on protons are used. The cross-sections
tabulated are measured values taken from the CERN HERA compilations. The
values for K$^{0}_{s}$/K$^{0}_{l}$ are updated as of July 1980. 
Strange baryon cross-section
are calculated using a parametrisation in terms of quark-quark forward
scattering amplitudes and optical theorem. The additive quark quark scattering
model is used.
All the cross-sections are contained in data statements so no
external file is needed for {\tt GHEISHA}.
\Shubr{GPGHEI}{}
This routine returns the distance to the next hadronic interaction
according to the {\tt GHEISHA} cross-sections. It calls \Rind{GHESIG}
and is called by \Rind{GUPHAD}. The default copy provided in
the {\tt GEANT} library uses the {\tt GHEISHA} shower code:
\begin{verbatim}

       SUBROUTINE GUPHAD
* 
*****************************************************************
*                                                               *
*   GEANT  user routine called at each step to evaluate         *
*   the remaining distance to the hadronic interaction point    *
*                                                               *
*****************************************************************
*
      CALL GPGHEI
*
      END

      SUBROUTINE GUHADR
*.
****************************************************************
*                                                              *
*       GEANT  user routine called when a hadronic process     *
*       has been selected in the current step, in order to     *
*       generate the final particle's state                    *
*                                                              *
****************************************************************
* 
       CALL GHEISH
*
       END
\end{verbatim}

\Shubr{GHEISH}{}
This is the main steering routine for the hadronic interactions and is
a fan-out to the various {\it cascade} routines of
{\tt GHEISHA} which treat the particular hadronic interaction. Here the kind
of interaction is stored in the {\tt INT}
flag, with the following meaning:
\begin{DLtt}{MMMMMMMM}
\item[INT=0] no interaction            ({\tt NONE});
\item[INT=1] elastic scattering occurs ({\tt ECOH});
\item[INT=2] inelastic incoherent interaction occurs ({\tt INHE}),
    1 and 2 include also nuclear reaction processes at very low energies;
\item[INT=3] nuclear fission with inelastic scattering occurs ({\tt FISS});
\item[INT=4] neutron nuclear capture occurs ({\tt CAPT}).
\end{DLtt}

After the interaction has been selected, 
the appropriate cascade routine is called. Upon exit from this
there is a check whether the interaction has generated new particles or not. If
yes, the new particles are copied in the {\tt GEANT} temporary stack
{\tt (GKING)}. If the particle is a heavy fragment or a proton and it is below
the energy cut specified via the {\tt CUTS} data record, it is not stored in the
stack but the kinetic energy is collected. The size of the {\tt GHEISHA}
stack is parametrised, and its limit is currently set to 100 in sequence
{\tt GCKMAX}. The
user is left to decide in \Rind{GUSTEP} what to do with
the new tracks. This routine is called also in case of a stopping hadron,
i.e. with kinetic energy below the {\tt GEANT} cuts.
In this case the routine
\Rind{GHSTOP} (see later)
is called to handle the stopping hadron. The printing flags for
\Rind{GHEISHA} are also set in this routine according to the current value
of {\tt IDEBUG}.
 
This routine must be called by the user routine \Rind{GUHADR} (see above).
 
As explained above, for inorganic scintillators such as the BGO, it is
possible to activate a correction to the hadronic cross-section. This is
done via the routine \Rind{GSTPAR} in the following way:
\begin{verbatim}
      CALL GSTPAR(ITMED,'GHCOR1',1.0)
\end{verbatim}
The parameter is actually a flag which, if different from 0, triggers
the calculation of the cross-section correction, but in view of future
developments it is good practice set it to 1.0 when those corrections are
required. {\tt ITMED} is the tracking medium number as used in
\Rind{GSTMED} for which corrections are requested. This routine has
to be called before \Rind{GPHYSI}.
\Shubr{GHSTOP}{}
This is an internal routine used to handle stopping particles, called
by \Rind{GHEISH}. Here again we have a switch to the various routines
handling stopping particles. In particular this routine can lead to
nuclear absorption for negative pions and negative kaons ({\tt ABSO}) 
and to
annihilation for antineutrons, and antiprotons ({\tt ANNH}). The kinetic
energy is completely absorbed, and in the case of an unstable particle, the
particle is decayed at rest via the standard {\tt GEANT} decay routine
\Rind{GDECAY}.
 
