\documentstyle{book}
\oddsidemargin 0.54cm
\evensidemargin 0.0cm
\topmargin -50pt
\textheight 22.5cm \textwidth 15.cm

\begin{document}
\pagestyle{plain}
\rm
\Large
{\bf The Ray-tracing}
\\[2em]
\large
\rm
Running interactively under X11, a new visualization tool is provided in
GEANT 3.21: the ray-tracing package. Based on the new Geant tracking (see
GEOM and TRACK), a set of routines doing light processing is provided to
visualize the detectors (useful also to visualize the results of boolean
operations).
Basically, visible light particles are tracked
throughout the detector until when they do not hit the surface of a volume
declared not transparent; then, the intersection point is transformed to 
the screen coordinates and the corresponding pixel is drawn with a computed 
hue and luminosity. 
In case the command (DOPT RAYT ON) is executed,
the drawings are performed by the Geant ray-tracing;
automatically, the color is assigned according to the tracking medium of each
volume and the volumes with a density lower/equal than the air are considered
transparent; if the option (USER) is set (ON) (again via the command (DOPT)),
the user can set color and visibility for the desired volumes via the command
(SATT), as usual, relatively to the attributes (COLO) and (SEEN).
The resolution can be set via the command (SATT * FILL VALUE), where (VALUE)
is the ratio between the number of pixels drawn and 20 (user coordinates).
Parallel view and perspective view are possible (DOPT PROJ PARA/PERS); in the
first case, we assume that the first mother volume of the tree is a box with
dimensions 10000 X 10000 X 10000 cm and the view point (infinetely far) is
5000 cm far from the origin along the Z axis of the user coordinates; in the
second case, the distance between the observer and the origin of the world
reference system is set in cm by the command (PERSP NAME VALUE); grand-angle
or telescopic effects can be achieved changing the scale factors in the command
(DRAW). (Please, note that in case of perspective views, fixed the distance of 
the viewer, the zooming factors for the raytracing option must be 8.25 times
bigger than the ones used for the HIDE options in order to get the same
picture).
When the final picture does not occupy the full window,
mapping the space before tracing can speed up the drawing, but can also
produce less precise results; values from 1 to 4 are allowed in the command
(DOPT MAPP VALUE), the mapping being more precise for increasing (VALUE); for
(VALUE = 0) no mapping is performed (therefore max precision and lowest speed).
The command (VALCUT) allows the cutting of the detector by three planes
ortogonal to the x,y,z axis. The attribute (LSTY) can be set by the command
SATT for any desired volume and can assume values from 0 to 7; it determines
the different light processing to be performed for different materials:
0 = dark-matt, 1 = bright-matt, 2 = plastic, 3 = ceramic, 4 = rough-metals,
5 = shiny-metals, 6 = glass, 7 = mirror. The detector is assumed to be in the
dark, the ambient light luminosity is 0.2 for each basic hue (the saturation
is 0.9) and the observer is assumed to have a light source (therefore he will
produce parallel light in the case of parallel view and point-like-source
light in the case of perspective view). Finally, a second light source can 
be positioned in the space (with a desired intensity) via the command (SPOT).








   


   






\end{document}


