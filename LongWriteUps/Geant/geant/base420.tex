%%%%%%%%%%%%%%%%%%%%%%%%%%%%%%%%%%%%%%%%%%%%%%%%%%%%%%%%%%%%%%%%%%%
%                                                                 %
%  GEANT manual in LaTeX form                                     %
%                                                                 %
%  Michel Goossens (for translation into LaTeX)                   %
%  Version 1.00                                                   %
%  Last Mod. Jan 24 1991  1300   MG + IB                          %
%                                                                 %
%%%%%%%%%%%%%%%%%%%%%%%%%%%%%%%%%%%%%%%%%%%%%%%%%%%%%%%%%%%%%%%%%%%
\Origin{R.Brun, F.Carminati}   
\Submitted{27.07.93}      \Revised{14.12.93}
\Version{Geant 3.16}      \Routid{BASE420}
\Makehead{The random number generator}
 
\Shubr{GRNDM}{(VEC*,LEN)}
\begin{DLtt}{MMMMMMMM}
\item[VEC] ({\tt REAL}) vector containing the generated random number;
\item[LEN] ({\tt INTEGER}) number of random numbers to generate.
\end{DLtt}

\Rind{GRNDM} generates a sequence of uniformly distributed random numbers in the
interval (0,1). The numbers are returned in a vector.
The code is a copy
of the CERN Program Library routine \Rind{RANECU}~\cite{bib-LECU,bib-JAM1} 
(entry V114).

Several independent sequences can be defined and used. Each sequence {\bf must}
be initialised by the user, otherwise the result is unpredictable. 
Two integer seeds are used to initialise a sequence. Not all pairs of
integers define a good random sequence or one which is independent from
others. Sections of the same random sequence can be defined as independent
sequences. The period of the generator is $2^{60} \approx 10^{18}$. 
A generation has
been performed in order to provide the seeds to start any of the generated
sections. There are 215 possible seed pairs and they are all $10{^9}$  numbers
apart. Thus a sequence started from one of the seed pairs, after $10{^9}$
numbers will start generating the next one. 

\Shubr{GRNDMQ}{(ISEED1,ISEED2,ISEQ,CHOPT)}
\begin{DLtt}{MMMMMMMM}
\item[ISEED1] ({\tt INTEGER}) first seed of the sequence;
\item[ISEED2] ({\tt INTEGER}) second seed of the sequence;
\item[ISEQ] ({\tt INTEGER}) number of the independent
sequence of random numbers referred to by this call. If
{\tt ISEQ}$\leq$0, then the last valid sequence used will be addressed
either for a save or a store. In case the option {\tt 'G'} is
specified, on output the variable will contain the sequence
actually used;
\item[CHOPT] ({\tt CHARACTER*(*)}) the action to be taken:
\begin{DLtt}{MMMM}
\item[' ']  if 1$\leq${\tt ISEQ}$\leq$215, sequence number {\tt ISEQ} will be
initialised with the seeds of the pre-computed
independent sequence {\tt ISEQ}.

If {\tt ISEQ}$\leq$0 or {\tt ISEQ}$>$215, sequence number 1 will be
initialised with the default seeds. {\tt ISEED1} and {\tt ISEED2} are
ignored;
\item['G']  get the present status of the generator: the two integer
seeds {\tt ISEED1} and {\tt ISEED2} will be returned for sequence
{\tt ISEQ};
\item['S']  set the status of the generator.
The two integer seeds {\tt ISEED1} and {\tt ISEED2} will be
used to restart the generator for sequence {\tt ISEQ}.
\item['SH']  the same action as for {\tt 'S'} and store the two
integer seeds {\tt ISEED1} and {\tt ISEED2} in the event header bank.
\item['Q']  Get the pre-generated seeds for {\tt ISEQ} 
(1$\leq${\tt ISEQ}$\leq$215).
There are 215 pre-generated sequences each one will
generate $10^{9}$  numbers before reproducing the following
one.
\end{DLtt}
\end{DLtt}
 
Initialises the random number generator.
 
\Sfunc{GARNDM}{VALUE = GARNDM(DUMMY)}

\begin{DLtt}{MMMMMMMM}
\item[DUMMY] ({\tt REAL}) dummy parameter, ignored;
\end{DLtt}

Returns a random number $r$ distributed as $e^{-x}$. $r = -\ln(\eta)$
where $\eta$ is a random number extracted by \Rind{GRNDM}.

\Shubr{GPOISS}{(AMVEC,NPVEC*,LEN)}

\begin{DLtt}{MMMMMMMM}
\item[AMVEC] ({\tt REAL}) array of length {\tt LEN} containing the average
values of the Poisson distributions requested;
\item[NPVEC] ({\tt INTEGER}) array of length {\tt LEN} containing the random
numbers: {\tt NPVEC(I)} is a random number with a Poisson distribution of
average {\tt AMVEC(I)};
\item[LEN] ({\tt INTEGER}) number of random numbers requested.
\end{DLtt}

This routine extracts integer random numbers according to the
Poisson distribution. Given a Poisson distribution of average $\lambda \leq 16$
and $r$ uniformly distributed between 0 and 1, the method used is the 
following:
\begin{enumerate}
\item let $P = \exp(-\lambda)$, $N=0$ and $S=P$;
\item if $r\leq S$ accept $N$;
\item let $N=N+1$, $P=P \lambda/N$, $S=S+P$ and go back to 2;
\end{enumerate}

If $\lambda > 16$, a gaussian distribution with average $\lambda$ and 
$\sigma = \sqrt{\lambda}$ is generated.
