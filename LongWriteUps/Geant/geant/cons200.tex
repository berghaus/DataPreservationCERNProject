%%%%%%%%%%%%%%%%%%%%%%%%%%%%%%%%%%%%%%%%%%%%%%%%%%%%%%%%%%%%%%%%%%%
%                                                                 %
%  GEANT manual in LaTeX form                                     %
%                                                                 %
%  Version 1.00                                                   %
%                                                                 %
%  Last Mod. 12 June 1993 1700   MG                               %
%                                                                 %
%%%%%%%%%%%%%%%%%%%%%%%%%%%%%%%%%%%%%%%%%%%%%%%%%%%%%%%%%%%%%%%%%%%
\Origin{R.Brun}            
\Revision{F.Carminati, M.Maire}
\Version{Geant 3.16}           \Routid{CONS200}
\Submitted{01.06.83}           \Revised{08.12.93}
\Makehead{Tracking medium parameters}
\section*{Description of the routines}
\Shubr{GSTMED}{(ITMED,NATMED,NMAT,ISVOL,IFIELD,FIELDM,TMAXFD,STEMAX,
                DEEMAX,EPSIL,STMIN,UBUF,NWBUF)}
This routine associates a set of tracking parameters to a material, defining
a so-called {\it tracking medium}. Every {\tt GEANT} volume must be 
associated to an existing tracking medium. The routine
stores the parameters of the tracking
medium {\tt ITMED} in the data structure {\tt JTMED}.
\begin{DLtt}{MMMMMMMM}
\item[ITMED]      ({\tt INTEGER}) tracking medium number;
\item[NATMED]     ({\tt CHARACTER*20}) tracking medium name;
\item[NMAT]       ({\tt INTEGER}) material number corresponding to {\tt ITMED};
\item[ISVOL]      ({\tt INTEGER}) {\it sensitivity} flag (see later):
\begin{DLtt}{MMMM}
\item[$\leq$0] not a sensitive volume;
\item[$>$0] sensitive volume;
\end{DLtt}
\item[IFIELD]     ({\tt INTEGER}) magnetic field flag:
\begin{DLtt}{MMMM}
\item[=0]         no magnetic field;
\item[=1]         strongly 
inhomogeneous magnetic field (returned by the user function
\Rind{GUFLD}): tracking performed with the Runge-Kutta method by the
routine \Rind{GRKUTA};
\item[=2]         inhomogeneous magnetic field (returned by the user function
\Rind{GUFLD}), tracking along a helix performed by the routine \Rind{GHELIX};
\item[=3]         uniform magnetic field along the {\tt z} axis of strength
{\tt FIELDM}, tracking performed along a helix by the routine \Rind{GHELX3};
\end{DLtt}
\item[FIELDM]     ({\tt REAL}) maximum field value (in Kilogauss);
\item[TMAXFD]     ({\tt REAL}) maximum angular deviation due to the magnetic
field permitted in one step (in degrees);
\item[STEMAX]     ({\tt REAL}) maximum step permitted (cm);
\item[DEEMAX]     ({\tt REAL}) maximum fractional energy loss in one step ($0<$
                  {\tt DEEMAX} $\leq 1$);
\item[EPSIL]      ({\tt REAL}) boundary crossing precision (cm);
\item[STMIN]      ({\tt REAL}) minimum value for the maximum step imposed by 
energy loss, multiple scattering, \v{C}erenkov or magnetic field effects (cm);
\item[UBUF]       ({\tt REAL}) array of {\tt NWBUF} additional user parameters;
\item[NWBUF]      ({\tt INTEGER}) number of additional user parameters.
\end{DLtt}
 
{\bf Note:} the tracking medium number can in principle be any positive
integer from 1 to 65,535. However this number is used directly by {\tt GEANT}
as the number of the pointer to the data structure where the information
is stored. When a pointer is defined,
all pointers from 1 to the one allocated are created, if not already existing.
Every time data structures are moved in memory, all the
links are potentially scanned for update. This can be time consuming and
it can seriously affect performances. So a continuous
range of numbers should be used for tracking media.

\Shubr{GFTMED}{(ITMED,NATMED*,NMAT*,ISVOL*,IFIELD*,FIELDM*,TMAXFD*,
                STEMAX*,DEEMAX*,EPSIL*,STMIN*,UBUF*,NWBUF*)}
 
Extracts the parameters of the tracking medium {\tt ITMED}
from the data structure {\tt JTMED}.
 
\Shubr{GPTMED}{(ITMED)}
 
Prints the tracking medium parameters for the given medium.
\begin{DLtt}{MMMMM}
\item[ITMED]      ({\tt INTEGER}) tracking medium to be printed,
all tracking media if {\tt ITMED}=0.
\end{DLtt}

\section*{Automatic calculation of the tracking parameters}

The results of the simulation depend critically on the choice of the tracking
medium parameters. To make of {\tt GEANT} a reliable and consistent predictive
tool, the calculation of these parameters has been automated.
In a normal GEANT run, the variable {\tt IGAUTO} in common
\FCind{/GCTRAK/} is set to 1. In this case the following parameters 
are calculated by the program:
\begin{eqnarray*}
\mbox{\tt TMAXFD} & = & \left\{
\begin{array}{LL}
20^{\circ} & \mbox{if {\tt TMAXFD} $\leq 0$ or {\tt TMAXFD} $ > 20$} \\
\mbox{\it input value} & \mbox{if $0 <$ {\tt TMAXFD} $\leq 20$}
\end{array} \right . \\
\mbox{\tt STEMAX} & = & \phantom{\{} 
\begin{array}{LL} \mbox{\tt BIG} (=10^{10}) \end{array} \\
\mbox{\tt DEEMAX} & = & \left \{
\begin{array}{LL}
0.25 & \parbox[t]{10cm}{if {\tt ISVOL} $= 0$ and $X_{0} \leq 2cm$, 
where $X_{0}$ is the radiation length of the material} \\
0.25-\frac{0.2}{\sqrt{X_{0}}} & \mbox{otherwise}
\end{array} \right . \\
\mbox{\tt STMIN} & = & \left\{
\begin{array}{LL}
\frac{2 R}{\sqrt{X_{0}}} & \mbox{if {\tt ISVOL} $=0$} \\ [.2cm]
\frac{5 R}{\sqrt{X_{0}}} & \mbox{if {\tt ISVOL} $\neq 0$}
\end{array} \right . 
\end{eqnarray*}

where $R$ is the range in cm of an electron of energy {\tt CUTELE}+200keV.

If the {\tt IGAUTO} variable is set to 0 via the {\tt AUTO} data record, 
then value given by the user for the above parameters is accepted as the
true parameter value if $>0$, while automatic calculation still takes place
in case the input value is negative.

Users are strongly recommended to begin their simulation with the parameters
as calculated by {\tt GEANT}. Users who want to modify any of the above
parameters must be sure they understand their function in the program and
the implications of a change.

The {\tt EPSIL} parameter represents the absolute precision with which the
tracking is performed. It has a double meaning. When tracking in magnetic
field, {\tt EPSIL} is the minimum distance for which boundaries are
checked. A particle nearer than {\tt EPSIL} to the boundary is considered
as exiting the current volume. If the end point of the step of a particle in
magnetic field is distant less than {\tt EPSIL} along each axis 
from what would be the end point in absence of magnetic field, then no boundary 
checking is performed. 

In all cases, when a particle is going to reach the
boundary of the current volume with the current step, the step length is 
increased by a quantity ({\tt PREC}, common \FCind{/GCTMED/}) which is set to the 
minimum between $0.1 \times {\tt EPSIL}$
and 10 micron at the beginning of the tracking. This quantity is recalculated
at every step according to the formula:
\begin{equation}
\mbox{\tt PREC} 
=\max \left [ \min \left(\frac{\mbox{\tt EPSIL}}{10}, 10 \mu \right ) ,
\max \left [ | x |, | y |, | z |, S \right ) \times \epsilon \right ]
\end{equation}
Where $x, y, z$ are the current coordinates of the particle, $S$ is the length
of the track, and $\epsilon$ is the assumed machine precision.
$\epsilon$ (called {\tt EPSMAC} in the program) is initially set to
$10^{-6}$ for 32 bits machines and $10^{-11}$ for 64 bits machines.
During the tracking, every fifth time that a particle tries unsuccessfully
to exit from the same volume, the machine precision is multiplied by the
number of trials. This accounts for additional losses of precision due to
transformation of coordinates and region of the floating point range where
the real machine precision is different from the above (this happens 
in particular on IBM mainframes with 370 floating point number representation).
