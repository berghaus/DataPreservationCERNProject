\documentstyle{book}
\oddsidemargin 0.54cm
\evensidemargin 0.0cm
\topmargin -50pt
\textheight 23.5cm \textwidth 15.cm

\begin{document}
\pagestyle{plain}
\rm
\Large
`MANY' Volumes in GEANT 3.21  
\\[2em]
\large
\rm
The topology of Geant is given by 16 general shapes which are used to represent
solids in the design of detectors. These shapes are bounded by 2-nd order
surfaces for reasons of speed at tracking time. New tools have been 
introduced to extend the power of the Geant geometry: boolean operations 
(union, intersection and subtraction) between such shapes are now allowed, so
that an infinity of new shapes can be built; divisions along arbitrary axis
are now possible, allowing to exploit any simmetry in the design.
The tracking is now able to handle overlapping volumes in the most general
sense: the concept of MANY volumes has been extended to automatic clipping 
for protuding objects. Thanks to this MANY technique, any 3D or 2D geometrical
pattern can be reduced to a single dimensional structure, basically cancelling
the need for GUNEAR at tracking time (the Accordeon calorimeter has been
succesfully redesigned with this logic). 
Finally, more than one order of magnitude in speed has been gained compared
to the old step search algorithm used to track in MANY volumes. \\[.5cm]
\begin{itemize}
\item Topological extensions via boolean operations.
\item Divisions along arbitrary axis.
\item Reduction of 2D structure to 1D structure.
\item Logic and algorithm.
\end{itemize}  
\begin{center}
Topological extensions via boolean operations.\\[.5cm]
\end{center}
Boolean operations between a given set of shapes allow the creation of an
infinite number of shapes. Adding a single new primitive (for ex. a toroid) to 
the basic set of shapes, all the possible combinations with all the old shapes
become moreover possible.\\[.5cm]
\begin{itemize}
\item UNION: the union of two volumes B and C is obtained positioning two
             overlapping `many' daughters B and C inside the same mother A.
             To identify the result of the union as a single volume is enough
             to create a DET associated to the volumes B and C. 
\item SUBTRACTION: the subtraction of a volume B from a volume C is obtained
                   positioning B as `only' and C as a `many' overlapping B.
                   The result of the subtraction is automatically what the
                   tracking sees as the C volume, so GSPOS is the only user
                   interface needed. The volume B can have the same material as
                   the mother A or not. The normal positioning technique used
                   for `onlys' is a particular case of boolean subtraction (a
                   volume is contained inside the other).
\item INTERSECTION: the intersection of two volumes B and C is obtained 
                    positioning a protuding `many' object C in a `many' object
                    B which is normally placed in a `only' mother A. A and B
                    must have the same material, while C has the material asked
                    for the intersection. The intersection is given by the part
                    of C which is not protuding from B, therefore by what the
                    new tracking sees as the volume C: again the only needed
                    user interface is GSPOS. If not interacting with other
                    daughters of A, B could also be `only'. Without any
                    ambiguity, even further `many' daughters of A can overlap
                    the protuding part of C outside B, because it is really
                    invisible to the new tracking. 
\end{itemize} 
\begin{center}
Divisions along arbitrary axis.\\[.5cm]          
\end{center}
As protuding MANY daughters are automatically clipped by their mother, it is
possible to divide objects along arbitrary directions. Suppose for example
that we want to divide a PCON into planar slices parallel to the plane
X + Y = 0; it is enough to position a MANY BOX embedding completely the
PCON `inside' the PCON itself; the BOX must be rotated of 45 degrees in X and
Y and then divided along one of its natural axis. Another example can be the
following: suppose we want to divide 24 times in PHI a PGON with only 6 natural
segmentations in PHI; then we have to position a MANY TUBE, completely embedding
the PGON, `inside' the PGON itself; again, we can now divide the tube in 24
slices in PHI and the tracking will see only the PGON divided 24 times in PHI.
\begin{center}
Reduction of 2D structure to 1D structure.
\end{center}
Sometimes there are complicated multidimensional geometrical structures (like
honeycombs, spaghetti, etc.) which can be reduced to the overlap of several
one-dimensional patterns (for which divisions can be used or virtual divisions
are particularly efficient). The MANY option allows to superimpose such 1D
structures to reproduce the multi-dimensional ones in a very effective way,
instead of positioning the volumes one by one in the 2D or 3D pattern.
The following picture will explain better this concept.

\begin{center}
Logic and algorithm.\\[.5cm]
\end{center}
In GTMEDI, all the `many' volumes for which the point is found inside (and not
in any of its daughters) is put in a stack saving also its branch of the tree.
The point is considered to be in the deepest level `many'. Each time the point
is also found in an intermediate `only' volume the stack is cleared (to insure
boolean subtraction). In GTNEXT, if stepping inside a `many' volume, not only
its daughters and its own boundaries are checked, but also the daughters and
the boundaries of all the others `many' in the list have to be checked. This
is not yet enough: the logic requires that the boundaries of all the possible
overlapping volumes are checked, so it is also necessary for each `many' volume 
of the list to go up to the first `only' mother in the tree repeating the same 
procedure. This allows also boolean intersection. Obviously, volumes already
checked are not checked again. 





\end{document}


