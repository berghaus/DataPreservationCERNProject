%%%%%%%%%%%%%%%%%%%%%%%%%%%%%%%%%%%%%%%%%%%%%%%%%%%%%%%%%%%%%%%%%%%
%                                                                 %
%  GEANT manual in LaTeX form                              %
%                                                                 %
%  Michel Goossens (for translation into LaTeX)                   %
%  Version 1.00                                                   %
%  Last Mod. Jan 24 1991  1300   MG + IB                          %
%                                                                 %
%%%%%%%%%%%%%%%%%%%%%%%%%%%%%%%%%%%%%%%%%%%%%%%%%%%%%%%%%%%%%%%%%%%
\Origin{F.Bruyant}
\Submitted{13.06. 85}      \Revised{16.12.93}
\Version{Geant 3.16}       \Routid{HITS105}

\Makehead{Detector aliases}
Detector {\tt aliases} can be specified for any sensitive detector for which
the user either needs to store more than one type of hit
or wants to define additional detector entries.

\Shubr{GSDETA}{(CHSET,CHDET,CHALI,NWHI,NWDI,IALI*)}
\begin{DLtt}{MMMMMMMM}
\item[CHSET] ({\tt CHARACTER*4}) set name;
\item[CHDET] ({\tt CHARACTER*4}) name of the detector for which an alias
is being defined;
\item[CHALI] ({\tt CHARACTER*4}) alias name;
\item[NWHI] ({\tt INTEGER}) initial number of words of {\tt HITS} banks;
\item[NWDI] ({\tt INTEGER}) initial number of words of {\tt DIGI} banks;
\item[IALI] ({\tt INTEGER}) position of alias in bank {\tt JS=LQ(JSET-ISET)}.
\end{DLtt}

Defines an alias {\tt CHALI} for detector {\tt CHDET} of set {\tt CHSET} and
enters it in the {\tt JSET} structure as an additional detector in the
corresponding set, at the position {\tt IALI}. The new detector will be
a copy at position {\tt IALI} of the original detector
{\tt CHDET}, with empty links to the \Rind{GSDETH}, \Rind{GSDETD} and
\Rind{GSDETU} parameter banks.
The user can therefore call these three routines again with the
arguments appropriate to the detector {\tt CHALI}.
Several aliases can be defined for
the same detector through calls to \Rind{GSDETA}.
