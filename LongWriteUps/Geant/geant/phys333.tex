%%%%%%%%%%%%%%%%%%%%%%%%%%%%%%%%%%%%%%%%%%%%%%%%%%%%%%%%%%%%%%%%%%%
%                                                                 %
%  GEANT manual in LaTeX form                                     %
%                                                                 %
%  Michel Goossens (for translation into LaTeX)                   %
%  Version 1.00                                                   %
%  Last Mod. Jan 24 1991  1300   MG + IB                          %
%                                                                 %
%%%%%%%%%%%%%%%%%%%%%%%%%%%%%%%%%%%%%%%%%%%%%%%%%%%%%%%%%%%%%%%%%%%
\Origin{M.Maire}
\Revision{F.Carminati}
\Version{Geant 3.16}\Routid{PHYS333}
\Submitted {30.05.86}\Revised{16.12.93}
\Makehead {Information about energy loss fluctuations}
\section{Subroutines}
\Shubr{GDRPRT}{(IPART,IMATE,STEP,NPOINT)}
\begin{DLtt}{MMMMMMMMMM}
\item[IPART] ({\tt INTEGER}) {\tt GEANT} particle number;
\item[IMATE] ({\tt INTEGER}) {\tt GEANT} material number;
\item[STEP]  ({\tt REAL}) step in cm;
\item[NPOINT] ({\tt INTEGER}) number of logarithmically spaced energy points
between {\tt EKMIN} (default 10 keV) and {\tt EKMAX} (default 10 TeV). See
{\tt [BASE040]} for more information on these value and how to change them
with the {\tt ERAN} input record;
\end{DLtt}

This routine calculates and prints the relevant parameters
relative to the simulation of energy loss fluctuations due to ionisation and
$\delta$-ray production in {\tt STEP} cm:
\begin{DLtt}{MMMMMMMM}
\item[N d-rays] average number of $\delta$-rays produced;
\item[dE/I] average energy loss divided by the
average ionisation potential;
\item[xi/I] variable $\xi/I$ of the Landau/Vavilov
theory;
\item[xi/Emax] variable $\xi/E_{Max}$ of the Landau/Vavilov
theory;
\item[regime] valid model to simulate energy loss fluctuations.
Possible values are:
\begin{DLtt}{MMMMMMMM}
\item[Gauss] Gauss distribution;
\item[Vavilov] Vavilov distribution;
\item[Landau] Landau distribution;
\item[Urb\'{a}n] \Rind{GLANDZ} routine of Urban (see {\tt [PHYS332]});
\item[ASHO] atomic harmonic oscillator approximation (see {\tt [PHYS334]});
\item[PAI]  Photo-Absorption Ionisation model (see {\tt [PHYS334]});
\end{DLtt}
\end{DLtt}

This routine has no action on the {\tt GEANT} 
system. It is intended as a help to users understanding the
model which {\tt GEANT} may be asked to use, 
for instance via the routine \Rind{GSTPAR} {\tt [CONS210]}.

\section{Method}
When a charged particle traverses a portion of matter, 
it interacts with the electrons
and the nuclei of the atoms. Most of 
these interactions are
electromagnetic (quasi-)
elastic collisions in which the incoming particle loses energy in the
laboratory reference frame. The amount of energy loss in a thickness of 
material
$t$ is subject to two sources of fluctuations. The {\it number} of
collisions can fluctuate, and at the same time the {\it energy} lost in each
collision varies statistically. Both distributions are characterised
by a Poissonian-like behaviour. We can distinguish between collisions
where the energy transferred to the atomic electrons is enough to 
extract them from the atoms (ionisation with production of $\delta$-rays)
and collisions where the atomic structure is excited, without a complete
ionisation (excitation). It has to be noted that the energy
transferred to the nuclei is usually negligible. Momentum conservation
considerations show that the ratio of the energy transferred to 
electrons to the energy transferred to nucleus in Coulomb interactions
is of the order of $m_{e}Z A^{-1}m_{p}^{-1}$.

Several theories have been proposed to describe
this important mechanism in the transport of charged particles.
The main difference of these theories is in the greater or lesser
detail with which the scattering centres are described. 
The detail of these theories as they are implemented in {\tt GEANT} can
be found in {\tt [PHYS332]}, {\tt [PHYS334]} for energy loss fluctuations
and in {\tt [PHYS430]}, {\tt [PHYS431]} for energy loss.

In general, the greater the thickness of the layer traversed
in terms of the number of atoms encountered, the larger
is the number of collisions. In this case a detailed description of
the atomic structure may be irrelevant to account for the form of the
fluctuation. Landau and Vavilov have proposed theories in this region
and these are implemented in {\tt GEANT}. 

When the thickness of the material is such that the number of collisions
becomes smaller, the detailed nature of the atomic structure becomes 
important in determining the fluctuations of the energy loss. In this
case the coupling of the various atomic energy levels to the to the
Coulomb field must be taken into account.

When simulating the energy loss by ionisation, the average path-length
between collisions is a function of the cross section $\sigma(E)$.
This cross section becomes very large
when $E \rightarrow 0$, so that it is necessary
to establish a threshold energy below which the process is described in a
{\it condensed} way. Above this threshold, ionisations are
described in a detailed way, with the production of $\delta$-rays.
Thus, when a charged particle is moving in a medium, there are
in general two ways to simulate the energy loss by ionisation:
\begin{enumerate}
\item calculate the average value the energy loss 
via the full Bethe-Bloch equation. This takes into account
$\delta$-rays generation. The fluctuations are then explicitly
introduced via an appropriate distribution.
In {\tt GEANT} 
this method is selected by the value {\tt ILOSS = 2} which is
controlled by the {\tt LOSS} data record.
\item explicitly generate the $\delta$-rays {\it above} a given 
threshold ({\tt DCUTE} for electrons, {\tt DCUTM} for others particle).
In this case the average energy
loss is computed from a restricted formula and
both the value of the energy loss
and the number of $\delta$-rays
produced are function of the threshold cut (hereafter {\tt DRCUT}).
In {\tt GEANT} this method is selected by the value {\tt ILOSS = 1} which is
controlled by the {\tt LOSS} data record.
\end{enumerate}

\subsection{Validity ranges for the different models}
The limits of the validity for
the various models are somewhat arbitrary
within a certain range of material densities and particle energies.
To help determine these limits, some characteristic parameters are used:

\begin{DLtt}{MMMMMMMMMMMMM}
\item[$\xi$] typical energy loss of
a particle in a given layer of material. Its value is:

\[
\xi = 2 \pi r_{e}^{2} m_{e} c^{2} N_{Av} \left ( \frac{Z_{inc}}
{\beta} \right )^{2} \frac{Z \rho t}{A} = 0.1535 \left ( \frac{Z_{inc}}
{\beta} \right )^{2} \frac{Z \rho t}{A} \: (MeV)
\]

\item[$E_{Max}$] maximum transferable energy in a single collision:

\[
E_{Max} = \frac{2 m_{e} \beta^{2} \gamma^{2}}{1 +2 \gamma \frac{m_{e}}
{M} +\left ( \frac{m_{e}}{M} \right ) ^{2}}
\]

where $M$ is the mass of the incoming particle. Note that for incoming
electrons this value has to be divided by 2 due to the impossibility
distinguishing the two electrons in the final state.

\item[$I$] typical electron binding energy. This is a value which 
characterises the average energy levels of the atomic electrons. In
{\tt GEANT} it is parametrised as:

\[
I = 16 Z^{0.9} \: (eV)
\]

\item[$\displaystyle \frac{dE}{dx} \: t$] 
average energy loss by the particle in a
layer of thickness $t$ according to the {\tt GEANT} energy loss tables.

\item[$\xi/E_{Max}$] (hereafter $\kappa$) 
relative importance of high energy transfer collisions 
in the ionisation process

\item[$\min \left ( \frac{dE}{dx} \: t,\xi \right )/I$] 
(hereafter $\Delta$) estimation of the number of collisions
with energy close to the ionisation energy.
\end{DLtt}

In {\tt GEANT} two variables control the model used to describe the
energy loss fluctuations: 
\begin{DLtt}{MMMMMMMMMM}
\item[ILOSS] fluctuation model:
\begin{DLtt}{MMMMM}
\item[0] no energy loss;
\item[1] $\delta$-rays are produced above the threshold,
{\it reduced}
fluctuations from $\delta$-rays below the threshold are added to the
energy loss;
\item[2] no $\delta$-rays are produced, complete fluctuations are calculated;
\item[3] same as {\tt 1};
\item[4] no fluctuations;
\end{DLtt}
\item[ISTRA] energy loss model for {\it thin} layer (see below):
\begin{DLtt}{MMMMM}
\item[0] Urb\'{a}n model;
\item[1] PAI model;
\item[2] ASHO model for $1 < \Delta \leq 30$,(not yet available) 
PAI model otherwise;
\end{DLtt}
\end{DLtt}
The validity limits of the different models are estimated
as follows:
\begin{enumerate}
\item {\bf large number of low-energy collisions: $\Delta \geq 50$}

if $\delta$-rays generation is requested ({\tt ILOSS = 1} or {\tt 3} and 
{\tt IDRAY = 1}) the Urb\'{a}n model is used. If $\delta$-rays are not
produced ({\tt ILOSS = 2} and {\tt IDRAY = 0}) we distinguish three regions:
\begin{enumerate}
\item {\bf very few energy transfers close to the 
maximum: $\kappa \leq 10^{-2}$}

the Landau distribution is used;

\item {\bf few energy transfers close to the 
maximum: $10^{-2} < \kappa \leq 10$}

the Vavilov distribution is used;

\item {\bf many energy transfers close to the 
maximum: $\kappa > 10$}

the Gauss distribution is used;

\end{enumerate}

\item {\bf small number of low-energy collisions: $ \Delta < 50$}

in this region the same model is used for $1 \leq \mbox{\tt ILOSS}
\leq 3$ irrespective of the value of {\tt IDRAY}.
The model used depends on the value of {\tt ISTRA}.

\begin{enumerate}
\item {\bf $ 30 < \Delta \leq 50$}

Urb\'{a}n model if {\tt ISTRA} $= 0$, PAI model otherwise.

\item {\bf $ 1 < \Delta \leq 30$}

Urb\'{a}n model if {\tt ISTRA} $= 0$, PAI model if {\tt ISTRA}
$=1$ and ASHO model if {\tt ISTRA} $=2$.

\item {\bf $ 0.01 < \Delta \leq 1$}

Urb\'{a}n model if {\tt ISTRA} $= 0$, PAI model otherwise.

\item {\bf $ \Delta \leq 0.01$}

Urb\'{a}n model if {\tt ISTRA} $=0$, PAI model otherwise.

\end{enumerate}
\end{enumerate}

\subsection{Recommendations}
The full energy loss fluctuations or the restricted fluctuations
complemented by the production of $\delta$-rays give equivalent
results if the number of $\delta$-rays generated 
is sufficient, e.g. a few tens along
the full trajectory of the particle in the medium.
 
For a relativistic particle, the number of delta-rays
produced per cm can be estimated integrating the
cross section provided in~\cite{bib-PDGB}:
\begin{equation}
\frac{dN}{dx} \approx \frac{D}{2} \frac{\rho Z}{A} 
\frac{Z_{inc}^2}{\mbox{\tt DRCUT}}
\end{equation}
where
\begin{DLtt}{MMMMMMM}
\item[$D$] 0.307 MeV cm$^2$ g$^{-1}$;
\item[$\rho$] density of the medium;
\item[$Z,A$] atomic number and atomic weight of the medium;
\item[$Z_{inc}$] charge of the particle;
\item[DRCUT] energy threshold for emitted delta-rays.
\end{DLtt}
This formula holds for electrons/positrons as well as for other particles.
The number of $\delta$-rays must be sufficient 
to reproduce the statistical fluctuations in the energy loss,
but not too large, as in this case
a large amount of time could be spent in 
tracking them
without a corresponding improvement of the energy loss distribution.

On the other hand, the distribution of the
energy {\it lost} by a particle does not necessarily correspond to the
distribution of the energy {\it deposited} by the same particle in the
medium traversed. This is particularly true for light or {\it thin}
materials where $\delta$-rays can escape into the
next material.

It is the responsibility of the user to estimate the number of $\delta$-rays 
per cm that are needed, according to the considerations above. Then the 
correct value for {\tt DRCUT} should be set with the help of \Rind{GDRPRT}. 
It is recommended to use the same value for {\tt DCUTE} and {\tt DCUTM}.
As the {\tt GEANT} tables for cross-sections are not generated below 
{\tt EKMIN}, {\tt DRCUT} cannot be smaller than {\tt EKMIN}. The value of 
{\tt EKMIN} can be changed via the {\tt ERAN} data record, and its default
value is $10$ keV.

In a {\it light} material like gas, the number of emitted $\delta$-rays for
{\tt DRCUT} $=10$ keV may not be sufficient to ensure a correct energy loss 
distribution. If the user is mainly interested in the energy lost by the 
particle, no discrete $\delta$-ray should be produced setting {\tt ILOSS} 
$=2$ (see {\tt [PHYS332]}).

In case the user is more interested in a precise simulation of the energy 
deposited, then the explicit $\delta$-ray generation should be tried.

As far as {\it thin} materials are concerned ($\Delta < 50$), the default
in {\tt GEANT} is to use the Urb\'{a}n model. Comparison with experimental
data have shown that this model gives very good results, and it is
considerably faster than both the ASHO and the PAI models. Users should
understand well their setup and their results before they try with
different models, which, in principle, should give very similar results.

\subsection{Default values in {\tt GEANT}}
 
In order to avoid double counting, there is an automatic protection in
the code:
{\it if {\tt ILOSS=2}, {\tt IDRAY} is set to 0 and the generation of 
$\delta$-rays is disabled}.
 
The following table summarises the different cases:
\begin{center}
\begin{tabular}{|c|c|c|}\hline
                    &\tt LOSS=2          &\tt LOSS=1     \\
                    &(default)            &               \\
                    \cline{2-3}
     \tt IDRAY      & 0                   &     1          \\
     \tt DCUTE      &  $10^4$ GeV         & \tt CUTELE     \\
     \tt DCUTM      &  $10^4$ GeV         & \tt CUTELE     \\
\hline
\end{tabular}
\end{center}
Where {\tt LOSS}$=1$ means that the restricted fluctuation is 
activated and {\tt LOSS}$=2$ means that the complete Landau/Vavilov/Gauss
fluctuations are used in the region $\Delta \geq 50$. 
