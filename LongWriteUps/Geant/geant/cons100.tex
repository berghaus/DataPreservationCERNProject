%%%%%%%%%%%%%%%%%%%%%%%%%%%%%%%%%%%%%%%%%%%%%%%%%%%%%%%%%%%%%%%%%%%
%                                                                 %
%  GEANT manual in LaTeX form                                     %
%                                                                 %
%  Version 1.00                                                   %
%  Last Mod.  8 June 1993 1300   MG                               %
%                                                                 %
%%%%%%%%%%%%%%%%%%%%%%%%%%%%%%%%%%%%%%%%%%%%%%%%%%%%%%%%%%%%%%%%%%%
\Origin{R.Brun, G.N.Patrick} 
\Version{Geant 3.16}\Routid{CONS100}
\Submitted{06.06.83}             \Revised{01.11.93}
\Makehead{Material definition}
\Shubr{GMATE}{}
 
Stores the following standard material constants
in the data structure \cite{bib-PDGD}
{\tt JMATE}.
\begin{center}\begin{tabular}{|l|r|r|r|r|r|}
 \hline
Material   &No. &A         &Z       &Density   &Radiat L   \\
\hline
Hydrogen   &1   &1.010     &1.000   &0.071     &865.000    \\
Deuterium  &2   &2.010     &1.000   &0.162     &757.000    \\
Helium     &3   &4.000     &2.000   &0.125     &755.000    \\
Lithium    &4   &6.940     &3.000   &0.534     &155.000    \\
Beryllium  &5   &9.010     &4.000   &1.848     &35.300     \\
Carbon     &6   &12.010    &6.000   &2.265     &18.8       \\
Nitrogen   &7   &14.010    &7.000   &0.808     &44.500     \\
Neon       &8   &20.180    &10.000  &1.207     &24.000     \\
Aluminium  &9   &26.980    &13.000  &2.700     &8.900      \\
Iron       &10  &55.850    &26.000  &7.870     &1.760      \\
Copper     &11  &63.540    &29.000  &8.960     &1.430      \\
Tungsten   &12  &183.850   &74.000  &19.300    &0.350      \\
Lead       &13  &207.190   &82.000  &11.350    &0.560      \\
Uranium    &14  &238.030   &92.000  &18.950    &0.320      \\
Air        &15  &14.610    &7.300   &$1.205 \times 10^{-3}$ &30423  \\
Vacuum     &16  &$10^{-16}$ &$10^{-16}$ &$10^{-16}$ &$10^{16}$ \\
\hline
\end{tabular} \end{center}
 
{\bf Note:} this table provides some material definition with standard
properties. Advanced user will want to define their own table of materials,
which can be done with the routine \Rind{GSMATE}, described below.
 
\Shubr{GSMATE}{(IMATE,CHNAMA,A,Z,DENS,RADL,ABSL,UBUF,NWBUF)}
Stores the constants for the material {\tt IMATE} in the data structure
{\tt JMATE}.
\begin{DLtt}{MMMMMMMM}
\item[IMATE]      ({\tt INTEGER}) material number;
\item[CHNAMA]     ({\tt CHARACTER*20}) material name;
\item[A]           ({\tt REAL}) atomic weight;
\item[Z]           ({\tt REAL})  atomic number;
\item[DENS]        ({\tt REAL}) density in g cm$^{-3}$;
\item[RADL]        ({\tt REAL}) radiation length in cm;
\item[ABSL]        ({\tt REAL}) absorption length in cm. This parameter
is ignored by {\tt GEANT}, but it has been kept in the calling sequence for
backward compatibility;
\item[UBUF]       ({\tt REAL}) array of {\tt NWBUF} additional user parameters;
\item[NWBUF]      ({\tt INTEGER}) number of user words in {\tt UBUF}.
\end{DLtt}
\Shubr{GFMATE}{(IMATE,CHNAMA*,A*,Z*,DENS*,RADL*,ABSL*,UBUF*,NWBUF*)}
Retrieves the constants of
material {\tt IMATE} from the data structure {\tt JMATE}. The parameters
are the same as for \Rind{GSMATE}.
\Shubr{GPMATE}{(IMATE)}
Prints the material constants:
\begin{DLtt}{MMMMMMMM}
\item[IMATE]   {\tt (INTEGER)} material number. If {\tt IMATE=0} then all
materials will be printed. If {\tt IMATE<0} the material number {\tt -IMATE}
will be printed without the header describing the value of the various fields.
\end{DLtt}
