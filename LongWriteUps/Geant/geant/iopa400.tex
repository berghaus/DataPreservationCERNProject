%%%%%%%%%%%%%%%%%%%%%%%%%%%%%%%%%%%%%%%%%%%%%%%%%%%%%%%%%%%%%%%%%%%
%                                                                 %
%  GEANT manual in LaTeX form                              %
%                                                                 %
%  Michel Goossens (for translation into LaTeX)                   %
%  Version 1.00                                                   %
%  Last Mod. Jan 24 1991  1300   MG + IB                          %
%                                                                 %
%%%%%%%%%%%%%%%%%%%%%%%%%%%%%%%%%%%%%%%%%%%%%%%%%%%%%%%%%%%%%%%%%%%
\Origin{R.Brun}
\Submitted{20.08.87}        \Revised{17.12.93}
\Version{Geant 3.16}        \Routid{IOPA400}
\Makehead{ZEBRA direct access files handling}

I/O with direct access files is the most efficient way to use direct
access devices. All the routines described in this section and in
({\tt [IOPA500]}) are based on the {\tt ZEBRA-RZ} package and the
user is referred to the related documentation~\cite{bib-ZEBRA}
for an understanding of the basic principles.

\Shubr{GRFILE}{(LUN,CHFILE,CHOPT)}
\begin{DLtt}{MMMMMMMM}
\item[LUN] ({\tt INTEGER}) logical unit number;
\item[CHFILE] ({\tt CHARACTER*(*)}) file name;
\item[CHOPT] ({\tt CHARACTER*(*)}) any valid option for \Rind{RZOPEN},
\Rind{RZMAKE} and \Rind{RZFILE} and in particular:
\begin{DLtt}{MMMM}
\item[N] create a new file;
\item[U] open an existing file for update;
\item[Q] the initial allocation (default 1000 records)
is in {\tt IQUEST(10)};
\item[I] read all initialisation data structures (see description of 
\Rind{GRIN}/\Rind{GROUT}) from file to memory;
\item[O] write all initialisation data structures from memory to file;
\end{DLtt}
\end{DLtt}

This routine opens a file for direct access I/O of {\tt ZEBRA} data
structures. By default the file is opened in exchange mode and the option
{\tt W} is added, which creates an {\tt RZ} top level directory with the
name {\tt //LUNnn} where {\tt nn} is the logical unit number.

The default allocation of an {\tt RZ} directory is enough for 1000 records,
which allows for a 4Mb file with the standard record length of 1024 words. 
If a larger file is needed, the size of the directory should be changed 
when creating the file. This can be done by assigning to the variable
{\tt IQUEST(10)} in the common \FCind{/QUEST/} the requested number of records,
and calling \Rind{GRFILE} with the option {\tt Q} as shown in the following
example:
\begin{verbatim}
      COMMON / QUEST / IQUEST(100)
      .
      .
      .
      IQUEST(10) = 5000
      CALL GRFILE(1,'mygeom.geom','Q')
\end{verbatim}

\Shubr{GRMDIR}{(CHDIR,CHOPT)}

\begin{DLtt}{MMMMMMMM}
\item[CHDIR] ({\tt CHARACTER*(*)}) subdirectory name to create;
\item[CHOPT] option string:
\begin{DLtt}{MMMM}
\item[' '] create a subdirectory;
\item[S] create a subdirectory and set the current directory to it;
\end{DLtt}
\end{DLtt}
This routine creates an {\tt RZ} subdirectory.

\Shubr{GREND}{(LUN)}
\begin{DLtt}{MMMMMMMM}
\item[LUN] ({\tt INTEGER}) logical unit number (see \Rind{RZEND} in
the {\tt ZEBRA} manual~\cite{bib-ZEBRA});
\end{DLtt}
 
Routine to close an {\tt RZ} file. It is very important to call this
routine for all {\tt RZ} files open in write mode. Failure to do may
result in the files being incompletely written or corrupted.
