%%%%%%%%%%%%%%%%%%%%%%%%%%%%%%%%%%%%%%%%%%%%%%%%%%%%%%%%%%%%%%%%%%%
%                                                                 %
%  GEANT manual in LaTeX form                              %
%                                                                 %
%  Michel Goossens (for translation into LaTeX)                   %
%  Version 1.00                                                   %
%  Last Mod. Jan 24 1991  1300   MG + IB                          %
%                                                                 %
%%%%%%%%%%%%%%%%%%%%%%%%%%%%%%%%%%%%%%%%%%%%%%%%%%%%%%%%%%%%%%%%%%%
\Authors{R.Brun,P.Zanarini}    \Origin{Same}
\Submitted{15.06.84}            \Revised{22.04.86}
\Version{GEANT3.10}             \Routid{XINT 110}
\Makehead{Drawing Commands}
\Ssubr{DRAW  name [theta phi psi u0 v0 su sv] }
{\tt CALL GDRAW}(name,theta,phi,psi,u0,v0,su,sv)
If optional parameters are missing, the current values in /GCDRAW/
are taken
\Ssubr{DVOLUME  n lnames lnumbs nrs [theta phi psi u0 v0 su sv]}
{\tt CALL GDRVOL}(n,lnames,lnumbs,nrs,theta,phi,psi,u0,v0,su,sv)
N is number of levels from the top of the geometry structure
until the volume lnames(n),lnumbs(n) to be drawn
{\tt LNAMES} and {\tt LNUMBS} are arrays containing the names and numbers
identifying the path
{\tt NRS} is the reference system used: {\tt NRS=0} for {\tt MARS} or
{\tt NRS<>0} for {\tt DRS}
If optional parameters are missing, the current values in\Cind{ /GCDRAW/}
are taken
\Ssubr{DCUT name iaxis cutval [u0 v0 su sv]}
{\tt CALL GDRAWC}(name,iaxis,cutval,u0,v0,su,sv)
the cut plane is normal to iaxis (1,2,3)
and placed at the distance cutval from the origin.
The resulting picture is seen from the the same axis.
If optional parameters are missing, the current values in \Cind{/GCDRAW/}
are taken
\Ssubr{DXCUT name cutthe cutphi cutval [theta phi u0 v0 su sv]}
{\tt CALL GDRAWX}(name,cutthe,cutphi,cutval,theta,phi,u0,v0,su,sv)
the cut plane is normal to the line given by the
cut angles cutthe,cutphi and placed at the distance cutval from the
origin.
The resulting picture is seen from the viewing angles theta,phi.
If optional parameters are missing, the current values in \Cind{/GCDRAW/}
are taken
\Ssubr{DTREE [name levmax isel]}
{\tt CALL GDTREE}(name,levmax,isel)
isel is used to select options in the picture of the tree:
\begin{verbatim}
if 0, then we draw only node name
if xxxxx1, then we add multiplicity
if xxxx1x, then we add 'ONLY' information
if xxx1xx, then we add 'DET ' information
if xx1xxx, then we add 'SEEN' information
if x1xxxx, then we add a little picture of the volume, above each node
if 1xxxxx, then we add the graphics cursor picking capability :
          after the cursor has been moved press ...
 
          ' ' (space bar) to redraw the tree from the pointed node
          '2' to set the SEEN attribute  2
          '-' to set the SEEN attribute -1
          '0' to set the SEEN attribute  0
          '1' to set the SEEN attribute  1
\end{verbatim}
Defaults are : {\tt name=<global detector name> levmax=0 isel=111}
\Ssubr{DSPEC name }
{\tt CALL GDSPEC (name)}
\Ssubr{DFSPC name [isort inter]}
{\tt CALL GDFSPC (name,isort,inter)}
 
If {\tt isort=1} all the pictures will be drawn in
ascending alphabetic order.
If {\tt inter=1} the routine will prompt the user at each plot
before doing a clear screen, otherwise it will
clear automatically the screen before starting a new frame.
 
Defaults are : {\tt isort=0 inter=1}
\Ssubr{DTEXT  x0 y0 text size angle lwidth iopt}
{\tt CALL GDRAWT (x0,y0,text,size,angle,lwidth,iopt)}
\Ssubr{DVECTOR x vect y vect npoint}
{\tt CALL GDRAWV (x\_vect,y\_ vect,npoint)}
 
where {\tt x\_vect} and {\tt y\_vect} are 2 {\tt ZCEDEX} vectors
\Ssubr{DSCALE   u  v }
{\tt CALL GDSCAL (u,v)}
\Ssubr{DAXIS  x0 y0 z0 dx}
{\tt CALL GDAXIS (x0,y0,z0,dx)}
\Ssubr{DMAN  u v }
{\tt CALL GDMAN}(u,v)
\Ssubr{DHEAD  [isel name chrsiz]}
{\tt CALL GDHEA (isel,name,chrsiz)}
 
{\tt isel} is an option to be selected for the title name :
\begin{DL}{MMMMMM}
\item[isel=0] to have only the header lines
\item[isel=xxxxx1] to add the text name centered on top of header
\item[isel=xxxx1x] to add global detector name (first volume) on left
\item[isel=xxx1xx] to add date on right
\item[isel=xx1xxx] to select thick characters for text in top of header
\item[isel=x1xxxx] to add the text {\tt EVENT NR x} on top of header
\item[isel=1xxxxx] to add the text {\tt RUN NR x} on top of header
\end{DL}
Note that {\tt isel=x1xxx1} or {\tt isel=1xxxx1} are illegal choices,
i.e. they generate overwritten text.
name is the title (string terminated by a dollar sign)
and chrsiz the character size in cm of text name
Defaults are : {\tt isel=111110 name='\$' chrsiz=0.6}
\Ssubr{MEASURE }
Position the cursor on first point {\tt(u1,v1)} and press the space bar.
Position the cursor on second point {\tt(u2,v2)} and press the space bar.
The command will compute the distance in space separating
the 2 points on the projection view.
\Ssubr{ZOOM  [zfu zfv isel uz0 vz0 u0 v0] }
{\tt CALL GDZOOM (zfu,zfv,uz0,vz0,u0,v0)}
This command sets the zoom parameters that will be used by
next calls to the drawing routines. Each zoom operation is always
relative to the status of the current zoom parameters.
The scale factors in {\tt u,v} are respectively  {\tt zfu,zfv}.
{\tt zfu=0} (or {\tt zfv=0}) will act as a reset (i.e. unzoomed viewing).
The zoom is computed around
{\tt uz0,vz0} (user coordinates), and the
resulting picture will be centered at {\tt u0,v0}.
\begin{verbatim}
 If isel=0 :
    1. position the cursor at (uz0,vz0)
    2. press the space bar
 
      (u0,v0 are chosen at centre of screen)
 If isel=1 :
    1. position the cursor at first corner of zoom rectangle
    2. press the space bar
    3. position the cursor at second corner of zoom rectangle
    4. press the space bar
 
      (zfu,zfv are chosen according to zoom rectangle;
       uz0,vz0 are chosen at centre of zoom rectangle;
       u0,v0 are chosen at centre of screen)
 
 If isel=2 :
    1. position the cursor at (uz0,vz0)
    2. press the space bar
    3. position the cursor at (u0,v0)
    4. press the space bar
Defaults are : zfu=2.  zfv=zfu
               uz0,vz0 = <centre of screen>
               u0,v0   = <centre of screen>
\end{verbatim}
\Ssubr{DXYZ  [itra] }
{\tt CALL GDXYZ (itra)}
\Ssubr{KXYZ  [epsilo]}
{\tt CALL GKXYZ (epsilo)}
 
Defaults are : {\tt epsilo=0.25}
 
The picking of track points requires the JXYZ data structure
and is  repeated until the character typed is 'Q' or 'q'.
{\tt EPSILO} is the delta angle used for pick; if {\tt EPSILO=0}
there is no optimization performed and
over all the track points the one nearest to the pick
point is taken.
\Ssubr{DPART  [itra isel size]}
{\tt CALL GDPART (itra,isel,size)}
 
{\tt isel=x1} to draw the track number\\
{\tt isel=1x} to draw the particle name
 
Defaults are : {\tt itra=0  isel=11  size=0.25}
 
\Ssubr{DHITS  [iuset iudet itra isymb ssymb]}
{\tt CALL GDHITS (iuset,iudet,itra,isymb,ssymb)}
The character plotted at each hit point may be chosen by {\tt isymb}:
\begin{center}\begin{tabular} {lcr}
-1              &(small) hardware points            &(fast)\\
0               &software crosses                   &(default)\\
840,850         &empty/full circles                 &(slow) \\
841,851         &empty/full squares                 &(slow)\\
842,852         &empty/full triangles (up)          &(slow)\\
843,853         &empty diamond/full triangle (down) &(slow)\\
844,854         &empty/full stars                   &(slow)\\
\end{tabular} \end{center}
 
Except for {\tt isymb = 1} the size of the character on the screen can be
chosen by
 
{\tt ssymb cm (default=0.1)}.
 
\Ssubr{KHITS  [iuset iudet epsilo]}
{\tt CALL GKHITS (iuset,iudet,epsilo)}
 
Defaults are : {\tt iuset=0 iudet=0 epsilo=0.1}
 
The picking of hit points requires the appropriate {\tt JSET} data structure
and is  repeated until the character typed is 'Q' or 'q'.
If the character typed to pick is 'K' or 'k' then the
kinematics of the corresponding track is also printed.
The search is done over all the hits of all tracks in
detector {\tt IUDET} of set {\tt IUSET}.
{\tt EPSILO} is the pick aperture; if {\tt EPSILO<0} its absolute value is taken
and in addition the pick aperture is drawn; if {\tt EPSILO=0 }
there is an infinite pick aperture and
over all the hits the one nearest to the pick point is taken.
\Ssubr{DCHIT  [iuset iudet itra isymb sizmax ihit hitmin hitmax] }
{\tt CALL GDCHIT (iuset,iudet,itra,isymb,sizmax,ihit,hitmin,hitmax)}
 
The character plotted at each hit point may be chosen by {\tt isymb}:
\begin{center}\begin{tabular}{lcr}
   -1    &   (small) hardware points            &   (fast) \\
    0    &   software crosses                   &   (default)\\
 840,850 &   empty/full circles                 &   (slow) \\
 841,851 &   empty/full squares                 &   (slow)  \\
 842,852 &   empty/full triangles               &   (up)  (slow)\\
 843,853 &   empty diamond/full triangle  (down)&   (slow)\\
 844,854 &   empty/full stars                   &   (slow) \\
\end{tabular}  \end{center}
 
Except for {\tt isymb = 1} the {\tt SIZE} of the character on the screen
is a function of
{\tt HITS(IHIT)}, with {\tt HITMIN} and {\tt HITMAX} defining the range.
The maximum character size (used in overflow) is {\tt SIZMAX}.
 
{\tt SIZE = sizmax * ( hits(ihit) - hitmin ) / hitmax }
 
Defaults are:
 
{\tt iuset=0 iudet=0 itra=0 isymb=0 sizmax=1 ihit=4 hitmin=0 hitmax=0   }
\Ssubr{DUVIEW  name type cpxtyp [iview] }
{\tt CALL GUVIEW (name,type,cpxtyp,iview)}
 
