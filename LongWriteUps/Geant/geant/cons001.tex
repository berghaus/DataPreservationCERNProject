%%%%%%%%%%%%%%%%%%%%%%%%%%%%%%%%%%%%%%%%%%%%%%%%%%%%%%%%%%%%%%%%%%%
%                                                                 %
%  GEANT manual in LaTeX form                                     %
%                                                                 %
%  Michel Goossens (for translation into LaTeX)                   %
%  Version 1.00                                                   %
%  Last Mod. Jan 24 1991  1300   MG + IB                          %
%                                                                 %
%%%%%%%%%%%%%%%%%%%%%%%%%%%%%%%%%%%%%%%%%%%%%%%%%%%%%%%%%%%%%%%%%%%
\Documentation{F.Bruyant}   
\Version{Geant 3.16}\Routid{CONS001}
\Submitted{01.10.84}    \Revised{17.12.93}
\Makehead{Introduction to the section CONS}
\section{Introduction}
 
The setup where the particles are transported is represented by a 
structure of geometrical volumes. Each volume is
filled with matter (which can have the properties of the vacuum in case
of it contains no matter).

The matter composing the volumes is described via
two sets of attributes. The first set is relative to the nature
of the {\it material} composing the volume, and contains information
such as the atomic number, the atomic weight, the density and so on
(see the description of the routine \Rind{GSMATE} in this section for
more information). 

The second set of attributes is relevant to the
process of particle transport 
and they define a so-called {\it tracking medium}. These are
parameters such as the material composing the volume,
the magnetic field, the required tracking precision,
the maximum energy that can be lost in one step and so on (see the
description of the routine \Rind{GSTMED} in this section for more
information).

Each tracking medium refers to a material via a {\it material
number} which is assigned by the user. Different tracking media can, with
some limitation, refer to the same material.

Each volume is filled by a tracking medium identified by 
a {\it medium number}.
Different volumes may have the same medium number (see {\tt [GEOM]}).
 
The transport of particles through a
setup ({\tt [TRAK]}) requires access to
data which describe:
\begin{itemize}
\item   the geometry of the setup;
\item   the {\it material} and {\it tracking medium} parameters;
\item   the particle properties.
\end{itemize}
The section {\tt [CONS]} contains the routines
for the storage and retrieval of information related to materials,
tracking media and the particles.

{\bf Important note:} many entities in {\tt GEANT} are user-defined
via a subroutine call. One of the arguments of this subroutine is 
a integer number which identifies the entity. Examples are materials,
tracking media, particles and so on. It can be tempting, for booking
purposes, to use very large numbers. For instance, in a large detector,
the number of all the materials in the hadronic calorimeter could be
$1001 \leq I \leq 2000$. Even if these conventions are very handy, they
can introduce a performance penalty.

For reasons of speed, the number provided by the user is used directly
as the number of the link in the {\tt ZEBRA} data structure indicating
where to store the pointer to
the bank containing the information on the entity. {\tt ZEBRA} 
pointers are contiguous. Defining an object with a user number of $1000$
forces {\tt ZEBRA} to allow space for $1000$ links. This entails a loss
of space ($999$ words), but much worse than that, induces {\tt ZEBRA} to
believe that there are in fact $999$ more banks. At every operation which 
causes a {\it relocation} of banks, {\tt ZEBRA} will check all the possible 
links, which can be very time consuming.

So values of user-defined entities must be kept as small as possible and
contiguous. In large applications one could write a routine which returns
the next free number to be allocated, which can then be stored in a variable
and always referenced symbolically, freeing the user from the need to define
ranges. As an example we give here a function performing this operation
for the material number:
\begin{verbatim}
      FUNCTION NEXTMA()
+SEQ, GCBANK
      IF(JMATE.LE.0) THEN
         NEXTMA = 1
      ELSE
         DO 10 IMAT = 1, IQ(JMATE-2)
            IF(LQ(JMATE-IMAT).EQ.0) GOTO 20
  10     CONTINUE
  20     NEXTMA = IMAT
      ENDIF
      END
\end{verbatim}

\section{Materials}
The material constants are stored in the data structure {\tt JMATE} via the 
routine \Rind{GSMATE} and can be retrieved via the routine \Rind{GFMATE} and 
printed via the routine \Rind{GPMATE}. The routine \Rind{GMATE} calls
repeatedly the routine \Rind{GSMATE} to define a standard set of materials, 
but its use is not mandatory. Mixtures of basic materials, compounds or
molecules with atoms
from different basic materials, may also be defined and their characteristics
are stored in the structure {\tt JMATE}, through calls to the routine
\Rind{GSMIXT}. Mixtures of compounds are allowed.

Quantities such as energy loss and cross-section tables (see {\tt [PHYS]}),
computed during the initialisation of the program are stored in the 
structure {\tt JMATE}. These can be accessed through the routine \Rind{GFTMAT}
and plotted or printed through the routines \Rind{GPLMAT} and \Rind{GPRMAT} 
respectively.

\section{Tracking media}
 
The tracking medium
parameters are stored in the data
structure {\tt JTMED} by the routine \Rind{GSTMED}. Details
about these parameters are given
in {\tt [TRAK]}. They can be accessed with the routine
\Rind{GFTMED} and printed with the routine \Rind{GPTMED}.

The correctness of the transport process and thus the reliability
of the results obtained with {\tt GEANT} depend critically on the
values of the {\it tracking medium parameters}. To help the user,
most of the parameters are recalculated by default by {\tt GEANT}
irrespective of the value assigned by the user. Advanced users
can control the value of the parameters thus bypassing the automatic
computation. See the description of the routine \Rind{GSTMED} for
more information.

The tracking thresholds, and other parameters and flags that control
the physics processes, defined in \Rind{GINIT} and possibly
modified through the relevant data records, are also stored in the structure
{\tt JTMED}. These parameters can be redefined for each tracking medium
via the routine \Rind{GSTPAR}.

\section{Particles}
The parameters of the particles transported by {\tt GEANT}
are stored in the data structure
{\tt JPART} via the routine \Rind{GSPART}. 
The decay properties of particles are defined via the routine \Rind{GSDK}.
The particle constants can be
accessed with the routine \Rind{GFPART} and printed with the routine
\Rind{GPPART}.

The routine \Rind{GPART} defines the standard table of particles,
branching ratios and decay modes. This routine needs to be called for 
the {\tt GEANT} system to work properly. 
\Rind{GPART} calls the routine \Rind{GSPART} for the
particle properties and \Rind{GSDK} for the decays for
each particle in the standard {\tt GEANT} table. 
The user may call \Rind{GSPART} and
\Rind{GSDK} to extend or modify the table defined by \Rind {GPART}.
