%%%%%%%%%%%%%%%%%%%%%%%%%%%%%%%%%%%%%%%%%%%%%%%%%%%%%%%%%%%%%%%%%%%
%                                                                 %
%  GEANT manual in LaTeX form                                     %
%                                                                 %
%  Michel Goossens (for translation into LaTeX)                   %
%  Version 1.00                                                   %
%  Last Mod. Jan 24 1991  1300   MG + IB                          %
%                                                                 %
%%%%%%%%%%%%%%%%%%%%%%%%%%%%%%%%%%%%%%%%%%%%%%%%%%%%%%%%%%%%%%%%%%%
\Origin{T.Gabriel, C.Zeitnitz, K.Lassila-Perini}
\Documentation{K.Lassila-Perini}
\Submitted{17.12.93} \Revised{17.12.93}
\Version{Geant 3.16}\Routid{PHYS530}
\Makehead{The GEANT/MICAP interface}
\section{Subroutines}

\Shubr{GMORIN}{}
\Rind{GMORIN} initialises the {\tt MICAP} variables and
reads the cross-section file for the materials that have been
defined. It is called from {\tt GFMDIS} when a neutron with
kinetic energy below 20 MeV enters there first time.


\Shubr{GFMDIS}{}
\Rind{GFMDIS} computes the distance to the next interaction
points. It calls the {\tt FLUKA} routines to compute the
cross-sections for all particles except neutrons with
kinetic energy below 20 MeV for which {\tt MICAP}
function \Rind{SIGMOR} is called.
\Rind{GFMDIS} is called from the user routine {\tt GUPHAD}
where the hadronic package can be chosen.
The only difference between \Rind{GFLDIS} (see {\tt [PHYS520]})
and \Rind{GFMDIS}
is that for the former, {\tt GHEISHA} hadronic package
is called for the neutrons below 20 MeV, and for the latter,
the low-energy neutrons are handled by {\tt MICAP}.

\Shubr{GFMFIN}{}
\Rind{GFMFIN} calls the {\tt FLUKA} routines to
generate the hadronic interaction. For the neutrons
with kinetic energy below 20 MeV \Rind{GMICAP}
is called. 
The only difference between \Rind{GFLFIN} (see {\tt [PHYS520]})
and \Rind{GFMFIN}
is that for the former, {\tt GHEISHA} hadronic package
is called for the neutrons below 20 MeV, and for the latter,
the low-energy neutrons are handled by {\tt MICAP}.
\Rind{GFMFIN} is called from the user routine {\tt GUHADR}.

\Shubr{GMICAP}{}
\Rind{GMICAP} calls the {\tt MICAP} routines to
handle the low-energy interaction of neutrons.
It writes the eventual secondaries to the {\tt GEANT}
stack. \Rind{GMICAP} is called from \Rind{GFMFIN}.

\section{Method}
{\tt MICAP} (A Monte Carlo Ionization Chamber Analysis Package)
\cite{bib-MICA}, \cite{bib-MIC2} is a Monte Carlo
system to analyze ionisation chamber responses.
As a standalone program it contains the code for
formatting the cross-section files, neutron and photon
transport, the geometry definitions and the code for the
chamber response. In {\tt GEANT}, only the sampling
of the neutron interactions from the already prepared
cross-section file is included.
The interface between {\tt GEANT} and {\tt MICAP} has been extracted from
{\tt GCALOR} package\cite{bib-ZEIT} by C.Zeitnitz and T.
Gabriel. 

When using {\tt GEANT-MICAP} interface the low-energy
neutrons are handled in {\tt MICAP} routines. 
Other hadrons and high-energy neutrons are
passed to {\tt FLUKA} interaction routines.
For low-energy neutrons,
the total cross-section is given by {\tt MICAP}
and if the neutron interaction is chosen by {\tt GEANT}
tracking routine,
{\tt GMICAP} reads the cross-section for neutron interactions
processes,
samples and generates the interaction and the returns the
secondary particles (nucleons, heavy fragments, or
photons) to {\tt GEANT}. Information on the recoil
nucleus (atomic number {\tt AMED}, charge {\tt ZMED} and
kinetic energy {\tt ERMED}) can be found in {\tt MCRECO}
common block. The program flow is shown in figures
\ref{fmufin} and \ref{fmdist}.

\begin{figure}
\normalsize{
\begin{picture}(400,300)

\put(140,290){\framebox(70,20)}
\put(175,300){\makebox(0,0){GTNEUT}}
\put(175,290){\vector(0,-1){20}}

\put(140,250){\framebox(70,20)}
\put(175,260){\makebox(0,0){GUPHAD}}
\put(175,250){\vector(0,-1){30}}

\put(0,80){\dashbox{3}(350,150)[tl]{\small{Interface routines}}}

\put(140,200){\framebox(70,20)}
\put(175,210){\makebox(0,0){GFMDIS}}
\put(175,200){\line(0,-1){20}}
\put(80,185){$n$, $E_{kin} <$ 20 MeV}
\put(80,180){\line(1,0){190}}
\put(80,180){\line(0,-1){20}}
\put(270,180){\line(0,-1){20}}

\put(35,165){\small{Only once}}
\put(35,160){\line(1,0){90}}
\put(35,160){\vector(0,-1){100}}
\put(0,20){\framebox(70,40)}
\put(35,50){\makebox(0,0){GMORIN}}
\put(35,40){\makebox(0,0){\small{initialisation}}}
\put(35,30){\makebox(0,0){\small{of MICAP}}}

\put(125,160){\vector(0,-1){100}}
\put(90,20){\framebox(70,40)}
\put(125,50){\makebox(0,0){SIGMOR}}
\put(125,40){\makebox(0,0){\small{cross sections}}}
\put(125,30){\makebox(0,0){\small{from MICAP}}}

\put(225,165){\small{Only once}}
\put(225,160){\line(1,0){90}}
\put(225,160){\vector(0,-1){40}}
\put(190,100){\framebox(70,20)}
\put(225,110){\makebox(0,0){FLINIT}}
\put(225,100){\vector(0,-1){40}}
\put(190,20){\dashbox{3}(70,40)}
\put(225,50){\makebox(0,0){FLUKA}}
\put(225,40){\makebox(0,0){\small{initialisation}}}
\put(225,30){\makebox(0,0){\small{routines}}}

\put(315,160){\vector(0,-1){100}}
\put(280,20){\dashbox{3}(70,40)}
\put(315,50){\makebox(0,0){FLUKA}}
\put(315,40){\makebox(0,0){\small{cross sections}}}
\put(315,30){\makebox(0,0){\small{routines}}}

\end{picture}}
\parbox{\textwidth}{
\begin{minipage} [b]{\textwidth} {
\caption{\label{fmdist}Program flow for calculation of the distance to
the next interaction point}}
\end{minipage}}

\end{figure}

\begin{figure}
\normalsize{
\begin{picture}(400,250)

\put(155,220){\framebox(70,20)}
\put(190,230){\makebox(0,0){GTNEUT}}
\put(190,220){\vector(0,-1){20}}

\put(155,180){\framebox(70,20)}
\put(190,190){\makebox(0,0){GUHADR}}
\put(190,180){\vector(0,-1){20}}

\put(155,120){\framebox(70,40)}
\put(190,150){\makebox(0,0){GFMFIN}}
\put(190,140){\makebox(0,0){\small{Interface}}}
\put(190,130){\makebox(0,0){\small{routine}}}
\put(190,120){\line(0,-1){30}}

\put(105,90){\line(1,0){170}}
\put(105,95){$n$, $E_{kin} <$ 20 MeV}
\put(105,90){\vector(0,-1){30}}
\put( 65,20){\framebox(80,40)}
\put(105,50){\makebox(0,0){GMICAP}}
\put(105,40){\makebox(0,0){\small{low-energy neutron}}}
\put(105,30){\makebox(0,0){\small{interactions}}}

\put(275,90){\vector(0,-1){30}}
\put(240,20){\dashbox{3}(70,40)}
\put(275,50){\makebox(0,0){FLUKA}}
\put(275,40){\makebox(0,0){\small{hadronic}}}
\put(275,30){\makebox(0,0){\small{interactions}}}


\end{picture}}
\parbox{\textwidth}{
\begin{minipage} [b]{\textwidth} {
\vspace{.5cm}
\caption{\label{fmufin}Program flow for generating secondary particles}}
\end{minipage}}

\end{figure}

{\tt MICAP} uses pointwise cross-section data 
(as a difference to so called group
cross-sections where the data are averaged over certain
energy intervals). This method has the advantage that
the resonances are not smoothed by averaging the data.
The neutron cross-section are available for the following
isotopes:

\begin{center}
\begin{tabular}{l@{\hspace{2cm}}l@{\hspace{2cm}}l}
Hydrogen (bound) & Sodium     & Copper         \\
Hydrogen (free)  & Magnesium  & Molybdenum     \\
Lithium (5)      & Aluminium  & Barium         \\
Lithium (6)      & Silicon    & Tantalum       \\
Boron (10)       & Chlorine   & Tungsten       \\
Boron (11)       & Argon      & Lead           \\
Carbon           & Calcium    & Uranium (235)  \\ 
Nitrogen         & Chromium   & Uranium (238)  \\
Oxygen           & Iron       &                \\
Fluorine         & Nickel     &                \\
\end{tabular}
\end{center}

If the cross-sections are not found for some of the
defined materials, a warning is printed first at
the initialisation time telling which cross-section
are used (the closest Z available) instead. Then, an 
additional warning is printed each tracking step.
