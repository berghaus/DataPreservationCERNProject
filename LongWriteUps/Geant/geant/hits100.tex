%%%%%%%%%%%%%%%%%%%%%%%%%%%%%%%%%%%%%%%%%%%%%%%%%%%%%%%%%%%%%%%%%%%
%                                                                 %
%  GEANT manual in LaTeX form                              %
%                                                                 %
%  Michel Goossens (for translation into LaTeX)                   %
%  Version 1.00                                                   %
%  Last Mod. Jan 24 1991  1300   MG + IB                          %
%                                                                 %
%%%%%%%%%%%%%%%%%%%%%%%%%%%%%%%%%%%%%%%%%%%%%%%%%%%%%%%%%%%%%%%%%%%
\Origin{R.Brun,F.Bruyant}
\Submitted{01.11.83}                \Revised{17.12.93}
\Version{Geant 3.16}                \Routid{HITS100}
\Makehead{Sensitive DETector definition}
\Shubr{GSDET}{(CHSET,CHDET,NV,CHNMSV,NBITSV,IDTYP,NWHI,NWDI,ISET*,IDET*)}
 
\begin{DLtt}{MMMMMMMM}
\item[CHSET] ({\tt CHARACTER*4}) set identifier, user defined;
\item[CHDET] ({\tt CHARACTER*4}) detector identifier,
has to be the name of an existing volume;
\item[NV] ({\tt INTEGER}) number of volume descriptors;
\item[CHNMSV] ({\tt CHARACTER*4}) array of {\tt NV} volume descriptors;
\item[NBITSV] ({\tt INTEGER}) array of {\tt NV}, {\tt NBITSV(I)} 
({\tt I=1,...,NV}) is the number of bits in which to pack the copy 
number of volume {\tt CHNMSV(I)};
\item[IDTYP] ({\tt INTEGER}) detector type, user defined;
\item[NWHI] ({\tt INTEGER}) initial size of {\tt HITS} banks;
\item[NWDI] ({\tt INTEGER}) initial size of {\tt DIGI} banks;
\item[ISET] ({\tt INTEGER}) position of set in bank {\tt JSET};
\item[IDET] ({\tt INTEGER}) position of detector in bank {\tt JS=LQ(JSET-ISET)}.
\end{DLtt}

Assigns detector {\tt CHDET} to the set {\tt CHSET}
and defines its basic parameters.
 
{\bf Note:} The vector {\tt CHNMSV} (length {\tt NV)} contains the list of
volume names which permit unambiguous identification of all copies of
volume {\tt CHDET} [see example in {\tt [HITS110]}.
Each element of the vector {\tt NBITSV} (length {\tt NV}) is the number
of bits used for packing the number of the corresponding volume, when building
the packed identifier of a given physical detector.

For more details see the example given in {\tt [HITS110]}.
The detector type {\tt IDTYP} is not used internally by {\tt GEANT}
and can be used to distinguish quickly between various
kinds of detectors, in the routine \Rind{GUSTEP} for example.

\Shubr{GSDETV}{(CHSET,CHDET,IDTYP,NWHI,NWDI,ISET*,IDET*)}
The arguments of this routine are the same than the previous one,
but {\tt NAMES, NBITSV} will be computed by \Rind{GGDETV} called by
\Rind{GGCLOS}) (see {\tt [HITS001]}).

\Shubr{GFDET}{(CHSET,CHDET,NV*,CHNMSV*,NBITSV*,IDTYP*,NWHI*,NWDI*,ISET*,IDET*)}
Returns the parameters for detector {\tt CHDET} of set {\tt CHSET}, the
arguments have the same meaning than for routine \Rind{GSDET}.

\Shubr{GPSETS}{(CHSET,CHDET)}
Prints {\tt SET} and {\tt DET}ector parameters.
\begin{DLtt}{MMMMMMMM}
\item[CHSET] ({\tt CHARACTER*4}) set to be printed, if {\tt *} prints 
all detectors of all sets;
\item[CHDET] ({\tt CHARACTER*4}) detector to be printed, if {\tt *} prints
all detectors of set {\tt CHSET}.
\end{DLtt}
 
