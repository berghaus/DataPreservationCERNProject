%%%%%%%%%%%%%%%%%%%%%%%%%%%%%%%%%%%%%%%%%%%%%%%%%%%%%%%%%%%%%%%%%%%
%                                                                 %
%  GEANT manual in LaTeX form                              %
%                                                                 %
%  Michel Goossens (for translation into LaTeX)                   %
%  Version 1.00                                                   %
%  Last Mod. Jan 24 1991  1300   MG + IB                          %
%                                                                 %
%%%%%%%%%%%%%%%%%%%%%%%%%%%%%%%%%%%%%%%%%%%%%%%%%%%%%%%%%%%%%%%%%%%
 
 
 
 
\Origin{J.Vuoskoski, N. H\mbox{\o}imyr}
\Submitted{22.03.94}         \Revised{23.09.94}
\Version{Geant 3.21}         \Routid{GEOM910}
 
\Makehead{The CADINT Interface}
 
\section*{1 Introduction}
 
The engineering work in HEP for design,
analysis and manufacturing of detectors
requires Computer Aided Engineering (CAE) tools.
To ensure coherent design the same detector descriptions should
be used by physicists, engineers and, eventually, manufacturers. Efficient
exchange of data between CAE tools and
{\tt GEANT} becomes crucial.
 
 
The CADINT interface allows to export detector models from {\tt GEANT}
to Computer Aided Design (CAD) systems.
CADINT outputs detector models in the SET (Standard d'Echange
et de Transfert)~\cite{set} neutral file format.
A detector model is written as an assembly
of solid volumes, with respect to a global coordinate system. The geometric
representation used is
Constructed Solid Geometry (CSG)~\cite{mortenson}.
 
A {\tt GEANT} detector geometry exported through CADINT
into a CAD system can be used to
check the coherence of the model used in simulation.
The CADDFAS service at CERN can be used to convert SET files into IGES
(Initial Graphics Exchange Specification, an American
standard for exchange of CAD data)~\cite{iges} format, which is readable
by most CAD systems.
For more information about the CADDFAS service, send an e-mail to {\it
caddfas@cadd.cern.ch} with subject: {\it info}
 
 
 
 
 
\section*{2 SET File Format}
 
 
SET is a French standard
to exchange and archive CAD data.  It was developed as a neutral file format for
exchanging data between different CAD systems at Aerospatiale in 1983.
The aim was to develop
a more reliable alternative to IGES.  It became an official
French standard,
Afnor Z68-300, SET, in 1985.  SET was revised and extended in 1989.  The
version (89-06) used in CADINT, supports wireframe, surface and solid entities.
Entities for drafting and connectivity applications, as well as scientific
data and FEM (Finite Element Method) modelling are also included.
The philosophy behind the standard is that it should be able to
handle the complete information that may occur.
It is also considered to be important to have an unambiguously defined format,
that is compact in size,
and is flexible enough to handle future demands from the CAD/CAM industry.
 
 
The structure of SET is based upon a three level hierarchical structure,
which consists of data assemblies, data blocks, and data sub-blocks.
Information that is common to several blocks or assemblies, is stored in a
so-called dictionary.
 
\begin{itemize}
\item{\bf Assembly:} A SET data assembly is a collection of data defining
a certain piece of information, such as a mechanical part.
A SET file can contain one or more data assemblies;
 
\item{\bf Block:} In the nomenclature of the SET files,
the blocks are identified by an {\bf @},
followed by the number of the block identifying its type.
This number is followed by the block's reference number in the file.
(i.e. in a file with {\it n} blocks, the block sequence numbers will be from 0
 to {\it n}.)
After this, sub-blocks and dictionary entries used in the block follows.
A SET data block is an elementary entity, which consists of
definition or control data that are used in different applications.
Such entities could be geometric objects as points or lines, or other entities
like matrices, drawings and views or SET file identifiers;
 
 
\item{\bf Sub-block:} A sub-block consists of an identifier, and a list of data
that contributes to the description of the entity defined by the data block.
The different parameters inside a sub-block, such as coordinates, are
represented by their values. A sub-block has the identifier ``\#'', followed by
 its
type number, possible references to the dictionary, and parameters applicable in
the sub-block;
 
\item{\bf Dictionary:} The dictionary is a set of predefined parameters in the
specifications of the standard. They are accessible as dictionary entries which
are assigned an identifier (a colon ``{\bf :}'' and the dictionary number), and
 a
precise meaning given by authorised and/or default values.
 
\end{itemize}
 
References among entities in a SET file are made by using pointers,
either directly
from the blocks, or via sub-blocks or dictionary entries.  Pointers
to other blocks are identified by an ``{\bf !}'', followed by the sequence
numbers of the blocks.
SET has no unique mapping of  sub-block types and dictionary
entries into blocks. Several combinations are allowed for each different
block.  Definitions of possible combinations and guidelines for implementation
are given in the SET standard~\cite{set}.
 
 
 
\section*{3 Implementation}
 
 
In the CADINT interface,
a user can indicate a subtree of a detector by giving the name of a volume.
All the contents of this volume will be written into a SET file.
A user can also choose the contents by setting the daughter volumes
visible or invisible using the visibility attribute. Only the visible
volumes will be written into the SET file.
In order to avoid repeated volumes, a number of divisions can be chosen.
The rest of the divisions can be reproduced in a CAD system, if needed.
The colors of volumes defined by the {\tt GEANT} color attribute are
also transmitted into SET files.
 
 
The CADINT interface writes the geometry of a {\tt GEANT} detector model
in the SET file format in two basic phases:\\[-0.7cm]
 
\begin{enumerate}
\item write the volume information;\\[-0.7cm]
\item write the position information.\\[-0.5cm]
\end{enumerate}
 
The interface writes volumes into the SET format as
CSG solids i.e. the volumes are created using extrusions, revolutions,
ruled solids, rectangular parallelepipeds, etc.
Divisions are written as normal volumes.
Every division of a divided volume is
a distinct volume in a SET file. An index is attached to the end
of the name of each of
the divisions in order to distinguish each division instance.
The indexing of instances is reset in each division.
An index is attached to the end of the volume name of multiple copies
of normal volumes as well.
 
In SET format, a volume must be first written, after that it can
be positioned. All volumes are positioned with respect to
the global coordinate system. Thus, the tree structure
in the SET file is flat. There will be overlaps between volumes,
since no Boolean operations
are used to define the tree structure. However, the tree structure
is written into a separate file. The material information
is written into this same text file.
 
The interface  decodes the {\tt JVOLUM} data structure and
computes the necessary parameters if needed (negative parameters,
dimensions of divisions, etc.).
These computations are based on the {\tt GEANT} drawing routines.
The interface decodes a detector model starting from the global
mother volume. After that it decodes the first daughter volume in the
left side of the tree. The daughter volume can be a normal volume or a division.
It continues decoding daughter volumes until it
is at the bottom of the tree. After that the interface
returns one level and decodes the next daughter of the current volume
if any exists. In a case of a divided volume, the divisions are treated
in the same way as the daughter volumes.
The creation of the SET file is performed simultaneously.
 
 
\subsection*{3.1 Volume Information}
 
 
The tree decoding routine transfers the volume information
during decoding to the relevant
shape routine which writes the volumes into the SET file.
Every different shape has its own routine.
For example, the simplest shape, {\tt BOX}, is written
in the  SET format as follows:
 
\noindent{\tt@50,N1,:5,2\#60,X,Y,Z\\
@302,N2\#317,-DX,-DY,-DZ\\
@100,N3,:5,2,:9,'Name'\#101,!N1,!N2}\\
 
\newpage
 
\noindent Where\\[-0.7cm]
\begin{itemize}
\item {\bf{\tt @50}} is the block number which defines a primitive
 solid;\\[-0.7cm]
\item {\bf{\tt N1,N2,N3}} are the sequence numbers of each block;\\[-0.7cm]
\item {\bf{\tt :5,2}}  is the dictionary entry number which gives a total
subordination to one block;\\[-0.7cm]
\item {\bf{\tt \#60}} is the sub-block number which defines the geometric
parameters of a solid rectangular parallelepiped;\\[-0.7cm]
\item {\bf{\tt X,Y,Z}} are the dimensions of the rectangular
 parallelepiped;\\[-0.7cm]
\item {\bf{\tt @302}} is the block number for geometric
 transformation;\\[-0.7cm]
\item {\bf{\tt \#317}} is the sub-block number for translation;\\[-0.7cm]
\item {\bf{\tt DX,DY,DZ}} are the coefficients of the translation;\\[-0.7cm]
\item {\bf{\tt @100}} is the block number which defines a constructed
 solid;\\[-0.7cm]
\item {\bf{\tt :9}}  is the dictionary entry number for a name associated
with the block;\\[-0.7cm]
\item {\bf \tt Name} is the name of the volume;\\[-0.7cm]
\item {\bf{\tt \#101}} is the sub-block number for transformation
 operation.\\[-0.5cm]
\end{itemize}
 
The rectangular parallelepiped is first defined with the given parameters.
Then the translation of coordinates is defined. The translation is necessary
since {\tt GEANT} and SET use origins in different places. Finally,
the transformation operation is defined for a constructed solid which
in this case is the translation of the coordinates of a primitive solid.
The GEANT shapes are converted to SET as follows:
 
\begin{center}
 \begin{tabular}{|l|p{9.0 cm}|}
 \hline
 \bf SHAPE           &  \bf DEFINITION IN SET \\
 \hline
BOX  &   rectangular parallelepiped \\
TRD1 &   ruled solid between 2 faces   \\
TRD2 &   ruled solid between 2 faces \\
TRAP &   ruled solid between 2 faces \\
TUBE &   solid of revolution of a face \\
TUBS &   solid of revolution of a face \\
CONE &   solid of revolution of a face \\
CONS &   solid of revolution of a face \\
SPHE &   sphere \\
PARA &   ruled solid between 2 faces \\
PGON &   first segment ruled solid, other segments copied and rotated,
         finally boolean union for all the segments \\
PCON &   solid of revolution of a face \\
ELTU &   solid of linear extrusion of a face \\
HYPE &   not implemented \\
GTRA &   ruled solid between 2 faces \\
CTUB &   solid of revolution and boolean subtraction by 2 half-spaces \\
 \hline
 \end{tabular}
\end{center}
 
 
\subsection*{3.2 Position Information}
 
The position information of
a volume is handled by a routine,
which writes the coordinates and the rotation matrix. The tree decoding
routine transforms the coordinates and the rotation matrix of a volume 
to the global coordinate system, and then transfers the data
to the routine which writes the translation and rotation of a volume into
the SET format as follows:
 
\noindent{\tt @302,N1\#301,M1,M2,M3,M4,M5,M6,M7,M8,M9,XC,YC,ZC\\
@100,N2,:9,'Name'\#101,!N1(of the shape),!N1}\\
 
\newpage
 
\noindent Where\\[-0.7cm]
\begin{itemize}
\item {\bf{\tt @302}} is the block number for geometric
 transformation;\\[-0.7cm]
\item {\bf{\tt N1,N2}} are the sequence numbers of each block;\\[-0.7cm]
\item {\bf{\tt \#301}} is the sub-block number which defines the coefficients
of the rotation-translation matrix;\\[-0.7cm]
\item {\bf{\tt M1...M9}} are the  coefficients of the rotation matrix;\\[-0.7cm]
\item {\bf{\tt XC,YC,ZC}} are the components of the translation
 vector;\\[-0.7cm]
\item {\bf{\tt @100}} is the block number which defines a constructed
 solid;\\[-0.7cm]
\item {\bf{\tt :9}}  is the dictionary entry number for a name associated
with the block;\\[-0.7cm]
\item {\bf{\tt Name}} is the name of the volume;\\[-0.7cm]
\item {\bf{\tt \#101}} is the sub-block number for transformation
 operation;\\[-0.7cm]
\item {\bf{\tt !N1}} is the reference to sequence number of the mother
volume.\\[-0.5cm]
\end{itemize}
 
The geometric transformation is first defined with the given parameters
and then the transformation operation is defined for a constructed solid.
The principle of writing divisions into the SET format is
equivalent to the case of normal volumes.
 
\subsection*{3.3 Material and Tree Information}
 
The material file is written into a separate file simultaneously with the
SET file. It contains information on tracking medium, material and density for
each defined volume. The {\tt GEANT} tree is written into this same file. After
 the
volume name there follows the number of daughters and daughter names.
In a case of divided volume, the negative number of divisions is given after
the name of the divided volume, and followed by the offset, step
and the name of the division instance.
An output of a material file of a {\tt GEANT} example program:
 
\footnotesize{
\begin{verbatim}
 GEANT-SET MATERIAL LISTING FILE
 --------------------------------
 Materials in the geometry described in
 the .SET file:  gexam4.set
 
Volume name  Tracking media   Material         Density
 
CALO         1 AIR            1 AIR            0.12000001E-02
CAL1         1 AIR            1 AIR            0.12000001E-02
MOD1         1 AIR            1 AIR            0.12000001E-02
CAL2         1 AIR            1 AIR            0.12000001E-02
MOD2         1 AIR            1 AIR            0.12000001E-02
CAL3         1 AIR            1 AIR            0.12000001E-02
MOD3         1 AIR            1 AIR            0.12000001E-02
EPO1         4 CARBON         4 CARBON         0.22650001E+01
CHA1         6 BRASS          6 BRASS          0.85600004E+01
TUB1         6 BRASS          6 BRASS          0.85600004E+01
GAS1         5 GAS            5 ARG/ISOBU      0.21360000E-02
 
 GEANT TREE
 ----------
  The GEANT tree starting from the given volume
 
CALO    6  CAL1  CAL2  CAL3  EPO1  CHA1  EPO1
CAL1  -64  -0.48000000E+02   0.15000000E+01  MOD1
CAL2  -35  -0.43750000E+02   0.25000000E+01  MOD2
CAL3  -13  -0.24050001E+02   0.37000003E+01  MOD3
CHA1  -40  -0.25000000E+02   0.12500000E+01  TUB1
MOD1    6  SHIL  URPL  SHIL  EPO1  CHA1  EPO1
MOD2    4  SHIL  URPL  SHIL  CHA2
MOD3    2  COPL  CHA2
TUB1    1  GAS1
CHA2  -72  -0.23500000E+02   0.65277779E+00  TUB2
TUB2    1  GAS2
 
        ------ end of file -------
\end{verbatim}
}
\normalsize
 
\section*{4 Usage}
 
 
 
In order to carry out a successful transfer of a detector model,
a user needs to be
aware of certain limitations of the interface. The user is recommended
to communicate
with the engineers using CAD systems to establish a proper transfer methodology.
If a CAD system is not capable of receiving the whole detector geometry
in a single SET file, a user can transfer a detector model in smaller parts.
These parts can then be joined together in the CAD system.
Normally it is a good idea to just export one instance of each division
in order to limit the number of exported volumes.
The divisions can be reproduced in CAD systems.
It is impossible to give an exact size-limit for a detector model
which can be transferred in one file, because the limitation
depends on the structure of the detector model.
 
 
The {\tt GEANT} graphics package  provides excellent capabilities for
viewing parts of a detector model which is to be exported. Using
the visibility attribute and the number of division instances, a
user will be able to see what is to be exported. A user can use different
colors in order to help to distinguish different volumes (eg division
instances, etc.) in a CAD system.
Dummy volumes, such as mothers made of air or vacuum should be suppressed using
the visibility attribute, since the interface is currently not able to discard
such information.
A user must remember to specify the drawing parameters again if he
wants to draw the detector model after using the interface.
This is because the interface uses modified drawing routines.
 
A plot of the {\tt GEANT} tree can be useful for engineers to understand the
structure of the {\tt GEANT} detector model. {\tt GEANT}
volume names are useful references
regarding the material and tree information. The tree information is also
provided in the material file, but a graphical presentation is often useful.
 
 
 
 
 
\begin{thebibliography}{99}
 
 
 
 
\bibitem{set} French Standardization, Data exchange and transfer
standard specification, Afnor 1989.
 
 
\bibitem{mortenson} Michael E. Mortenson, Geometric Modeling, John
Wiley \& Sons, Inc., 1985.
 
 
 
\bibitem{iges} US department of Commerce, Initial Graphics Exchange
Specification, 1988.
 
 
 
\end{thebibliography}
 
 
 
 
 
 
