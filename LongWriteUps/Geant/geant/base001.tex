%%%%%%%%%%%%%%%%%%%%%%%%%%%%%%%%%%%%%%%%%%%%%%%%%%%%%%%%%%%%%%%%%%%
%                                                                 %
%  GEANT manual in LaTeX form                                     %
%                                                                 %
%  Michel Goossens (for translation into LaTeX)                   %
%  Version 1.00                                                   %
%  Last Mod. Jan 24 1991  1300   MG + IB                          %
%                                                                 %
%%%%%%%%%%%%%%%%%%%%%%%%%%%%%%%%%%%%%%%%%%%%%%%%%%%%%%%%%%%%%%%%%%%
\Documentation {F.Bruyant}       \Origin{GEANT}
\Submitted{01.10.84}       \Revised{08.11.93}
\Version{Geant 3.16}\Routid{BASE001}
\Makehead{Introduction to GEANT}
\section{{\tt GEANT} applications}
 
The {\tt GEANT} program simulates the passage of elementary particles through
the matter. Originally designed for the High Energy Physics experiments,
it has today found applications also outside this domain in areas such as
medical and biological sciences, radio-protection and astronautics.

The principal applications of {\tt GEANT} in High Energy Physics are:
\begin{itemize}
\item the {\it transport} 
of particles ({\it tracking} in this manual)
through an experimental setup for the simulation
of detector response;
\item
the graphical representation of the setup and of the particle trajectories.
\end{itemize}

The two functions are combined in the interactive version of {\tt GEANT}.
This is very useful,
since the direct
observation of what happens to a particle inside the detector
makes the debugging easier and may reveal possible weakness of the
setup (also sometimes of the program!).

In view of these applications, the {\tt GEANT} system allows you to:
\begin{itemize}
\item
describe an experimental setup
by a structure of geometrical volumes.
A {\tt MEDIUM} number 
is assigned to each volume
by the user ({\tt [GEOM]}). Different volumes may have
the same medium number. A medium is defined by
the so-called {\tt TRACKING MEDIUM} parameters, which
include reference to the {\tt MATERIAL}
filling the volume {\tt [CONS]};
\item
accept events simulated by Monte Carlo generators {\tt [KINE]};
\item transport particles through the various
regions of the setup, taking
into account geometrical volume boundaries and
physical effects according to the nature of the
particles themselves, their interactions
with matter and the magnetic field {\tt [TRAK], [PHYS]};
\item
record particle trajectories and the response of
the sensitive detectors {\tt [TRAK], [HITS]};
\item visualise the
detectors and the particle trajectories {\tt [DRAW], [XINT]}.
\end{itemize}

The program contains {\it dummy} and {\it default}
user subroutines called whenever application-dependent
actions are expected.

It is the responsibility of the user to:
\begin{itemize}
\item code the relevant user subroutines providing the data describing the
experimental environment;
\item assemble the appropriate program segments and utilities into an 
executable program;
\item compose the appropriate data records
which control the execution of the program. 
\end{itemize}
The
section {\tt [BASE]} of this manual gives more information on the above.

{\bf Note:}
the names of the dummy or default user subroutines have {\tt GU} or {\tt UG} 
as their first two letters.
 
\section{Event simulation framework}

The framework offered by {\tt GEANT} for event simulation is described in
the following paragraphs, in order
to familiarise the reader with the areas where user
interventions are expected.
For each item we will indicate in brackets the relevant section where more
information can be found.

At the same time, the {\tt GEANT} data structures are introduced. 
This is
important as the coding to be provided by the user
most often consists of storing and retrieving information from
data structures, or reading or writing data structures.
For simple applications
user routines are provided as an interface to the data structures
partially hiding them from the users. For advanced users of 
{\tt GEANT}, some idea of the layout of the data in memory is helpful.
{\tt GEANT} data structures are logically related set of data which are
physically stored in the \FCind{/GCBANK/} common block. The position
of each structure is contained in an {\tt INTEGER} variable which
is constantly kept up-to-date by {\tt ZEBRA}. By convention the names
of these variable, called {\it pointers} begin with {\tt J}, and they
are used in this manual to designate the structure they point to.

A main program has to be provided by the user ({\tt [BASE100]}) for
{\it batch} type operation. For interactive operation a main program is
provided, both binary and source, in the library directory
both at CERN and in the standard distribution tape of the CERN Program
Library. The file is called {\tt gxint<ver>.<ext>}, where {\tt <ver>}
is the version of {\tt GEANT} to which this file belongs and {\tt <ext>}
is the system-dependent file-name extension to denote a FORTRAN source
or an object file. This file should be loaded in front of all other
files when assembling a {\tt GEANT} application. The source is provided
in case the user wants to modify it, in particular changing the size
of the commons \FCind{/GCBANK/} or \FCind{/PAWC/}.

The main program allocates the dynamic memory for {\tt ZEBRA} and 
{\tt HBOOK} and passes control to the three phases of the run:
 
\begin{enumerate}
\item initialisation
\item event processing
\item termination
\end{enumerate}

where in each of the three phases the user can implement his own code
in the appropriate routines.
 
\section{Initialisation}

The initialisation is controlled by the user in the subroutine {\tt UGINIT}  
who has the responsibility to call the appropriate routines
({\tt [BASE100]}).
It consists of the following steps, most of them
performed through calls to {\tt GEANT} subroutines:
 
\begin{DLtt}{MMMMMMMMMM}
\item[\Rind{GINIT}]
initialise the {\tt GEANT} common blocks
with default values ({\tt [BASE030], [BASE110]});
\item[\Rind{GFFGO}]
read {\it free format} data records either to
modify the default options ({\tt [BASE040],
[BASE110]}) or to provide information on the current run;
\item[\Rind{GZINIT}]
initialise the memory manager, the link areas
and the run header bank {\tt JRUNG} ({\tt [BASE110]});
\item[\Rind{GDINIT}]
initialise the drawing package ({\tt [DRAW]}), calling this routine
without having initialised the graphics package via a call to \Rind{IGINIT}
or \Rind{HPLINT} will cause {\tt GEANT} to abort;
\item[\Rind{GPART}]
(or \Rind{GSPART}) fill the data structure {\tt JPART}
with the particle properties ({\tt [CONS]});
\item[\Rind{GMATE}] (or \Rind{GSMATE})
fill the data structure {\tt JMATE}
with the characteristics of the materials used
({\tt [CONS]});
\item[User code] it is the responsibility of the user to:
\begin{itemize}
\item define the geometry of the different components of
the setup ({\tt [GEOM]}), stored in the {\tt JROTM} and {\tt JVOLUM} data
structures; 
\item define tracking medium parameters ({\tt [CONS],[TRAK]}), stored
in the {\tt JTMED} data structure;
\item specify which elements of the geometrical
setup should be considered as {\it sensitive detectors}, giving
a {\it response} when hit by a particle ({\tt [HITS]});
\item usually all done in a user routine called \Rind{UGEOM};
\end{itemize}
\item[\Rind{GGCLOS}]
process all the geometrical information
provided by the user and prepare for particle transport;
\item[\Rind{GBHSTA}]
book standard {\tt GEANT} histograms if required by
the user with the data record {\tt HSTA} ({\tt [BASE040], [BASE110]});
\item[\Rind{GPHYSI}]
compute energy loss and cross-section tables
and store them in the data structure {\tt JMATE} ({\tt [CONS], [PHYS]}).
\end{DLtt}
 
\section{Event processing}
 
The processing phase is triggered by a call to the subroutine \Rind{GRUN} which,
for each event to be processed, gives control to the subroutines:
 
\begin{DLtt}{MMMMMMMM}
\item[\Rind{GTRIGI}] initialise event processing and create the
event header bank {\tt JHEAD};
\item[\Rind{GTRIG}] process one event;
\item[\Rind{GTRIGC}] clean up the portion of memory used by the event
and check that enough time is left for the next event ({\tt [BASE200]}).
\end{DLtt}
\Rind{GTRIG} calls the following user routines:
\begin{DLtt}{MMMMMMMM}
\item[\Rind{GUKINE}]
generates or reads ({\tt [IOPA]})
the kinematics of the event and stores it
in the data structures {\tt JVERTX} and {\tt JKINE} ({\tt [KINE]});

\item[\Rind{GUTREV}]
calling \Rind{GTREVE} which performs the following operations for each
vertex in turn:
\begin{enumerate}
\item moves all the particles attached to the vertex from the permanent 
stack {\tt JKINE} to the temporary stack {\tt JSTAK};
\item controls the propagation of each particle though the setup by 
calling \Rind{GUTRAK}/\Rind{GTRACK} ({\tt [TRAK]}); 
each particle is tracked in turn and when a sensitive detector
is traversed, the user may store
any useful information in the data structure {\tt JHITS} 
via the routines described in the section {\tt [HITS]};
\end{enumerate}
 
The {\tt JSTAK} data structure is a LIFO (Last In -- First Out) stack.
Secondary products generated by the current particle transported
are processed before proceeding to the next particle. 
It is very important to understand that by default {\tt GEANT} does not
follow the secondary particles generated. It is the responsibility of the
user to indicate which particles should be followed via the
routines \Rind{GSKING}/\Rind{GSKPHO}.

The data structure {\tt JXYZ}, containing the
coordinates of space points along the tracks,
can be filled by the user during tracking ({\tt [TRAK]}).

\item[\Rind{GUDIGI}]
simulate the detector responses for the event, using 
the information recorded in the data
structure {\tt JHITS} during particle transport, and store
the results in the data structure {\tt JDIGI} ({\tt [HITS]});

\item[\Rind{GUOUT}]
perform all the processing at the end of the event and 
output the relevant data structures ({\tt [IOPA]}).
\end{DLtt}
 
Other routines called during the tracking phase triggered
by {\tt GTREVE} should be mentioned for completeness:
\begin{itemize}
\item
hadronic interactions can be simulated via either the 
{\tt GHEISHA}~\cite{bib-GHEI} or 
{\tt FLUKA}~\cite{bib-FLUK,bib-FLU1,bib-FLU2,bib-FLU3,bib-FLU4}
hadronic shower generator.
In the
subroutines \Rind{GUPHAD} and \Rind{GUHADR} ({\tt [TRAK]}) the user
may select the hadronic shower generation program to be used.
The default for {\tt GEANT} is {\tt GHEISHA};
\item after each tracking step along the
track, control is
given to the subroutine \Rind{GUSTEP}. From the information
available in labelled common blocks the user
is able to take the appropriate action, such as
storing a hit or transferring a secondary product either in the
stack {\tt JSTAK} or in the event structure {\tt JVERTX/JKINE} via
the subroutine 
\Rind{GSKING}. In the subroutine \Rind{GSSTAK},
called by \Rind{GSKING}, a user routine \Rind{GUSKIP} is called which permits
skipping any unwanted track before entering it in the stack for subsequent
transport;
\item the subroutine
\Rind{GUSWIM} is called by the the routines which transport charged particles
when in a magnetic field; it selects and calls the appropriate routine to
transport the particle. Although formally a user routine, the default version
provided by {\tt GEANT} is usually appropriate for most situations.
The magnetic field, unless it is constant along the {\tt Z} axis,
has to be described via the subroutine \Rind{GUFLD}.
\end{itemize}
 
\section{Termination}
 
The termination phase is under the control of the user ({\tt [BASE300]})
via the routine \Rind{GULAST}. In 
simple cases it may consist of a call to the subroutine \Rind{GLAST} 
which computes and prints some statistical information (time per
event, use of dynamic memory, etc.).
