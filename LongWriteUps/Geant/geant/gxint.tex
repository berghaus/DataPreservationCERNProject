\documentstyle{book}
\oddsidemargin 0.54cm
\evensidemargin 0.0cm
\topmargin -50pt
\textheight 22.5cm \textwidth 15.cm

\begin{document}
\pagestyle{plain}
\rm
\Large
{\bf The Interactive version}
\\[2em]
\large
\rm
Instead of writing a MAIN program to initialize the Geant batch
version, it is possible to use the MAIN interactive program (GXINT321.F)
provided in /cern/pro/lib exactly as the library LIBGEANT321.A. The user
has simply to insert in his link file the file GXINT321.F as the main
program, followed by the user object code and by the library LIBGEANT321.A.
\\[.5em] Example: \\[.1em]
       PROGRAM GXINT \\[.1em]
* \\[.1em]
*     GEANT main program. To link with the MOTIF user interface \\[.1em]
*     the routine GPAWPP(NWGEAN,NWPAW) should be called, whereas \\[.1em]
*     the routine GPAW(NWGEAN,NWPAW) gives access to the basic \\[.1em]
*     graphics version. \\[.1em]
* \\[.1em]
      PARAMETER (NWGEAN=3000000,NWPAW=1000000) \\[.1em]
      COMMON/GCBANK/GEANT(NWGEAN) \\[.1em]
      COMMON/PAWC/PAW(NWPAW) \\[.1em]
* \\[.1em]
      CALL GPAWPP(NWGEAN,NWPAW) \\[.1em]
* \\[.1em]
      END \\[.5em]
The user has only to set the desired value of NWGEAN and NWPAW for the
GEANT and PAW Zebra stores, and to call the desired initialization routine:
\begin{itemize}
\item GPAW to initialize, besides GEANT, also HIGZ and to include the full
      functionality of PAW;
\item GPAWPP to initialize, besides GEANT and HIGZ, also the Motif version 
      and to include the full functionality of Paw++;
\item USER initialization routine, to do anything else (for example a UGINIT-
      like routine or a gxint315.f--like routine). 
\end{itemize}
The interactive version, after the initialization, gives the control to the
user at the prompt GEANT $>$ ; then it is possible to type and execute 
commands (corresponding to batch routines) to edit the geometry, the materials
or the tracking media at run time; it is also possible to execute commands
to visualize the detectors, to set the kinematics and to run events; again
interactively, one can spy the histograms, change the kinematics, and run
more events (visualizing the tracks and the hits, for example). The GXINT
chapter contains in the following pages a full description of the available
GEANT commands. See the PAW, KUIP, DZDOC, HIGZ manuals for a description of
the relative commands executable from GEANT. All the commands are also
documented by an on-line help. (Try to type HELP at the first GEANT $>$ prompt).
In the interactive version, a COMIS interface is also available: COMIS is a
FORTRAN interpreter which allows:
\begin{itemize}
\item to edit at run time important routines like UGEOM, GUSTEP, GUKINE, etc.
\item to `execute' them from the interactive session, without having to 
      recompile and relink, by typing commands like CALL UGEOM.F.
\end{itemize}.
Of course the interpreter is slower than the compiled object code,
then, since GEANT321, it is also possible to invoke the native compiler and
to link dinamically to the executable the compiled routine (see the COMIS 
manual for further details).











   


   






\end{document}


