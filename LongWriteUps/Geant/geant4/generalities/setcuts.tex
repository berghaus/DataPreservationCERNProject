\chapter[SetCuts\mbox{ : } conversion from range to kinetic energy]
         {SetCuts\mbox{ : } conversion from 
              range to kinetic energy }


\section{Charged particles}
 Starting from a particle cut given in stopping range, the method
 constructs a vector holding the cut values in kinetic energy for every
 material. The main steps of the algorithm are the same for all the
 charged particles, the only difference is that different energy loss
 formulae are used for electrons, positrons and heavy charged particles
 ( muon,pion,proton,...).  

  In the case of charged hadrons this procedure is applied for the proton
 and antiproton only , the cut values in kinetic energy for other hadrons 
 are computed
 using the proton and antiproton energy loss and range tables . 
     \subsection{General scheme}
        \begin{enumerate}
           \item An energy loss table is created and filled for all the elements
                 in the element table.
           \item For every material in the material table the following steps are
                 performed:
              \begin{enumerate}
                 \item a range vector is constructed using the energy loss table and
                       formulae for the low energy part of the calculations,
                 \item the conversion from stopping range to kinetic energy is performed and
                       the corresponding element of the {\tt KineticEnergyCuts} vector is set
                       (It contains the particle cut value in kinetic energy for the
                        actual material.),
                 \item the range vector is deleted.
               \end{enumerate}
          \item The energy loss table is deleted at the end of the process.
       \end{enumerate}
    \subsection{Energy loss formula for heavy charged particles}
     The energy loss of the particle is calculated from a simplified Bethe-Bloch
     formula if the kinetic energy of the particle is above the value :
     \[
     T_{lim}=2 \,MeV \times \left( \frac { \mbox{particle mass}}{\mbox{proton mass}} \right)
     \]
     The simplified here means that the low energy shell correction term and
     the high energy Sternheimer density correction term have been omitted.
     Below the energy value $T_{lim}$ a simple parametrized energy loss formula
     is used to compute the loss, which reproduces the energy loss values
     of the stopping power tables fairly well. The main reason of using a
     parametrized formula for low energy is that the Bethe-Bloch breaks down
     at low energy. The formula has the following form :
     \[
    \frac{dE}{dx} = \left \{ \begin{array}{ll}
                   a*\sqrt{\frac{T}{M}}+b*\frac{T}{M} & \mbox{for $T \in [0, \, T_0]$}
                   \\ \\
                   c*\sqrt{\frac{T}{M}} & \mbox{for $T \in [T_0, \, T_{lim}]$} 
                      \end{array}
            \right.
     \]
   \begin{tabbing}
      whereb:bbb\= \kill
      where :  \> M = particle mass, \\
               \> T = kinetic energy, \\
               \> $ T_0=0.1 \, MeV \times Z^{1/3} \times \mbox{ M/(proton mass)} $\\
               \> Z = atomic number.
    \end{tabbing}
     the paramaters $a,b,c$ have been chosen in such a way that $dE/dx$ is a
     continuous function of T at $T=T_{lim}$ and $T=T_0$ , and $dE/dx$ reaches its maximum
     at the correct T value.

    \subsection{Energy loss of electrons and positrons}
     The Berger-Seltzer energy loss formula has been used for T $>$ 10 keV to
     compute the energy loss comes from the ionization . This formula plays the
     role of the Bethe-Bloch for electrons ( see e.g. the GEANT3 manual).Below
     10 keV the simple c/(T/mass of electron) parametrization has been used,
     where c can be determined from the requirement of continuity at T = 10 keV.
     As in the case of electrons the radiation loss is important even at 
     relative low (few MeV ) energy, a second term has been added to the
     energy loss formula which accounts for the radiation losses (losses due to
     bremsstrahlung). This second term is an empirical parametrized formula.
     For positrons a different formula is used to calculate the ionization
     loss, the term accounting for the radiation losses is the same than that
     for electrons.

    \subsection{Range calculation}
     The stopping range is defined as : 
              \[ R(T)= \int_0^T \frac{1}{(dE/dx)} \, dE \]
     The integration has been done analytically for the low energy part and 
     numerically above an energy limit.
  

\section{Photons}
 Starting from a particle cut given in absorption length, the method
 constructs a vector holding the cut values in kinetic energy for every
 material. The main steps of the algorithm are the following :
    \subsection{General scheme}
       \begin{enumerate}
          \item A cross section table is created and filled for all the elements
                in the element table.
          \item For every material in the material table the following steps are
                performed:
                \begin{enumerate}
                    \item an absorption length  vector is constructed using the 
                          cross section table 
                    \item the conversion from absorption length to kinetic energy is
                          performed and the corresponding element of the 
                          {\tt KineticEnergyCuts} vector is set
                          (It contains the particle cut value in kinetic energy 
                           for the actual material.),
                    \item the absorption length vector is deleted.
                \end{enumerate}
           \item The cross section table is deleted at the end of the process.
       \end{enumerate}
    \subsection{Cross section formula for elements}
        An approximate empirical formula is used to compute the {\em absorption
        cross section} of a photon in an element.
        The {\em absorption cross section} means here the sum of the cross sections
        of the gamma conversion, Compton scattering and photoelectric effect. 
        These processes are the 'destructive' processes
        for photons , they destroy the photon or decrease its energy.
        (The coherent or Rayleigh scattering changes the direction of the gamma
        only , its cross section is not included in the {\em absorption cross section}.)
    \subsection{Absorption length vector}
        The {\tt AbsorptionLength} vector is calculated for every material as :
         \begin{center}
                 {\tt AbsorptionLength} = 5/(macroscopic absorption cross section)
         \end{center}
        The factor 5 comes from the requirement that the probability of having
        no 'destructive' interaction should be small, hence 
        \begin{eqnarray*} 
        \exp(-\mbox{{\tt AbsorptionLength * MacroscopicCrossSection}}) &=&\exp(-5) \\
         &=& 6.7 \times 10^{-3}
        \end{eqnarray*}
     \subsection{Meaningful cuts in absorption length}
         The photon cross section for a material has a minimum at a certain kinetic
         energy $T_{min}$. The {\tt AbsorptionLength} has a maximum at $T=T_{min}$, the value of
         the maximal {\tt AbsorptionLength} is the biggest "meaningful" cut in absorption
         length . If the cut given by the user is bigger than this maximum, a warning
         is printed and the cut in kinetic energy is set to the maximum gamma energy
         (i.e. all the photons will be killed in the material).  

\section{Status of this document}
   9.10.98  created by L. Urb\'an.
