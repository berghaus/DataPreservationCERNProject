 \chapter[Computing the occurrence of a process]
   {Computing the occurrence of a process} 

\section{Particle transport using the {\em differential} approach}
\subsection{The interaction length}
\begin{itemize}
\item[*]
         In a simple material the number of atoms per volume is:
         \[n  = \frac{\mathcal{N}\rho}{A}\]
         where:
         \begin{eqnarray*}
          \mathcal{N} &  & \mbox{Avogadro's number} \\
          \rho        &  & \mbox{density of the medium} \\
          A           &  & \mbox{mass of mole} 
         \end{eqnarray*}
\item[*]
         In a compound material the number of atoms of Element elm per volume is:
         \[n_{elm}  = \frac{\mathcal{N}\rho w_{elm}}{A_{elm}}\]
         where:
         \begin{eqnarray*}
          \mathcal{N} &  & \mbox{Avogadro's number} \\
          \rho        &  & \mbox{density of the medium} \\
          w_{elm}     &  & \mbox{proportion by mass of the Element elm}\\
          A_{elm}     &  & \mbox{mass of mole of the Element elm} 
         \end{eqnarray*} 
\item[*] 
         The mean free path, $\lambda$, of a process can be given in terms of its 
 total cross section. 
         \[
           \lambda(E) \equiv \frac{1}{\Sigma (E)} 
             = \frac{1}{\sum_{elm}{\lbrack n_{elm} \sigma(Z_{elm},E)\rbrack}}
         \]
         where $\sigma(Z,E)$  is the total cross section per atom of the
         process and
          $\sum_{elm}$ runs over all Elements the material is made of.
\end{itemize}

\noindent
Cross sections per atom and mean free path values are tabulated during 
initialisation.

\subsection{Determination of the interaction point}
The mean free path of a particle for a given process,
$\lambda$, depends on the medium and cannot be used
directly to sample the probability of an interaction in a heterogeneous
detector. The number of mean free paths which a particle travels is:

\begin{equation}
\label{int.c}
n_\lambda =\int \frac{dx}{\lambda(x)}
\end{equation}

and it is independent of the material traversed. If $n_r$ is
a random variable denoting the number of mean free paths from a given
point until the point of interaction, it can be shown that $n_r$ has the
distribution function
\begin{equation}
\label{int.d}
P( n_r < n_\lambda ) = 1-e^{-n_\lambda}
\end{equation}
The total number of mean free paths the particle travels before
the interaction point, $n_\lambda$, is sampled at the beginning
of the trajectory as:
\begin{equation} 
\label{int.e}
n_\lambda = -\log \left ( \eta \right )
\end{equation}   
where $\eta$ is a random number uniformly distributed
in the range $(0,1)$.
$n_\lambda$ is updated after each step $\Delta x$ according the formula:
\begin{equation}
\label{int.f}
n'_\lambda=n_\lambda -\frac{\Delta x }{\lambda(x)}
\end{equation}
until the step originating from $s(x) = n_\lambda \lambda(x)$ is
the shortest and this triggers the specific process.

The short description given above is the {\em differential approach} 
 of the particle transport , which is used in most of the simulation
codes( e.g. \cite{int.geant3},\cite{int.egs4}).

 In this approach besides the other ({\em discrete}) processes  
 the continuous energy loss imposes a limit on the stepsize, too.
 The reason of this is the
energy dependence of the cross sections. It is assumed in this approach
that
the cross sections of the particles are  constant during a step , i.e.
 the step size should be so small that the relative difference of the cross 
 sections
at the beginning of the step and at the end can be small enough.In principle
one
has to use very small steps in order to have an accurate simulation , but the 
computing
time increases if the stepsize decreases. As a good compromise the stepsize is
limited in GEANT4 by the requirement that the stopping range of the particle can
decrease by not more than 20 \% during the step. This condition works fine for
a particle of kinetic energy \(> 0.5 MeV - 1. MeV\) , but for low energy it
gives very short step sizes.
 To cure this problem a lower limitation on the stepsize is also introduced.
The lower limit of the stepsize is
 the {\em cut in range} parameter of the program. 

 There is another disadvantage of this usual differential algorithm , 
 which can be a serious problem in the case of certain processes. If the
 interaction process has a cross section with narrow peaks in it , 
in this approach the particle can easily skip over these peaks.(This can happen
 in the case of a number of hadronic processes.)

 To overcome these shortcomings the {\em integral approach} has been implemented
 in GEANT4 as an option .

\section{Particle transport with {\em integral} approach}

 In this algorithm the integral in eq. \ref{int.c} is really computed for every
 process and another equation is used instead of eq. \ref{int.f} when 
 $n_\lambda$ is updated. This section gives a short overview on the formulae
 used in this approach.

  The change in the number of mean free path after a {\em small} step
 can be written as

\begin{equation}
\label{int.g}
   \Delta n_\lambda = \frac{\Delta x}{\lambda (x)} .
\end{equation} 

This formula can be rewritten in the following form

\begin{equation}
\label{int.h}
   \Delta n_\lambda = \frac{dx}{dT}\cdot \frac{1}{\lambda (T)} \cdot dT  ,
\end{equation} 

where 
\begin{tabular}[t]{ll}
$T$         & kinetic energy of the particle \\  
$\frac{dx}{dT} = \frac{1}{\frac{dT}{dx}}$ & inverse of the well-known quantity {\em $\frac{dE}{dx}$}  . \\
\end{tabular}

After these preliminaires the number of mean free paths (eq. \ref{int.c})
 can be rewritten as 
\begin{equation}
\label{int.i}
n_\lambda (T) = \int_{0}^{T}\frac{d\tau}{f(\tau)\cdot \lambda (\tau)}   
\end{equation}

 or

\begin{equation}
\label{int.j}
n_\lambda (T) = \int_{0}^{T}\frac{\sigma (\tau ) \cdot d\tau}{f(\tau) }  
\end{equation}

where $\frac{dE}{dx}$ has been denoted by $f()$ . Eqs. \ref{int.i} and \ref{int.j} are
the basic equations of the integral approach . The meaning of the integrals
on the right hand sides of the equations is the number of mean free paths
 for a given process between the initial state of energy $T$ and the stopping
 of the particle ( energy $0$). 

 The steps of the algorithm using the integral approach are the following:
\begin{itemize}
\item the number of mean free paths the particle travels to the interaction
      point , $\Delta n_\lambda$ is sampled using eq. \ref{int.e}
\item initial value of the quantity $n_\lambda (T_0)$ is computed 
\item the (possible) final energy $T$ is calculated from 
  \begin{equation}
  \label{int.k}
   \Delta n_\lambda = n_\lambda (T_0) - n_\lambda (T)
  \end{equation}

\item the step limitation originating from the process is
   \begin{equation}
  \label{int.l}
   steplimit = r(T_0) - r(T)      
  \end{equation},
  where the function $r(T)$ gives the range of the particle for a kinetic
 energy value T.
\end{itemize}
 
 If the interaction occurred at the end of the step was  some other
process or the step was limited by 
the geometry (at medium boundary) , the number of mean free
paths $\Delta n_\lambda$ should be updated. This update can be easily done
 using the function $n_\lambda (T)$
\begin{equation}
\label{int.m}
 \Delta n_{\lambda,updated} =  \Delta n_\lambda - [n_\lambda (T_0)-
                                    n_\lambda (T) ]  
\end{equation}
  
\section{Updating of the proper and laboratory time of the particle}

 The proper and lab. time of the particle should be updated after each step.
This update is done differently in the two approaches.

 In the differential approach the following formula is used to update the
particle time in the laboratory system

\begin{equation}
\label{int.n}
  \Delta t_{lab} = \frac{\Delta x}{\frac{v_0 + v}{2}}
\end{equation}
 
  , where 
\[
\begin{array}{ll}
\Delta x    & \mbox{step travelled by the particle,} \\
 v_0  &
 \mbox{particle velocity at the beginning  of the step} \\
 v  &
 \mbox{particle velocity at the end of the step .} \\
\end{array}
\]
This expression is a good approximation for use in the differential
approach , because in this scheme the velocity is not allowed to
change too much during the step.This is not the case in the integral approach,
the kinetic energy and the velocity of the particle can change a lot here,
therefore the update of the laboratory time is done by using the integral
expression
\begin{equation}
\label{int.o}
 t = \int_{0}^{s} \frac{dx}{v(x)}
\end{equation}
which can be written as
\begin{equation}
\label{int.p}
 t = \int_{0}^{T} \frac{f(\tau ) \cdot d\tau }{v(\tau )}.
\end{equation}
In the eqs. \ref{int.o} and \ref{int.p} 
\[
\begin{array}{ll}
t  &  \mbox{time}  \\
v  &  \mbox{velocity of the particle} \\
s  &  \mbox{path length can be travelled by the particle before stopping} \\
T  &  \mbox{kinetic energy} \\
f(T) & dE/dx \\
\end{array}
\]

 Using eq. \ref{int.p} the expression for $\Delta t_{lab}$ is
\begin{equation}
\label{int.q}
 \Delta t_{lab} = t(T_0)-t(T)  .
\end{equation}

 The update of the proper time is performed similarly in the two approaches.

\section{Status of this document}
  9.10.98   created by L. Urb\'an.

\begin{thebibliography}{99}
\bibitem[EGS4]{int.egs4} W.R. Nelson et al.:The EGS4 Code System.
   {\em SLAC-Report-265 , December 1985 }
\bibitem[GEANT3]{int.geant3}
  GEANT3 manual ,CERN Program Library Long Writeup W5013 (October 1994).
\end{thebibliography}
