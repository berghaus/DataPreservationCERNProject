\section{The high energy $\gamma$-nucleon and $\gamma$-nucleus
 interactions.}
 
\hspace{1.0em}We consider the following kinematic 
variables for $\gamma$-nucleon
scattering: the Bjorken-$x$ variable defined as $x=Q^2/2m\nu$ with $Q^2$, $\nu$
and $m$ the photon virtuality, the photon energy and nucleon mass,
respectively.
The the squared total energy of the $\gamma$-nucleon system is given by
$s=Q^2(1-x)/x + m^2$. We restrict consideration to
 the range of small $x$-values and  $Q^2$ is much
less than $s$.

The Generalized Vector Dominance Model (GVDM) \cite{BSY78}
 assumes that the virtual photon 
fluctuates into intermediate $q\bar{q}$-states $V$ of mass $M$ which
 subsequently may 
interact with a nucleon $N$. 
Thus the total photon-nucleon cross section
 can be expressed by a relation \cite{PRW95}:
\begin{equation}
\begin{array}{c}
\label{HEGI1}\sigma_{\gamma N}(s,Q^2)=4\pi\alpha_{em}\int_{M^2_0}^{M^2_{1}}
dM^2D(M^2)\times \\
\times (\frac{M^2}{M^2+Q^2})^2(1+\epsilon\frac{Q^2}{M^2})\sigma_{VN}(s,Q^2),
\end{array}
\end{equation}
where integration over $M^2$ should be performed between $M^2_0=4m^2_{\pi}$
 and $M^2=s$.
Here $\alpha_{em} = e^2/4\pi = 1/137$ and the density
 of $q\bar{q}$-system per
unit mass-squared is given by
\begin{equation}
\label{HEGI2}D(M^2)= \frac{R_{e^{+}e^{-}}(M^2)}{12\pi^2M^2},
\end{equation}
\begin{equation}
\label{HEGI3} R_{e^{+}e^{-}}(M^2)=\frac{\sigma_{e^{+}e^{-}\rightarrow
hadrons}(M^2)}{\sigma_{e^{+}e^{-}\rightarrow
\mu^{+}\mu^{-}}(M^2)}\approx 3\Sigma_{f}e^2_{f},
\end{equation}
where $e^2_{f}$ the squared charge of quark with flavor $f$. 
$\epsilon$ is the
ratio between the fluxes of longitudinally
 and transversally polarized photons.

Similarly the
 inelastic cross section for the scattering of a $\gamma$ with virtuality
$Q^2$ and with  a nucleus $A$ at impact parameter $B$ 
and the $\gamma$-nucleon c.m.
energy squared $s$ is given by \cite{ERR97}:
\begin{equation}
\begin{array}{c}
\label{HEGI4}\sigma_{\gamma A}(s,Q^2,B)=4\pi\alpha_{em}\int_{M^2_0}^{M^2_{1}}
dM^2D(M^2)\times \\
\times (\frac{M^2}{M^2+Q^2})^2(1+\epsilon\frac{Q^2}{M^2})\sigma_{VA}(s,Q^2,B),
\end{array}
\end{equation}

To calculate $\gamma$-nucleon or $\gamma$-nucleus inelastic cross sections 
we need model for the $M^2$-, 
 $Q^2$- and $s$-dependence of the $\sigma_{VN}$ or $\sigma_{VA}$. For 
 these we  apply the 
Gribov-Regge approach, similarly as it was done for $h$-nucleon or $h$-nucleus 
inelastic cross sections.

The 
 effective cross section for the interaction of a $q\bar{q}$-system with
squared mass $M^2$ with nucleus for the coherence length
\begin{equation}
\label{HEGI5} d=\frac{2\nu}{M^2+Q^2}
\end{equation}
exceeding the average distance between two nucleons
can be written as follows
\begin{equation}
\begin{array}{c}
\label{HEGI6}\sigma_{V A}(s,Q^2,B)=\int \prod_{i=1}^{A} 
d^3 r_i\rho_A({\bf r}_i)
\times \\
\times (1 - |\prod_{i=1}^{A}[1-u(s,Q^2,M^2, b^2_i)]|^2).
\end{array}
\end{equation}
Here the amplitude (eikonal) 
$u(s,Q^2,M^2, b^2_i)$ for the interaction of 
the hadronic fluctuation with $i$-th nucleon 
is given by \cite{ERR97}
\begin{equation}
\begin{array}{c}
\label{HEGI7}u(s,Q^2,M^2,{\bf b}_i)=
\frac{\sigma_{VN}(s,Q^2,M^2)} {8 \pi \lambda(s,Q^2,M^2)} \times \\
\times (1-i \rho \exp{[-\frac{b^2}{4\lambda(s,Q^2,M^2)}]},
\end{array}
\end{equation}
where $\rho\approx 0$ is the ratio of real and imaginary parts of scattering 
amplitude at $0$ angle.
The amplitude parameters: the effective $q\bar{q}$-nucleon cross section  
\begin{equation}
\label{HEGI8} \sigma_{VN}(s,Q^2,M^2)=\frac{\tilde{\sigma}_{VN}(s,Q^2)}{M^2+Q^2+C^2},
\end{equation}
where $C^2=2$ \ GeV$^2$,
and
\begin{equation}
\label{HEGI9}\lambda(s,Q^2,M^2)=2+\frac{m^2_{\rho}}{M^2+Q^2} + 
\alpha_{P}^{\prime}
\ln{(\frac{s}{M^2+Q^2})}. 
\end{equation}
The values of $\tilde{\sigma}_{VN}(s,Q^2)$ are calculated in paper
\cite{ERR97}.
It was shown \cite{ERR97} that $Q^2$ dependence of $\sigma_{VN}(s,Q^2)$ 
is very week at $Q^2 < m^2_{rho} + C^2$, where $m_{\rho}$ is 
$\rho$-meson mass, and we omitted this dependence. We also use 
$\sigma_{VN}(s,Q^2)$ calculated  in \cite{ERR97} at $M^2=m^2_{rho}$.

 If coherence length is smaller that an internuclear distance integrated 
over $B$ then cross section 
$\sigma_{VA}=A\sigma_{VN}$.

