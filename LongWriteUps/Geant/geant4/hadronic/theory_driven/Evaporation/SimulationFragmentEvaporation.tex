\section{Simulation of fragment evaporation.}

\hspace{1.0em} The evaporation of
 neutron, proton, deutron, thritium and alpha fragments 
 are taken into account. 

\subsection{Evaporation threshold.} 

\hspace{1.0em}One should take into account the energy condition
for fragment emission, i. e. the nucleus 
excitation energy should be higher than
the reaction threshold:
\begin{equation}
\label{SFE1}T_b^{\max }=E^{*}-Q_b-V_b > 0. 
\end{equation}
Here $T_b^{\max }$ is the maximal kinetic energy carried by emitted
fragment $b$, $Q_b = M(A,Z)-M(A_f,Z_f)-M_b$ is the fragment $b$ binding
energy. $V_b$ is the Coulomb
potential energy, i. e. the Coulomb barrier for fragment $b$.
 $M(A,Z)$ is the mass of the initial nucleus, $M(A_f,Z_f)$ is
the mass of the nucleus after emission of fragment $b$ and $M_b$ is the
fragment $b$ mass. It should be noted that expression ($\ref{SFE1}$) is 
only valid, when the recoil kinetic energy equals zero. In our code we 
apply the condition:
\begin{equation}
\label{SFE2}T_b^{\max }=E_b^{\max}-M_b-V_b > 0, 
\end{equation}  
where
\begin{equation}
\label{SFE3}E_b^{\max }=\frac{(M(A,Z)+E^{*})^2+M^2_b-M^2(A_f,Z_f)}
{2(M(A,Z)+E^{*})}. 
\end{equation}

\subsection{Nuclear and fragment mass defects.} 

\hspace{1.0em}We use neither the tabulated \cite{AW95} or  
 calculated according to
the  Cameron's liquid drop model formula \cite{CAM57} values of
mass defects $dM(A,Z)$, $dM(A_f,Z_f)$ and $dM_b$, respectively
%
\begin{equation}
\label{SFE4} 
\begin{array}{c}
dM(A,Z)=M_{Volume}(A,Z)+M_{Surface}(A,Z)+M_{Coulomb}(A,Z)+ \\ 
+M_{Exchange}(A,Z)+\Delta^{C}_{Shell}(A,Z)+\Delta^{C}_{Pair}(A,Z)+ \\
+(A - Z)m_N + Zm_H, 
\end{array}
\end{equation}
where 
\begin{equation}
\label{SFE5}M_{Volume}(A,Z)=-17.035[1-1.846(\frac{A-2Z}A)^2]A 
\end{equation}
is the volume energy; 
\begin{equation}
\label{SFE6}M_{Surface}(A,Z)=25.8357[1-1.712(\frac{A-2Z}A)^2]
[1-\frac{0.62025}{A^{2/3}}]^2A^{2/3} 
\end{equation}
is the surface energy; 
\begin{equation}
\label{SFE7}M_{Coulomb}(A,Z)=0.779\frac{Z(Z-1)}{A^{1/3}}[1-\frac{1.5849 
}{A^{2/3}}+\frac{1.2273}A+\frac{1.5772}{A^{4/3}} 
\end{equation}
is the Coulomb energy;
%
\begin{equation}
\label{SFE8}M_{Exchange}(A,Z)=-0.4323\frac{Z^{4/3}}{A^{1/3}}[1-\frac{%
0.57811}{A^{1/3}}-\frac{0.14518}{A^{2/3}}+\frac{0.49597}A 
\end{equation}
is the exchange energy and $m_N=8.07169$ MeV and $m_H= 7.8922$ MeV are
the mass excess of neutron and the hydrogen atom, respectively. The
values of $\Delta^{C}(A,Z)=\Delta^{C}_{Shell}(A,Z)+\Delta^{C}_{Pair}(A,Z)$ 
is connected with
influences of shell structure and pairing effect and their values are
tabulated.

\subsection{Coulomb barrier calculation.} 

\hspace{1.0em}The  Coulomb barrier:
\begin{equation}
\label{SFE9}V_b = C_b\frac{Z_bZ_f}{R_f+R_b},
\end{equation}
where $C_b = 1.44$ \ MeVfm, $Z_b$ and $R_b$ are charge and
 radius of nucleus after fragment
emission, $Z_f$ and $R_f$ are charge and radius of fragment. The value
$V_b$  varies with the excitation energy $E^{*}$,
  as $1/(1+\sqrt{\frac{E^{*}}{2A}})$ \cite{Charity88}.
The radii of nuclei are
approximated by $R = r_{C}A^{1/3}$,
 where \cite{IKP94}
\begin{equation}
\label{SFE8}r_{C} = 2.173\frac{1+0.006103Z_bZ_f}{1+0.009443Z_bZ_f} \ fm. 
\end{equation}

\subsection{The fragment evaporation probability.} 

\hspace{1.0em}Evaporation process has been  predicted by the statistical
Weisskopf-Ewing model \cite{WE40}.  Probability to evaporate particle
$b$ in the energy interval $(T_b, T_b+dT_b)$ per unit of time is given
\begin{equation}
\label{SFE9}W_{b}(T_b) = \sigma_{b}(T_b)\frac{(2s_b+1)m_b}{\pi^2 h^3}
\frac{\rho_b(U_b - Q_b-T_b)}{\rho_c(U_c)}T_b,
\end{equation}
where $\sigma_{b}(T_b)$ is the inverse (absorption of particle $b$)
reaction cross section, $s_b$ and $m_b$ are particle spin and mass,
  $\rho_c$ and $\rho_b$ are level densities of compound nucleus  and
nucleus after particle evaporation, respectively.
The energies $U_b$ and $U_c$ are defined as $U_b = E^{*} - \Delta_b$ 
and $U_c = E^{*} - \Delta_c$, where $\Delta_{b,c}$ are pairing energies 
$\Delta_{Pair}$
of the compound and residual nuclei, respectively.
The pairing energy $\Delta_{Pair}$ is calculated according to
\begin{equation}
\label{SFE10} \Delta_{Pair} = \kappa \frac{12}{\sqrt{A}} \ [MeV]
\end{equation}
with $\kappa = 0$, $1$, or $2$ for odd-odd, odd-even or even-even 
nuclei, respectively.

We should note, that Eq. ($\ref{SFE9}$) is written for the case, 
when we have neglected angular momenta and parities
of the compound and residual nuclei as well as the spin 
of the evaporated fragment.
 
\subsection{The inverse reaction cross section.}

\hspace{1.0em}To calculate inverse   reaction cross section it is
assumed to have the form \cite{Dostr59}
\begin{equation}
\label{SFE11} 
\sigma _b(T_b ) = (1+C_b)(1-k_bV_b/T_b )\pi R^2 
\end{equation}
for charged charged fragments with $A_f \leq 4$ interaction, where the
$k_b$ is the barrier penetration coefficient ( its tabulated values are
used), and
\begin{equation}
\label{SFE12} 
\sigma _b(T_b ) = \alpha (1 + \beta /T_b )\pi R^2 
\end{equation}
for neutrons. Here $R = r_{0} A^{1/3}$ denotes the absorption radius, where 
$r_0 = 1.5$ \ fm, 
$\alpha =0.76 + 2.2 A^{-1/3}$ and $\beta=(2.12 A^{-2/3} - 0.05)/(0.76 +
2.2 A^{-1/3})$.

\subsection{The level density.}

\hspace{1.0em}The level 
density is approximated by Fermi-gas approach \cite{IKP94} for the nuclear 
level density: 
\begin{equation}
\label{SFE13} \rho(E^{*}) = C\exp{(2\sqrt{aE^{*}})},
\end{equation}
where $C$ is a constant, which does not depend from nucleus properties
and excitation energy $E^{*}$ and $a$ is the level density parameter (note,
the system entropy is defined as $S=2\sqrt{aU}$). 


\subsection{Level density parameter calculation.}

\hspace{1.0em}This parameter plays a major role in level density models.
The parametrization $a=A/k$ MeV$^{-1}$, where $k\approx 8$,
approximation ( this approximation is used), which is frequently
employed in equilibrium calculations, is not adequate in the
neighborhood of magic nuclei.  Instead, marked shell effects appear for
these nuclei and these effects manifest themselves by an associated
decrease of the level density parameter at the binding energy. It has
been argued by Ignatyuk  \cite{Ignat75} that these shell
effects disappear with increasing excitation energy so that at
sufficiently high excitation energy a simple linear mass dependence of
the level density parameter is recovered. In our calculations we use 
the Iljinov's systematics for $a(Z,A,E^{*})$ \cite{Iljinov92}.
The authors of \cite{Iljinov92} used a functional form proposed 
by Ignatyuk  \cite{Ignat75}:
\begin{equation}
\label{SFE14}a(Z,A,E^{*}) = a_0(A)[1+
\Delta^{C}_{Shell}(Z,A)\frac{f(E^{*}-\Delta_{Pair})}
{E^{*}-\Delta_{Pair}}],
\end{equation}
where 
\begin{equation}
\label{SFE15}a_0(A)=\alpha A + \beta A^{2/3}B_s
\end{equation}
is the Fermi-gas value of the level density parameter at high excitation 
energies and 
\begin{equation}
\label{SFE16} f(U) = 1 - \exp{(-\gamma E^{*})}.
\end{equation}
 $B_s$ is the ratio of the surface are of the nucleus to the 
surface are of the sphere of the same volume. It was taken $B_s = 1$.
$\Delta_{Shell}(Z,A)$ is 
the shell correction in the nuclear mass formula \cite{CAM57}. 
The parameters $\alpha = 0.072$, $\beta = 0.257$ and $\gamma = 0.052$ 
\ MeV$^{-1}$ were found in \cite{Iljinov92}.

\subsection{The total evaporation probability.} 

\hspace{1.0em}The total probability $W_b$ or total partial width
$\Gamma_b=\hbar W_b$ to evaporate particle $b$ can be obtained from 
Eq. ($\ref{SFE9}$) by integration over $T_b$:
\begin{equation}
\label{SFE17} 
W_{b}=\int_{V_b}^{U-Q_b} W_b(T_b)dT_b.
\end{equation}
Here the summation is carried out over all excited states of the
fragment.
 
Integration in Eq. ($\ref{SFE17}$) for probability to emit fragment $b$
can be performed analiticaly, if we will use Eq. ($\ref{SFE11}$) for level
density and the Eqs. ($\ref{SFE11}$)-($\ref{SFE12}$) for inverse cross
section.
The probability
to emit a charged particle:
\begin{equation}
\label{SFE18} 
\begin{array}{c}
W_{b}=\gamma _bA_b^{2/3}B\exp [-2 
\sqrt{aU}]\frac{(1+C_b)}{a_b^2}\{a_bT_b^{\max }[2\exp (2\sqrt{%
a_bT_b^{\max }})+1]- \\
- 3\sqrt{a_bT_b^{\max }}\exp (2\sqrt{a_bT_b^{\max }}%
)-3[1-\exp (2\sqrt{a_bT_b^{\max }})]/2\}, 
\end{array}
\end{equation}
where $T_b^{max}$ is defined by the equation ($\ref{SFE1}$).
The following notations were introduced: $A_b=A-\Delta A_b$, $%
B=m_Nr_0^2/(2\pi h^2)$, $\gamma _b=(2s_b+1)m_b/m_N$. $\Delta A_b$ is the
number of nucleons in $b$ particle. $m_b$, $m_N$ and $s_b$ are mass of
particle $b$, mass of nucleon and spin of particle $b$ respectively. 
  The $a_b$ is level density parameter for
nucleus after emission of fragment $b$.
Similarly for the neutron evaporation probability we obtain the
following equation:
\begin{equation}
\begin{array}{c}
\label{SFE19}W_{n}
=\gamma _nA_n^{2/3}B\frac \alpha {2a_n^2} \\
\exp [-2\sqrt{aE^{*}}+2\sqrt{a_nT_n^{\max }}]
[4a_nT_n^{\max }+(2a_n\beta -3)(2\exp (2\sqrt{a_nT_n^{\max }}-1)].
\end{array} 
\end{equation}
Using probabilities Eq. ($\ref{SFE18}$) and Eq. ($\ref{SFE19}$) we are 
able to
choose randomly the type of emitted fragment.

\subsection{Kinetic energy of emitted fragment.}

\hspace{1.0em}The equation ($\ref{SFE9}$) can be used to sample kinetic
energies of evaporated fragments. 
 For example, keeping terms in Eq. ($\ref{SFE9}$),
which depend from $T_b$ and using the approximations for inverse cross
section is given by Eq. ($\ref{SFE11}$) and for level densities are given
by Eq. ($\ref{SFE13}$), we obtain for charged fragments
\begin{equation}
\label{SFE20}W(x)=C_1 x\exp (2\sqrt{a(T_b^{\max }-x)}= C_2 
T_b\exp (2\sqrt{aE^{*}}), 
\end{equation}
where $C_1$ and $C_2$ do not depend from $T_b$, $x=T_b-V_b$. To generate
values of $x$ we can use the next procedure, changing the expression for
$W(x)$ to have $W(x^{\max })=1$ ($x^{\max }=[(a_b+T_b^{\max
}+1/4)^{1/2}-1/2]/a_b$). Choose two random numbers $\xi _1$ and $\xi _2$
(distributed with equal probabilities between 0 and 1) and find kinetic
energy of particle $b$ as $T_b=T_b^{\max }\xi _1+V_b$ at condition $\xi
_2\leq W(\xi _1T_b^{\max })$. If this condition is not fulfilled we
should choose another pair of random numbers.


\subsection{Angular distribution of evaporated fragments.}

\hspace{1.0em}So far we consider the angular distribution for emitted 
fragments as
isotropical in spite of that decaing nucleus has an angular momentum.

\subsection{The angular momenta of 
evaporated fragments.} 
\hspace{1.0em} The angular momenta of and evaporated particles
are considered as classical vectors ${\bf l_i}$ and estimated in the sharp
cut-off approximation \cite{IT69}, \cite{IT73} according to 
\begin{equation}
\label{SF21}P(l_b)dl_b \sim l_bdm_b, 0 \leq l_b \leq l_b^{max}
\end{equation}
where
\begin{equation}
\label{SF22} l_b^{max}=\sqrt{2\mu_b(E_b - V_b)}R_b/\hbar.
\end{equation}
Here $R_b$ is the radius of the interaction of the $b$th 
emmitted particle with the 
residual nucleus, $E_b$,$V_b$ and $\mu_b$ are the energy 
in the center-of-mass system,
Coulomb barrier and reduced mass of particle, respectively. 
The spins of the emmitted 
particles are not taken into account when estimating the 
angular momentum of the
residual nuclei. 
Angular momenta of residual nuclei are calculated 
without taking into
account the spin of initial target nucleus and of 
intermediate nuclei during emission
of particles.

\subsection{Parameters of residual nucleus.}

\hspace{1.0em}After fragment emission we  update parameter
of decaying nucleus:
\begin{equation}
\label{SFE21} 
\begin{array}{c}
A_f=A-A_b; Z_f=Z-Z_b; P_f = P_0 - p_b; \\ 
E_f^{*}=\sqrt{E_f^2-\vec{P}^2_f} - M(A_f,Z_f); \vec{L}_f = \vec{L}_0-\vec{l_b}. 
\end{array}
\end{equation}
Here $p_b$ is the evaporated fragment four momentum.
Angular momenta of residual nuclei are calculated without taking into
account the spin of initial target nucleus and
 of intermediate nuclei during emission
of fragments.

\subsection{The angular momentum of excited nuclei.}

\hspace{1.0em}The angular momentum influence on evaporation and 
fission processes can be approximately taken into account. 
Thus the angular momentum $L$ 
dependence of the level density can be approximated \cite{ICC80}
by 
\begin{equation}
\label{SFE22}\rho(U-\Delta_{Pair},L)=\rho(E^{*},0),
\end{equation}
 where $E^{*}=U-\Delta_{Pair}-E_R$
 and $E_R$ are 
the thermal and rotational energies of the nucleus, respectively. 
The fission barrier $B_{fis}(L)$ for a fissioning nucleus with the angular
momentum $L$ can be approximated as
\begin{equation}
\label{SFE23} B_{fis}(L)=B_{fis}(0) - (E^{GS}_{R} - E^{SP}_R),
\end{equation}
where $E^{GS}_{R}$ and $E^{SP}_R$ are nuclear rotational energies for 
the ground state and at the saddle-point, respectively.
The rotational energies is calculated according to
\begin{equation}
\label{SFE24} E_R=\frac{L(L+1)}{2\theta},
\end{equation}
where $\theta$ is the moment inertia of a nucleus. The moment of inertia for the 
compound and residual nucleus are calculated according to a rigid-body 
expression:
\begin{equation}
\label{SFE25} \theta=0.4m_Nr^2_0A^{5/3},
\end{equation}
where $m_N$ is nucleon mass and $r_0 = 1.2$ \ fm.
The moment inertia of a nucleus at saddle-point $J_{SP}$ is calculated in 
 in \cite{CS63}.
