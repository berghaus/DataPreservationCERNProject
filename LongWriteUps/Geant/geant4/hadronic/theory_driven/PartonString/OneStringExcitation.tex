\section{Quark or diquark annihilation in hadronic processes.}

\hspace{1.0em}
We consider also hadron-hadron inelastic processes when antiquark or 
antidiquark from hadron projectile annihilate with corresponding quark 
or diquark from hadron target.
In this case excitation of one baryonic (string with quark and diquark 
ends) or mesonic (string with quark and antiquark ends) is created, 
respectively. These processes in the Regge theory correspond to cut 
reggeon exchange diagrams. Initial energy $\sqrt{s}$ 
dependences of these processes 
cross sections are defined by  intercepts of reggeon exchange trajectories.
For example $\sigma_{\pi^{+}p\rightarrow S(s)} \sim s^{\alpha_{\rho}(0)-1}$, 
$S$ notes string and $\alpha_{\rho}(0)$ is the intercept of $\rho$ reggeon 
trajectory. Thus $\sigma_{\pi^{+}p\rightarrow S(s)}
$ decreases with energy 
rise. Cross sections for other quark and diquark proccesses have simiar 
as $\sigma_{\pi^{+}p\rightarrow S(s)}$ initial energy dependences. 
Thus quark and diquark annihilation processes are important at 
relative low initial energies. Another example of these processes is 
$\bar{p}p \rightarrow S$, which is used in the kinetic model to describe 
final state of $\bar{p}p$ annihilation.
Simulation of such kind process is rather simple. We should randomly 
(according to weight calculated using hadron wave function)
choose quark (antiquark) or diquark (antidiquark) from projectile and 
find suitable (with the same flavor content) partner for annihilation 
from target. The created string four-momentum will be equal total reaction 
four-momentum since annihilated system has small neglected momentum (only 
low momenta quarks are able to annihilate).
 
To determine statistical weights for 
 quark annihilation processes are leading to a string production 
and separate them from processes, when two or more strings can be produced we 
use the Regge motivated total cross section parametrization suggested by
Donnachie and Landshoff \cite{DL92}. Using their parametrization the
statistical weight for the one string production process is given by
\begin{equation}
\label{OSE1} W_{1} = \frac{Y_{hN}s^{-\eta}}{\sigma^{tot}_{hN}(s)}
\end{equation}
and statistical weight to produce two and more strings is given by 
\begin{equation}
\label{OSE2} W_{2} = \frac{X_{hN}s^{\epsilon}}{\sigma^{tot}_{hN}(s)},
\end{equation}
where hadron-nucleon total cross sections  $\sigma^{tot}_{hN}(s)$ and its 
fit parameters $Y_{hN}$, $X_{hN}$, which do not depend 
from the total c.m. energy squared $s$ and depend on type of
projectile hadron $h$ and target nucleon $N$ can be found in \cite{PDG96}. 
The reggeon intercept $\eta \approx 
0.45$ and the pomeron intercept $\epsilon \approx 0.08$.
