The final state of radiative capture is described by either photon 
multiplicities, or photon production cross-sections, and the discrete and 
continuous contributions to the photon energy spectra, along with the angular 
distributions of the emitted photons.

For the description of the photon multiplicity there are two supported data
representations.
It can either be tabulated as a function of the energy of
the incoming neutron for each discrete photon as well as the eventual
continuum contribution, or the full transition probability array is known, and
used to determine the photon yields. If photon production cross-sections are
used, only a tabulated form is supported.

The photon energies $E_\gamma$ are associated to the multiplicities or the
cross-sections for
all discrete photon emissions. For the continuum contribution, the normalised
emission probability $f$ is broken down into a weighted 
sum of normalised distributions $g$.
$$f\left(E\rightarrow E_\gamma\right)~=~
\sum_{i}p_i(E)g_i(E\rightarrow E_\gamma)$$
The weights $p_i$ are tabulated as a function of the energy $E$
of the incoming neutron. 
For each neutron energy, the distributions $g$ are tabulated as a function 
of the photon energy. As in the ENDF/B-VI data formats\cite{ENDF}, several 
interpolation laws are used to minimise the amount of data, and optimise
the descriptive power. All data are derived from evaluated data libraries.

The techniques used to describe and sample the angular distributions are 
identical to the case of elastic scattering, with the difference that there 
is either a tabulation or a set of legendre coefficients for each photon 
energy and continuum distribution.

As an example of the results is shown in figure\ref{capture} the energy
distribution of the emitted photons for the radiative
capture of 15~MeV neutrons on Uranium ($^{238}$U). 
Similar comparisons for photon yields, energy and angular distributions have
been performed for capture on
${\rm^{238}U}$, 
${\rm^{235}U}$, 
${\rm^{23}Na}$, and 
${\rm^{14}N}$
for a set of incoming neutron energies.
In all cases investigated
the agreement between evaluated data and Monte Carlo is very good.

\begin{figure}[b!] % fig 1
\centerline{\epsfig{file=hadronic/lowEnergyNeutron/neutrons/plots/cap92u238.energy.fine.epsi,height=5.5in,width=3.5in}}
\vspace{10pt}
\caption{Comparison of data and Monte Carlo for photon energy
distributions for radiative capture of 15~MeV neutrons on Uranium ($^{238}U$). 
The points are evaluated data, the histogram is
the Monte Carlo prediction.}
\label{capture}
\end{figure}
