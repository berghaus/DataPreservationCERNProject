\chapter[Photoabsorption cross section at low energies]
{Photoabsorption cross section at low energies}

\section{Method}


The photoabsorption cross section, 
$\sigma_{\gamma}(\omega)$, where $\omega$ is the photon energy , is used in
GEANT-4 package for description of X-ray transportation and ionization effects in
very thin absorbers.
For practical calculations it is convenient to use the 
representation of the photoabsorption cross section as polinom of $\omega^{-1}$ as
was proposed in \cite{sandia}(Sandia table):

\[
\sigma_{\gamma}(\omega) = \sum_{k=1}^{4}a_{k}^{(i)}\omega^{-k} ,
\]

where the coefficients, $a_{k}^{(i)}$ are fitted with experimental data by the
least square method separately in each i-th energy interval. The interval borders
are equal as a rule to the corresponding photoabsorption edges.

Calculations of primary ionization and energy loss distributions produced by
relativistic charged particles in gaseous detectors based on original data
of Sandia table showed that there is clear disagreement with experimental data, 
 especially for gas mixtures with xenon.

The special investigation was performed \cite{grich} for fitting of coefficients 
$a_{k}^{(i)}$  in the energy range of $ 10 - 50 \ eV$ based on modern experimental
data from synchrotron radiation experiments. The elements usually used in
operation mixture of gaseous detectors were checked.

For hydrogen, fluorine, carbon, nitrogen and oxygen the data from the synchrotron
radiation experiments with molecular gases such as $N_2$, $O_2$, $CO_2$, 
 $CH_4$, and $CF_4$ were used: \cite{lee73},\cite{lee77}. The noble gases were
checked using data given in the tables \cite{marr} and \cite{west}.



\section{Status of  this document}

18.11.98 created by V. Grichine .

\begin{thebibliography}{99}
\bibitem[Biggs90]{sandia} Biggs F., and Lighthill R.
{Preprint Sandia Laboratory, SAND 87-0070} (1990)
\bibitem[Grich94]{grich} Grichine V.M., Kostin A.P., Kotelnikov S.K. at el
{Bulletin of the Lebedev Institute no. 2-3, 34} (1994)
\bibitem[Lee73]{lee73} Lee L.C. at el
{J.Q.S.R.T., v. 13, p. 1023} (1973)
\bibitem[Lee77]{lee77} Lee L.C. at el
{Journ. of Chem. Phys., v. 67, p. 1237} (1977)
\bibitem[Marr76]{marr} Marr G.V. and West J.B.
{Atom. Data Nucl. Data Tabl., v. 18, p. 497} (1976)
\bibitem[West80]{west} West J.B. and Morton J.
{Atom. Data Nucl. Data Tabl., v. 30, p. 253} (1980)
\end{thebibliography}