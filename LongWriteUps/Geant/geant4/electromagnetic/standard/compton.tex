
\chapter[Compton scattering]{Compton scattering}
\section{Cross section and Mean free path}
\subsection{Cross section per atom}
An empirical cross-section formula is used,
which reproduces rather well the cross-section data down to 10 keV:
\[
\sigma(Z,E_{\gamma}) = \left [ P_{1}(Z) \frac{\log(1+2X)}{X} +
\frac{P_{2}(Z)+P_{3}(Z) X + P_{4}(Z) X^{2}}{1+aX+bX^{2}+cX^{3}} \right ] 
\]
where:
\begin{eqnarray*}
Z          & = & \mbox{atomic number of the medium} \\
E_{\gamma} & = & \mbox{energy of the photon}  \\
m          & = & \mbox{electron mass}  \\
X          & = & \frac{E_{\gamma}}{mc^2}  \\
P_{i}(Z)   & = & Z (d_{i} + e_{i}Z + f_{i}Z^{2})
\end{eqnarray*}
\\ 
The values of the parameters are defined within the method which computes
the cross section per atome.
\\
The fit was made over 511 data points chosen between:
\[
1 \leq Z \leq 100 \makebox[2cm]{;} 10 \: \mbox{keV} \leq E_{\gamma} \leq 100 \: 
\mbox{GeV}
\]
\\ 
The accuracy of the fit is estimated to be:
 
\vspace{.3cm}
\[
\frac{\Delta\sigma}{\sigma} = \left\{ \begin{array}{lcl}
\approx 10\% & \makebox[2cm][r]{\rm for E} & \simeq 10\: \mbox{keV}-20\: 
\mbox{keV} \vspace{.3cm} \\
\leq 5-6\% & \makebox[2cm][r]{\rm for E} & > 20 \: \mbox{keV}
\end{array} \right.
\]
\subsection{Mean free path}

\begin{itemize}
\item[*]
         In a simple material the number of atoms per volume is:
         \[n  = \frac{\mathcal{N}\rho}{A}\]
         where:
         \begin{eqnarray*}
          \mathcal{N} &  & \mbox{Avogadro's number} \\
          \rho        &  & \mbox{density of the medium} \\
          A           &  & \mbox{mass of mole} 
         \end{eqnarray*}
\item[*]
         In a compound material the number of atoms of Element elm per volume is:
         \[n_{elm}  = \frac{\mathcal{N}\rho w_{elm}}{A_{elm}}\]
         where:
         \begin{eqnarray*}
          \mathcal{N} &  & \mbox{Avogadro's number} \\
          \rho        &  & \mbox{density of the medium} \\
          w_{elm}     &  & \mbox{proportion by mass of the Element elm}\\
          A_{elm}     &  & \mbox{mass of mole of the Element elm} 
         \end{eqnarray*} 
\item[*] 
         The mean free path, $\lambda$, for a photon to interact via Compton
         scattering is given by
         \[
           \lambda(E_{\gamma}) \equiv \frac{1}{\Sigma (E_{\gamma})} 
             = \frac{1}{\sum_{elm}{\lbrack n_{elm} \sigma(Z_{elm},E_{\gamma})\rbrack}}
         \]
         where $\sum_{elm}$ runs over all Elements the material is made of.
\end{itemize}
 
\section {Sampling the final state}
 
For a complete account of the Monte Carlo methods used the
interested user is referred to the publications of Butcher and Messel 
, Messel and Crawford or Ford and Nelson \cite{butch,messel,egs4}.
Only the basic formalism is outlined here.
\\ 
The quantum  mechanical Klein-Nishina differential cross-section
is \cite{klein,egs4}:
\[
\Phi(E_0,E_1) =\frac{X_0 n \pi r_0^2 m_{\rm e}c^2}{E_0^2}
     \left[\frac{1}{\epsilon}+\epsilon\right]
     \left[1 - \frac{\epsilon \sin^2 \theta}{1+\epsilon^2}\right]
\]
where: \quad
\begin{tabular}[t]{l@{\ = \ }l}
$E_0$       & energy of the incident photon   \\
$E_1$       & energy of the scattered photon  \\
$\epsilon$  & $E_0/E_1$                       \\
$n$         & nb of electron per volume       \\
$X_0$       & radiation length                \\
$m_{e}$     & electron mass                   \\
$r_0$       & classical electron radius       
\end{tabular}

\noindent 
Assuming an elastic collision, the scattering angle $\theta$ is
defined by the Compton formula: 
\[
   E_1   = E_0 \frac{m_{\rm e}c^2}{ m_{\rm e}c^2 + E_0(1-\cos\theta )}
\]

\subsection{sample the gamma energy}
The value of $\epsilon$ corresponding to the minimum
photon energy (backward scattering) is given by:
\[ \epsilon_0 = \frac{m_{\rm e}c^2}{m_{\rm e}c^2+2E_0} \]
Therefore we have: \qquad $\epsilon \in [\epsilon_0, 1]$
 
\noindent
Using the combined ``composition and rejection'' Monte Carlo methods
described in the Gamma Conversion chapter, we may set:
\[
\Phi(\epsilon)
     \simeq  \lbrack \frac{1}{\epsilon}+\epsilon\rbrack
                \lbrack 1 - \frac{\epsilon \sin^2 \theta}{1+\epsilon^2}\rbrack
     = f(\epsilon) g(\epsilon)
     = \lbrack \alpha_1 f_1(\epsilon) + \alpha_2 f_2(\epsilon)\rbrack 
               g(\epsilon)
\]
where: 
\[
\begin{array}{lcl}
\alpha_1      = \ln (1/\epsilon_0) & ; & 
f_1(\epsilon) = 1/(\alpha_1\epsilon)       \\
\alpha_2      = (1-\epsilon_0^2)/2 & ; &
f_2(\epsilon) = \epsilon/\alpha_2           \\
   & &
g(\epsilon)   = \left[ 1 - \frac{\epsilon}{1+\epsilon^2} \sin^2\theta \right]
\end{array}
\]
\\
$f_1$ and $f_2$ are probability density functions on $\lbrack\epsilon_0, 1\rbrack$ 
and $g$ is a rejection function: 
  $\forall \epsilon \in [\epsilon_0, 1] : 0 < g(\epsilon) \leq 1$
  
\noindent 
Given a set of 3 random numbers $r, r', r''$ uniformly distributed in [0,1],
the sampling procedure for $\epsilon$ is the following:
\begin{enumerate}
\item
decide which element of the $f(\epsilon)$ distribution to sample from. \\
if $ r < \frac{\alpha_1}{\alpha_1+\alpha_2}$ select $f_1(\epsilon)$,
                                   otherwise select $f_2(\epsilon)$; 
\item 
sample $\epsilon$ from the distributions corresponding to $f_1$ or $f_2$. \\ 
For $f_1 : \epsilon   = \epsilon_0^{r'} \qquad (\equiv \exp(-r' \alpha_1))$ \\ 
For $f_2 : \epsilon^2 = \epsilon_0^2 + (1-\epsilon_0^2)r'$
 
\item 
calculate $\sin^2\theta = t(2-t)$ 
where $t \equiv (1-\cos\theta) = \frac{m_{\rm e}c^2}{E_0}
                                 \frac{1-\epsilon}{\epsilon}$ 

\item 
test the rejection function: if $g(\epsilon) \geq r''$ accept
$\epsilon$, otherwise return to step 1.
\end{enumerate}

\subsection{compute the final kinematic}
After the successful sampling of $\epsilon$, the polar angles of the 
scattered photon with respect to the direction of the parent photon
are generated.
The azimuthal angle, $\phi$, is generated isotropically and
$\theta$ is as defined in the previous section. 
The momentum vector of the scattered photon, $\overrightarrow{P_{\gamma1}}$,
is then transformed into the {\tt World} coordinate system.

\noindent The kinetic energy and momentum of the recoil electron are then:
\begin{eqnarray*}
T_{el} & = & E_0 - E_1 \\
\overrightarrow{P_{el}} & = & 
              \overrightarrow{P_{\gamma0}} - \overrightarrow{P_{\gamma1}}
\end{eqnarray*}

\section{Restriction}
 
The differential cross-section is only valid for those
collisions in which the energy of the recoil electron is large compared
with its binding energy (which is ignored). However, as pointed out by
Rossi \cite{rossi}, this has a negligible effect 
because of the small number of
recoil electrons produced at very low energies.
\section{Status of this document}
 9.10.98  created by M.Maire.
  
\begin{thebibliography}{99}
\bibitem[Klein29]{klein} O. Klein and Y. Nishina.
   {\em Z. Physik 52} 853 (1929)
\bibitem[Rossi52]{rossi} B. Rossi.
   {\em High energy particles, Prentice-Hall} 77-79 (1952)       
\bibitem[Butch60]{butch} J.C. Butcher and H. Messel.
   {\em Nucl. Phys. 20} 15 (1960)   
\bibitem[Mess70]{messel} H. Messel and D. Crawford.
   {\em Electron-Photon shower distribution, Pergamon Press} (1970)
\bibitem[egs4]{egs4} R. Ford and W. Nelson.
   {\em SLAC-265, UC-32} (1985)    
\end{thebibliography}

