\chapter[Bremstrahlung]{Bremstrahlung}

Bremstrahlung dominates other muon interaction processes in the region of
catastrophic collisions ($v \geq 0.1$ ); at "moderate" muon 
energies - above knock onelectron production kinematic limit.

\section{Differential cross section }

The differential cross section for muon bremstrahlung (in units $cm^{2}/(g~GeV)$)
can be written as :


\begin{eqnarray}
    \frac{d \sigma (E, \epsilon, Z, A )} {d \epsilon} &=&
  \frac{16}{3} \alpha N_{A} (\frac{m}{\mu} r_{e} )^2 \frac{1}{ \epsilon A} Z
  (Z \Phi_{n}+ \Phi_{e})(1-v+\frac{3}{4}v^{2})   \\  
    \frac{d \sigma (E, \epsilon, Z, A )} {d \epsilon} &=& 0 \quad
    if \quad \epsilon \geq \epsilon_{max} = E- \mu  \nonumber
\end{eqnarray}
%
where $\mu$ and $m$ are the muon and  electron masses,
$Z$ and $A$ are the atomic number and atomic weight of the material
and $N_{A}$ is Avogadro's number.
 If we indicate with $E$ and $T$ the initial total and kinetic energy of muon,
$\epsilon$--emitted photon energy we have: $\epsilon = E - E'$ and
relative energy transfer $v = \epsilon /E$.

Functions $\Phi_{n}$ and $\Phi_{e}$ represent contribution 
of nuclear and electrons and can be expressed as :

\begin{eqnarray*}
 \Phi_{n} &=& \ln \frac {BZ^{-1/3}(\mu + \delta (D_{n}' \sqrt{e} -2))}
    {D_{n}'(m+ \delta \sqrt{e}BZ^{-1/3})} ; \\
   &=&  0 \quad if \quad negative;
\end{eqnarray*}

\begin{eqnarray*}
  \Phi_{e} &=& \ln \frac {B'Z^{-2/3} \mu }
   {(1+ \frac{\delta \mu}{m^{2} \sqrt{e}})(m+ \delta \sqrt{e} B'Z^{-2/3}) }  \\
   &=&  0 \quad if \quad \epsilon \geq \epsilon'_{max} = E/(1+ \mu^{2}/2mE) \\
   &=&  0 \quad if \quad negative    
\end{eqnarray*}
%
\begin{eqnarray*}
\delta &=& \mu^{2} \epsilon /2EE' = \mu^{2} v/2(E- \epsilon) \\
D'_{n} &=& D_{n}^{(1-1/Z)}, \quad D_{n}= 1.54A^{0.27}  \\
B &=& 183, \quad B'=1429, \quad \sqrt{e}=1.648(721271).
\end{eqnarray*}
%
Special case for hydrogen ($Z$=1): $B = 202.4,\: B' = 446, \: D_{n}' = D_{n}$.


Formulae are mostly taken from (\cite{brem.kel95}) and (\cite{brem.kel97}).
They include improved (in the region $v \sim 1$ and low $Z$, 
 in comparison with
(\cite{brem.petr68}) correction for nuclear size. Bremstrahlung on atomic 
electrons (taking into account target recoil and atomic binding) is
introduced instead of a rough substitution $Z(Z+1) $. Also  (\cite{brem.andr94})
correction for processes with nuclear excitation is included here.

\subsection{Aplicability and restrictions of the method}

1. $E \gg \mu $, ultrarelativistic approximation used in this method; \\
2. $E \leq 10^{20}$ eV, above LPM cross section suppression can be expected;\\
3. $v \geq 10^{-6}$ , below , Ter-Mikaelyan suppression can be expected.
However, in this region cross section of muon bremstrahlung is many
orders of magnitude less then for other processes.\\
4. Coulomb corrections is not included. However Bugaev's calculations
show that it is small for muon bremstrahlung.

\section{Continuous energy loss}

The restricted energy loss for muon bremsrahlung $(d E/ dx)_{rest}$
 with relative transfers $v = \epsilon / (T+ \mu) \leq v_{cut}$
can be calculated as follows :
$$
\frac{d E}{d x}
= \int_{0}^{\epsilon_{cut}}  \epsilon \sigma (E,\epsilon )  \: d \epsilon  =
(T+ \mu ) \int_{0}^{v_{cut}}  v \sigma (E, v )  \: d v
$$
% 
If user's cut $v_{cut} \geq v_{max}=T/(T+ \mu)$, total average 
energy loss is calculated. Integration is done on the basis of Gauss quadratures, 
binnig provides  the accuracy better
than about 0.03\% for $T = 1$ GeV, $Z=1$ (rapidly improving with increasing $T$
and $Z$).


\section{Total cross-section}

The integration of differential cross section  (1) over
$d\epsilon$ gives the total cross-section for muon bremstrahlung:

\begin{equation}
\sigma_{tot} (E, \epsilon_{cut})
= \int_{\epsilon_{cut}}^{\epsilon_{max}}  \sigma (E,\epsilon )  \: d \epsilon  =
 \int_{\ln v_{cut}}^{\ln v_{max}}  v \sigma (E, v )  \: d (\ln v)\:,
\end{equation}
where $v_{max}=T/(T+ \mu)$. If $v_{cut} \geq v_{max}$ , $\sigma_{tot}$ = 0.

\section{Sampling}

Random photon energy $\epsilon_{p}$ is found from the numerical solution
of the equation :
$$ P \:= \int_{\epsilon_{p}}^{\epsilon_{max}}  \sigma (E,\epsilon,Z,A )  \: d \epsilon \: 
 / \: \int_{\epsilon_{cut.}}^{\epsilon_{max}}  \sigma (E,\epsilon,Z,A )  \: d \epsilon .
$$
%
Here $P$ is random uniform probability, $\epsilon_{max}=T$,
$\epsilon_{cut}=(T+\mu) \cdot v_{min.cut}$;
$v_{min.cut}=10^{-5}$ being a relative transfer cut adopted 
in the algorithm.

For fast sampling, solution of (2) is tabulated at initialization 
for a set of $Z,T,P$. In simulation, sampling subroutine returns random 
$\epsilon_{p}$ corresponding to probability $P$,
interpolation in the table being used.

Tabulation subroutine uses function (1) for differential 
cross section. The table contains the values :
\begin{equation}
 x_{p} = \ln (v_{p} / v_{max})/\ln (v_{max}/v_{min.cut}),
\end{equation}
where $v_{p}= \epsilon_{p}/(T+ \mu),\: v_{max} = T/(T+ \mu).$ 
The table parameters are : $Z_{min}=1, \: Z_{max}=128; \:
T_{min}=1$ GeV, $T_{max}=10^{6}$ GeV ;
$P_{min}=10^{-5}, \: P_{max}=1.$

Atomic weight (which is the necessary parameters in (1)) is estimated
here with iterative solution of the known relation (approximate) :
$$ A = Z (2+0.015 A^{2/3}),$$
for $Z=1$ it is used $A=1$ .
%
To  find $x_{p}$ (and, respectively, $\epsilon_{p}$) corresponding to 
a given probability $P$, in sampling subroutine the linear
interpolation in $\ln Z, \: \ln T$, and cubic (4 points Lagrange) interpolation 
in $\ln P$ is realised. For $P \leq P_{min}$ , linear interpolation 
in $(P,x)$ - coordinates is used, noticeably, $x = 0$ at $P = 0$.
Then, random energy $\epsilon_{p}$ is obtained from the inverse
transformation of (3) :
%
$$\epsilon_{p} = (T+ \mu ) v_{max} (v_{max}/v_{min})^{x_{p}} $$
%
The algorithm with the parameters described above has been
tested for various $Z$ and $T$. It provides the reproduction 
of the differential cross section with about 0.2 - 0.7 \% 
accuracy for $T \geq 10$ GeV, the average total energy loss 
being kept within 0.3\%. Accuracy improves with the increase
of $T$, satisfactory results are obtained also for
$1 \leq T \leq 10$ GeV.

It is important to note that this sampling scheme allows
to generate $\epsilon_{p}$ for different user's cuts on $v$
grater then $v_{min.cut}$. 
To realize such simulation, it is sufficient to
re-define probability variable :
%
$$P' = P \: \sigma_{tot} \: (v_{user.cut}) / \sigma_{tot} (v_{min.cut})$$
%
and just use $P'$ in sampling subroutine.
Thus, time consuming re-calculation of 3-dimensional table
is not required (only calculation 
of $\sigma_{tot}(v_{user.cut})$ is needed).
  

After the successful sampling of $\epsilon$,  the direction
 of the emitted photon is generated with respect to the direction of the
incident particle. The azimuthal angle $\phi$ is generated isotropically;
the polar angle
$\theta$ is calculated from the energy momentum conservation.
This information
is used to calculate the energy and momentum of both scattered
particles and to transform them into the {\em global} coordinate system.

\section{Status of this document}
 9.10.98 created by R.Kokoulin and A.Rybin .

\begin{thebibliography}{99}

\bibitem[Keln95]{brem.kel95}
  S.R.Kelner, R.P.Kokoulin, A.A.Petrukhin. Preprint MEPhI 024-95, Moscow, 1995.
\bibitem[Keln97]{brem.kel97}
  S.R.Kelner, R.P.Kokoulin, A.A.Petrukhin. Phys. Atomic Nuclei, 60(1997)576.
\bibitem[Petr68]{brem.petr68}
  A.A.Petrukhin, V.V.Shestakov. Canad.J.Phys., 46(1968)S377.
\bibitem[Andr94]{brem.andr94}
  Yu.M.Andreyev, L.B.Bezrukov, E.V.Bugaev. Phys. Atomic Nuclei, 57(1994)2066.
\end{thebibliography}


