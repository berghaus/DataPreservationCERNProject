%%%%%%%%%%%%%%%%%%%%%%%%%%%%%%%%%%%%%%%%%%%%%%%%%%%%%%%%%%%%%%%%%%%
%                                                                 %
%   COMIS - Reference Manual -- LaTeX Source                      %
%                                                                 %
%   Chapter 4                                                     %
%                                                                 %
%   Files referenced: none                                        %
%                                                                 %
%   Editor: Michel Goossens / CN-AS                               %
%   Last Mod.: 30 Jun 1993  08:40  mg                             %
%                                                                 %
%%%%%%%%%%%%%%%%%%%%%%%%%%%%%%%%%%%%%%%%%%%%%%%%%%%%%%%%%%%%%%%%%%%

\Filename{H1comis-system-directives}
\chapter{System directives}
\label{sec:system-directives}
 
Some of the directives listed below have the optional
parameter ``\Lit{LUN}''. It is the logical unit number for input/output
streams. If this parameter is omitted, the default value
is used (see appendix~\ref{sec:ap-io-unit-numbers}).
All \COMIS{} directives start with a special character "\excl{}" to avoid
a conflict with user defined routines.
 
\SKUIP[HELP]{!HELP}{}
  
This directive ouputs to terminal short help about directives.
 
\SKUIP[FILE]{!FILE}{ [lun,] file\_name}
 
By default the system input device is the terminal. 
This directive sets the input stream to the file \Lit{file_name}.
The same action can be performed in a user routine by calling:

\Shubrii{CSOFIL}{(lun,'file\_name')}{CSFILE}{('file\_name')}
 
The system switches to the terminal again when the file
is read or an \Lit{!EOF} directive is reached.
 
\SKUIP[EOF]{!EOF}{}
 
This directive closes the input file.
The next string will be accepted from the terminal.
 
\SKUIP[LOGFILE]{!LOGFILE}{[lun,] file\_name}
 
The transcript of the interactive session will be collected in the file
\Lit{file_name}. This file can be used in the \Lit{FILE} directive
later (to repeat the same dialogue session, for example).
The same action can be performed in a user routine by calling:

\Shubrii{CSOLOG}{(lun,'file\_name')}{CSLOG}{('file\_name')}
 
\SKUIP[FORTRAN]{!FORTRAN}{}
 
This directive sets the mode of compilation to ``Fortran''.
This mode should be selected avoid
syntactic conflicts when you want to process standard Fortran
sources with the \COMIS{} compiler. After getting this directive
the compiler treats a 'C' character in the first column
as a comment mark and every character in the sixth column
as a continuation mark (unlike \COMIS{}' free format syntax).
This mode is the default.
 
\SKUIP[COMIS]{!COMIS{}}{}
 
This directive sets the ``\Lit{COMIS}'' mode of compilation.
 
\SKUIP[SHELL]{!SHELL}{command}
 
passes an operand line to the operating system for command
processing. Does not cause a break.
 
\SKUIP[SHOW-MEMORY]{!SHOW}{memory}

This directive shows the \COMIS{} internal memory usage.
 
\SKUIP[SHOW-ROUTINES]{!SHOW}{routines}
 
This directive shows a list of routines currently known to \COMIS{}.
 
\SKUIP[SHOW-COMMONS]{!SHOW}{commons}
 
This directive shows list of common and global blocks
currently known to \COMIS{}.
 
\SKUIP[SHOW-NAMES]{!SHOW}{names common\_name}
 
This directive shows declaration of common or global block
with name \Lit{common_name}.
 
\SKUIP[REMOVE]{!REMOVE}{cs\_routine\_name}
 
This directive removes from the internal \COMIS{} memory the comis routine
with name \Lit{cs_routine_name}.
 
\SKUIP[CLEAR]{!CLEAR}{}
 
This directive clears the internal \COMIS{} memory: it removes all comis
routines and all common/global blocks declarations.
 
\SKUIP[CHECKB]{!CHECKB}{}
 
When the directive \Lit{CHECKB} is given \COMIS{} will check during routines
interpretation that the evaluated result of a array's subscript
expression is greater than or equal to the corresponding lower dimension
bound and does not exceed the corresponding upper dimension bound.
 
\SKUIP[NOCHECKB]{!NOCHECKB}{}

The directive \Lit{NOCHECKB} causes no check to be made during routines
interpretation (default is \Lit{CHECKB}).
 
\SKUIP[PARAM]{!PARAM}{}
 
The directive \Lit{PARAM} causes the \COMIS{} compiler to insert additional
code to provide all facilities for the treatment of actual arguments.
 
\SKUIP[NOPARAM]{!NOPARAM}{}

With the directive \Lit{NOPARAM} you cannot obtain the argument's text
and the Algol like manner of argument processing is not available
(default is \Lit{NOPARAM}).

