%%%%%%%%%%%%%%%%%%%%%%%%%%%%%%%%%%%%%%%%%%%%%%%%%%%%%%%%%%%%%%%%%%%
%                                                                 %
%   HEBDB    - Reference Manual -- LaTeX Source                   %
%                                                                 %
%   Overview                                                      %
%                                                                 %
%   Original: Jamie Shiers (from DBL3, OPCAL doc)                 %
%                                                                 %
%   Last Mod.: 02 Sep  1991 10:50 mg                              %
%                                                                 %
%%%%%%%%%%%%%%%%%%%%%%%%%%%%%%%%%%%%%%%%%%%%%%%%%%%%%%%%%%%%%%%%%%%

\Filename{H1hdbover-Overview}
\chapter{Overview}

High Energy Physics experiments require the use of powerful
data base systems in which to store information such
as detector geometry and calibration constants.
The data stored usually consist of a part which is largely
time independent, for instance the parameters which describe the
experimental setup, and of another part whose contents may vary with
time, with different frequencies, and have therefore to be recorded
repeatedly, with proper time or other validity ranges.

The latter may represent quite a large amount of data. As an example,
for a LEP Detector, one has to store between tens and hundreds of megabytes
of data per year of operation.

At program execution time, fast access to the data base contents is
essential, and an efficient system which limits the rate of transactions
between the directly addressable storage medium, where the data base
resides, and the computer memory available to the user, has to be
implemented.

In a multi-user, multi-computer environment, keeping up to date a
centralized data base and optimizing the data flow is not a trivial
matter. This can only be achieved through dedicated 'service' machines
under control of a data base 'server'.

Last, but not least, the system has to be robust and safe, and
equipped with facilities which minimize the inconveniences of a
possible program crash or of a computer hang-up.

The package HEPDB, described in this manual, is an attempt to
solve the above requirements. It has greatly benefitted from
similar packages developed by other experiments at CERN and elsewhere,
notably the DBL3 and OPCAL packages,
written by the L3 and OPAL collaborations respectively.
The DBL3 package is described in~\cite{bib-DBL3}.

HEPDB is based on the ZEBRA system~\cite{bib-ZEBRA} and relies heavily on
the Random Access I/O package RZ~\cite{bib-ZEBRARZ}, and also
MZ (memory management) \cite{bib-ZEBRAMZ},
DZ~(debug and dump)~\cite{bib-ZEBRADZ} and
FZ~(sequential I/O)~\cite{bib-ZEBRAFZ} packages.

The choice of the ZEBRA RZ package as a subsystem on which
to build HEPDB was natural, as it is used by other widely used
programs such as GEANT \cite{bib-GEANT}, HBOOK \cite{bib-HBOOK},
PAW \cite{bib-PAW} and FATMEN \cite{bib-FATMEN}.
The ZEBRA package provides all of basic features required for the HEPDB
system, including interchange mechanism (which in fact uses
ZEBRA FZ sequential exchange mode files).
\Filename{H1hdbover-HEPDB-concepts-and-basic-functionality}
\chapter{HEPDB concepts and basic functionality}

Built upon RZ, the HEPDB package provides facilities and utilities
which are sufficiently general as to be of use to all High Energy
Physics experiments.
It increases the functionality of RZ in a number of areas, such
as  data compression, for more
economical storage on the random access devices and via
partitioned directories, which helps optimize performance.

\Filename{H2hdbover-Overview-of-the-Zebra-RZ-package}
\section{Overview of the Zebra RZ package}

The ZEBRA RZ \cite{bib-ZEBRARZ} package uses random access files,
which can reside on direct access devices or in memory.
RZ files are organised in a hierarchical manner, much like
the Unix file system. Thus, data is accessed or stored using a Unix-like
{\bf pathname}, plus one or more identifiers known as {\bf keys}.

The ZEBRA RZ package was designed for use in High Energy Physics experiments,
and thus is well suited to applications such as HEPDB.

In HEPDB, data is stored in one or more ZEBRA RZ files.
Typically, one might use separate subdirectories for the different
detector components. Again, for any given subdetector, separate
subdirectories could be used for the various types of information
that are to be stored for that component. Thus, one might have
subdirectories such as {\tt TEMPERATURE}, {\tt GAS}, {\tt ALIGN}
and so on. For efficiency reasons, it is also recommended to
store information with widely differing update frequencies
in different directories. Thus, one might have a subdirectory
per subdetector for constant or pseudo-constant information,
such as the number of wires
in a cell, and other directories, as shown above, for
information which varies more rapidly, such as gas pressures.

\Filename{H2hdbover-Data-base-representation-in-memory}
\section{Data base representation in memory}

The HEPDB package uses a Zebra dynamic store (which is
in fact a Fortran labelled common block).
This can be any of the stores used by the application program. HEPDB
operates inside two divisions in that store: a HEPDB system division, of
no concern for the user, and the division where, at user's request, the
data objects are dumped from the random access medium and kept in memory
as long as required by the user.

In the latter division, data objects from a given data base file are
stored as Zebra banks, or Zebra data structures, within a main data
structure which reproduces, partially and only for the directories in
use, the Node/Key structure of the data base files. As many main
structures can be simultaneously handled in memory as there are data
base files declared by the user; however, they are expected to all share
the same division. The skeleton of NODE and KEY banks grows up,
following the successive user requests, and stays permanently available
in memory. The data banks, appended to the corresponding KEY
banks, can be either refreshed when the time dependence of their
contents requires it, or dropped when they are no longer needed by the
user. They can also be marked as temporarily unwanted, so that the
system can drop them in case of shortage of memory, at the cost of
reaccessing them from the random access device if and when required
again.

\Filename{H2hdbover-HEPDB-keys}
\section{HEPDB keys}

In order to minimize the disk-memory transfers and to permit a tight
control of the whole system, a few general standard keys have been
defined. The HEPDB package assumes that all key vectors consist of at
least 10 keys. These keys are defined as shown in table
\ref{table-keys}.

\begin{table}[t]
\begin{center}
\begin{tabular}{|l|l|l|}
\hline
\multicolumn{3}{|c|}{\bf System keys}\\
\hline
Key number & Meaning & Parameter \\
\hline
1 & serial number & \Lit{IDHKSN}  \\
\hline
2 & pointer       & \Lit{IDHPTR}  \\
\hline
3 & flags         & \Lit{IDHFLG}  \\
\hline
4 & insertion time & \Lit{IDHINS} \\
\hline
5 & {\it reserved} & \\
\hline
\multicolumn{3}{|c|}{\bf Special User keys}\\
\hline
6 & unique source identifier& \Lit{IDHUSI} \\
\hline
7 & software reference number& \Lit{IDHSRN} \\
\hline
8 & {\it reserved}  & \\
\hline
9 & {\it reserved}  & \\
\hline
10 & {\it reserved} & \\
\hline
\multicolumn{3}{|c|}{\bf Validity range pairs (Repeated \Lit{NPAIR} times)}\\
\hline
11 & start range 1 & \\
\hline
12 & end range 1   & \\
\hline
\multicolumn{3}{|c|}{Normal user keys (up to RZ limit of 100)}\\
\hline
\Lit{11+NPAIR*2} & first normal user key& \\
\hline
\Lit{12+NPAIR*2} & second normal user key& \\
\hline
\end{tabular}
\end{center}
\caption{HEPDB keys}
\label{table-keys}
\end{table}

The system keys should not be touched by the user or application program.
The normal user keys may contain any information.

After the standard keys come a number of key pairs. The number of pairs
is a database constant. For example, the L3 collaboration uses only
one key pair, the start and end times for which a database object is
valid. The OPAL collaboration uses 3 key pairs - the start and end
experiment, run and event number for which a database object is valid.

Additional keys, the so-called user keys, can be declared by the user
(within the overall RZ \cite{bib-ZEBRARZ} limit of 100 keys).
Their role will become more
transparent when reviewing the functionality of the storage and
retrieval of data objects, described below.

\Filename{H2hdbover-Data-compression}
\section{Data compression}

In order to minimise the usage of disk storage, data objects may be packed.
Several packing mechanisms are provided - one that is particularly useful
is bit packing. In this case, and provided that all elements of a data
object must be of the same type (integer or real), HEPDB
searches for the optimum bit-length that permits the majority of the data
elements to be packed. The remaining elements are stored using two words
each. Another
packing algorithm, activated at user's request, stores only the data
elements whose absolute values are greater than a predefined number.
Packing and unpacking are automatic and not seen by the user, who always
deals with unpacked data in memory. Furthermore, the user is in any case
given the possibility to inhibit the packing.

In order to store data as compactly as possible,
one may also store only the differences from another object
in the same directory.  Like packing,
the storing of difference records is completely automatic and transparent
to the user.

Further information on packing and the storing of difference records
is given in the Appendix.

\Filename{H2hdbover-Real-time-optimization}
\section{Real time optimization}

The RZ package is not optimized for handling directories with large numbers of
data objects; the memory resident part of the data describing the keys
becomes very large and storage and retrieval become more expensive, both
in CPU time and real time, when the number of data objects exceeds a few
thousands. The concept of partitioned directory has been introduced to
attenuate these effects. Subdirectories are automatically created when
needed, with keys which permit to keep track of their contents and to
quickly access the right partition. Partitioning is invisible to the
user, who just has to give its characteristics when creating the
directory. Subsequent storage and retrieval of data objects in a
partitioned directory follow the same rules as for a normal directory,
as far as the calling sequences of the user interface subroutines are
concerned. It should be noted however that direct calls to RZ routines
for handling such directories would be extremely hazardous and should
be avoided.

To further optimize real time usage in storing data, HEPDB also
provides a facility to store several objects, belonging to the same
directory, in one pass. In real time, the gain is about proportional to
the number of objects one stores at once.

\Filename{H2hdbover-Dictionary}
\section{Dictionary}
\index{dictionary}

The HEPDB package supports a dictionary in a special directory called
DICTIONARY. In the dictionary the user can enter, for any of the
directories, simple aliases for the pathnames as well as mnemonic names
for the data object elements (both limited to 8 characters). The
DICTIONARY directory, which does not make use of the standard HEPDB keys,
is automatically generated, at the top level, when the data base is
created. To start with, it has only one data object, with key 1 set to
\Lit{-1}. The data object contains the number of directories in the data base,
followed by 25 words for each directory, 3 integers and 22 hollerith.
The integers contain the unique numeric identifier assigned to the
directory, the number of characters in the pathname and the date of the
last modification. The 2 first hollerith are reserved to store an alias
name for the pathname (up to 8 characters) and the rest to store the
pathname itself (top directory name excluded).

\Filename{H2hdbover-Aliases}
\section{Aliases}
\index{aliases}
\label{HDB-ALIAS}

Aliases are names of up to 8 characters for standard {\tt HEPDB}
directories. They provide a convenient way of accessing frequently
accessed directories or directories with long names.
Aliases may be manipulated by the \Rind{CDALIA} routine or the
\Rind{ALIAS} command.

\Filename{H2hdbover-Mnemonic-names}
\section{Mnemonic names}
\index{mnemonics}
\label{HDB-MNEMONIC}

Mnemonic names are character names of up to 8 characters for the
various elements of the Zebra banks stored in {\tt HEPDB}.

\Filename{H2hdbover-ASCII-data-objects-and-HELP-directory}
\section{ASCII data objects and HELP directory}
\index{ascii files}
\index{help files}

ASCII files, with up to 80 characters per record, may be stored in the database.
An example of the use of such a facility is for
Production Control, when the logs of individual production jobs
are stored in the database.
Another example is related to the storage of help information in a special HELP
directory generated at the top level when a data base file is created.
HELP directories do not make use of the standard keys. One key is used
to specify the node to which the help information is relevant.

\Filename{H2hdbover-Journaling}
\section{Journaling}
\index{journaling}
\index{dictionary}

For restoring the integrity of a data base after a crash, as well as
for establishing communication with other servers, a journaling
facility has been implemented. This facility is used exclusively
by the database servers. It consists of recording on an FZ
sequential file all kinds of updates which affect the data base, namely
the addition or deletion of directory trees, the addition or deletion of
data objects, the alteration of key values, the deletion of some
partitions in a partitioned directory, the definition of aliases and
mnemonics in the \Lit{DICTIONARY} directory or the insertion of items in the
\Lit{HELP} directory. The information is recorded in different types of FZ
records, consisting of at least an FZ header, with or without a data
part, depending on the kind of action, recorded as first word in the FZ
header. In case of several data base files, each corresponding top
directory is associated with an RZ file number. The same FZ file can
however serve several top directories, if the user has decided so. It is
possible to update a database on a selective number of directories from
a journal file.

Two journal files are supported - a 'standard' journal file, which
is used to communicate updates to other servers, and a 'special
backup' file, which is retained locally. In either case, the journal
files are handled automatically by the servers and are no concern
of general users.

%%%%%%%%%%%%%%%%%%%%%%%%%%%%%%%%%%%%%%%%%%%%%%%%%%%%%%%

\Filename{H1hdbover-Using-the-HEPDB-Fortran-interface}
\chapter{Using the HEPDB Fortran interface}

In most of the HEPDB calling sequences, various character options can
be preset by the user and transmitted as a character string, the
argument \Rarg{CHOPT}.
These will be referred to as ``options'' in the following paragraphs.
\index{option}

\Filename{H2hdbover-Initialisation}
\section{Initialisation}

The user must initialise the ZEBRA memory management
\index{ZEBRA}%
system before calling any of the HEPDB routines. This may
either be done explicitly, e.g. via a call to \Rind{MZEBRA}, or
implicitly, e.g. via a call to \Rind{HLIMIT}.
Alternatively, the routine \Rind{CDSTART} may be used.

Then, for each data base file, a call to
\Rind{CDOPEN} initializes the HEPDB/RZ control for the corresponding file. When
using several data base files, the user should be careful always to give
the complete pathnames for all subsequent references to the directories.

For an optimum usage of the RZ system, the user is advised to have a
reasonably large allocation for the system division.
This is performed automatically if the option \Ropt{E} is specified when calling
\Rind{CDOPEN}.

A unique numeric identifier is associated to every given top directory
(every file). The HEPDB package defines by default an identifier
increasing monotonically as the user makes the successive calls to
\Rind{CDOPEN}. The choice of the identifier can however be imposed by the user,
by presetting \Lit{IQUEST(1)}, a Zebra short term communication variable, to
\index{quest@{\tt QUEST}!{\tt IQUEST}}\index{iquest@{\tt IQUEST}}%
the desired value, before calling \Rind{CDOPEN}. The numeric identifier
assigned to each directory packed with that of the top directory, is
stored in the corresponding \Lit{NODE} bank. This permits to record in a
simple way which data objects have been used in a given program
execution.

The \Lit{DICTIONARY} directory, initially generated under control of \Rind{CDNEW},
is dumped into memory at every call to \Rind{CDOPEN}. It can be expanded at
user's request to store mnemonics (up to 8 characters) for the elements
of the data objects. These are mainly used in HEPDB interactive
applications. For each directory, the mnemonics are stored in a data
object of the \Lit{DICTIONARY} directory, with key 1 set equal to the numeric
identifier of the corresponding directory. The routine which can be
used for entering or retrieving the information is \Rind{CDNAME}.
The aliases, in the \Lit{DICTIONARY} directory, can be
stored and retrieved through the routine \Rind{CDALIA}.

The HELP directory, also generated automatically under control of
\Rind{CDNEW}, at the time of creating the data base, is initially empty. The
user should enter the help information as an ASCII file through the
routine \Rind{CDHELP}. The content of the file is encoded as a computer
independent single data object with the key 1 set equal to the numeric
identifier of the relevant directory. The help information can be
subsequently retrieved, decoded and displayed, also through the routine
\Rind{CDHELP}.

\Filename{H2hdbover-Creation-of-standard-directories}
\section{Creation of standard directories}

Before storing any data, the user has to create a directory with all
specifications of the key vector. The routine \Rind{CDMDIR} permits the creation of
HEPDB standard directories, with any number of additional user keys. The
option 'P' (partitioned) can be used to create a partitioned directory.
The dictionary directory is automatically updated upon creation of a new
directory. In case the directories at the intermediate nodes do not
exist, \Rind{CDMDIR} will create them automatically with 9 integer keys.
Even if a directory is created as a non-partitioned directory it may
be converted at any time using the routine \Rind{CDPART}.
\Filename{H2hdbover-Storage-of-data}
\section{Storage of data}
The user can store data from memory to disk with the
routine \Rind{CDSTOR} , the data in memory being simple Zebra banks
or Zebra data structures.
The pathname of the directory as well as the key
vector have to be supplied when storing the data (the system keys 1-5
do not however need to be filled in). In order to retain complete
flexibility, the directory structure must already have been created
before an object is stored.

The contents of an ASCII file can converted into a packed, machine independant
format and stored in a ZEBRA bank suitable for storing in the data base
through the routine \Rind{CDTEXT}. This bank can then be stored in the
database using \Rind{CDSTOR} as above.

\Filename{H2hdbover-Retrieval-of-data}
\section{Retrieval of data}

The user can retrieve Zebra data structures from any directory on a
disk file into memory through the routine \Rind{CDUSE}. The data objects are
selected according to the given directory pathname and to the contents
of the key vector. Selection is made on the pairs of validity ranges and
on the specific user keys selected, if any.
If not yet done, the routine creates
in memory a tree structure of NODE banks, according to the pathname
elements. At the lowest node, \Rind{CDUSE} inserts the information relevant to
the keys in KEY banks (one or several, created as a linear chain at the
next-of-same-type link of the corresponding NODE banks) and the
information relevant to the data itself, in DATA banks supported by the
first link of the corresponding KEY banks. As a general rule, when \Rind{CDUSE}
finds that more than one data object satisfy the validity criteria, the
default action is to return the most recently inserted object
that matches the selection criteria. Special actions can
however be taken, by using the mask {\tt IMASK} to select on values of
system or standard user keys, such as the {\tt source identifier} or
{\tt insertion time}, or indeed any of the normal user keys.
If any element {\it n} of the vector {\tt IMASK} is set, then only objects
with this key element equal to the corresponding element of the input
key vector will be selected. In the special case of the key pairs,
setting this element will select only objects with start validity
less than, or end validity greater than, the corresponding value specified
in the input selection vector {\tt ISEL}.
This permits selection of all objects valid before or after
a given run, event or time, depending on the key pair definitions.

Independently, the user can set a global selection on insertion time,
through the routine
\Rind{CDBE4}, for instance to ignore modifications to the data base past a
certain date and be able to reproduce the conditions of a previous
program execution.

In some special cases, one may wish to retrieve all valid objects that
satisfy the specified selection criteria, and not just the most recently inserted one.
In this case, or if one wishes to
make multiple selections in a single call,
the routine \Rind{CDUSEM} should be used.

With the option 'K' (key), \Rind{CDUSE} also permits retrieval of the keys
only, without loading the data objects.

In case \Rind{CDUSE} finds that the required data objects already exist in
memory, it does not bother to transfer them again from disk.

For an optimum use of the memory, and in order to minimize the disk
accesses, the user should call the routine \Rind{CDFREE}, in conjunction with
\Rind{CDUSE}. \Rind{CDFREE} sets a flag in the specified KEY bank to signal that the
corresponding data object is no longer needed, until a subsequent call
to \Rind{CDUSE} requires it again. The data object bank is not dropped
immediately, but only if and when any other call to \Rind{CDUSE} need more
space than currently available in memory. If the 'frozen' data object
has not been dropped when it is required again, the flag is cleared and
no further disk transfer will take place.

To read data objects where ASCII information has been encoded, the
user can call \Rind{CDTEXT}, similar to \Rind{CDUSE}, except that the logical unit
number of the file reserved to write the ASCII information has to be
passed as an argument.

Simple Fortran vectors may be stored and retrieved using the
routine \Rind{CDVECT}.

\Filename{H2hdbover-Error-handling}
\section{Error handling}

\index{return code!{\tt IRC}}
All HEPDB routines have an integer return code {\tt IRC} as last
argument. If, on return, this contains the value zero then
the routine was successful. A non-zero return code indicates
an error. The error codes may be translated to a meaningful
error message using the routine \Rind{CDERR}.
An explicit message is printed out when the debug log level has been
set to 1 or a higher value, through a call to \Rind{CDLOGL} (the debug level
is set by default to 0 in \Rind{CDOPEN}).

Some routines return additional status information in the
Zebra common block \Lit{/QUEST/IQUEST(100)}.
However, the user is not required
to test on the values of the \Lit{IQUEST} vector on return.
\index{quest@{\tt QUEST}!{\tt IQUEST}}\index{iquest@{\tt IQUEST}}%

\Filename{H2hdbover-Termination}
\section{Termination}

The user should always close the data base files at the end of the
job, through a call to \Rind{CDEND}. This routine closes all the RZ files
opened during the session, or just a single file, depending on the
options specified.
The user should not make reference to any HEPDB routines
(other than \Rind{CDOPEN} !) after the call to \Rind{CDEND}.

\Filename{H2hdbover-Other-HEPDB-facilities}
\section{Other HEPDB facilities}

In this section a few other user callable subroutines of general
interest are mentioned, as well as some additional facilities which
extend the functionality of the HEPDB package and make it more user
friendly.

\subsection{Print and trace-back}

The user can print the contents of a directory with the routine
\Rind{CDLDIR}, only the keys with the option 'K' (keys), or both the data and
the keys with the option 'D' (data).

The user can print the summary of data base usage through the routine
\Rind{CDSTAT}. The summary consists of the numbers of calls to \Rind{CDUSE} and actual
disk accesses, for each set of user key values.

Information on the data objects used in a given program execution can
be obtained through the routine \Rind{CDRINFO}.

\subsection{Time related routines}

Two sets of routines are available to pack/unpack the date and time
to/from one single word - the routines \Rind{CDPKTM} and \Rind{CDUPTM}
for times accurate to one minute and
\Rind{CDPKTS} and \Rind{CDUPTS} when one second accuracy is required.

The maximum of the start validity and the minimum of the end validity
of all data base objects used in a given program execution can be
obtained through the routine \Rind{CDVALID}.

The insertion time of the last inserted object in a given directory
can be enquired through the routine \Rind{CDLKEY}, and the time when a
directory has been last modified, through the routine \Rind{CDLMOD}.

\subsection{Purging operations}

With the automatic updating mode, deleting data objects from a given
directory is rather critical, because of the possibility of deleting a
master object and leaving alive its updated version(s). Therefore, any
direct call to RZ deletion routines should be avoided. The operation can
be taken care of by the routines \Rind{CDPURK} and \Rind{CDPURG}. The latter provides
a wide range of actions through a number of character options, while
\Rind{CDPURK} can be used to purge data objects with selection on user keys
like \Rind{CDUSE}.

The user can delete a complete tree of directory starting from a given
node, using the routine \Rind{CDDELT}. In the situation where a normal
directory has been 'a posteriori' partitioned, through the use of
\Rind{CDPART}, the user should call \Rind{CDDELT} to delete the original directory.

With the routine \Rind{CDPURGP}, it is possible to delete all but the last few
partitions, as specified by the user, from a partitioned directory.

The user can delete all but a few directory trees from the data base
using the routine \Rind{CDKEEP}.
