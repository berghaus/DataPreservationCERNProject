\Version{STUDIS}                     \Routid{G104}
\Keywords{DISTRIBUTION INVERSE STUDENT T-}
\Author{K.S. K\"olbig}                   \Library{MATHLIB}
\Submitter{}                             \Submitted{15.02.1994}
\Language{Fortran}                    %  \Revised{}
\Cernhead{Student's \ t-Distribution and Its Inverse}
Function subprogram {\tt STUDIS} calculates the value of the Student
$t$-distribution function
$$F(t,n) \ = \ \frac{\Gamma(\textstyle \frac{1}{2}(n+1))}{\sqrt{\pi n}
\ \Gamma(\textstyle \frac{1}{2}n)} \
\int_{-\infty}^t \ \left( 1+\frac{x^2}{n} \right)^{-\frac{1}{2}(n+1)}dx$$
for a given degrees of freedom $n \geq 1$.
\par
Function subprogram {\tt STUDIN} calculates the inverse $t(F,n)$.
\Structure
{\tt FUNCTION} subprogram \\
User Entry Names: \Rdef{STUDIS}, \Rdef{STUDIN}\\
Files Referenced: Printer\\
External References: \Rind{GAUSIN}{G105}, \Rind{MTLMTR}{N002},
                     \Rind{ABEND}{Z035}
\Usage
In any arithmetic expression,
\begin{center}
{\tt STUDIS(T,N)} \quad or \quad {\tt STUDIN(F,N)} \quad has the value
\quad $F(\mathtt{T,N})$ \quad or \quad $t(\mathtt{F,N})$,
\end{center}
respectively. {\tt STUDIS}, {\tt STUDIN}, {\tt F} and {\tt T} are of
type {\tt REAL}, {\tt N} is of type {\tt INTEGER}.
\Errorh
Error {\tt G104.1}: $\mathtt{N \le 0}$. \\
Error {\tt G104.2}: $\mathtt{F < 0}$ or $\mathtt{F > 1}$. \\
In both cases, a message is written on {\tt Unit 6}, unless
subroutine {\tt MTLSET} (N002) has been called.
\Accuracy
About six decimal places are usually correct. Accuracy is lost
for {\tt STUDIS} when $\mathtt{T << 0}$ and $\mathtt{N > 4}$.
\Notes
The subprograms are based on algorithms given in the references.
\Refer
\begin{enumerate}
\item B.E. Cooper, Algorithm AS3 - Applied Statistics
{\bf 17} (1968) 189.
\item G.W. Hill, Algorithm 396, Student's $t$-quantiles,
Collected algorithms from CACM (1970).
\end{enumerate}
$\bullet$
