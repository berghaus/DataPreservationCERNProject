\Version {VMIO}                    \Routid{Z309}
\Keywords{SUBROUTINE ROUTINE PACKAGE}
\Keywords{VMCLOS VMEND VMINIT VMNEXT VMOPEN VMREAD VMRITE VMRNDM}
\Keywords{VMDKOPEN VMFINAL VMFORMAT VMINFO VMITIS VMLABEL VMOPIO}
\Keywords{VMTAPEND VMTPOPEN JOB$VM}
\Keywords{ASSEMBLER READ TAPE CART CARTRIDGE}
\Author{A. Cass}                  \Library{PACKLIB, IBM  VM/CMS only}
\Submitter{}                \Submitted{08.06.1989}
\Language{IBM Assembler}           %\Revised{}
\Cernhead {CMS Macro I/O Package}
{\tt VMIO} is an I/O package written using CMS Macros to perform the I/O
rather than using OS Macros as Fortran and {\tt IOPACK}.
The use of native CMS I/O services allows:
\begin{DLtt}{12}
\item[$\bullet$]  Blocksizes of up to 64K;
\item[$\bullet$]  Random (read) access to any CMS disk file --
not just preformatted {\tt RECFM F} files;
\item[$\bullet$]  Random access to blocks on 3480 cartridges;
\item[$\bullet$]  Better tape handling -- for example no scratch
tapes are ever mounted.
\end{DLtt}
\Structure
{\tt SUBROUTINE} subprograms \\
User Entry Names:
\begin{tabular}[t]{l*{7}{@{\hspace{4pt}}l}}
\Rdef{VMCLOS}, & \Rdef{VMEND},  & \Rdef{VMINIT}, & \Rdef{VMNEXT}, &
\Rdef{VMOPEN}, & \Rdef{VMREAD}, & \Rdef{VMRITE}, & \Rdef{VMRNDM}, \\
\Rdef{VMUPDT}
\end{tabular}\\
Internal Entry Names:
\begin{tabular}[t]{l*{6}{@{\hspace{4pt}}l}}
{\tt VMDKOPEN}, & {\tt VMFINAL}, & {\tt VMFORMAT}, & {\tt VMINFO}, &
{\tt VMITIS},   & {\tt VMLABEL}, & {\tt VMOPIO}, \\
{\tt VMTAPEND}, & {\tt VMTPOPEN}
\end{tabular} \\
External References: \Rind{JOB\$VM}
\Usage
See {\bf Long Write-up}.
\\ $\bullet$
