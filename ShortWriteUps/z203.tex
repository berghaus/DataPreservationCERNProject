\Version {XINOUT}                            \Routid{Z203}
\Keywords{INOUT LIST READ XINB XINBF XINBS XINCF XINOUT XOUTB XOUTBF XOUTBS}
\Keywords{XOUTCF}
\Author{R. Matthews, J. Zoll}                \Library{KERNLIB}
\Submitter{}                                 \Submitted{15.07.1978}
\Language{Fortran}                     \Revised{18.09.1991}
\Cernhead {Short List Reading and Writing}
The 'long list' form \verb" WRITE(LUN) (A(J),J=1,N) " is translated into
slow object code by some compilers.
Normally, these compilers handle the 'short list' form
\begin{verbatim}
    DIMENSION A(N)
    WRITE(LUN) A
\end{verbatim}
correctly, compiling just one system request, rather than {\tt N}
requests.
\par
Furthermore, some machines require the calling program to know the
record size beforehand, if reading is done in Fortran. The problem
can be solved by adding the record size as the first word of the record,
thus for \\[2mm]
\begin{tabular}{@{\hspace*{8mm}}l@{\qquad}l}
writing: &  {\tt WRITE(LUN) N,(B(J),J=1,N)} \\
reading: &  {\tt READ (LUN) N,(B(J),J=1,N)}
\end{tabular} \\[2mm]
This way of reading and writing is an extra convention; it is called
'variable length' in the descriptions below.
\par
Sometimes it is convenient to prefix each record with some
identifiers, always the same number of words, say {\tt NA} words: \\[2mm]
\begin{tabular}{@{\hspace*{8mm}}l@{\qquad}l}
writing: &  {\tt WRITE(LUN) N,(A(J),J=1,NA),(B(J),J=1,N)} \\
reading: &  {\tt READ (LUN) N,(A(J),J=1,NA),(B(J),J=1,N)}
\end{tabular} \\[2mm]
This mode is called 'split mode' in the descriptions below.
\par
The routines of {\tt XINOUT} provide 'short list' reading and writing for
split mode, variable length mode and also for fixed length mode.
\Structure
{\tt SUBROUTINE} subprograms \\
User Entry Names:
\Rdef{XINB}, \Rdef{XINBF}, \Rdef{XINBS},
\Rdef{XOUTB}, \Rdef{XOUTBF}, \Rdef{XOUTBS} \\
{\tt COMMON} Block Names and Lengths: {\tt /SLATE/ NR,DUMMY(39)} \\
Files Referenced: Parameter
\Notes
The routines {\tt XINCF} and {\tt XOUTCF} to handle formatted files
are obsolete.
\newpage
\Usage
{\bf Reading:} \\[2mm]
The vectors to be read are {\tt XAV} and {\tt XV} of length
{\tt NA} and {\tt NX}; the read routines contain effectively
\begin{verbatim}
    DIMENSION XV(NX) [,XAV(NA)]
\end{verbatim}
Before calling, {\tt NX} must be preset to the maximum number of words
to be accepted into {\tt XV} with, say, $\mathtt{NX=NWMAX}$. \\[5mm]
\begin{tabular}{@{\hspace*{5mm}}l@{\hspace{10mm}}l}
{\tt CALL XINB(LUN,XV,NX)}         & Read binary, variable length: \\
                                   & {\tt READ(LUN) NR,(XV(J),J=1,MIN(NR,NX))} \\ [2mm]
{\tt CALL XINBF(LUN,XV,NX)}        & Read binary, fixed length:    \\
                                   & {\tt READ(LUN) XV}            \\ [2mm]
{\tt CALL XINBS(LUN,XAV,NA,XV,NX)} & Read binary, split mode:      \\
                                   & {\tt READ(LUN) NR,XAV,(XV(J),J=1,MIN(NR,NX))} \\
\end{tabular} \\[2mm]
On return {\tt NX} contains:
\begin{DLtt}{1234}
\item[NX] $\mathtt{> 0:}$ Read successful,
number of words transmitted into {\tt XV}. \\
$\mathtt{= 0:}$ End-of-file. \\
$\mathtt{< 0:}$ Read error,
its value contains the {\tt IOSTAT} error code on most machines.
\end{DLtt}
For {\tt XINB} and {\tt XINBS} the record length {\tt NR} read
from the file is stored into the first word of {\tt /SLATE/}. \\[5mm]
{\bf Writing:} \\[2mm]
The vectors to be written are {\tt AV} and {\tt V} of
length {\tt NA} and {\tt N}; the write routines contain \\[3mm]
\begin{tabular}{@{\hspace*{5mm}}l@{\hspace{10mm}}l}
{\tt DIMENSION V(N) [,AV(NA)]} \\[3mm]
{\tt CALL XOUTB(LUN,V,N)}        & Write binary, variable length:\\
                                 & {\tt WRITE(LUN) N,V}          \\ [2mm]
{\tt CALL XOUTBF(LUN,V,N)}       & Write binary, fixed length:   \\
                                 & {\tt WRITE(LUN) V}            \\ [2mm]
{\tt CALL XOUTBS(LUN,AV,NA,V,N)} & Write binary, split mode:     \\
                                 & {\tt WRITE(LUN) N,AV,V}       \\
\end{tabular}
\\ $\bullet$
