% 20 oct 94  ksk
\Version{RDEQMR}                          \Routid{D202}
\Keywords{FIRST-ORDER DIFFERENTIAL EQUATION RUNGE KUTTA MERSON}
\Authors{K.S. K\"olbig}                 \Library{MATHLIB}
\Submitter{}                             \Submitted{15.02.1989}
\Language{Fortran}                 \Revised{01.12.1994}
\Cernhead{First-order Differential Equations (Runge--Kutta--Merson)}
Subroutine subprograms {\tt RDEQMR}  and {\tt DDEQMR} advance the
solution of the system of $n \ge 1$ simultaneous first-order
differential equations
$$ \frac{dy_i}{dx} \ = \
f_i(x,y_1,\ldots,y_n), \qquad (i = 1,2,\ldots,n)$$
from a specified value $x_1$ to a specified value $x_2$ of the
independent variable $x$. The integration step-length is automatically
adjusted to keep the estimated error per step less than a specified
value.
\par
On computers other than CDC and Cray, only the double-precision version
{\tt DDEQBS} is available. On CDC and Cray computers,
only the single-precision version {\tt RDEQBS} is available.
\Structure
{\tt SUBROUTINE} subprograms \\
User Entry Names : \Rdef{RDEQMR}, \Rdef{DDEQMR}\\
Obsolete User Entry Names: \Rdef{DEQMR} $\equiv$ {\tt RDEQMR} \\
Files  Referenced : {\tt Unit 6} \\
External References: \Rind{MTLMTR}{N002}, \Rind{ABEND}{Z035},
User-supplied {\tt SUBROUTINE} subprogram
\Usage
For $\mathtt{t=R}$ (type {\tt REAL}), $\mathtt{t=D}$ (type
{\tt DOUBLE PRECISION}),
\begin{verbatim}
    CALL tDEQMR(N,X1,X2,Y,H0,EPS,SUB,W)
\end{verbatim}
\begin{DLtt}{123456}
\item[N] ({\tt INTEGER}) Number $n$ of equations.
\item[X1] (type according to {\tt t}) Initial value $x_1$ of the
independent variable $x$.
\item[X2] (type according to {\tt t})
Final value $x_2$ of the independent variable $x$.
\item[Y] (type according to {\tt t})
One-dimensional array of length $\ge $ {\tt N}.
On entry, $\mathtt{Y(i),(i=1,\ldots,N)}$, must contain $y_i(x_1)$.
On exit, $\mathtt{Y(i),(i=1,\ldots,N)}$, contains approximate values
 $y_i(x_2)$.
\item[H0] (type according to {\tt t}) On entry, {\tt H0} must contain the
proposed initial step-length $h_0$. On exit, {\tt H0} contains the
last computed step-length (See also {\bf Method} and {\bf Notes}).
\item[EPS] (type according to {\tt t}) The requested absolute
accuracy $\varepsilon$. ({\tt EPS} should not be smaller than
approximately $10^3$ times the machine precision).
\item[SUB] Name of a user-supplied {\tt SUBROUTINE} subprogram,
declared {\tt EXTERNAL} in the calling program.
\item[W] (type according to {\tt t}) Array containing at least
{\tt 6*N} elements required as working-space.
\end{DLtt}
The user-supplied subroutine {\tt SUB} should be of the form
\begin{verbatim}
    SUBROUTINE SUB(X,Y,F)
\end{verbatim}
where the variable {\tt X} and the one-dimensional arrays {\tt Y(*)}
and {\tt F(*)} are of type {\tt t}. This subroutine must set
$$\mathtt{F(I)} = f_{\mathtt{I}}(\mathtt{X,Y(1),\ldots,Y(N))} \qquad
(\mathtt{I = 1,2,\ldots,N}).$$
\newpage
\Method
The method is a modification by Merson of the Runge--Kutta method.
The initial value of the step-length $h$ is taken to be the first of the
numbers $h_0, h_0/2, h_0/4, \ldots$ for which the estimated relative
error at the end of the step is less than $\varepsilon$.
At each susequent
step, an estimate of the integration error for that step (proportional to
$h^5$) is computed. If the corresponding relative error exceeds
$\varepsilon$, the current step-length is halfed; if it is less than
$\varepsilon/32$ the step-length is doubled. This process is
continued
until $x_2$ is reached. (For details, see Ref. 1).
\Errorh
Error {\tt D202.1}: If the requestec accuracy cannot obtained,
a message is written on
{\tt Unit 6}, unless subroutine {\tt MTLSET} (N002) has been called. \\
For $\mathtt{N<1}$, or $\mathtt{X1=X2}$ or $\mathtt{H0=0}$, control
is returned to the calling program without any change in {\tt Y}.
\Notes
For well-conditioned systems of equations any reasonable value of the
initial step length $h_0$ may be chosen. For ill-conditioned systems,
the initial value of $h_0$ may be important, and tests with different
values are advised. An inappropriate choice may lead to wrong results in
such cases.
\Refer
\begin{enumerate}
\item  G.N. Lance, Numerical methods for high-speed computers,
(Iliffe \& Sons, London 1960) 56
\end{enumerate}
$\bullet$
