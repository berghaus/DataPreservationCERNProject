% 20 may 1992  mg
\Version {CHTOI}                     \Routid{M400}
\Keywords{CHARACTER INTEGER CONVERSION PORTABLE}
\Author{H. Renshall}                 \Library{KERNLIB}
\Submitter{M. Metcalf}               \Submitted{27.11.1984}
\Language{Fortran}                   \Revised{12.03.1985}
\Cernhead {Portable Conversion Between Type CHARACTER and Type INTEGER}
{\tt CHTOI} converts between a {\tt CHARACTER*1} value in a
95--character set and {\tt INTEGER} values in the range 32--126 via a
look-up table.
\Structure
{\tt SUBROUTINE} subprogram \\
User Entry Names: \Rdef{CHTOI}, \Rdef{ITOCH}
\Usage
\begin{verbatim}
    CALL CHTOI(CHAR,INTGR,*label)
\end{verbatim}
\begin{DLtt}{12345678}
\item [CHAR] {\tt (CHARACTER*1)} Variable or constant (may be a
substring of a longer string) containing on input the character for
which the integer equivalent is required.
\item [INTGR] {\tt (INTEGER)} Variable which will contain on output the
integer equivalent from a look-up table of the input character argument.
A zero will be returned if the character was not found in the table.
\item [label] {\tt (INTEGER)} Label of an executable statement within the
calling program to which control will be transferred should the input
character not be found in the table.
\end{DLtt}
\begin{verbatim}
    CALL ITOCH(INTGR,CHAR,*label)
\end{verbatim}
\begin{DLtt}{12345678}
\item [CHAR] {\tt (CHARACTER*1)} variable which will contain on output
the character equivalent from a look-up table of the input integer
argument. A question mark will be returned if the integer is outside the
range $\mathtt{32-126}$ inclusive.
\item [INTGR] {\tt (INTEGER)} variable or constant containing on input an
integer in the range $\mathtt{32-126}$ for which the character equivalent is
required.
\item [label] {\tt (INTEGER)} Label of an executable statement in the
calling program to which control will be transferred should the input
integer be outside the range $\mathtt{32-126}$.
\end{DLtt}
\Method
A look-up table containing 95 entries is mapped
consecutively into integers $\mathtt{32-126}$. The table is as follows:
\begin{verbatim}
 32- 47:   ! " # $ % & ' ( ) * + , - . / (32 is a blank)
 48- 57: 0 ... 9
 58- 64: : ; < = > ? @
 65- 90: A ... Z
 91- 96: [ \ ] ^ _ `
 97-122: a ... z
123-126: { | } ~
\end{verbatim}
\newpage
\Restrict
This routine is typed in Fortran on a system
which includes all the above characters. Systems with fewer
characters available usually make some local translation when they read
the source for example on CDC NOS/BE the lower case letters are
translated to upper case. Exact reproducibility of other than the subset
of characters is not guaranteed.
\Notes
These integer values are the same as for the 8-bit {\tt ASCII} set.
\\ $\bullet$
