%11 jan 1994 ksk
\Version {PROB}                             \Routid{G100}
\Keywords{DISTRIBUTION PROBABILITY CHI-SQUARE TAIL UPPER}
\Author{G. Folger, K.S. K\"olbig}          \Library{MATHLIB}
\Submitter{}                         \Submitted{21.08.1971}
\Language{Fortran}                       \Revised{15.01.1994}
\Cernhead{Upper Tail Probability of Chi-Squared Distribution}
Function subprogram {\tt PROB} computes the probability that a random
variable having a $\chi^2$-distribution with $N \geq 1$
degrees of freedom assumes a value which is larger than a given
value $X \ge 0$, i.e.
$$ Q(X|N) \ = \ \displaystyle
\frac{1}{\sqrt{2^N}\,\Gamma(\textstyle \frac{1}{2}N)}\,
\int_{X}^\infty e^{-\frac{1}{2}t}\,t^{\frac{1}{2}N-1}\,dt. $$
\Structure
{\tt FUNCTION} subprogram \\
User Entry Names: \Rdef{PROB}\\
External References: \Rind{ERFC}{C300}, \Rind{DERFC}{C300},
\Rind{MTLMTR}{N002}, \Rind{ABEND}{Z035}
\Usage
In any arithmetic expression,
\begin{center}
{\tt PROB(X,N)} \quad has the value \quad $Q(\mathtt{X,N})$.
\end{center}
{\tt PROB} and {\tt X} are of type
{\tt REAL} and {\tt N} is of type {\tt INTEGER}.
\Method
See Ref. 1, formulae Nr. 26.4.4, 26.4.5 and, for $N>300$, No. 26.4.14.
\Accuracy
For $\mathtt{N \le 300}$, {\tt PROB} has an accuracy of about six digits.
For $\mathtt{N>300}$, the accuracy decreases for $\mathtt{X>N}$ with
increasing {\tt X}.
\Errorh
Error {\tt G100.1}: $\mathtt{N<1}$. \\
Error {\tt G100.2}: $\mathtt{X<0}$. \\
In both cases,
the function value is set equal to zero, and a message is written on
{\tt Unit 6}, unless subroutine {\tt MTLSET} (N002) has been called.
\Refer
\begin{enumerate}
\item M. Abramowitz and I.A. Stegun (eds.), Handbook of mathematical
functions with formulas, graphs, and mathematcal tables, 9th printing
with corrections, (Dover, New York 1972).
\end{enumerate}
$\bullet$
