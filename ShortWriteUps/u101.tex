%  12 may 1995  ksk
\Version{LOREN4}                        \Routid{U101}
\Keywords{LORENTZ TRANSFORM}
\Author{TC}                             \Library{KERNLIB}
\Submitter{J. Zoll}                     \Submitted{01.03.1968}
\Language{Fortran}                \Revised{27.11.1984}
\Cernhead {Lorentz Transformation}
This routine transforms momentum and energy of a particle from one
Lorentz-frame to another.
\par
Seen from the reference system $\Sigma$, the other system
$\Sigma '$ has the velocity $\vec{\beta}$, with
$\vec{\eta} = \gamma \vec{\beta}$.
\par
If a rest mass $M$ is tied to system $\Sigma '$, with energy $E$
and momentum $\vec{P}$, we have:
$$\vec{\beta} = \vec{P}/E, \qquad \vec{\eta} = \vec{P}/M, \qquad
\gamma = E/M. $$
The momentum and energy of a particle with mass $m$ is \\[1mm]
\begin{tabular}{l@{\quad : \quad}lll}
in system $\Sigma$   &  $\vec{p}$   & and  & $e = \sqrt{p^2+m^2}$,\\
in system $\Sigma'$  &  $\vec{p'}$  & and  & $e' = \sqrt{{p'}^2+m^2}$.
\end{tabular}
\Structure
{\tt SUBROUTINE} subprogram \\
User Entry Names: \Rdef{LOREN4}
\Usage
\begin{verbatim}
    CALL LOREN4(S,A,X)
\end{verbatim}
with the 4--vectors $\mathtt{S} = (\vec{P},E)$ and
$\mathtt{A} = (\vec{p},e)$
calculates the transformed 4--vector $\mathtt{X} = (\vec{p'},e')$. \\
{\tt LOREN4} contains one square-root to derive $M$ from $P$ and $E$.
\Method
If we split $\vec{p} = \vec{p}_L + \vec{p}_T$
into components parallel and normal to $\vec{\beta}$, where
$$ \vec{p}_L = \displaystyle \frac{\vec{p}\, \vec{\eta}}{\eta^2}\,
 \vec{\eta}, \qquad \vec{p}_T = \vec{p}-\vec{p}_L,$$
we can write the transformations as
$$ \vec{p'}_L = \gamma \, \vec{p}_L-\vec{\eta} \,e, \qquad
 \vec{p'}_T = \vec{p}_T, \qquad e' = \gamma \,e - \vec{\eta}\,\vec{p}$$
and get
\begin{eqnarray*}
\vec{p'} & = & \vec{p} + (\gamma -1)\vec{p}_L- e\,\vec{\eta} \\
 & = & \vec{p} + \vec{\eta} \, ((\gamma-1)\vec{p}\,\vec{\eta}/\eta^2
  - e) \\
 & = & \vec{p} + \vec{\eta}\, (\vec{p}\,\vec{\eta}/(\gamma+1) - e)
\hspace{25mm} \mbox{(because of} \qquad \eta^2 = \gamma^2 -1) \\
 & = & \vec{p} + \vec{P}\, (\vec{p}\,\vec{P}/(E+M) -e)/M,\\
\vec{e'} & = & \gamma e - \vec{\eta}\,\vec{p} \\
         & = & ( eE-\vec{p}\,\vec{P})/M.
\end{eqnarray*}
$\bullet$
