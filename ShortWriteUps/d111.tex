\Version{GPINDP}                              \Routid{D111}
\Keywords{NUMERICAL INTEGRATION GENERAL}
\Author{T. H\aa vie}                             \Library{MATHLIB}
\Submitter{}                                  \Submitted{01.04.1969}
\Language{Fortran}                            \Revised{01.01.1979}
\Cernhead {General Purpose Integration in Double-Precision}
{\tt  GPINDP} computes an approximation to the integral
$$ \int^b_a F(x)dx $$
in double-precision mode.
$F(x)$ is calculated from a {\tt FUNCTION} subprogram
supplied by the user. The user may choose between three different
methods of integration. The calculation is performed in such a way
that at any step during the calculation lower and upper bounds
for the true value of the integral are computed. The integration
methods, though extremely safe, are rather slow and the main usage
should be to check the accuracy of other faster methods of integration.
\Structure
{\tt SUBROUTINE} subprogram  \\
User Entry Names: \Rdef{GPINDP}\\
Files Referenced: {\tt Unit 6}\\
External References: User supplied {\tt FUNCTION} subprogram \\
{\tt COMMON} Block Names and Lengths: \Rind {/GPINT/ 10}
\Usage
See {\bf Long Write-up}.
\\ $\bullet$
