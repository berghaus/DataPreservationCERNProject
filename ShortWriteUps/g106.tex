%  12 may 1995  ksk
\Version{GAMDIS}               \Routid{G106}
\Keywords{DISTRIBUTION GAMMA INCOMPLETE FUNCTION}
\Author{K.S. K\"olbig}            \Library{MATHLIB}
\Submitter{}               \Submitted{01.05.1990}
\Language{Fortran}            \Revised{15.03.1993}
\Cernhead{Gamma Distribution}
Function subprogram {\tt GAMDIS} calculates the gamma distribution
function (incomplete gamma function)
$$P(x,a) \ = \ \frac{1}{\Gamma(a)} \int_0^x e^{-t}\,t^{a-1}\,dt$$
for real arguments $x \geq 0$ and $a > 0$.
\Structure
{\tt FUNCTION} subprogram \\
User Entry Name: \Rdef{GAMDIS} \\
Files Referenced: {\tt Unit 6} \\
External References: \Rind{GAMMA}{C302}, \Rind{ALGAMA}{C304},
\Rind{MTLMTR}{N002}, \Rind{ABEND}{Z035}
\Usage
In any arithmetic expression,
\begin{center}
{\tt GAMDIS(X,A)} \quad has the value \quad $P(\mathtt{X,A})$,
\end{center}
where {\tt GAMDIS}, {\tt X} and {\tt A} are of type {\tt REAL}.
\Method
The method is described in Ref. 1.
\Accuracy
Approximately six digits are correct.
\Errorh
Error {\tt G106.1}: $\mathtt{X<0}$ or $\mathtt{A \le 0}$. \\
Error {\tt G106.2}: Difficulties of convergence (unlikely). \\
The function value is set equal to zero, and a message is written on
{\tt Unit 6}, unless subroutine {\tt MTLSET} (N002) has been called.
\Notes
\begin{enumerate}
\item For greater accuracy, or for the case $a \le 0$,
use {\tt GAPNC} (C334).
Note, however, that in this case the arguments {\tt X}
and {\tt A} must be interchanged.
\item Note that, for integer $\mathtt{N \geq 1}$,
$\mathtt{GAMDIS(X,N/2.) = 1-PROB(2*X,N)}$,
where {\tt PROB} (G100) is the upper tail probability of the chi-squared
distribution function. {\tt PROB} (G100) is faster
than {\tt GAMDIS} (G106) in   this case.
\end{enumerate}
\Source
This subprogram is based on a Fortran program for the incomplete gamma
functions published in Ref. 2.
\Refer
\begin{enumerate}
\item W. Gautschi, A computational procedure for incomplete gamma
functions, ACM Trans. Math. Software {\bf 5} (1979) 466--481.
\item W. Gautschi,Algorithm 542, Incomplete gamma functions,
Collected Algorithms from CACM (1979).
\end{enumerate}
$\bullet$
