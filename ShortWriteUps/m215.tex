\Version{PSCALE}             \Routid{M215}
\Keywords{SCALING FIND POWER PRINT OUTPUT NUMBER}
\Author{J. Zoll}               \Library{KERNLIB}
\Submitter{C. Letertre}         \Submitted{01.09.1969}
\Language{Fortran}             \Revised{15.09.1978}
\Cernhead{Find Power-of-Ten Scale for Printing}
{\tt PSCALE} gives the power of ten by which it is necessary to
multiply a {\tt REAL} number $A$ for the purpose of obtaining a new
{\tt REAL} number $B$ having a fixed number of digits on the left of the
decimal point.
\Structure
{\tt FUNCTION} subprogram \\
User Entry Names: \Rdef{PSCALE}
\Usage
\begin{verbatim}
    FACT=PSCALE(N,NMAX,A,IDIG)
\end{verbatim}
returns the largest {\tt N} and its power {\tt FACT $=$ 10.0**N},
such that {\tt FACT*A} has at most {\tt IDIG} digits to the left of the
decimal point. {\tt N} is limited to $\mathtt{\leq NMAX}$, however.
\Examples
Suppose we have an array {\tt B(100)}, which we want to
print with a {\tt FORMAT(10F10.3)}. Using {\tt VMAXA} (F121)
we find the smallest number {\tt BMAX}, such that
$\mathtt{BMAX \geq |B(I)|}$ for all {\tt I}. Then
\begin{verbatim}
    FACT=PSCALE(N,9,BMAX,4)
\end{verbatim}
allows us to print the vector {\tt FACT*B(I)} with the above
{\tt FORMAT}.
The following sample values of {\tt BMAX} give values for
{\tt FACT} as indicated below:
\begin{verbatim}
          BMAX                      FACT
    1234567800.                     10.0**(-6)
       1234567.8                    10.0**(-3)
          1234.5678                 1
             1.2345678              10.0**3
             0.0012345678           10.0**6
          1234.5678*10.0**(-9)      10.0**9
          1234.5678*10.0**(-12)     10.0**9
             0.0                    10.0**9
\end{verbatim}
All {\tt FACT*BMAX} but the two last ones, will be printed as
{\tt 1234.567}.
\\ $\bullet$
