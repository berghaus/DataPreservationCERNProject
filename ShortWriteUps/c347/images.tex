\nonstopmode
\documentclass{cernman}
\makeatletter
\usepackage{amssymb,amsmath}
\usepackage{cernlib1}
\usepackage{html}

\newcommand{\binomv}[2]{\genfrac{}{}{0pt}{}{#1}{#2}}

\newcommand{\binomg}[2]{\genfrac{\{}{\}}{0pt}{}{#1}{#2}}

\newcommand{\binoms}[2]{\genfrac{[}{]}{0pt}{}{#1}{#2}}

\newcommand{\Title}{CERN Program Library}

\providecommand{\Writekeys}{}

\makeatother
\ifx\AtBeginDocument\undefined \newcommand{\AtBeginDocument}[1]{}\fi
\newenvironment{tex2html_wrap}{}{}
\newbox\sizebox
\setlength{\hoffset}{0pt}\setlength{\voffset}{0pt}
\addtolength{\textheight}{\footskip}\setlength{\footskip}{0pt}
\addtolength{\textheight}{\topmargin}\setlength{\topmargin}{0pt}
\addtolength{\textheight}{\headheight}\setlength{\headheight}{0pt}
\addtolength{\textheight}{\headsep}\setlength{\headsep}{0pt}
\setlength{\textwidth}{349pt}
\newwrite\lthtmlwrite
\makeatletter
\let\realnormalsize=\normalsize
\topskip=0pt
\def\preveqno{}\let\real@float=\@float \let\realend@float=\end@float
\def\@float{\let\@savefreelist\@freelist\real@float}
\def\end@float{\realend@float\global\let\@freelist\@savefreelist}
\let\real@dbflt=\@dbflt \let\end@dblfloat=\end@float
\let\@largefloatcheck=\relax
\def\@dbflt{\let\@savefreelist\@freelist\real@dbflt}
\def\adjustnormalsize{\def\normalsize{\mathsurround=0pt \realnormalsize\parindent=0pt\abovedisplayskip=0pt\belowdisplayskip=0pt}\normalsize}
\def\lthtmltypeout#1{{\let\protect\string\immediate\write\lthtmlwrite{#1}}}%
\newcommand\lthtmlhboxmathA{\adjustnormalsize\setbox\sizebox=\hbox\bgroup}%
\newcommand\lthtmlvboxmathA{\adjustnormalsize\setbox\sizebox=\vbox\bgroup%
 \let\ifinner=\iffalse }%
\newcommand\lthtmlboxmathZ{\@next\next\@currlist{}{\def\next{\voidb@x}}%
 \expandafter\box\next\egroup}%
\newcommand\lthtmlmathtype[1]{\def\lthtmlmathenv{#1}}%
\newcommand\lthtmllogmath{\lthtmltypeout{l2hSize %
:\lthtmlmathenv:\the\ht\sizebox::\the\dp\sizebox::\the\wd\sizebox.\preveqno}}%
\newcommand\lthtmlfigureA[1]{\let\@savefreelist\@freelist
       \lthtmlmathtype{#1}\lthtmlvboxmathA}%
\newcommand\lthtmlfigureZ{\lthtmlboxmathZ\lthtmllogmath\copy\sizebox
       \global\let\@freelist\@savefreelist}%
\newcommand\lthtmldisplayA[1]{\lthtmlmathtype{#1}\lthtmlvboxmathA}%
\newcommand\lthtmldisplayB[1]{\edef\preveqno{(\theequation)}%
  \lthtmldisplayA{#1}\let\@eqnnum\relax}%
\newcommand\lthtmldisplayZ{\lthtmlboxmathZ\lthtmllogmath\lthtmlsetmath}%
\newcommand\lthtmlinlinemathA[1]{\lthtmlmathtype{#1}\lthtmlhboxmathA  \vrule height1.5ex width0pt }%
\newcommand\lthtmlinlinemathZ{\egroup\expandafter\ifdim\dp\sizebox>0pt %
  \expandafter\centerinlinemath\fi\lthtmllogmath\lthtmlsetmath}
\def\lthtmlsetmath{\hbox{\vrule width.5pt\vtop{\vbox{%
  \kern.5pt\kern1.2 pt\hbox{\hglue.5pt\copy\sizebox\hglue1.2 pt}\kern.5pt%
  \ifdim\dp\sizebox>0pt\kern1.2 pt\fi}%
  \ifdim\hsize>\wd\sizebox \hrule depth1pt\fi}}}
\def\centerinlinemath{\dimen1=\ht\sizebox
  \ifdim\dimen1<\dp\sizebox \ht\sizebox=\dp\sizebox
  \else \dp\sizebox=\ht\sizebox \fi}

\def\lthtmlcheckvsize{\ifdim\ht\sizebox<\vsize\expandafter\vfill
  \else\expandafter\vss\fi}%
\makeatletter
\providecommand{\Mu}{\textrm{M}}
\providecommand{\Nu}{\textrm{N}}
\providecommand{\Chi}{\textrm{X}}
\providecommand{\Zeta}{\textrm{Z}}
\providecommand{\Alpha}{\textrm{A}}
\providecommand{\Omicron}{\textrm{O}}
\providecommand{\omicron}{\textrm{o}}
\providecommand{\Rho}{\textrm{R}}
\providecommand{\Tau}{\textrm{T}}
\providecommand{\Epsilon}{\textrm{E}}
\providecommand{\Eta}{\textrm{H}}
\providecommand{\Beta}{\textrm{B}}
\providecommand{\Iota}{\textrm{J}}
\providecommand{\Kappa}{\textrm{K}}


\begin{document}
\pagestyle{empty}\thispagestyle{empty}%
\lthtmltypeout{latex2htmlLength hsize=\the\hsize}%
\lthtmltypeout{latex2htmlLength vsize=\the\vsize}%
\lthtmltypeout{latex2htmlLength hoffset=\the\hoffset}%
\lthtmltypeout{latex2htmlLength voffset=\the\voffset}%
\lthtmltypeout{latex2htmlLength topmargin=\the\topmargin}%
\lthtmltypeout{latex2htmlLength topskip=\the\topskip}%
\lthtmltypeout{latex2htmlLength headheight=\the\headheight}%
\lthtmltypeout{latex2htmlLength headsep=\the\headsep}%
\lthtmltypeout{latex2htmlLength parskip=\the\parskip}%
\lthtmltypeout{latex2htmlLength oddsidemargin=\the\oddsidemargin}%
\makeatletter
\if@twoside\lthtmltypeout{latex2htmlLength evensidemargin=\the\evensidemargin}%
\else\lthtmltypeout{latex2htmlLength evensidemargin=\the\oddsidemargin}\fi%
\makeatother
{\newpage\clearpage
% contents=\begin{tex2html_wrap_indisplay}$\displaystyle \int_{0}^{\infty}$\end{tex2html_wrap_indisplay}
\lthtmlinlinemathA{tex2html_wrap_indisplay471}%
$\displaystyle \int_{0}^{\infty}$%
\lthtmlinlinemathZ
\hfill\lthtmlcheckvsize\clearpage}

{\newpage\clearpage
% contents=\begin{tex2html_wrap_indisplay}$\displaystyle {\frac{d\xi}{\sqrt<#311#>(1+\xi^2)(1+<#59#>k'<#59#>^2\xi^2)<#311#>}}$\end{tex2html_wrap_indisplay}
\lthtmlinlinemathA{tex2html_wrap_indisplay472}%
$\displaystyle {\frac{d\xi}{\sqrt{(1+\xi^2)(1+{k'}^2\xi^2)}}}$%
\lthtmlinlinemathZ
\hfill\lthtmlcheckvsize\clearpage}

{\newpage\clearpage
% contents=\begin{tex2html_wrap_indisplay}$\displaystyle \qquad$\end{tex2html_wrap_indisplay}
\lthtmlinlinemathA{tex2html_wrap_indisplay473}%
$\displaystyle \qquad$%
\lthtmlinlinemathZ
\hfill\lthtmlcheckvsize\clearpage}

{\newpage\clearpage
% contents=\begin{tex2html_wrap_indisplay}$\displaystyle {\frac{a+b\xi^2}{(1+\xi^2)\sqrt<#312#>(1+\xi^2)(1+<#66#>k'<#66#>^2\xi^2)<#312#>}}$\end{tex2html_wrap_indisplay}
\lthtmlinlinemathA{tex2html_wrap_indisplay477}%
$\displaystyle {\frac{a+b\xi^2}{(1+\xi^2)\sqrt{(1+\xi^2)(1+{k'}^2\xi^2)}}}$%
\lthtmlinlinemathZ
\hfill\lthtmlcheckvsize\clearpage}

{\newpage\clearpage
% contents=\begin{tex2html_wrap_indisplay}$\displaystyle \xi$\end{tex2html_wrap_indisplay}
\lthtmlinlinemathA{tex2html_wrap_indisplay478}%
$\displaystyle \xi$%
\lthtmlinlinemathZ
\hfill\lthtmlcheckvsize\clearpage}

{\newpage\clearpage
% contents=\begin{tex2html_wrap_indisplay}$\displaystyle {\frac{1+\xi^2}{(1+p\xi^2)\sqrt<#313#>(1+\xi^2)(1+<#73#>k'<#73#>^2\xi^2)<#313#>}}$\end{tex2html_wrap_indisplay}
\lthtmlinlinemathA{tex2html_wrap_indisplay483}%
$\displaystyle {\frac{1+\xi^2}{(1+p\xi^2)\sqrt{(1+\xi^2)(1+{k'}^2\xi^2)}}}$%
\lthtmlinlinemathZ
\hfill\lthtmlcheckvsize\clearpage}

{\newpage\clearpage
% contents=\begin{tex2html_wrap_indisplay}$\displaystyle \ne$\end{tex2html_wrap_indisplay}
\lthtmlinlinemathA{tex2html_wrap_indisplay486}%
$\displaystyle \ne$%
\lthtmlinlinemathZ
\hfill\lthtmlcheckvsize\clearpage}

{\newpage\clearpage
% contents=\begin{tex2html_wrap_indisplay}$\displaystyle {\frac{a+b\xi^2}{(1+p\xi^2)\sqrt<#314#>(1+\xi^2)(1+<#84#>k'<#84#>^2\xi^2)<#314#>}}$\end{tex2html_wrap_indisplay}
\lthtmlinlinemathA{tex2html_wrap_indisplay494}%
$\displaystyle {\frac{a+b\xi^2}{(1+p\xi^2)\sqrt{(1+\xi^2)(1+{k'}^2\xi^2)}}}$%
\lthtmlinlinemathZ
\hfill\lthtmlcheckvsize\clearpage}

{\newpage\clearpage
% contents=\begin{tex2html_wrap_indisplay}$\displaystyle \int_{0}^{\pi/2}$\end{tex2html_wrap_indisplay}
\lthtmlinlinemathA{tex2html_wrap_indisplay497}%
$\displaystyle \int_{0}^{\pi/2}$%
\lthtmlinlinemathZ
\hfill\lthtmlcheckvsize\clearpage}

{\newpage\clearpage
% contents=\begin{tex2html_wrap_indisplay}$\displaystyle {\frac{a \cos^2 \phi + b \sin^2 \phi}{\cos^2 \phi + p \sin^2 \phi}}$\end{tex2html_wrap_indisplay}
\lthtmlinlinemathA{tex2html_wrap_indisplay498}%
$\displaystyle {\frac{a \cos^2 \phi + b \sin^2 \phi}{\cos^2 \phi + p \sin^2 \phi}}$%
\lthtmlinlinemathZ
\hfill\lthtmlcheckvsize\clearpage}

{\newpage\clearpage
% contents=\begin{tex2html_wrap_indisplay}$\displaystyle {\frac{d\phi}{\sqrt<#315#>\cos^2 \phi + <#89#>k'<#89#>^2 \sin^2 \phi<#315#>}}$\end{tex2html_wrap_indisplay}
\lthtmlinlinemathA{tex2html_wrap_indisplay499}%
$\displaystyle {\frac{d\phi}{\sqrt{\cos^2 \phi + {k'}^2 \sin^2 \phi}}}$%
\lthtmlinlinemathZ
\hfill\lthtmlcheckvsize\clearpage}

{\newpage\clearpage
% contents=\begin{tex2html_wrap_indisplay}$\displaystyle {\frac{d\psi}{\sqrt<#98#>1-k^2\sin^2 \psi<#98#>}}$\end{tex2html_wrap_indisplay}
\lthtmlinlinemathA{tex2html_wrap_indisplay505}%
$\displaystyle {\frac{d\psi}{\sqrt{1-k^2\sin^2 \psi}}}$%
\lthtmlinlinemathZ
\hfill\lthtmlcheckvsize\clearpage}

{\newpage\clearpage
% contents=\begin{tex2html_wrap_indisplay}$\displaystyle \sqrt{1-k^2\sin^2 \psi}$\end{tex2html_wrap_indisplay}
\lthtmlinlinemathA{tex2html_wrap_indisplay509}%
$\displaystyle \sqrt{1-k^2\sin^2 \psi}$%
\lthtmlinlinemathZ
\hfill\lthtmlcheckvsize\clearpage}

{\newpage\clearpage
% contents=\begin{tex2html_wrap_indisplay}$\displaystyle \psi$\end{tex2html_wrap_indisplay}
\lthtmlinlinemathA{tex2html_wrap_indisplay510}%
$\displaystyle \psi$%
\lthtmlinlinemathZ
\hfill\lthtmlcheckvsize\clearpage}

{\newpage\clearpage
% contents=\begin{tex2html_wrap_indisplay}$\displaystyle \le$\end{tex2html_wrap_indisplay}
\lthtmlinlinemathA{tex2html_wrap_indisplay512}%
$\displaystyle \le$%
\lthtmlinlinemathZ
\hfill\lthtmlcheckvsize\clearpage}

{\newpage\clearpage
% contents=\begin{tex2html_wrap_indisplay}$\displaystyle \begin{array}<#109#>rcl<#109#> F(k,\pi/2) ;SPMamp; = ;SPMamp; \mathrm<#110#>K<#110#>(k) \quad = \quad \mathbf<#111#>F<#111#>_1^*(k') \qquad (|k| ;SPMlt; 1...
\lthtmlinlinemathA{tex2html_wrap_indisplay518}%
$\displaystyle \begin{array}{rcl}
F(k,\pi/2) & = & \mathrm{K}(k) \quad = \quad \mathbf{F}_1^*(k') \qquad
(|k| < 1), \\[6mm]
\widehat{F}(1,k) & = & \displaystyle \int_0^1
\frac{d\eta}{\sqrt{(1-\eta^2)(1-k^2\eta^2)}} \quad = \quad
\mathbf{F}_1^*(k') \qquad (|k| < 1).
\end{array}\end{array}$%
\lthtmlinlinemathZ
\hfill\lthtmlcheckvsize\clearpage}

{\newpage\clearpage
% contents=\begin{tex2html_wrap_indisplay}$\displaystyle \begin{array}<#119#>rcl<#119#> E(k,\pi/2) ;SPMamp; = ;SPMamp; \mathrm<#120#>E<#120#>(k) \quad = \quad \mathbf<#121#>F<#121#>_2^*(k',1,<#122#>k'<#122#>^2) ...
\lthtmlinlinemathA{tex2html_wrap_indisplay520}%
$\displaystyle \begin{array}{rcl}
E(k,\pi/2) & = & \mathrm{E}(k) \quad = \quad \mathbf{F}_2^*(k',1,{k'}^2)
\qquad (|k| \le 1), \\[6mm]
\widehat{E}(1,k) & = & \displaystyle \int_0^1
\sqrt{\frac{1-k^2 \eta^2}{1-\eta^2}} \, d\eta \quad = \quad
\mathbf{F}_2^*(k',1,{k'}^2) \qquad (|k| \le 1).
\end{array}\end{array}$%
\lthtmlinlinemathZ
\hfill\lthtmlcheckvsize\clearpage}

{\newpage\clearpage
% contents=\begin{tex2html_wrap_indisplay}$\displaystyle \begin{array}<#131#>rclcl<#131#> \Pi(\pi/2,h,k) ;SPMamp; = ;SPMamp; \displaystyle \int_0^<#132#>\pi/2<#132#> \frac<#133#>d\psi<#133#><#319#>(1+h\sin^2 \ps...
\lthtmlinlinemathA{tex2html_wrap_indisplay522}%
$\displaystyle \begin{array}{rclcl}
\Pi(\pi/2,h,k) & = & \displaystyle \int_0^{\pi/2}
\frac{d\psi}{(1+h\sin^2 \psi)\sqrt{1-k^2\sin^2 \psi}} & = &
\mathbf{F}_3^*(k',h+1) \qquad (|k| < 1), \\[6mm]
\widehat{\Pi}(1,h,k) & = & \displaystyle \int_0^1
\frac{d\eta}{(1+h\eta^2)\sqrt{(1-\eta^2)(1-k^2\eta^2)}} & = &
\mathbf{F}_3^*(k',h+1) \qquad (|k| < 1).
\end{array}\end{array}$%
\lthtmlinlinemathZ
\hfill\lthtmlcheckvsize\clearpage}

{\newpage\clearpage
% contents=\begin{tex2html_wrap_inline}$ \equiv$\end{tex2html_wrap_inline}
\lthtmlinlinemathA{tex2html_wrap_inline524}%
$ \equiv$%
\lthtmlinlinemathZ
\hfill\lthtmlcheckvsize\clearpage}

{\newpage\clearpage
% contents=\begin{tex2html_wrap_inline}$ \approx$\end{tex2html_wrap_inline}
\lthtmlinlinemathA{tex2html_wrap_inline541}%
$ \approx$%
\lthtmlinlinemathZ
\hfill\lthtmlcheckvsize\clearpage}

{\newpage\clearpage
% contents=\begin{tex2html_wrap_inline}$ \ne$\end{tex2html_wrap_inline}
\lthtmlinlinemathA{tex2html_wrap_inline550}%
$ \ne$%
\lthtmlinlinemathZ
\hfill\lthtmlcheckvsize\clearpage}

{\newpage\clearpage
% contents=\begin{tex2html_wrap_inline}$ \le$\end{tex2html_wrap_inline}
\lthtmlinlinemathA{tex2html_wrap_inline560}%
$ \le$%
\lthtmlinlinemathZ
\hfill\lthtmlcheckvsize\clearpage}

{\newpage\clearpage
% contents=\begin{tex2html_wrap_inline}$ \Pi$\end{tex2html_wrap_inline}
\lthtmlinlinemathA{tex2html_wrap_inline565}%
$ \Pi$%
\lthtmlinlinemathZ
\hfill\lthtmlcheckvsize\clearpage}

{\newpage\clearpage
% contents=\begin{tex2html_wrap_indisplay}$\displaystyle \begin{array}<#250#>rcl<#250#> \lambda \mathrm<#251#>K<#251#>(k) + \mu \mathrm<#252#>E<#252#>(k) ;SPMamp; = ;SPMamp; \mathbf<#253#>G<#253#>(k',1,\lambda+\...
\lthtmlinlinemathA{tex2html_wrap_indisplay568}%
$\displaystyle \begin{array}{rcl}
\lambda \mathrm{K}(k) + \mu \mathrm{E}(k) & = &
\mathbf{G}(k',1,\lambda+\mu,\lambda+\mu {k'}^2) \\\lambda \mathrm{K}(k) + \mu \Pi(h,k) & = &
\mathbf{G}(k',h+1,\lambda+\mu,\lambda (h+1)+\mu)
\end{array}\end{array}$%
\lthtmlinlinemathZ
\hfill\lthtmlcheckvsize\clearpage}

{\newpage\clearpage
% contents=\begin{tex2html_wrap_indisplay}$\displaystyle \begin{array}<#259#>rcl<#259#> \mathrm<#260#>K<#260#>(k)                           ;SPMamp; = ;SPMamp; \mathbf<#261#>G<#261#>(k',1,1,1), \\  \mathrm<#262#...
\lthtmlinlinemathA{tex2html_wrap_indisplay570}%
$\displaystyle \begin{array}{rcl}
\mathrm{K}(k)                           & = & \mathbf{G}(k',1,1,1), \\\mathrm{E}(k)                           & = & \mathbf{G}(k',1,1,{k'}^2)\\(\mathrm{K}(k)-\mathrm{E}(k))/k^2       & = & \mathbf{G}(k',1,0,1), \\(\mathrm{K}(k)-{k'}^2\mathrm{E}(k))/k^2 & = & \mathbf{G}(k',1,1,0), \\\Pi(h,k)                                & = & \mathbf{G}(k',h+1,1,1),\\(\mathrm{K}(k)-\Pi(h,k))/h              & = & \mathbf{G}(k',h+1,0,1),\\\end{array}\end{array}$%
\lthtmlinlinemathZ
\hfill\lthtmlcheckvsize\clearpage}

{\newpage\clearpage
% contents=\begin{tex2html_wrap_inline}$ \ge$\end{tex2html_wrap_inline}
\lthtmlinlinemathA{tex2html_wrap_inline572}%
$ \ge$%
\lthtmlinlinemathZ
\hfill\lthtmlcheckvsize\clearpage}

{\newpage\clearpage
% contents=\begin{tex2html_wrap_inline}$ \Phi$\end{tex2html_wrap_inline}
\lthtmlinlinemathA{tex2html_wrap_inline584}%
$ \Phi$%
\lthtmlinlinemathZ
\hfill\lthtmlcheckvsize\clearpage}

{\newpage\clearpage
% contents=\begin{tex2html_wrap_inline}$ \Lambda_{0}^{}$\end{tex2html_wrap_inline}
\lthtmlinlinemathA{tex2html_wrap_inline586}%
$ \Lambda_{0}^{}$%
\lthtmlinlinemathZ
\hfill\lthtmlcheckvsize\clearpage}

{\newpage\clearpage
% contents=\begin{tex2html_wrap_indisplay}$\displaystyle \begin{array}<#323#>rcl@<#280#>\qquad<#280#>l<#323#> \mathbf<#281#>Z<#281#>(\Phi,k) ;SPMamp; = ;SPMamp; \displaystyle k^2 ;SPMthinsp; \frac<#282#>\sin \Ph...
\lthtmlinlinemathA{tex2html_wrap_indisplay589}%
$\displaystyle \begin{array}{rcl@{\qquad}l}
\mathbf{Z}(\Phi,k) & = & \displaystyle k^2 \,
\frac{\sin \Phi \cos \Phi}{\mathrm{K}(k)} \, \mathbf{G}(k',q,0,\sqrt{q})
& (q = \cos^2 \Phi + {k'}^2 \sin^2 \Phi) \\[4mm]
\Lambda_0(\Phi,k) & = & \displaystyle \frac{2}{\pi}
\sqrt{q} \sin \Phi \ \mathbf{G}(k',q,1,{k'}^2) &
(q = 1 + k^2 \tan^2 \Phi).
\end{array}\end{array}$%
\lthtmlinlinemathZ
\hfill\lthtmlcheckvsize\clearpage}

{\newpage\clearpage
% contents=\begin{tex2html_wrap_inline}$ \bullet$\end{tex2html_wrap_inline}
\lthtmlinlinemathA{tex2html_wrap_inline597}%
$ \bullet$%
\lthtmlinlinemathZ
\hfill\lthtmlcheckvsize\clearpage}


\end{document}
