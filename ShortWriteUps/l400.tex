\Version{PATCHY}                        \Routid{L400}
\Keywords{SOURCE FILE CODE CARD MAINTAIN PATCHY UPDATE}
\Author{J. Zoll}                        \Library{none}
\Submitter{}                             \Submitted{31.01.1972}
\Language{Fortran}                        \Revised{15.01.1977}
\Cernhead{Source Code Maintenance}
{\tt PATCHY} and the associated auxiliary programs serve in development,
maintenance, and inter-computer transport of source programs. Suitably
structured source files containing several versions of a given program
permit code selection and code modification (down to
single-statement-level) by simple control cards to {\tt YPATCHY}.
Compacting and structuring of card files for efficiency {\tt (YTOBIN)},
maintenance of compacted files at the deck level {\tt (YEDIT)},
creation of machine-independent, transportable files {\tt (YTOCETA)} and
listing of compacted files {\tt (YLIST)} and others are simple auxiliary
operations in this environment.
\Structure
Complete programs; executable modules exist on all machines at CERN
where the CERN Program Library is installed, normally in the directory
{\tt /cern/pro/bin}. \\
User Entry Names:
\begin{htmlonly}
\begin{tabular}{llllllll}
\end{htmlonly}
\begin{latexonly}
\begin{tabular}[t]{l*{7}{@{\hspace{4pt}}l}}
\end{latexonly}
\Rdef{YPATCHY}, & \Rdef{YEDIT},  & \Rdef{YTOBIN}, & \Rdef{YTOBCD}, &
\Rdef{YLIST},   & \Rdef{YTOCETA},& \Rdef{YFRCETA},& \Rdef{YCOMPAR}, \\
\Rdef{YSEARCH}, & \Rdef{YSHIFT}
\end{tabular} \\
\Usage
See {\bf Long Write-up} ({\tt PATCHY} Reference Manual).
\\ $\bullet$
