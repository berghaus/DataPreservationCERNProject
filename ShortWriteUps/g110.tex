% 24 oct 1994  ksk
\Version{LANDAU}               \Routid{G110}
\Keywords{DISTRIBUTION LANDAU RANDOM NUMBER INVERSE}
\Author{K.S. K\"olbig}             \Library{MATHLIB}
\Submitter{}              \Submitted{30.08.1985}
\Language{Fortran}            \Revised{15.03.1993}
\Cernhead{Landau Distribution}
The {\tt LANDAU} function subprogram package contains six independent
subprograms for the calculation of the following functions related to
the Landau distribution:
$$\begin{array}{ll@{ \quad = \quad }l}
\mbox{{\rm The density}}           & \phi(\lambda)
& \displaystyle \frac{1}{2\pi i} \
\int_{c-i\infty}^{c+i\infty} \exp(\lambda s + s \ln s) ds,
\\ [4mm]
\mbox{{\rm the distribution}}      & \Phi(\lambda)
& \displaystyle \int_{-\infty}^\lambda \phi(\lambda) d \lambda, \\[4mm]
\mbox{{\rm the derivative}}        & \phi'(\lambda)
& \displaystyle \frac{d\phi(\lambda)}{d\lambda},\\ [4mm]
\mbox{{\rm the first moment}}     & \Phi_1(x)
& \displaystyle \frac{1}{\Phi(x)} \ \int_{-\infty}^x \lambda
\phi(\lambda) d \lambda, \\ [4mm]
\mbox{{\rm the second moment}}     & \Phi_2(x)
& \displaystyle \frac{1}{\Phi(x)} \ \int_{-\infty}^x \lambda^2
\phi(\lambda) d \lambda,\\  [4mm]
\mbox{{\rm the inverse of} $\Phi(x)$}& \Psi(x)
& \displaystyle \Phi^{-1}(x).
\end{array}$$
The function $\Psi(x)$ can be used to generate Landau random numbers
(see {\bf Usage}).
\Structure
{\tt FUNCTION} subprograms \\
User Entry Names: \Rdef{DENLAN}, \Rdef{DISLAN}, \Rdef{DIFLAN},
                  \Rdef{XM1LAN}, \Rdef{XM2LAN}, \Rdef{RANLAN} \\
Obsolete User Entry Names: \Rdef{DSTLAN} $\equiv$ {\tt DISLAN}
\Usage
In any arithmetic expression,
\begin{center} \begin{tabular}{r@{\qquad has the value \qquad}l}
{\tt DENLAN(X)} & $\phi(\mathtt{X})$, \\
{\tt DISLAN(X)} & $\Phi(\mathtt{X})$, \\
{\tt DIFLAN(X)} & $\phi'(\mathtt{X})$, \\
{\tt XM1LAN(X)} & $\Phi_1(\mathtt{X})$,\\
{\tt XM2LAN(X)} & $\Phi_2(\mathtt{X})$,\\
{\tt RANLAN(X)} & $\Psi(\mathtt{X})$,
\end{tabular} \end{center}
where {\tt DENLAN}, {\tt DISLAN}, {\tt DIFLAN}, {\tt XM1LAN},
{\tt XM2LAN}, {\tt RANLAN} and {\tt X} are of type {\tt REAL}.
\par
To generate a set of Landau random numbers, {\tt RANLAN} should
be referenced repeatedly, using as argument a random number from a
uniform distribution over the interval (0,1).
\Method
Approximation by rational functions. For reason of speed, {\tt RANLAN}
proceeds mainly by table look-up and quadratic interpolation.
\Accuracy
At least six significant digits (five for {\tt RANLAN}) are correct.
\newpage
\Restrict
\begin{enumerate}
\item Underflow may occur for {\tt DENLAN}, {\tt DISLAN} and
{\tt DIFLAN} if {\tt X} is negative and (moderately) large.
\item No test is made whether {\tt X} for {\tt RANLAN} lies
outside the interval (0,1), and hence no error message is printed.
\end{enumerate}
\Notes
This program package is a version of the
{\it CPC Program Library} package {\tt LANDAU} (Ref. 1).
\Refer
\begin{enumerate}
\item K.S. K\"olbig and B. Schorr, A program package for the Landau
distribution, Computer Phys. Comm. {\bf 31} (1984) 97--111.
\end{enumerate}
$\bullet$
