\Version {GOPARM}                              \Routid{Z262}
\Keywords{GOPARM PROVIDE STEP STRING USER}
\Author{J. Ehrman}                         \Library{KERNLIB, IBM only}
\Submitter{}                                \Submitted{15.09.1978}
\Language{Assembler}                         \Revised{20.06.1985}
\Cernhead {Provide the User with the G Step PARM-String (IBM)}
{\tt GOPARM} returns to a Fortran program the character string
specified in the {\tt PARM} field of the {\tt JCL} corresponding to the
step being executed. (This is the {\tt G}-step when using the CERN
procedure {\tt JFORTCLG}).
\Structure
{\tt SUBROUTINE} subprogram \\
User Entry Names: \Rdef{GOPARM}
\Usage
\begin{verbatim}
    CALL GOPARM(LENGTH,PARMS)
\end{verbatim}
\begin{DLtt}{12345678}
\item [LENGTH]  ({\tt INTEGER}) Contains the length in
characters of the parameter string specified in the {\tt JCL} of the
step in which this routine is executed. $({\tt 0 \leq LENGTH \leq 100})$.
\item [PARMS] ({\tt CHARACTER*100}) Contains the {\tt LENGTH} characters
of the string.
\end{DLtt}
\Restrict
In the VM-CMS operating system {\tt GOPARM} must be called before any
Input/Output operations have been performed. The mechanism for passing
parameters depends on how a program is loaded. For a module type:
\par
Module--name parameter--string
\par
After a {\tt LOAD} the parameters must be placed on the subsequent
{\tt START} command:
\par
{\tt START} entry--name parameter--string
\par
If {\tt START} is used as an option of {\tt LOAD} then no parameters are
passed and {\tt LENGTH} is returned as zero. Note that any {\tt '('}
will be passed as part of the parameter-string.
\Notes
The form with parentheses must be used if the parameter string is to be
continued on a second {\tt JCL} card, since a value or subvalue cannot
be split over two {\tt JCL} cards.
\Examples
The following are all equivalent:
\begin{verbatim}
//EXEC JFORTCLG,GPRM='FIRST,SECOND,THIRD=YES'
//EXEC JFORTCLG,GPRM=(FIRST,SECOND,'THIRD=YES')
//EXEC JFORTCLG,GPRM=('FIRST','SECOND','THIRD=YES')
\end{verbatim}
and give the following parm-string of length {\tt 22}:
\begin{verbatim}
    FIRST,SECOND,THIRD=YES
\end{verbatim}
$\bullet$
