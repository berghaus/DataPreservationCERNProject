\Version {FNZERO}                            \Routid{Z204}
\Keywords{BLANK CDC FILE FNBLAN FNNUM FNZERO NAME ZERO}
\Author{J. Zoll}                           \Library{KERNLIB, Cray only}
\Submitter{}                                \Submitted{01.02.1979}
\Language{Fortran}                     \Revised{01.12.1981}
\Cernhead{Cray File Name with Blank or Zero Fill}
{\tt FNBLAN, FNZERO} produce blank- or zero-fill of
a word containing a Cray file name, 7 characters left-justified
with zero- or blank-fill. The initial contents of characters
8-10 are ignored. Filling operates right to left, replacing
zero (blank) by blank (zero) until the first non-zero
(non-blank) character is found.
\par
{\tt FNNUM} acts as do-nothing if the parameter
is already a file-name. If the parameter is an integer $n$ with
$0 < n < 99$, the routine interpretes $n$ as a logical unit number
and converts it to the file name {\tt TAPEn} with zero-fill.
\Structure
{\tt SUBROUTINE} subprograms \\
User Entry Names: \Rdef{FNBLAN}, \Rdef{FNZERO}, \Rdef{FNNUM}
\Usage
\begin{verbatim}
    CALL FNBLAN(NAME)
\end{verbatim}
creates blank-fill in {\tt NAME} as described above.
\begin{verbatim}
    CALL FNZERO(NAME)
\end{verbatim}
creates zero-fill in {\tt NAME} as described above.
\begin{verbatim}
    CALL FNNUM(NAME)
\end{verbatim}
converts logical unit number {\tt NAME} to a CDC file name as described
above.
\Notes
For {\tt FNBLAN} and {\tt FNZERO} the initial content of {\tt NAME}
must be a
file name, not a logical unit number.
\\ $\bullet$
