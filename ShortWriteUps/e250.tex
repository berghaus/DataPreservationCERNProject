\Version {LFIT}                        \Routid{E250}
\Keywords{FIT FITTING LEAST-SQUARE STRAIGHT LINE}
\Author{M. Metcalf}                     \Library{MATHLIB}
 \Submitter{}                           \Submitted{01.05.1977}
\Language{Fortran}                \Revised{27.11.1984}
\Cernhead {Least-Squares Fit to Straight Line}
Given a vector of values $Y$ measured at the points $X$, {\tt LFIT}
and {\tt LFITW} find the best least-squares fit to the linear
relationship $Y=aX+b$. {\tt LFIT} performs an unweighted fit and
{\tt LFITW} takes account of a given vector of weights.
Both subroutines have an option for skipping missing
points without shifting the points of the vector $X$.
\Structure
{\tt SUBROUTINE} subprogram\\
User Entry Names: \Rdef{LFIT}, \Rdef{LFITW}
\Usage
\begin{DLtt}{1234}
\item[] {\tt CALL LFIT(X,Y,L,KEY,A,B,VAR)} \qquad or
\item[] {\tt CALL LFITW(X,Y,W,L,KEY,A,B,VAR)}
\end{DLtt}
\begin{DLtt}{123456}
\item[X] ({\tt REAL}) Vector of abscissae.
\item[Y] ({\tt REAL}) Vector of values corresponding to points {\tt X}.
\item[W] ({\tt REAL}) Vector of weights (for {\tt LFITW} only).
\item[L] ({\tt INTEGER}) Length of vectors {\tt X}, {\tt Y} and {\tt W}.
\item[KEY] ({\tt INTEGER}) \\
$\mathtt{= 0:}$ indicates that any points where $\mathtt{Y=0}$ are to be
skipped, \\
$\mathtt{= 1:}$ indicates that all {\tt L} points are to be used.
\item[A] ({\tt REAL}) Fitted slope $a$.
\item[B] ({\tt REAL}) Fitted constant term $b$.
\item[VAR] ({\tt REAL}) Residual sum of squares divided by
($\mathtt{L-2}$) indicating the badness of fit.
\end{DLtt}
\Refer
\begin{enumerate}
\item D.H. Menzel, Fundamental Formulas of Physics, Dover Publ., New York
(1960) 116.
\end{enumerate}
$\bullet$
