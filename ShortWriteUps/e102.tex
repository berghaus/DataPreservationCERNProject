% 03 nov 94 ksk
\Version {MAXIZE}                       \Routid{E102}
\Keywords{ARRAY ELEMENT MAXIMUM MINIMUM}
\Author{K.S. K\"olbig, H. Lipps}      \Library{MATHLIB}
\Submitter{}                            \Submitted{29.08.1968}
\Language{Fortran}                      \Revised{01.12.1994}
\Cernhead {Maximum and Minimum Elements of Arrays}
Function subprograms {\tt MAXIZE}, {\tt MAXRZE}, {\tt MAXDFZ} and
{\tt MINIZE}, {\tt MINRZE}, {\tt MINDFZ} give give the positions of
the maximum and minimum elements of a one-dimensional array.
\par
On CDC and Cray computers, the double-precision versions
{\tt MAXDZE} and {\tt MINDZE} are not available.
\Structure
{\tt FUNCTION} subprograms \\
User Entry Names: \Rdef{MAXIZE}, \Rdef{MAXRZE}, \Rdef{MAXDZE},
                  \Rdef{MINIZE}, \Rdef{MINRZE}, \Rdef{MINDZE} \\
Obsolete User Entry Names: \Rdef{MAXFZE} $\equiv$ {\tt MAXRZE},
                           \Rdef{MINFZE} $\equiv$ {\tt MINRZE}
\Usage
In any arithmetic expression, for $\mathtt{t=I}$ (type {\tt INTEGER}),
$\mathtt{t=R}$ (type {\tt REAL}), $\mathtt{t=D}$
(type {\tt DOUBLE PRECISION}),
\begin{center}
{\tt MAXtZE(A(J),N)} \qquad and \qquad {\tt MINtZE(A(J),N)}
\end{center}
has the {\tt INTEGER} value of the location of, respectively,
the maximum and minimum elements of the {\tt N} successive elements of
the array {\tt A}, {\bf relative to the element} {\tt A(J)}, where
{\tt A} is of type {\tt t}.
\Notes
\begin{enumerate}
\item If there is more than one location at which the maximum or
minimum is attained, the first location is returned as the function
value in each case.
\item If $\mathtt{N < 1}$ the function value is {\tt 1}.
\item Clearly, {\tt N+J} should not exceed the dimension of the array
{\tt A}.
\item The obsolete older entries {\tt MAXFZE} (for {\tt MAXRZE}) and
{\tt MINFZE} (for {\tt MINRZE}) are kept for a transitional period.
They will eventually disappear.
\end{enumerate}
$\bullet$
