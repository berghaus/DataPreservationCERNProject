%  18.10.94  ksk
\Version {RTCLGN}                    \Routid{U112}
\Keywords{CLEBSCH-GORDAN COEFFICIENTS RATIONAL PRIME REPRESENTATION}
\Author{K.S. K\"olbig}   \Library{MATHLIB}
\Submitter{}             \Submitted{15.10.1994}
\Language{Fortran}                    %  \Revised{}
\Cernhead{Clebsch-Gordan Coefficients in Rational Form}
Function subprogram {\tt RTCLGN} calculates the (signed) square of
the Clebsch-Gordan coefficient in rational form and in powers of
prime numbers. In terms of the Wigner-3$j$ symbol, this
coefficient is defined by
\begin{eqnarray*}
C & = & (j_1\,j_2\,m_1\,m_2\,|\,j_1\,j_2\,j_3\,m_3)
\ = \ \displaystyle (-1)^{j_1-j_2+m_3}\,\sqrt{2j_3+1}
\left(\begin{array}{ccc}
j_1 & j_2 & j_3 \\ m_1 & m_2 & -m_3 \end{array}\right).
\end{eqnarray*}
All $j_i$ and $m_i$ must have integral or half-integral values
(see {\bf Notes}). For definitions and notations see Ref. 1.
\par
On computers other than CDC and Cray, only the double-precision version
{\tt DTCLGN} is available. On CDC and Cray computers, only the
single-precision version {\tt RTCLGN} is available.
\Structure
{\tt SUBROUTINE} subprogram \\
User Entry Names: \Rdef{RTCLGN} \\
Files Referenced: {\tt Unit 6}
\Usage
For $\mathtt{t=R}$ (type {\tt REAL}), $\mathtt{t=D}$ (type
{\tt DOUBLE PRECISION}),
\begin{verbatim}
    CALL tTCLGN(JJ1,JJ2,JJ3,MM1,MM2,MM3,RNUM,RDEN,KPEX)
\end{verbatim}
\begin{DLtt}{12345678912}
\item[JJ1,JJ2,JJ3] ({\tt INTEGER}) The $j$-parameters multiplied by
two, i.e. {\tt JJ1}$=2j_1$ etc.
\item[MM1,MM2,MM3] ({\tt INTEGER}) The $m$-parameters multiplied by
two, i.e. {\tt MM1}$=2m_1$ etc.
\item[RNUM] (type according to {\tt t}) Contains, on exit,
the signed numerator of $C^2$.
\item[RDEN] (type according to {\tt t}) Contains, on exit, the
denominator of $C^2$.
\item[KPEX] ({\tt INTEGER}) Array of length 40 at least. Contains, on
exit, the exponents $k_n$ in the expression
\begin{eqnarray*}
C^2 & = & \displaystyle \prod_{n=1}^{40} p_n^{k_n},
\end{eqnarray*}
where $p_1=2,\,p_2=3,\,p_3=5,\ldots,\,p_{40}=173$ are the first 40
prime numbers.
\end{DLtt}
\Notes
A Clebsch-Gordan coefficient
$(j_1\,j_2\,m_1\,m_2\,|\,j_1\,j_2\,j_3\,m_3)$
is considered to be zero unless simultaneously \\
\begin{tabular}[t]{ll}
(i) & $j_i$ and $m_i$ have both either integral or half-integral
values (each $i$), \\
(ii) & $j_i \ge |m_i| \ge 0$ \ (each $i$), \\
(iii) & $m_1+m_2=m_3$, \\
(iv) & $j_1+j_2+j_3$ \ is an integer and \
$j_1+j_2 \ge j_3, \quad j_2+j_3 \ge j_1, \quad j_3+j_1 \ge j_2$.
\end{tabular}
\par
In this case, $\mathtt{RNUM = 0}$, $\mathtt{RDEN = 1}$ or
$\mathtt{DNUM = 0}$, $\mathtt{DDEN = 1}$, respectively, and
$\mathtt{KPEX(n) = 0,\,(n=1,\ldots,40)}$.
\newpage
\Source
This subroutine is based on an earlier version by H. Yoshiki.
\Errorh
Error {\tt U112.1:} The calculation requires a prime number $p_n$ with
$n>40$. \\
In this case, $\mathtt{DNUM = 0}$, $\mathtt{DDEN = 1}$,
$\mathtt{KPEX(n) = 0,\,(n=1,\ldots,40)}$. A message is written on
{\tt Unit 6} unless subroutine {\tt MTLSET} (N002) has been called.
\Refer
\begin{enumerate}
\item R.D. Cowan, The theory of atomic structure and spectra,
(Univ. of California Press, Berkeley CA 1981) 142--144
\end{enumerate}
$\bullet$
