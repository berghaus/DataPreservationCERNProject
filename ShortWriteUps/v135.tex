% 26.09.94 ksk
\Version {RNGAMA}                               \Routid{V135}
\Keywords{CHISQUARE DISTRIBUTION NUMBER RANDOM}
\Author{F. James, K.S. K\"olbig}                 \Library{MATHLIB}
\Submitter{}                                     \Submitted{15.10.1994}
\Language{Fortran}                          %\Revised{}
\Cernhead {Gamma or Chi-Square Random Numbers}
Function subprogram {\tt RANGAM} generates a positive random number
$x$ according to the gamma distribution with parameter $p>0$,
i.e., according to the density
$$ P(t < x < t+dt) \ = \
\displaystyle \frac{1}{\Gamma (p)}\, t^{p-1}e^{-t}dt. $$
A special case is the $\chi ^2$-distribution with $N$ degrees
of freedom
$$ \chi^2(t < 2x < t+dt) \ = \ \displaystyle
\frac{1}{\sqrt{2^N}\,\Gamma(\textstyle \frac{1}{2}N)}\,
t^{\frac{1}{2}N-1}\,e^{-\frac{1}{2}t}\,dt.$$
\Structure
{\tt FUNCTION} subprogram   \\
User Entry Names: \Rdef{RNGAMA}\\
External References: \Rind {RANLUX}{V115}, \Rind{RNORMX}{V120}
\Usage
In any arithmetic expression,
\begin{verbatim}
    RNGAMA(P)
\end{verbatim}
has the value of a gamma-distributed random number, where
$\mathtt{P > 0}$ is of type {\tt REAL}. The value of {\tt P} may
vary from call to call without influencing the efficiency.
\Method
For integral values of $p \le 15$, the logarithm of the product of
$p$ uniform random numbers is used. For any value of $p > 15$,
the Wilson-Hilferty approximation (a transformed normal distribution)
is used. For all other $p$, Johnk's algorithm is used.
\Notes
The routine is fast for small integer values of $p$,  and for $p > 15$,
(one Gaussian random number and one square root, plus a few
multiplications). Non-integral values of $p < 15$ are rather slow.
\Examples
\begin{verbatim}
    CHI2 = 2*RNGAMA(0.5*N)
\end{verbatim}
sets {\tt CHI2} to a random number distributed as $\chi ^2$ with
{\tt N} degrees of freedom.
\\ $\bullet$
