%  05 jan 94  ksk
\Version{NORBAS}                                \Routid{E210}
\Keywords{POLYNOMIAL SPLINE NORMALIZED B-SPLINE APPROXIMATION INTERPOLATION LEAST SQUARES VARIATION DIMINISHING}
\Author{W. M\"onch, B. Schorr}                   \Library{MATHLIB}
\Submitter{W. M\"onch}                           \Submitted{15.03.1993}
\Language{Fortran}                               \Revised{}
\Cernhead{Polynomial Splines / Normalized B-Splines}
{\tt NORBAS} ({\tt NOR}malized {\tt BA}sis {\tt S}plines) is a portable
collection of subprograms for various calculations with polynomial
splines in one dimension ({\bf 1D}) and in two dimensions ({\bf 2D}).
The polynomial splines are represented as linear combinations of
normalized basis splines (B-splines).
\par
On computers other than CDC or Cray, only the double-precision versions
{\tt DSPKN1}, etc. are available. On CDC and Cray computers, only the
single-precision versions {\tt RSPKN1}, etc. are available.
\par
The following outline provides the background material and the
notations needed for describing the subprograms
and their parameters. For further information about splines and their
applications see {\bf References}, in particular Ref. 7. \\[3mm]
{\bf Case (1D):}
\begin{DL}{123456}
\item[$k$] Degree (order $-1$) of the B-spline \, $(0 \le k \le 25)$.
\item[$m$] Number of spline-knots \, $( m \ge 2 \, k+2 )$.
\item[$i$] Index of the B-spline \, $(1 \le i \le m-k-1)$.
\item[$\tau$] Set of $m$ spline-knots $\tau = \{t_1,t_2, \ldots ,t_m \}$,
 in non-decreasing order, with multiplicity $\le k+1$
 (i.e. no more than $k+1$ knots coincide).
\item[{$[a,b]$}] Interval, defined by $a=t_{k+1}$ , $b=t_{m-k}$.
\item[$B_i(x)$] Normalized B-spline of degree $k$ over $\tau$ with
 index $i$. The value of $B_i(x)$ is identically zero outside the
 interval $t_i \le x \le t_{i+k+1}$ , and the normalization of $B_i(x)$
 is such that
 $$ \int_{- \infty}^{+ \infty} B_i(x)\,dx \ = \ \frac{t_{i+k+1}-t_i}{k+1}
 \qquad (i = 1, \ldots , m-k-1). $$
\item[$s(x)$] Polynomial spline at $x \in [a,b]$ in B-spline
 representation
\end{DL}
$$ y \ = \ s(x) \ = \ \sum_{i=1}^{m-k-1} \, c_i\,B_i(x) \, .$$
\par
{\bf Spline interpolation} to a data set:
\par
Given a data set $\{x_l,y_l\}_{l=1, \ldots ,n}$ ; determine
coefficients $\{c_i\}_{i=1, \ldots ,n}$ of a polynomial interpolation
spline $y = s(x)$ in B-spline representation with degree $k$ over a set
$\tau$ of $m=n+k+1$ knots, such that the following relations
(interpolation conditions) hold:
$$s(x_l) \ = \ y_l \qquad (l=1, \ldots ,n).$$
The existence of a solution of this interpolation problem
depends on an appropriate choice of the spline-knots (Ref. 7,
Theorem XIII.1 (Schoenberg-Whitney)).
\par
{\bf Least squares spline approximation} to a data set:
\par
Given a data set $\{x_l,y_l\}_{l=1, \ldots ,n}$ ; determine
coefficients $\{c_i\}_{i=1, \ldots ,m-k-1}$ of a polynomial approximation
spline $y = s(x)$ in B-spline representation with degree $k$ over a set
$\tau$ of $m \le n+k+1$ knots, such that following least squares
 problem is solved:
$$\phi (c_1, \ldots ,c_{m-k-1}) \ = \ \sum_{l=1}^{n}\,
  \left(y_l - s(x_l)\right)^2 \ = \ \mathtt{min \,!}$$
\newpage
{\bf Variation diminishing spline approximation} to a function
(Schoenberg):
\par
For a given function $y = f(x)$ on $[a,b]$ this spline approximation is
defined by $y = s(x)$, with \\
(Ref. 7, p. 158-162)
$$c_i \ = \ f(x_i); \qquad x_i \ = \ (t_{i+1}+\cdots+t_{i+k})/k
\quad (i=1,\ldots,m-k-1;\,k \ge 1). $$
{\bf Case (2D):}
\begin{DL}{12345678}
\item[$kx$] Degree of one-dimensional B-splines in $x$-direction
 $(0 \le kx \le 25)$.
\item[$ky$] Degree of one-dimensional B-splines in $y$-direction
 $(0 \le ky \le 25)$.
\item[$mx$] Number of spline-knots in $x$-direction \,
 $(mx \ge 2 \, kx+2)$.
\item[$my$] Number of spline-knots in $y$-direction \,
 $(my \ge 2 \, ky+2)$.
\item[$i$] Index of B-spline \, $(1 \le i \le mx-kx-1)$ in
 $x$-direction.
\item[$j$] Index of B-spline \, $(1 \le j \le my-ky-1)$ in
 $y$-direction.
\item[$\tau_x$] Set of $mx$ spline-knots
 $\tau_x = \{ t_{x,1},t_{x,2}, \ldots ,t_{x,mx}\}$ in $x$-direction,
 in non-decreasing order, with multiplicity $\le kx+1$
 (i.e. no more than $kx+1$ knots coincide).
\item[$\tau_y$] Set of $my$ spline-knots
 $\tau_y = \{ t_{y,1},t_{y,2}, \ldots ,t_{y,my}\}$ in $y$-direction,
 in non-decreasing order, with multiplicity  $\le ky+1$
 (i.e. no more than $ky+1$ knots coincide).
\item[{$[a_x,b_x]$}] Interval in $x$-direction,
 defined by $a_x=t_{x,kx+1}$ , $b_x=t_{x,mx-kx}$.
\item[{$[a_y,b_y]$}] Interval in $y$-direction,
 defined by $a_y=t_{y,ky+1}$ , $b_y=t_{y,my-ky}$.
\item[$B_i(x)$] B-spline of degree $kx$ over $\tau_x$ with index $i$.
\item[$B_j(y)$] B-spline of degree $ky$ over $\tau_y$ with index $j$.
\item[$B_{i,j}(x,y)$] Product $B_{i,j}(x,y) = B_i(x) \, B_j(y)$ of
 two one-dimensional B-splines.
\item[$s(x,y)$] Two-dimensional polynomial spline at
 $(x,y) \in [a_x,b_x] \times [a_y,b_y]$
 in B-spline representation
\end{DL}
$$ z \ = \ s(x,y) \ = \ \sum_{i=1}^{mx-kx-1}\,\sum_{j=1}^{my-ky-1}\,
c_{i,j}\,B_{i,j}(x,y).$$

{\bf Spline interpolation} to a data set: \\
Given a data set $\{x_{lx},y_{ly},z_{lx,ly}\}_{lx=1, \ldots ,nx ;
ly=1, \ldots , ny}$ ; determine coefficients
$\{c_{i,j}\}_{i=1, \ldots ,nx ; j=1, \ldots ,ny}$ of a
two-dimensional polynomial interpolation spline $z = s(x,y)$ in B-spline
representation with degrees $kx$ , $ky$ over the sets $\tau_x$
of $mx=nx+kx+1$ knots in $x$-direction and $\tau_y$ of $my=ny+ky+1$
knots in $y$-direction, such that following relations
(interpolation conditions) hold:
$$s(x_{lx},y_{ly}) = z_{lx,ly} \qquad
(lx=1, \ldots ,nx \, ; \, ly=1, \ldots ,ny) .$$
The existence of a solution of this interpolation problem depends
on an appropriate choice of the spline-knots $\tau_x$ , $\tau_y$ in the
two-dimensional interpolation area $[a_x,b_x] \times [a_y,b_y]$ .
\par
{\bf Least squares spline approximation} to a data set:
\par
Given a data set $\{x_{lx},y_{ly},z_{lx,ly}\}_{lx=1, \ldots ,nx ;
ly=1, \ldots , ny}$ ; determine coefficients
$\{c_{i,j}\}_{i=1, \ldots ,nx ; j=1, \ldots ,ny}$ of a
two-dimensional polynomial approximation spline $z = s(x,y)$ in B-spline
representation with degrees $kx$, $ky$ over the sets $\tau_x$
of $mx \le nx+kx+1$ knots in $x$-direction and $\tau_y$ of
$my \le ny+ky+1$ knots in $y$-direction, such that following
least squares problem is solved:
$$\phi (c_{1,1}, \ldots ,c_{mx-kx-1,my-ky-1}) = \,
\sum_{lx=1}^{nx} \, \sum_{ly=1}^{ny} \,
\left(z_{lx,ly}-s(x_{lx},y_{ly})\right)^2 \ = \ \mathtt{min\,!}$$
{\bf Variation diminishing spline approximation} to a function:
\par
For a given function $z = f(x,y)$ on $[a_x,b_y] \times [a_y,b_y]$ this
two-dimensional spline approximation is defined by $z = s(x,y)$
on $[a_x,b_x] \times [a_y,b_y]$, with
\[ \begin{array}{r@{\ =\ }l@{\qquad}r@{\ =\ }l@{\quad}l}
c_{i,j} & f(x_i,y_j); & x_i & (t_{x,i+1}+\cdots+t_{x,i+kx})/kx &
                               (i=1,\ldots,mx-kx-1;\,kx \ge 1), \\
\multicolumn{2}{c}{}  & y_j & (t_{y,j+1}+\cdots+t_{y,j+ky})/ky &
                               (j=1,\ldots,my-ky-1;\,ky \ge 1).
\end{array} \]
\newpage
The package {\tt NORBAS} contains {\tt FUNCTION} and {\tt SUBROUTINE}
subprograms for solving the following problems.
{\bf To calculate:}
\begin{DL}{123}
\item[{\bf (K)}]
A set $\tau$ of $m$ spline-knots
in the interval $[a,b]$ for normalized B-splines $B_i(x)$ of
degree $k$, use subprogram {\tt tSPKN1} ({\bf 1D}).
The knots are in non-decreasing order and determined in such a way that
the first $k+1$ knots coincide with $a$, the last $k+1$ knots coincide
with $b$, and the remaining $(m-2\,k-2)$ knots are equidistant
in $(a,b)$.
\par
Two sets $\tau_x$ , $\tau_y$ of spline-knots
in $[a_x,b_x]$ and $[a_y,b_y]$ for B-splines $B_{i,j}(x,y)$
of degrees $kx$ and $ky$ in $x$- and $y$-direction, use subprogram
{\tt tSPKN2} ({\bf 2D}).
$\tau_x$ and $\tau_y$, are calculated by the same formulae
in $x$-, and $y$-direction, as in case ({\bf 1D}).
\item[{\bf (B)}]
The function $B_i(x)$,
\begin{center}
the $n$-th derivative \quad
$\displaystyle \frac{d^n B_i(x)}{dx^n}$, \qquad
or the integral \quad $\displaystyle \int_{-\infty}^x B_i(\xi)\,d\xi$
\end{center}
of a normalized B-spline $B_i(x)$,
with fixed degree $k$ and index $i$ over a set $\tau$ of
spline-knots, use subprogram: {\tt tSPNB1} ({\bf 1D}).
\par
The function $B_{i,j}(x,y)$,
\begin{center}
the partial derivative \quad
$\displaystyle \frac{\partial^{nx}\,\partial^{ny}\,B_{i,j}(x,y)}
{\partial x^{nx}\,\partial y^{ny}}$, \qquad
or the integral \quad $\displaystyle
\int_{-\infty}^x\,\int_{-\infty}^y\,B_{i,j}(\xi,\eta)\,d\eta\,d\xi$
\end{center}
of a two-dimensional B-spline $B_{i,j}(x,y)$, with fixed
degrees $kx$, $ky$ and indices $i$, $j$  over the sets $\tau_x$,
$\tau_y$ of spline-knots, use subprogram {\tt tSPNB2} ({\bf 2D}).
\item[{\bf (P)}]
The function $s(x)$,
\begin{center}
the $n$-th derivative \quad
$\displaystyle \frac{d^n s(x)}{dx^n}$, \qquad
or the integral \quad $\displaystyle \int_{-\infty}^x s(\xi)\,d\xi$
\end{center}
of a polynomial spline $y = s(x)$ in B-spline representation with
given coefficients $c_i$, use subprogram {\tt tSPPS1} ({\bf 1D}).
\par
The function $s(x,y)$,
\begin{center}
the partial derivative \quad
$\displaystyle \frac{\partial^{nx}\,\partial^{ny}\,s(x,y)}
{\partial x^{nx}\,\partial y^{ny}}$, \qquad
or the integral \quad $\displaystyle
\int_{-\infty}^x\,\int_{-\infty}^y\,s(\xi,\eta)\,d\eta\,d\xi$
\end{center}
of a two-dimensional polynomial spline $z = s(x,y)$
in B-spline representation with given coefficients $c_{i,j}$, use
subprogram {\tt tSPPS2} ({\bf 2D}).
\item[{\bf (I)}]
The coefficients $c_i$ of a one-dimensional
polynomial interpolation spline $y = s(x)$ in B-spline
representation to a user-supplied data set $\{x_l,y_l\}$, use
subprogram {\tt tSPIN1} ({\bf 1D}).
\par
The coefficients $c_{i,j}$ of a two-dimensional
polynomial interpolation spline $z = s(x,y)$ in B-spline
representation to a user-supplied data set
$\{x_{lx},y_{ly},z_{lx,ly}\}$, use subprogram
{\tt tSPIN2} ({\bf 2D}).
\item[{\bf (A)}]
The coefficients $c_i$ of a one-dimensional
polynomial least squares approximation spline $y=s(x)$ in
B-spline representation to a user-supplied data set $\{x_l,y_l\}$,
use subprogram {\tt tSPAP1} ({\bf 1D}).
\par
The coefficients $c_{i,j}$ of a two-dimensional
polynomial least squares approximation spline $z=s(x,y)$ in
B-spline representation to a user-supplied data set
$\{x_{lx},y_{ly},z_{lx,ly}\}$ ,
use subprogram {\tt tSPAP2} ({\bf 2D}).
\item[{\bf (V)}]
The coefficients $c_i$ of a one-dimensional
polynomial variation diminishing spline approximation $y = s(x)$
in B-spline representation to a user-supplied function $y = f(x)$,
use subprogram {\tt tSPVD1} ({\bf 1D}).
\par
The coefficients $c_{i,j}$ of a two-dimensional
polynomial variation diminishing spline approximation $z = s(x,y)$
in B-spline representation to a user-supplied function
$z = f(x,y)$, use subprogram {\tt tSPVD2} ({\bf 2D}).
\item[{\bf (D)}]
From given coefficients $c_i$ of a
one-dimensional polynomial spline $y = s(x)$ in B-spline
representation, the corresponding coefficients $d_i$ of its $n$-th
derivative $d^n s(x)/dx^n$ in B-spline representation,
use subprogram {\tt tSPCD1} ({\bf 1D}).
\par
From given coefficients $c_{i,j}$ of a
two-dimensional polynomial spline $z = s(x,y)$ in B-spline
representation, the corresponding coefficients $d_{i,j}$ of its
partial derivative
$\partial^{nx}\,\partial^{ny}\,s(x,y)/\{\partial x^{nx}\,
\partial y^{ny} \}$ in B-spline representation,
use subprogram {\tt tSPCD2} ({\bf 2D}).
\end{DL}
\newpage
\Structure
{\tt SUBROUTINE} and {\tt FUNCTION} subprograms\\
User Entry Names:
\begin{htmlonly}
\begin{tabular}{llllllll}
\end{htmlonly}
\begin{latexonly}
\begin{tabular}[t]{l*{7}{@{\hspace{4pt}}l}}
\end{latexonly}
\Rdef{RSPKN1}, & \Rdef{RSPKN2}, & \Rdef{RSPNB1}, & \Rdef{RSPNB2}, &
\Rdef{RSPPS1}, & \Rdef{RSPPS2}, & \Rdef{RSPIN1}, & \Rdef{RSPIN2}, \\
\Rdef{RSPAP1}, & \Rdef{RSPAP2}, & \Rdef{RSPVD1}, & \Rdef{RSPVD2}, &
\Rdef{RSPCD1}, & \Rdef{RSPCD2}, \\
\Rdef{DSPKN1}, & \Rdef{DSPKN2}, & \Rdef{DSPNB1}, & \Rdef{DSPNB2}, &
\Rdef{DSPPS1}, & \Rdef{DSPPS2}, & \Rdef{DSPIN1}, & \Rdef{DSPIN2}, \\
\Rdef{DSPAP1}, & \Rdef{DSPAP2}, & \Rdef{DSPVD1}, & \Rdef{DSPVD2}, &
\Rdef{DSPCD1}, & \Rdef{DSPCD2}, \\
\end{tabular} \\[1mm]
Internal Entry Names: {\tt RSPAS1}, {\tt RSPAS2}, {\tt RSPLKK},
                      {\tt DSPAS1}  {\tt DSPAS2}, {\tt DSPLKK} \\
Files Referenced: {\tt Unit 6} \\
External References:
\begin{htmlonly}
\begin{tabular}{llllllll}
\end{htmlonly}
\begin{latexonly}
  \begin{tabular}[t]{@{}llll}
\end{latexonly}
     \Rind{RGBTRF}{F001},&\Rind{RGBTRS}{F001},&\Rind{RGESVD}{F001}, \\
     \Rind{DGBTRF}{F001},&\Rind{DGBTRS}{F001},&\Rind{DGESVD}{F001}, \\
     \Rind{RVSET}{F002}, &\Rind{RVSUM}{F002}, &\Rind{RVCPY}{F002},&
     \Rind{RVMPY}{F002},                                            \\
     \Rind{DVSET}{F002}, &\Rind{DVSUM}{F002}, &\Rind{DVCPY}{F002},&
     \Rind{DVMPY}{F002},                                            \\
     \Rind{RMCPY}{F003}, &\Rind{RMMPY}{F003}, &\Rind{DMCPY}{F003},&
     \Rind{DMMPY}{F003},                                            \\
     \Rind{MTLMTR}{N002},&\Rind{ABEND}{Z035}.                       \\
     \multicolumn{4}{@{}l}{User-supplied {\tt FUNCTION} subprogram}
  \end{tabular}
\Usage
For $\mathtt{t = R}$ (type {\tt REAL}),
$\mathtt{t = D}$ (type {\tt DOUBLE PRECISION}):

{\bf (K) Knots}
\begin{verbatim}
    CALL tSPKN1(K,M,A,B,T,NERR)
    CALL tSPKN2(KX,KY,MX,MY,AX,BX,AY,BY,TX,TY,NERR)
\end{verbatim}
{\bf (B) Normalized B-splines}
\begin{verbatim}
    tSPNB1(K,M,I,NDER,X,T,NERR)
    tSPNB2(KX,KY,MX,MY,I,J,NDERX,NDERY,X,Y,TX,TY,NERR)
\end{verbatim}
{\bf (P) Polynomial spline}
\begin{verbatim}
    tSPPS1(K,M,NDER,X,T,C,NERR)
    tSPPS2(KX,KY,MX,MY,NDERX,NDERY,X,Y,TX,TY,C,NDIMC,W,NERR)
\end{verbatim}
{\bf (I) Spline interpolation}
\begin{verbatim}
    CALL tSPIN1(K,N,XI,YI,KNOT,T,C,W,IW,NERR)
    CALL tSPIN2(KX,KY,NX,NY,XI,YI,ZI,NDIMZ,KNOT,TX,TY,C,NDIMC,W,IW,NERR)
\end{verbatim}
{\bf (A) Least squares spline approximation}
\begin{verbatim}
    CALL tSPAP1(K,M,N,XI,YI,KNOT,T,C,W,NW,NERR)
    CALL tSPAP2(KX,KY,MX,MY,NX,NY,XI,YI,ZI,NDIMZ,KNOT,TX,TY,C,NDIMC,W,NW,NERR)
\end{verbatim}
{\bf (V) Variation diminishing spline approximation}
\begin{verbatim}
    CALL tSPVD1(F,K,M,T,C,NERR)
    CALL tSPVD2(F,KX,KY,MX,MY,TX,TY,C,NDIMC,NERR)
\end{verbatim}
{\bf (D) Coefficients of derivatives}
\begin{verbatim}
    CALL tSPCD1(K,M,NDER,T,C,D,NERR)
    CALL tSPCD2(KX,KY,MX,MY,NDERX,NDERY,TX,TY,C,NDIMC,D,NERR)
\end{verbatim}
\newpage
{\bf Case (1D):}
\begin{DLtt}{1234}
\item[F] Name of a user-upplied {\tt FUNCTION} subprogram,
declared {\tt EXTERNAL} in the calling program.
This subprogram must provide the value of the function $y = f(x)$
for variation diminishing spline approximation.
\item[K]  ({\tt INTEGER}) Degree of B-splines
($\mathtt{1 \le K \le 25}$ for {\tt tSPVD1},
 $\mathtt{0 \le K \le 25}$ otherwise).
\item[M]  ({\tt INTEGER}) Number of knots ($\mathtt{\ge 2*K+2}$).
\item[I]  ({\tt INTEGER}) Index of B-splines
 ($\mathtt{1 \le I \le M-K-1}$).
\item[N] ({\tt INTEGER}) Number of data points $\{x_l,y_l\}$
($\mathtt{\ge K+1}$ for spline interpolation ({\tt tSPIN1});
$\mathtt{\ge M-K-1}$ for spline approximation ({\tt tSPAP1})).
\item[NDER] ({\tt INTEGER}) Selects one out of
three cases: (i) integral, (ii) function value, or (iii) derivative. \\
($\mathtt{-1 \le NDER \le K}$ for {\tt tSPNB1} and {\tt tSPPS1};\,
 $\mathtt{1 \le NDER \le K}$ for {\tt tSPCD1}).
\item[]$\mathtt{=-1:}$  Calculation of the integral of $B_i(x)$
({\tt tSPNB1}), or the integral of $s(x)$ ({\tt tSPPS1}).
\item[] $\mathtt{=0:}$ Calculation of the function value $B_i(x)$
({\tt tSPNB1}), or the function value $s(x)$ ({\tt tSPPS1}).
\item[] $\mathtt{\ge 1:}$ \parbox[t]{140mm}{
 Calculation of the {\tt NDER}-th derivative of $B_i(x)$ ({\tt tSPNB1}),
 or the {\tt NDER}-th derivative of $s(x)$ ({\tt tSPPS1}).}
\item[X]  (Type according to {\tt t}) Independent variable $x$ of
 polynomial spline $s(x)$ or B-spline $B_i(x)$.
\item[XI] (Type according to {\tt t}) One-dimensional array of length
 $\mathtt{\ge N}$. On entry, $\mathtt{XI(L)}$ must contain the $l$-th
 data point $x_l$ for spline interpolation ({\tt tSPIN1}) or spline
 approximation ({\tt tSPAP1}). The data points must be in
 ascending order.
\item[YI] (Type according to {\tt t}) One-dimensional array of length
 $\mathtt{\ge N}$. On entry, $\mathtt{YI(L)}$ must contain the $l$-th
 data point $y_l$ for spline interpolation ({\tt tSPIN1}) or spline
 approximation ({\tt tSPAP1}).
\item[KNOT] ({\tt INTEGER}) Controls the mode of supplying the knots
for spline interpolation or approximation.
\item[] $\mathtt{=1,2:}$ \parbox[t]{140mm}{
The knots are computed by the subprograms {\tt tSPIN1} and {\tt tSPAP1}.
At the left and right end-point of the interpolation (approximation)
interval $[x_1,x_n]$ arise multiple knots.
The remaining knots are either equidistant ($\mathtt{KNOT=1}$) or are
computed by using the data points $\{x_l\}$ of interpolation
(approximation) ($\mathtt{KNOT=2})$.}
\item[] $\mathtt{\ne 1,2:}$ The knots must be supplied by the user.
\item[A,B] (Type according to {\tt t}) On entry, {\tt A} and {\tt B}
must contain the left (right) end-point of the interval $[a,b]$ for
the calculation of a set of spline  knots ({\tt tSPKN1}).
\item[T]  (Type according to {\tt t}) One-dimensional array of length
 $\mathtt{ \ge M}$ . \\
 For {\tt tSPKN1} and for {\tt tSPINT1}, {\tt tSPAP1} if
 $\mathtt{KNOT = 1,2:}$
 On exit, {\tt T(J)} contains the $j$-th knot.
 In the other cases, on entry, {\tt T(J)} must contain the $j$-th knot.
 The knots must be in non-decreasing order with multiplicity
 $\mathtt{\le K+1}$.
\item[C] (Type according to {\tt t}) One-dimensional
array of length $\mathtt{ \ge M-K-1}$. \\
For {\tt tSPPS1} and {\tt tSPCD1}: On entry, {\tt C(I)} must contain
the $i$-th coefficient $c_i$ of the polynomial spline $s(x)$ in B-spline
representation. \\
For {\tt tSPIN1}, {\tt tSPAP1} and {\tt tSPVD1}: On exit, {\tt C(I)}
contains the $i$-th coefficient $c_i$ of the polynomial interpolation
or approximation spline $s(x)$ in B-spline representation.
\item[D] (Type according to {\tt t}) One-dimensional
array of length $\mathtt{ \ge M-K-1}$. \\
On exit, {\tt D(I)} contains the coefficient $d_i$ of the {\tt NDER}-th
derivative of a polynomial spline $s(x)$ in B-spline representation.
\item[W] (Type according to {\tt t}) One-dimensional array of length
$\mathtt{\ge (3*K+1)*N}$ ({\tt tSPIN1}), and of length
$\mathtt{\ge NW}$ ({\tt tSPAP1}); used as working space.
\item[NW] ({\tt INTEGER}) Length of working array {\tt W},
($\mathtt{NW \ge N*(} n_0 \mathtt{+5)+} n_0 \mathtt{*(} n_0\mathtt{+1});
\,n_0 \mathtt{ = M-K-1}$).
\item[IW] ({\tt INTEGER}) One-dimensional array
of length $\mathtt{\ge N}$, used as working space.
\item[NERR] ({\tt INTEGER}) Error indicator. On exit:
\item[]$\mathtt{=0:}$ No error or warning detected.
\item[]$\mathtt{=1:}$
At least one of the parameters {\tt I}, {\tt K}, {\tt M}, {\tt N},
{\tt NDER} is not in range or $\mathtt{A \le B}$ is not true.
\item[]$\mathtt{=2:}$ \parbox[t]{139mm}{
The subprograms {\tt tGEQPF}, {\tt tORMQR}, {\tt tTRTRS} in the Linear
Algebra package {\tt LAPACK} (F001) were unable to solve the linear
system of equations for calculating the coefficients of the spline
interpolation to a given data set.}
\end{DLtt}
{\bf Case (2D):}
\begin{DLtt}{12345}
\item[F] Name of a user-upplied {\tt FUNCTION} subprogram,
declared {\tt EXTERNAL} in the calling program. This subprogram must
provide the value of the function $z = f(x,y)$ for
variation diminishing spline approximation.
\item[KX,KY] ({\tt INTEGER}) Degree of one-dimensional B-splines in
$x$-\,($y$-)direction ($\mathtt{1 \le KX \le 25,\,1 \le KY \le 25}$
for {\tt tSPVD2};\,
$\mathtt{0 \le KX \le 25,\,0 \le KY \le 25}$ otherwise).
\item[MX,MY] ({\tt INTEGER}) Number of knots in $x$-\,($y$-)direction
($\mathtt{MX \ge 2*KX+2,\,MY \ge 2*KY+2})$.
\item[I,J]  ({\tt INTEGER}) Indices of B-splines
($\mathtt{1 \le I \le MX-KX-1},\,\mathtt{1 \le J \le MY-KY-1}$).
\item[NX,NY] ({\tt INTEGER}) Number of data points $x_{lx}\,(y_{ly})$
in $x$-\,($y$-)direction ($\mathtt{NX \ge KX+1,\,NY \ge KY+1}$
for spline interpolation {\tt tSPIN2};\,
$\mathtt{NX \ge MX-KX-1,\,NY \ge MY-KY-1}$
for spline approximation {\tt tSPAP2}).
\item[NDERX,] ({\tt INTEGER}) Selects one out of
three cases: (i) integral, (ii) function value, or (iii) derivative.
\item[NDERY]
($\mathtt{-1 \le NDERX \le KX,\,-1 \le NDERY \le KY}$
for {\tt tSPNB2} and {\tt tSPPS2}; \\
\hspace*{3.3mm} $\mathtt{1 \le NDERX \le KX}$,
\hspace*{3.2mm}$\mathtt{1 \le NDERY \le KY}$ for {\tt tSPCD2}).
\item[]$\mathtt{=-1:}$
Calculation of the integral of $B_{i,j}(x,y)$
({\tt tSPNB2}), or the integral of $s(x,y)$ ({\tt tSPPS2}).
\item[] $\mathtt{=0:}$
Calculation of the function value $B_{i.j}(x,y)$
({\tt tSPNB2}), or the function value $s(x,y)$ ({\tt tSPPS2}).
\item[] $\mathtt{\ge 1:}$ \parbox[t]{141mm}{
Calculation of the {\tt NDERX}-th partial derivative
with respect to $x$ and the {\tt NDERY}-th partial derivative with
respect to $y$ of $B_{i,j}(x,y)$ ({\tt tSPNB2}),
or the calcultion of these derivatives of $s(x,y)$ ({\tt tSPPS2}).}
\item[] Note that in the first two cases
$\mathtt{NDERX=NDERY=-1,\,NDERX=NDERY=0}$, respectively.
\item[X,Y] (Type according to {\tt t}) Independent variables $x,y$ of
polynomial spline $s(x,y)$ or B-spline $B_{i,j}(x,y)$.
\item[XI,YI] (Type according to {\tt t}) One-dimensional arrays of
length $\mathtt{\ge NX}$ and $\mathtt{\ge NY}$, respectively.
On entry, $\mathtt{XI(LX)}$ and $\mathtt{YI(LY)}$ must contain the
$lx$-th data point $x_{lx}$ and the $ly$-th data point $y_{ly}$
for spline interpolation ({\tt tSPIN2}) or spline approximation
({\tt tSPAP2}). The data points must be in  ascending order.
\item[ZI] (Type according to {\tt t}) Two-dimensional array of dimension
 $\mathtt{(NDIMZ, \ge NY)}$. On entry, $\mathtt{ZI(LX,LY)}$ must contain
 the $(lx,ly)$-th data point $z_{lx,ly}$ for spline interpolation
 ({\tt tSPIN2}) or spline approximation ({\tt tSPAP2}).
\item[NDIMZ] ({\tt INTEGER}) Declared first dimension of a
two-dimensional array {\tt ZI} in the calling program
($\mathtt{\ge NX}$).
\item[KNOT] ({\tt INTEGER}) Controls the mode of supplying the knots
for spline interpolation or approximation.
\item[] $\mathtt{=1,2:}$ \parbox[t]{138mm}{
The set of knots are computed by subprograms {\tt tSPIN2} and
{\tt tSPAP2}.
At the left and right end-points of the interpolation (approximation)
intervals $[x_1,x_{nx}],\,[y_1,y_{ny}]$ arise multiple knots.
The remaining knots are either equidistant ($\mathtt{KNOT=1}$) or are
computed by using the data points $\{x_{lx},y_{ly}\}$ of
interpolation (approximation) ($\mathtt{KNOT=2}$).}
\item[] $\mathtt{\ne 1,2:}$
The knots must be supplied by the user.
\item[AX,BX;] (Type according to {\tt t}) On entry, {\tt AX}, {\tt BX};
{\tt AY}, {\tt BY} must contain the left (right)
end-points of the
\item[AY,BY]
intervals $[a_x,b_x];\,[a_y,b_y]$ for the calculation of a set
of spline knots in $x$-\,($y$-)direction, respectively, by {\tt tSPKN2}.
\item[TX,TY] (Type according to {\tt t}) One-dimensional arrays of
length $\mathtt{\ge MX}$ and $\mathtt{\ge MY}$, repectively. \\
For {\tt tSPKN2} and for {\tt tSPIN2}, {\tt tSPAP2}
if $\mathtt{KNOT = 1,2:}$
On exit, {\tt TX(J)} and {\tt TY(J)} contain the
$j$-th knot in $x$-\,($y$-)direction. In the other cases, on entry,
{\tt TX(J)} and {\tt TY(J)} must contain the $j$-th knot in
$x$-\,($y$-)direction. The knots must be in non-decreasing order with
multiplicity $\mathtt{\le KX+1}$ and $\mathtt{\le KY+1}$, respectively.
\item[C] (Type according to {\tt t}) Two-dimensional array of
dimension $\mathtt{(NDIMC , \ge MY-KY-1)}$. \\
For {\tt tSPPS2}, {\tt tSPCD2}: On entry, {\tt C(I,J)} must contain
the $(i,j)$-th coefficient $c_{i,j}$ of the polynomial spline $s(x,y)$
in B-spline representation. \\
For {\tt tSPIN2}, {\tt tSPAP2}, {\tt tSPVD2}:
On exit, {\tt C(I,J)} contains the
$(i,j)$-th coefficient $c_{i,j}$ of the polynomial interpolation or
approximation spline $s(x,y)$ in B-spline representation.
\item[NDIMC] ({\tt INTEGER}) Declared first dimension of a
two-dimensional array {\tt C} in the calling program \\
($\mathtt{\ge MX-KX-1}$).
%\newpage
\item[D] (Type according to {\tt t}) Two-dimensional array of
dimension $\mathtt{(NDIMC , \ge MY-KY-1)}$. \\
On exit, {\tt D(I,J)} contains the coefficient $d_{i,j}$ of the
partial derivative of order $nx,\,ny$ with respect to $x$ and $y$
of a polynomial spline $s(x,y)$ in B-spline representation.
\item[W] (Type according to {\tt t}) One-dimensional array of length
$\mathtt{\ge MY-KY-1}$ ({\tt tSPPS2}), \\
$\mathtt{\ge (3*KX*NY+2)*NX*NY}$ ({\tt tSPIN2}),
and of length $\mathtt{\ge NW}$ ({\tt tSPAP2}), used as
working space.
\item[NW] ({\tt INTEGER}) Length of a one-dimensional array {\tt W},
used as working space \\
($\mathtt{\ge NX*NY*(} n_0 \mathtt{+6)+} n_0 \mathtt{*(} n_0\mathtt{+1});
\,n_0 \mathtt{ = (MX-KX-1)*(MY-KY-1)}$).
\item[IW] ({\tt INTEGER}) One-dimensional array
of length $\mathtt{\ge NX*NY}$, used as working space.
\item[NERR] ({\tt INTEGER}) Error indicator. On exit:
\item[]$\mathtt{ = 0:}$ No error or warning detected.
\item[]$\mathtt{ = 1:}$ \parbox[t]{141mm}{
At least one of the parameters {\tt I}, {\tt J}, {\tt KX}, {\tt KY},
{\tt MX}, {\tt MY}, {\tt NX}, {\tt NY}, {\tt NDERX}, {\tt NDERY}
is not in range or at least one of the relations
$\mathtt{AX \le BX}$, $\mathtt{AY \le BY}$ is not true.}
\item[]$\mathtt{ = 2:}$ \parbox[t]{141mm}{
The routines {\tt tGEQPF}, {\tt tORMQR}, {\tt tTRTRS} in the Linear
Algebra package {\tt LAPACK} (F001) were unable to solve the linear
system of equations for calculation coefficients of spline interpolation
to a given data set.}
\end{DLtt}
\Examples
Calculate
\begin{enumerate}
\item
The coefficients $c_i$ of
a polynomial interpolation spline $y=s(x)$ of degree $k=2$ in B-spline
representation to a given data set $\{(x_l,y_l)\}_{l=1,\ldots,6}$;
\item
The corresponding coefficients $d_i$ of the first derivative \quad
$\displaystyle y'=\frac{ds(x)}{dx}$;
\item
The values of \quad $s(x), \quad \displaystyle \frac{ds(x)}{dx}$ \quad
and \quad $\displaystyle \int_0^x s(\xi)d\xi$ \quad for $x=0(0.1)1$.
\end{enumerate}
\begin{verbatim}
      IMPLICIT DOUBLE PRECISION (A-H,O-Z)
      DIMENSION XI(6),YI(6),T(9),C(6),D(6),W(42),IW(6)
      DATA K,N,NDER,KNOT / 2,6,1,1 /
      DATA XI / 0D0,0.20D0,0.40D0,0.60D0,0.80D0,1.00D0 /
      DATA YI / 1D0,0.66D0,0.47D0,0.38D0,0.35D0,0.34D0 /
      M=N+K+1

      CALL DSPIN1(K,N,XI,YI,KNOT,T,C,W,IW,NERR)
      CALL DSPCD1(K,M,NDER,T,C,D,NERR)

      WRITE(6,1000) K,N,(T(I),I=1,M)
      DO 20 J=0,10
      X=J/1D1

      SPL0=DSPPS1(K,M, 0,X,T,C,NERR)
      SPL1=DSPPS1(K,M, 1,X,T,D,NERR)
      SPLI=DSPPS1(K,M,-1,X,T,C,NERR)

   20 WRITE(6,1010) J,X,SPL0,SPL1,SPLI
 1000 FORMAT(...)
 1010 FORMAT(...)
      END
\end{verbatim}
\begin{verbatim}
 DEGREE OF POLYNOMIAL SPLINE:  2          NUMBER OF DATA POINTS:  6
 KNOTS T :        0.00  0.00  0.00  0.25  0.50  0.75  1.00  1.00  1.00

     J        X         S(X)          DS(X)         IN(X)
     0      0.00      1.000000     -2.119921      0.000000
     1      0.10      0.809004     -1.700000      0.090100
     2      0.20      0.660000     -1.280079      0.163201
     3      0.30      0.550992     -0.940017      0.223467
     4      0.40      0.470000     -0.679816      0.274299
     5      0.50      0.415028     -0.419615      0.318334
     6      0.60      0.380000     -0.280953      0.357970
     7      0.70      0.358838     -0.142290      0.394796
     8      0.80      0.350000     -0.065306      0.430174
     9      0.90      0.344235     -0.050000      0.464873
    10      1.00      0.340000     -0.034694      0.499072
\end{verbatim}
\Errorh
\begin{tabular}{l@{\qquad}l}
Error {\tt E210.1}:  {\tt K|KX,KY} not in range, &
Error {\tt E210.5}:  {\tt NDER|NDERX,NDERY} not in range,  \\
Error {\tt E210.2}:  {\tt M|MX,MY} not in range, &
Error {\tt E210.6}:  {\tt A,B|AX,BX;AY,BY} inconsistent,  \\
Error {\tt E210.3}:  {\tt I|I,J}   not in range, &
Error {\tt E210.7}:  {\tt NDERX|NDERY} inconsistent. \\
Error {\tt E210.4}:  {\tt N|NX,NY} not in range,
\end{tabular} \\[1mm]
In all cases, {\tt NERR} is set ${\tt =1}$ (see above). A message is
written on {\tt Unit 6}, unless subroutine {\tt MTLSET} (N002) has
been called.
\newpage
\Refer
\begin{enumerate}
\item J.H. Ahlberg, E.N. Nilson, J.L. Walsh, The Theory of Splines and
 their Applications, Academic Press, New York, 1967.
\item M.J. Marsden, An identity for spline functions with applications
 to variation diminishing spline approximation,
 J. Appr. Theory 3 (1970), 7-49.
\item C. de Boor, On calculating with B-splines,
 J. Appr. Theory 6 (1972), 50-62.
\item M.G. Cox, The numerical evaluation of B-splines,
 J. Inst. Maths Applics 10 (1972), 134-149.
\item J.G. Hayes, J. Halliday, The least-squares fitting of cubic spline
 surfaces to general data sets,
 J. Inst. Maths Applics 14 (1974), 89-103.
\item M.G. Cox, An algorithm for spline interpolation,
 J. Inst. Maths Applics 15 (1975), 95-108.
\item C. de Boor, A Practical Guide to Splines,
 Springer-Verlag, Berlin (1978).
\item P. Lancester, K. Salkauskas, Curve and Surface Fitting - An
 Introduction, Academic Press, New York, 1986.
\item J.C. Mason, M.S. Cox (Eds.), Algorithms for Approximation,
 Clarendon Press, Oxford, 1987.
\item J.W. Schmidt, H. Sp\"ath (Eds.), Splines in Numerical Analysis,
 Akademie-Verlag, Berlin, 1989.
\item H. Sp\"ath, Eindimensionale Spline-Interpolations-Algorithmen;
Zweidimensionale Spline-Interpola\-tions-Algorithmen, (2 Bd.)
 R. Oldenbourg, M\"unchen 1990/1991.
\end{enumerate}
$\bullet$
