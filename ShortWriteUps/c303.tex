\Version{GAMMF}                        \Routid{C303}
\Keywords{GAMMA FUNCTION}
\Author{K.S. K\"olbig}                \Library{MATHLIB}
\Submitter{}                           \Submitted{07.06 1992}
\Language{Fortran}                     \Revised{}
\Cernhead{Gamma Function for Real Argument}
Function subprograms {\tt GAMMF} and {\tt DGAMMF} calculate
the gamma function
$$ \Gamma(x) \ = \ \displaystyle \int_0 ^{\infty} e^{-t} t^{x-1}dt
\qquad (x > 0), \qquad
\Gamma(x) \ = \ \frac{\pi}{\Gamma(1-x)\, \sin \pi x} \qquad (x<0) $$
for real argument $x \neq -n,(n = 0,1,2,\cdots)$.
\par
On CDC and Cray computers,  the double-precision
version {\tt DGAMMF} is not available.
\Structure
{\tt FUNCTION} subprograms\\
User Entry Names: \Rdef{GAMMF}, \Rdef{DGAMMF}\\
Files Referenced: {\tt Unit 6} \\
External References: \Rind{MTLMTR}{N002}, \Rind{ABEND}{Z035}
\Usage
In any arithmetic expression,
\begin{center}
{\tt GAMMF(X)} \quad or \quad {\tt DGAMMF(X)} \quad has the value \quad
$\Gamma(\mathtt{X})$,
\end{center}
where {\tt GAMMF} is of type {\tt REAL}, {\tt DGAMMF} is of type
{\tt DOUBLE PRECISION}, and  {\tt X}
has the same type as the function name.
\Method
Approximation by truncated Chebyshev series and functional relations.
\Accuracy
{\tt GAMMF} (except on CDC and Cray computers) has full single-precision
accuracy. {\tt DGAMMF} (and of {\tt GAMMF} on CDC and
Cray computers) has an accuracy which is approximately
one digit less than machine precision.
\Errorh
Error {\tt C303.1}: $\mathtt{X} = -n,(n = 0,1,2,\cdots).$
The function value is set equal to zero, and a message is written on
{\tt Unit 6}, unless subroutine {\tt MTLSET} (N002) has been called.
\Refer
\begin{enumerate}
\item Y.L. Luke, Mathematical functions and their approximations,
(Academic Press, New York 1975) 4.
\end{enumerate}
$\bullet$
