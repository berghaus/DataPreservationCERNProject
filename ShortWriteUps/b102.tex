\Version{ASINH}                          \Routid{B102}
\Keywords{ARSINH HYPERBOLIC ARCSIN}
\Author{K.S. K\"olbig}                   \Library{MATHLIB}
\Submitter{}                              \Submitted{07.06.1992}
\Language{Fortran}                        \Revised{15.03.1993}
\Cernhead{Hyperbolic Arcsine}
Function subprograms {\tt ASINH} and {\tt DASINH} calculate
the hyperbolic arcsine
$$ \mathrm{arcsinh}(x) = \ln (x+\sqrt{x^2 +1})$$
for real argument $x$.
\par
On CDC and Cray computers, the double precision version
{\tt DASINH} is not available
\Structure
{\tt FUNCTION} subprograms\\
User Entry Names: \Rdef{ASINH}, \Rdef{DASINH}
\Usage
In any arithmetic expression,
\begin{center}
{\tt ASINH(X)} \quad or \quad {\tt DASINH(X)} \quad has the value
\quad $\mathrm{arcsinh}(\mathtt{X})$,
\end{center}
where {\tt ASINH} is of type {\tt REAL},
{\tt DASINH} is of type {\tt DOUBLE
PRECISION}, and {\tt X} has the same type as the function name.
\Method
Approximation by truncated Chebyshev series and functional relations.
\Accuracy
{\tt ASINH} (except on CDC and Cray computers)
has full single-precision accuracy. For most values of the argument
{\tt X}, {\tt DASINH} (and {\tt ASINH} on CDC and Cray
computers) has an accuracy of approximately one significant
digit less than the machine precision.
\Refer
\begin{enumerate}
\item Y.L. Luke, Mathematical functions and their
approximations, (Academic Press New York, 1975) 66.
\end{enumerate}
$\bullet$
