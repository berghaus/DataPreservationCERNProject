\Version{ELPAHY}                          \Routid{D302}
\Keywords{PARTIAL DIFFERENTIAL EQUATION}
\Author{R.C. Le Bail}                     \Library{MATHLIB}
\Submitter{}                              \Submitted{20.03.1972}
\Language{Fortran}                        \Revised{01.12.1981}
\Cernhead{Fast Partial Differential Equation Solver}
{\tt ELPAHY} uses fast Fourier transform techniques for the solution,
over a rectangular domain, of the following  elliptic,
parabolic or hyperbolic part differential equation:
 $$ \frac{d^2\phi(x,y)}
{dx^2}+c_1\frac{d^2\phi(x,y)}{dy^2}+c_2\frac{d\phi(x,y)}
{dy}+c_3\phi(x,y) \ = \ \rho(x,y)$$
where $\phi (x,y)$ is the unknown function, $\rho (x,y)$ the known source
term, and $c_1,c_2,c_3$ given coefficients. A large variety of
boundary conditions can be specified on the sides of the rectangle.
\Structure
{\tt SUBROUTINE} subprogram  \\
User Entry Names: \Rdef {ELPAHY}\\
Internal Entry Names:  {\tt NEWRO}, {\tt ELANAL}, {\tt ESOLVE},
{\tt SYNT}, {\tt MFT}  \\
External References: \Rind{RFT}{D700} \\
{\tt COMMON} Block Names and Lengths:
{\tt /FW1/ 774}, {\tt /FW2/ 100}
\Usage
\begin{verbatim}
    CALL ELPAHY(F,NX,NY,DX,DY,C,IBX,BWEST,BEAST,JBY,BSOUTH,BNORTH)
\end{verbatim}
\begin{DLtt}{12345678}
\item [F] ({\tt REAL}) Two-dimensional array, dimensioned {\tt (NX,NY)}
in the calling program. On input it contains the source term
$\rho(x,y)$ and on return it contains the unknown function $\phi (x,y)$.
\item [NX] ({\tt INTEGER}) Number of divisions along {\tt X}.
{\tt NX} must be of the form $2^n+1$.
\item [NY] ({\tt INTEGER}) Number of divisions along {\tt Y}.
\item [DX] ({\tt REAL}) Mesh spacing along {\tt X}.
\item [DY] ({\tt REAL}) Mesh spacing along {\tt Y}.
\item [C] ({\tt REAL}) One-dimensional array of dimension {\tt 3},
containing the coefficients $c_1,c_2,c_3$.
\item [IBX] ({\tt INTEGER}) Controls the type of boundary conditions on
the  left {\tt (BWEST)} and right {\tt (BEAST)} sides of the rectangular
domain:\\
$\mathtt{IBX=1}:$ Imposed periodicity along $x$; {\tt BWEST}, {\tt BEAST}
not given.\\
$\mathtt{IBX=2}:$ Given derivative on either vertical side.\\
$\mathtt{IBX=3}:$ Given value on either vertical side.\\
$\mathtt{IBX=4}:$ Given value on the left side, given derivative on the
right side.
\item [BWEST] ({\tt REAL}) One-dimensional array of size {\tt NY}
containing values or derivatives for the left side; the interpretation
depends on {\tt IBX}.
\item [BEAST] ({\tt REAL}) One-dimensional array of size {\tt NY}
containing values or derivatives for the right side; the interpretation
depends on {\tt IBX}.
\newpage
\item [JBY] ({\tt INTEGER}) Controls the
 type of boundary conditions on the lower {\tt (BSOUTH)} und upper
{\tt (BNORTH)} sides of the rectangular domain: \\
{\bf Elliptic} equation ($c_1>0$): \\
$\mathtt{JBY=1:}$ Given value on both lower and upper sides. \\
$\mathtt{JBY=2:}$ Given derivative on both lower and upper sides. \\
$\mathtt{JBY=3:}$ Given value on lower side, given derivative on upper
side.\\
$\mathtt{JBY=4:}$ Given derivative on lower side, given value on upper
side.\\
{\bf Parabolic} equation ($c_1=0$):\\
Specify {\tt BSOUTH} array only. (If y=time, {\tt BSOUTH} are initial
values and the future {\tt BNORTH} cannot be specified). \\
$\mathtt{JBY=1:}$ Given value on lower side.\\
$\mathtt{JBY=2:}$ Given derivative on lower side.\\
{\bf Hyperbolic} equation ($c_1 <0 $):\\
The {\tt BSOUTH} array specifies the value, the {\tt BNORTH} array
the derivative.\\
$\mathtt{JBY=1}$.
\item [BSOUTH] ({\tt REAL}) One-dimensional  array of size {\tt NX}
containing values or derivatives for the lower side; the interpretation
depends on {\tt JBY}.
\item [BNORTH] ({\tt REAL}) One-dimensional  array of size {\tt NX}
containing values or derivatives for the upper side; the interpretation
depends on {\tt JBY}.
\end{DLtt}
\Notes
If $\mathtt{NX > 65}$, specify {\tt COMMON /FWORK/} of length {\tt 6*NX}
and {\tt COMMON /FW1/} of length {\tt 6*NX} in the calling program.
If $\mathtt{NY > 50}$, specify {\tt COMMON /FW2/} of length {\tt 2*NY}.
In either case, make sure your program is loaded
 before {\tt ELPAHY} (D302) (this is automatic unless you recompile D302
in the same job).
\Refer
\begin{enumerate}
\item R.C. Le Bail, Use of fast Fourier transforms for solving partial
differential equations in physics, J. Comput. Phys. {\bf 9} (1972)
440--465
\end{enumerate}
A copy of Ref. 1 is available.
\\ $\bullet$
