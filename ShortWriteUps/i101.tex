\Version{EPIO}                              \Routid{I101}
\Keywords{EP STANDARD FORMAT INPUT OUTPUT}
\Author{H. Grote, I. McLaren}                 \Library{PACKLIB}
\Submitter{}                                  \Submitted{01.12.1981}
\Language{Fortran, Assembler}          \Revised{01.02.1982}
\Cernhead{EP Standard Format Input/Output Package}
The EP format off-line package is intended to be used for all data
(at least on tape) in an experiment, in such a way that from the raw
data tape to the DST, the tape (or file) format is identical. This makes
the transport of data between computers easier, and simplifies the task
of passing the files or tapes at different stages of the production chain
through any other part of the production chain. {\tt EPIO} is designed
to make almost all features of the very flexible EP format
available to the user.
\Structure
{\tt SUBROUTINE} package\\
User Entry Names:
\begin{htmlonly}
\begin{tabular}{llllllll}
\end{htmlonly}
\begin{latexonly}
\begin{tabular}[t]{l*{7}{@{\hspace{4pt}}l}}
\end{latexonly}
\Rdef{EPINIT}, & \Rdef{EPREAD}, & \Rdef{EPOUTS}, & \Rdef{EPOUTL}, &
\Rdef{EPCLOS}, & \Rdef{EPRWND}, & \Rdef{EPDROP}, & \Rdef{EPEND},  \\
\Rdef{EPUREF}, & \Rdef{EPGETW}, & \Rdef{EPGETA}, & \Rdef{EPGETC}, &
\Rdef{EPSETW}, & \Rdef{EPSETA}, & \Rdef{EPSETC}, & \Rdef{EPADDH}, \\
\Rdef{EPUPDH}, & \Rdef{EPSTAT}
\end{tabular} \\
Files Referenced: User defined \\
External References: \Rind{UZERO}{V300}, \Rind{UCOPY}{V301},
\Rind {IOPACK}{Z300} (IBM only) \\
{\tt COMMON} Block Names and Lengths:  {\tt /EPCOMM/ 136}
\Usage
See {\bf Long Write-up}.
\\ $\bullet$
