\Version{CFFT}                                 \Routid{D704}
\Keywords{COMPLEX FAST FOURIER TRANSFORM}
\Author{H.-H. Umst\"atter}                     \Library{KERNLIB}
\Submitter{}                                    \Submitted{01.12.1981}
\Language{Fortran}                       %\Revised{}
\Cernhead{Complex Fast Fourier Transform}
Subroutine {\tt CFFT} computes the finite Fourier transform of a complex
periodic sequence, whose period n must be a power of 2. The
expressions which may be computed are either the forward transform
\begin{equation}
A_m \ = \ \sum ^{n-1}_{k=0}a_k \exp \left( \frac{-i 2\pi km}{n} \right),
\qquad (m=0,1,\ldots,n-1),
\end{equation}
or the unscaled inverse transform
\begin{equation}
\alpha_k \ = \
\sum ^{n-1}_{m=0} A_m \exp \left( \frac{i 2\pi mk}{n} \right),
\qquad (k=0,1,\ldots,n-1),
\end{equation}
where $a_k$ and $A_m$ are complex numbers.
If the $A_m$ in (2) have the values defined by (1), then
$a_k=\alpha_k/n, (k=0,1,\ldots,n-1)$.
To ensure optimum use of storage, the same array is used for input and
output, and all intermediate calculations are carried out in this array.
\Structure
{\tt SUBROUTINE} subprogram \\
User Entry Names: \Rdef{CFFT}
\Usage
\begin{verbatim}
    CALL CFFT(A,M)
\end{verbatim}
\begin{DLtt}{1234}
\item[A] ({\tt COMPLEX}) Array of length not less than $n$.
\item[M] ({\tt INTEGER}) On entry, the absolute value of {\tt M}
determines the value of $n$ through the relation $n=2^{\mathtt{|M|}}$.
If {\tt M} is negative, expression (1) is evaluated. If {\tt M} is
non-negative, expression (2) is evaluated. Unchanged on exit. \\
$\mathtt{M<0:}$ \\
On entry, $\mathtt{A}(k+1)=a_k,\,(k=0,1,\ldots,n-1)$.\\
On exit, $\mathtt{A}(m+1)=A_m,\,(m=0,1,\ldots,n-1)$, as defined by
(1). \\
$\mathtt{M\geq 0:}$ \\
On entry, $\mathtt{A}(m+1)=A_m,\,(m=0,1,\ldots,n-1)$. \\
On exit, $\mathtt{A}(k+1)=a_k,\,(k=0,1,\ldots,n-1)$,
as defined by (2).
\end{DLtt}
\Method
The method is based on an algorithm of Cooley, Lewis and Welch
(see References), with the following modifications which reduce the
overhead time for small values of M: multiplications by
$\exp(im\pi)$
are replaced by addition or subtraction, and terms of the form
$\exp(i2\pi/m),(m=2,4,\ldots,n)$, are computed recursively using only
square roots and divisions.
\Refer
\begin{enumerate}
\item  G. Dahlquist and \AA. Bj\"orck, Numerical methods
(Prentice-Hall, Englewood Cliffs, 1974) 416.
\item  L.R. Rabiner and B. Gold, Theory and application of
digital signal processing (Prentice-Hall, Englewood Cliffs, 1975) 332.
\end{enumerate}
$\bullet$
