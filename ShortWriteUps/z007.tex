% 20 may 1992  mg
\Version {DATIME}                     \Routid{Z007}
\Keywords{DATE DATIME DATIMH JOB TIME TIMED TIMEL TIMEST TIMEX}
\Keywords{TIMING SHOWTIME}
\Author{See below}                    \Library{KERNLIB}
\Submitter{}                          \Submitted{15.01.1977}
\Language{Fortran or C or Assembler}  \Revised{18.09.1991}
\Cernhead {Job Time and Date}
{\bf Authors:} J. Harms, E. Jansen, A. Michalon, J. Zoll,
A. Berglund, T. Cass, C. Wood, H. Renshall. \\[3mm]
The {\tt DATIME} package interfaces with the system of any particular
machine to obtain the current calendar date and time, as well as the
central processor time used by and remaining to the job.
\Structure
{\tt SUBROUTINE} subprograms \\
User Entry Names: \Rdef{DATIME}, \Rdef{DATIMH}, \Rdef{TIMEX},
\Rdef{TIMEL}, \Rdef{TIMED}, \Rdef{TIMEST}\\
External References: Machine dependent\\
{\tt COMMON} Block Names: {\tt /SLATE/ ISL(40)}
\Usage
\begin{verbatim}
    CALL DATIME(ID,IT)
\end{verbatim}
returns decimal {\tt INTEGER} date and time: {\tt ID=yymmdd},
{\tt IH=hhmm}.
It also stores the components into {\tt /SLATE/} as small integers:
\begin{center}
$\mathtt{ISL(1) = 19yy, \quad ISL(2) = mm, \quad ISL(3) = dd, \quad
ISL(4) = hh, \quad ISL(5) = mm, \quad ISL(6) = ss}$
\end{center}
for convenience of further processing by the user.
\begin{verbatim}
    CALL DATIMH(ND,NT)
\end{verbatim}
returns Hollerith date and time: {\tt ND $=$ 8Hdd/mm/yy} and
{\tt NT $=$ 8Hhh.mm.ss}
\begin{verbatim}
    CALL TIMEX(T)
\end{verbatim}
returns the execution time used by the job so far; {\tt T} is the central
processor time in seconds, a {\tt REAL} number with fractional part.
In supported interactive systems the time returned is that relative
to the first call to {\tt TIMEST}.
\begin{verbatim}
    CALL TIMEL(T)
\end{verbatim}
returns the execution time remaining until time-limit; {\tt T} in seconds
as for {\tt TIMEX}. In supported interactive systems the time returned
is the time left until the time-limit set by the first call to
{\tt TIMEST}. See {\bf Note 4} below.
\begin{verbatim}
    CALL TIMED(T)
\end{verbatim}
returns the execution time interval since the last call to {\tt TIMED};
{\tt T} in seconds as for {\tt TIMEX}.
\newpage
\begin{verbatim}
    CALL TIMEST(TLIM)
\end{verbatim}
This routine is necessary to initialise the timing operations
in the interactive mode of VM-CMS. In other systems (including VM-CMS
batch) it is a dummy do-nothing routine.
\par
It must be called once (subsequent calls are ignored) before any
calls to {\tt TIMEX} and {\tt TIMEL}. Before this routine is called
{\tt TIMEX} will return zero and {\tt TIMEL} will return {\tt 999.0}.
{\tt TLIM} is an input floating point value which will be used inside
{\tt TIMEL} as if it were the job time-limit. The first call to
{\tt TIMEST} also establishes the time origin for subsequent calls to
{\tt TIMEX} and {\tt TIMEL}.
\Accuracy
IBM: The RMS error returned in consecutive calls to {\tt TIMED}
without any intermediate code is of the order of 3 $\mu$sec on the
the CERN IBM 3090 with a minimum time for one call of 20 $\mu$sec.
The timing distribution has a long tail, however, and any individual
measurement could take as long as four or five times this value.
{\tt TIMEX} is accurate to within a tenth of a second and {\tt TIMEL}
only to the nearest second.
\Notes
\begin{enumerate}
\item The symbols {\tt yy,mm,dd,hh,mm,ss} used above stand
for the two decimal digits of {\it year, month, day, hours, minutes,
seconds}.
\item {\tt NT} and {\tt ND} in the call to {\tt DATIMH} are
2-word vectors on machines with a character-capacity of less than 8
characters per word.
\item  The information returned by these routines is obtained
by a system request. On some machines this is expensive in real time,
so one should avoid too many calls, to {\tt TIMEL} in particular.
\item Some machine/operating system configurations do not have a
value for timelimit, for example interactive work under VM-CMS (IBM) or
VMS (VAX) or no-limit batch job classes under VMS. In these cases a
constant time-left of 999.0 seconds is returned,
unless the time limit has been set with {\tt TIMEST}.
\end{enumerate}
\Examples
Suppose the date is Sept 16, 1976, and the time of day 19h 24m 55s.
\begin{verbatim}
    CALL DATIME(ID,IT)
\end{verbatim}
returns $\mathtt{ID = 760916, \quad IT = 1924, \quad
ISL = 1976, 9, 16, 19, 24, 55}$
\begin{verbatim}
    CALL DATIMH(ND,NT)
\end{verbatim}
returns {\tt ND $=$ 8H16/09/76} and {\tt NT $=$ 8H19.24.55}.
\\ $\bullet$
