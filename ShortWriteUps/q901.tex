\Version{INDENT}                        \Routid{Q901}
\Keywords{FORTRAN INDENT SOURCE}
\Author{M. Metcalf}                      \Library{PGMLIB}
\Submitter{}                              \Submitted{01.04.1983}
\Language{Fortran}                  %\Revised{}
\Cernhead{Indent Fortran Source}
The program reads Fortran source from a specified input file
and writes the indented source code to a specified output file.
\Structure
Complete {\tt PROGRAM}\\
User Entry Names: \Rdef{INDENT}\\
Files Referenced: Input and output units, either default or user defined.
\Usage
{\tt INDENT} reads from the default input unit four integer values in a
single record. The default values are taken if this record is absent.
\\[1mm]
\begin{tabular}{@{\hspace*{5mm}}ll}
Indenting shift & (Default = {\tt 3}) \\
Maximal indenting level & (Default = {\tt 10}) \\
File number of source input & (Default = {\tt 5}) \\
File number of transformed source output & (Default = {\tt 6})
\end{tabular} \\[2mm]
Note that the first column of the output file will be taken as carriage
control information if the output unit is a line printer.
\Method
The program detects the beginning and end of each
{\tt DO}-- and {\tt IF}--block, and indents each following source line by
a shift corresponding to the nesting level. Continuation lines are
constructed when necessary, but variable names are never split
across two lines.
\par
{\tt PATCHY} control records are treated as comment lines, and so
complete {\tt PAM}s can be handled.
\Restrict
Lines containing {\tt FORMAT} statements, or character
strings with multiple embedded blanks are not indented.
\par
Sequences of more than 200 comment lines may have
their order with respect to the following statement modified.
\par
Assembler code gets destroyed.
\Errorh
Primitive syntax checks protect the program from most non-Fortran
source input.
\Refer
\begin{enumerate}
\item M. Metcalf, FORTRAN Optimization, Academic Press London (1982),
Appendix B.
\end{enumerate}
$\bullet$
