\Version {LEAMAX}                                \Routid{D501}
\Keywords{CONSTRAIN, NON, LINEAR, LEAST, SQUARES, MAXIMUM, LIKELIHOOD,
DATA, FIT, FITTING, ESTIMATION}
\Author{W. M\"onch, B. Schorr}                   \Library{MATHLIB}
\Submitter{W. M\"onch}                           \Submitted{15.03.1993}
\Language{Fortran}                              %\Revised{}
\Cernhead {Constrained Non-Linear Least Squares and Maximum
            Likelihood Estimation}
{\tt LEAMAX} is a portable collection of
subprograms for solving general non-linear least squares problems,
non-linear data fitting problems, and maximum likelihood estimations.
 
Subroutine subprograms
{\tt RSUMSQ}, {\tt RFUNFT}, {\tt RMAXLK} and
{\tt DSUMSQ}, {\tt DFUNFT}, {\tt DMAXLK}
calculate an approximation to a minimum of an  objective function
$\varphi$ , with respect to $n$ unknown parameters
$a = (a_1,...,a_n) \mathbf{^T} \in \mathbf{R^n}$ : \\ [3mm]
\begin{DL}{1234}
\item[{\bf (S)}] The general non-linear least squares problem
$$ \varphi_S(a) \ = \ \frac{1}{2}
\sum_{i=1}^m\, [f_i(a)]^2,$$
\item[{\bf (F)}] The least squares data fitting problem
$$ \varphi_F(a) \ = \ \frac{1}{2} \sum_{i=1}^m\,
\left[\frac{y_i-f(x_i,a)}{\sigma_i}\right]^2,$$
\item[{\bf (M)}] The maximum likelihood estimation
$$ \varphi_M(a) \ = \ -\sum_{i=1}^m\, \ln (f(x_i,a)),$$
\end{DL}
subject to bounds on the variables of the form
$$ \underline{a}_j \le a_j \le \overline{a}_j
   \mbox{\hspace{1cm}} (j=1,2,\ldots,n).$$
The functions $f_i : \mathbf{R^n} \to \mathbf{R^1}$ $(i=1,...,m)$
and $f : \mathbf{R^k} \times \mathbf{R^n} \to \mathbf{R^1}$ are
arbitrary non-linear functions with respect to the argument $a$.
In the case of the data fitting problem {\bf (F)}, a set of observation
data $\{(x_i,y_i) | x_i \in \mathbf{R^k},
y_i \in \mathbf{R^1},i=1,...,m \}$ with their corresponding
weights $\sigma_i$ $(i=1,...,m)$ has to be provided,
whereas for the maximum likelihood estimation {\bf (M)}, the set of data
points $\{(x_i) | x_i \in \mathbf{R^k},i=1,...,m\}$ belongs to the input
of the problem.
 
These subprograms are based on the algorithm described by
Mor\'e (Ref. 1) for finding the solution of a general non-linear least
squares problem by using the Levenberg-Marquardt algorithm.
The method is completed
by an active set strategy for handling simple box constraints to the
unknown parameters (see {\bf Long Write-up} for details).
The necessary derivatives can either be supplied by the user
(subprogram {\tt SUB}) or are approximated numerically.
In the case of a non-linear data fitting problem, approximations to
the covariance matrix and the standard deviations of the model
parameter estimators are also provided.
 
On computers other than CDC or Cray, only the double-precision versions
{\tt DSUMSQ}, {\tt DFUNFT}, {\tt DMAXLK} are available. On CDC and
Cray computers, only the single-precision versions
{\tt RSUMSQ}, {\tt RFUNFT}, {\tt RMAXLK} are available.
\newpage
\Structure
{\tt SUBROUTINE} subprograms\\
User Entry Names: \Rdef{RSUMSQ}, \Rdef{RFUNFT}, \Rdef{RMAXLK},
\Rdef{DSUMSQ}, \Rdef{DFUNFT}, \Rdef{DMAXLK} \\
Internal Entry Names: {\tt D501L1}, {\tt D501L2}, {\tt D501SF},
{\tt D501P1}, {\tt D501P2}, {\tt D501N1}, {\tt D501N2} \\
External References:
  \begin{tabular}[t]{@{}llll}
     \Rind{RGEQPF}{F001},&\Rind{RORMQR}{F001},&
       \Rind{RTRTRS}{F001}, &\Rind{DGEQPF}{F001},\\
     \Rind{DORMQR}{F001},&\Rind{DTRTRS}{F001},&
       \Rind{RVSET}{F002},  &\Rind{RVSCL}{F002}, \\
     \Rind{RVSUB}{F002}, &\Rind{RVCPY}{F002}, &
       \Rind{RVMPY}{F002},  &\Rind{DVSET}{F002}, \\
     \Rind{DVSCL}{F002}, &\Rind{DVSUB}{F002}, &
       \Rind{DVCPY}{F002},  &\Rind{DVMPY}{F002}, \\
     \Rind{RMSET}{F003}, &\Rind{RMSCL}{F003}, &
       \Rind{RMCPY}{F003},  &\Rind{RMMPY}{F003}, \\
     \Rind{RMBIL}{F003}, &\Rind{DMMLT}{F003}, &
       \Rind{DMSET}{F003},  &\Rind{DMSCL}{F003}, \\
     \Rind{DMCPY}{F003}, &\Rind{DMMPY}{F003}, &
       \Rind{DMBIL}{F003},                       \\
     \Rind{RMMLT}{F004}, &\Rind{DMMLT}{F004}, &
     \Rind{RSINV}{F012}, &\Rind{DSINV}{F012}     \\
     \multicolumn{4}{@{}l}{User-supplied {\tt SUBROUTINE} subprogram}
  \end{tabular}
\Usage
For $\mathtt{t = R}$ (type {\tt REAL}),
$\mathtt{t = D}$ (type {\tt DOUBLE PRECISION}):
 
{\bf (S) General non-linear least squares problem}
\begin{verbatim}
    CALL tSUMSQ(SUB,M,N,NC,A,AL,AU,MODE,EPS,MAXIT,IPRT,
   +            MFR,IAFR,PHI,DPHI,COV,STD,W,NERROR)
\end{verbatim}
{\bf (F) Least squares data fitting problem}
\begin{verbatim}
    CALL tFUNFT(SUB,K,M,N,NX,NC,X,Y,SY,A,AL,AU,MODE,EPS,MAXIT,IPRT,
   +            MFR,IAFR,PHI,DPHI,COV,STD,W,NERROR)
\end{verbatim}
{\bf (M) Maximum likelihood estimation}
\begin{verbatim}
    CALL tMAXLK(SUB,K,M,N,NX,X,A,AL,AU,MODE,EPS,MAXIT,IPRT,
   +            MFR,IAFR,PHI,DPHI,W,NERROR)
\end{verbatim}
\begin{DLtt}{123456}
\item[SUB] Name of user-supplied {\tt SUBROUTINE} subprogram,
declared {\tt EXTERNAL} in the calling program.
This subprogram must provide the values of the functions
$f_i(a) \; (i=1, \ldots , m)$,  $f( \cdot , a)$,
and, if $\mathtt{MODE = 1}$, the values of the derivatives
$\partial f_i(a) / \partial a_j  \; $ and
$\partial f(\cdot , a) / \partial a_j  \; $
$(i=1, \ldots , m ; j=1, \ldots ,n)$  (see {\bf Examples}).
\item[K] ({\tt INTEGER}) Cases {\bf (F)} and {\bf (M)}: dimension $k$
of a data point (observation) $x_i \in \mathbf{R^k}$.
\item[M] ({\tt INTEGER}) Case {\bf (S)}:
number of non-linear functions $f_i$;
cases {\bf (F)} and {\bf (M)}: number of data points (observations).
\item[N] ({\tt INTEGER}) Number of unknown parameters $a$.
\item[NX] ({\tt INTEGER}) Cases {\bf (F)} and {\bf (M)}: declared first
dimension of array {\tt X} in the calling program, $\mathtt{NX \ge K}$.
\item[NC] ({\tt INTEGER}) Cases {\bf (S)} and {\bf (F)}: declared
first dimension of array {\tt COV} in the calling program,
$\mathtt{ NC \ge N}$.
\item[X] (Type according to {\tt t}) Cases {\bf (F)} and {\bf (M)}:
two-dimensional array of dimension $\mathtt{(NX , \ge M)}$.
On entry, {\tt X} must contain the data set \{ $x_i$ \} (the $i$-th
column of {\tt X} belongs to the data point $x_i \in \mathbf{R^k}$ ,
    $i=1,...,m$).
\item[Y] (type according to {\tt t}) Case {\bf (F)}: one-dimensional
array of length $\mathtt{ \ge M}$, contains, on entry, the data set
 \{ $y_i$ \}.
\item[SY] (type according to {\tt t}) Case {\bf (F)}: one-dimensional
array of length $\mathtt{ \ge M}$, contains, on entry, the weights
\{ $\sigma_i$ \} of the data points.
\item[A] (Type according to {\tt t}) One-dimensional array of length
$\mathtt{\ge N}$. On entry, {\tt A(J)} must contain the starting
value of $a_j$ for the Levenberg-Marquardt algorithm.
On exit, {\tt A(J)} contains an approximation to $a_j$ of a minimum
point (if the algorithm was successful).
\item[AL] (Type according to {\tt t}) One-dimensional array of length
$\mathtt{\ge N}$. On entry, {\tt AL(J)} must contain the lower bound
$\underline{a}_j$   of $a_j$  .
\item[AU] (Type according to {\tt t}) One-dimensional array of length
$\mathtt{\ge N}$. On entry, {\tt AU(J)} must contain the upper
bound $\overline{a}_j$ of $a_j$  .
\item[MODE] ({\tt INTEGER})
\item[] $\mathtt{ = 0:}$ The derivatives are approximated by divided
                         differences.
\item[] $\mathtt{ = 1:}$ The derivatives are to be supplied by subprogram
                         {\tt SUB}.
\item[] Other values for {\tt MODE} are treated as {\tt MODE = 0}.
\item[EPS] (Type according to {\tt t}) User-supplied tolerance used to
control the termination criterion. {\tt EPS} should be chosen according
to the accuracy required by the problem and the machine
accuracy {\tt t} (recommended value on entry: between
$\mathtt{10^{-6}}$ for {\tt t = R} , and $\mathtt{ 10^{-12}}$ for
{\tt t = D}, respectively).
\item[MAXIT] ({\tt INTEGER}) Maximum permitted number of iterations.
\item[IPRT] ({\tt INTEGER}) Printing control.
\item[] $\mathtt{ = 0:}$ no printing of intermediate results,
\item[] $\mathtt{ = \pm L:}$ \parbox[t]{136mm}{
printing of intermediate results at
every $\mathtt{|L|}$-th iteration; if $\mathtt{IPRT < 0}$, printing of
all input parameters of {\tt tSUMSQ}, {\tt tFUNFT}, {\tt tMAXLK},
respectively, in addition.}
\item[MFR] ({\tt INTEGER}) On exit, {\tt MFR} contains the number of
free variables at the solution point.
\item[IAFR] ({\tt INTEGER}) One-dimensional array
of length $\mathtt{\ge 2*N}$ for cases {\bf (S)} and {\bf (F)}, and of
length $\mathtt{\ge N}$ for case {\bf (M)}, used as working space.
On exit, the first {\tt MFR} elements of {\tt IAFR} contain the
indices of the free variables at the solution point.
\item[PHI]  (Type according to {\tt t}) On exit, {\tt PHI} contains the
value of the objective function $\varphi$ at the solution point.
\item[DPHI] (Type according to {\tt t}) One-dimensional array of length
$\mathtt{\ge N}$.
On exit, {\tt DPHI(J)} contains the derivative
$\partial \varphi / \partial a_j$  of $\varphi$ with
respect to $a_j$ (j-th component of the gradient of $\varphi$)
at the solution point.
\item[COV] (Type according to {\tt t}) Cases {\bf (S)} and {\bf (F)}:
two-dimensional array of dimension $\mathtt{(NC , \ge N)}$. On exit,
{\tt COV} contains an approximation to the covariance matrix.
\item[STD] (Type according to {\tt t}) Cases {\bf (S)} and {\bf (F)}:
one-dimensional array of length $\mathtt{\ge N}$. On exit, {\tt STD(J)}
contains an approximation to the standard deviation of the estimator
of the model parameter $a_j$.
\item[W] (Type according to {\tt t}) One-dimensional array of length
$\mathtt{\ge 9*N+4*M+2*M*N+3*N*N}$ for cases {\bf (S)} and {\bf (F)},
and of length $\mathtt{\ge 7*N+2*N*N}$ for case {\bf (M)}, used as
working space.
\item[NERROR] ({\tt INTEGER}) Error indicator. On exit:
\item[]$\mathtt{ = 0:}$ No error or warning detected.
\item[]$\mathtt{ = 1:}$ \parbox[t]{139mm}{
At least one of the constants {\tt K}, {\tt M}, {\tt N}, {\tt NX},
{\tt NC}, {\tt MAXIT} is illegal or at least for one $j$ the relation
$ \underline{a}_j \leq \overline{a}_j$ is not true.}
\item[]$\mathtt{ = 2:}$
The maximum number {\tt MAXIT} of iterations has been reached.
\item[]$\mathtt{ = 3:}$ \parbox[t]{139mm}{
The objective function $\varphi$ or its derivative is not defined for
the current values of the parameter vector $a$.}
\item[]$\mathtt{ = 4:}$ \parbox[t]{139mm}{
Cases {\bf (S)} and {\bf (F)}:
The routines {\tt tGEQPF}, {\tt tORMQR}, {\tt tTRTRS} in the Linear
Algebra package {\tt LAPACK} (F001) were unable to solve the linear
least squares problem or the subprogram {\tt tSINV} (F012) was unable
to compute the covariance matrix. \\
Case {\bf (M)}: the routine {\tt tSINV} (F012) was unable to solve
the normal equations.}
\end{DLtt}
 
\newpage
\Examples
For the user-supplied {\tt SUBROUTINE} subprogram {\tt SUB} write for
instance in the case $\mathtt{t = D}$: \\
 
{\bf (S)} Objective function (Brown badly-scaled function, $n=2,\,m=3$):
$$ \varphi_S(a) \ = \ \frac{1}{2} \:
\left[(a_1-10^6)^2+(a_2-2\cdot 10^{-6})^2+(a_1a_2-2)^2 \right] $$ .
\begin{verbatim}
      SUBROUTINE SUB(N,A,M,F,DF,MODE,NERROR)
      IMPLICIT DOUBLE PRECISION (A-H,O-Z)
      PARAMETER (Z0 = 0)
      DIMENSION A(*),F(*),DF(M,*)
      NERROR=0
      F(1)=A(1)-1D6
      F(2)=A(2)-2D-6
      F(3)=A(1)*A(2)-2
      IF(MODE .NE. 1) RETURN
      CALL DMSET(M,N,Z0,DF(1,1),DF(1,2),DF(2,1))
      DF(1,1)=1
      DF(2,2)=1
      DF(3,1)=A(2)
      DF(3,2)=A(1)
      RETURN
      END
\end{verbatim}
 
{\bf (F)} Objective function (Bard function, $n=3,\,m=15,\,k=3$):
$$ \varphi_F(a) \ = \ \frac{1}{2} \, \sum_{i=1}^m \:
\left[y_i - \left(a_1+ \frac{x_{1,i}}{x_{2,i}\,a_2+x_{3,i}\,a_3}\right)
\right]^2 $$ .
\begin{verbatim}
      SUBROUTINE SUB(K,X,N,A,F,DF,MODE,NERROR)
      IMPLICIT DOUBLE PRECISION (A-H,O-Z)
      DIMENSION A(*),X(*),DF(*)
      T=X(2)*A(2)+X(3)*A(3)
      IF (T .EQ. 0) THEN
       NERROR=3
      ELSE
       NERROR=0
       F=A(1)+X(1)/T
       IF(MODE .NE. 1) RETURN
       DF(1)=1
       DF(2)=-X(1)*X(2)/T**2
       DF(3)=-X(1)*X(3)/T**2
      ENDIF
      RETURN
      END
\end{verbatim}
 
\newpage
{\bf (M)} Objective function ($n=1,\,m=100,\,k=1$):
$$ \varphi_M(a) \ = \ -\:\sum_{i=1}^m\, \ln \: \left\{
\frac{1}{a_1\sqrt{\pi}} \, \exp \left[- \,\frac{1}{2} \,
\left(\frac{x_{1,i}-1}{a_1} \right)^2 \right] \right\} $$ .
\begin{verbatim}
      SUBROUTINE SUB(K,X,N,A,F,DF,MODE,NERROR)
      IMPLICIT DOUBLE PRECISION (A-H,O-Z)
      DIMENSION A(*),X(*),DF(*)
      PARAMETER (PIR = 0.56418 95835 47756 287D0)
      NERROR=3
      IF(A(1) .LE. 0) RETURN
      T=0.5D0*((X(1)-1)/A(1))**2
      F=PIR*EXP(-T)/A(1)
      NERROR=0
      IF(MODE .EQ. 1) DF(1)=-F*(1-2*T)/A(1)**2
      RETURN
      END
\end{verbatim}
 
In all three cases the parameters {\tt K} , {\tt N} ,
{\tt A} , {\tt M} , {\tt MODE} , {\tt NERROR} are as declared
above. The other parameters are the following:
 
\begin{DLtt}{1234}
\item[F] (Type according to {\tt t})
Case {\bf (S)}: one-dimensional array of length $\mathtt{\ge M}$.
  {\tt F(I)} must contain the function value
  $f_i(a)$ at $a$ $(i=1, \ldots ,m)$, on exit. \\
Cases {\bf (F)} and {\bf (M)}:  {\tt F} must contain the function value
$f(x,a)$ at $(x,a)$, on exit.
\item[DF] (Type according to {\tt t}) If {\tt MODE = 1}     values of
{\tt DF} are supplied by {\tt SUB}. For other values of {\tt MODE} the
parameter {\tt DF} is not referenced. \\
Case {\bf (S)}: two-dimensional array of dimension $\mathtt{(M, \ge N)}$.
{\tt DF(I,J)} must contain the value of the partial derivative
  $\partial f_i(a) / \partial a_j $ at $a$,
  $(i=1, \ldots , m ; j=1, \ldots ,n)$, on exit. \\
Cases {\bf (F)} and {\bf (M)}: one-dimensional array of length
$\mathtt{ \ge N}$.
{\tt DF(J)} must contain the value of the partial   derivative
  $\partial f(x, a) / \partial a_j$, $(j=1, \ldots ,n)$,
  on exit.
\item[X] (Type according to {\tt t}) Cases {\bf (F)} and {\bf (M)}:
one-dimensional array of length $\mathtt{\ge K}$ for {\bf one}
data point $x_i \in \mathbf{R^k}$ (in contrast to above declaration).
\end{DLtt}
 
\Refer
\begin{enumerate}
\item J.J. Mor\'e, The Levenberg-Marquardt algorithm: Implementation
 and theory, In: Numerical Analysis, G.A. Watson (Ed.), Lecture Notes
 in Mathematics 630, Springer-Verlag, New York (1977) 105-116.
\item {\AA}.Bj\"orck: Solution of Equations in $\mathbf{R^n}$
(Part 1: Least Squares Methods).
      In: Handbook of Numerical Analysis,
      P.G.Ciarlet, J.L.Lions (Eds.),
      North-Holland, Amsterdam, New York, Oxford,
      Tokyo, 1990, 467-636.
\item R.Fletcher: Practical Methods of Optimization.
      John Wiley and Sons, Chichester, 2nd Edition, 1987.
\end{enumerate}
$\bullet$
