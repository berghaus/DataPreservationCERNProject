\Version {FLOARG}                           \Routid{M250}
\Keywords{FLOAT INTEGER REAL NUMBER}
\Author{R. Brun, R. Matthews}                \Library{KERNLIB}
\Submitter{}                                  \Submitted{27.01.1984}
\Language{CDC: Compass, IBM: Assembler}      %\Revised{}
\Cernhead {Assure Floating or Integer Representation}
\begin{center}
\fbox{\parbox{120mm}{
\begin{center}
{\bf OBSOLETE}
\end{center}
Please note that this routine has been obsoleted in CNL 204. Users are
advised not to use it any longer and to replace it in older programs.
No maintenance for it will take place and it will eventually disappear.
}}
\end{center}
{\tt FLOARG} returns the {\tt REAL} representation of its argument,
making a conversion from {\tt INTEGER} to {\tt REAL} if and only if
the argument is an {\bf integer}. Similarly, {\tt INTARG} returns the
{\tt INTEGER} representation of its argument.
\par
For the purpose of this package, the contents of a computer word will
be considered an {\bf integer} if and only if the exponent part of the
word is zero.
\Structure
{\tt FUNCTION} subprogram \\
User Entry Names: \Rdef{FLOARG}, \Rdef{INTARG}
\Usage
\begin{DLtt}{1234567890}
\item [FLOARG(A):] If {\tt A} contains an {\bf integer} then
{\tt FLOAT(A)} else {\tt A}.
\item [INTARG(I):] If {\tt I} contains an {\bf integer} then
{\tt I} else {\tt IFIX(I)}.
\end{DLtt}
\Restrict
\begin{tabular}{ll}
IBM: & \parbox[t]{146mm}{
Integers larger than 16,777,215 will be considered
floating point numbers without causing an error indication.} \\
CDC: & \parbox[t]{146mm}{
Integers larger than 40,737,488,355,327 will be considered
floating point numbers without causing an error indication.} \\
VAX: & \parbox[t]{146mm}{
These routines do not attempt conversion on a VAX and merely
return the input value unchanged.} \\
Apollo: & \parbox[t]{146mm}{
Integers larger than 8,388,607 will be considered
floating point numbers without causing an error indication.}
\end{tabular}
\Notes
{\tt FLOARG} and {\tt INTARG} will return an incorrect result if called
with an integer whose absolute value exceeds the above limit.
\\ $\bullet$
