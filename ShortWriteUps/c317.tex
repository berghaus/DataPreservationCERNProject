% 07 nov 95  ksk
\Version{CPSIPG}                       \Routid{C317}
\Keywords{PSI DIGAMMA POLYGAMMA GAMMA LOGARITHMIC DERIVATIVE FUNCTION COMPLEX}
\Author{K.S. K\"olbig}                 \Library{MATHLIB}
\Submitter{}                           \Submitted{15.11.1995}
\Language{Fortran}   %              \Revised{ }
\Cernhead{Psi (Digamma) and Polygamma Functions for Complex Argument}
Function subprograms {\tt CPSIPG} and {\tt WPSIPG} calculate either
the logarithmic derivative of the gamma function (the psi, or
digamma, function)
$$ \psi(z) \ \equiv \ \psi^{(0)}(z) \ = \ \frac{d\ln \Gamma(z)} {dx}$$
or one of the other polygamma functions
$$ \psi^{(k)}(z) \ = \ \frac{d^k}{dz^k} \, \psi(z) \ = \
\frac{d^{k+1}}{dz^{k+1}} \, \ln \Gamma(z) $$
for complex arguments $ z \neq -n,(n=0,1,2,\ldots)$ and $k = 0,1,2,3,4$.
\par
The double-precision version {\tt WPSIPG} is available only on
computers which support a {\tt COMPLEX*16} Fortran data type.
\Structure
{\tt FUNCTION} subprograms\\
User Entry Names: \Rdef{CPSIPG}, \Rdef{WPSIPG} \\
Files Referenced: {\tt Unit 6} \\
External References: \Rind{MTLMTR}{N002}, \Rind{ABEND}{Z035}
\Usage
In any arithmetic expression,
\begin{center}
{\tt CPSIPG(Z,K)} \quad or \quad {\tt WPSIPG(Z,K)} \quad has the value
\quad $\psi^{(\mathtt{K})}(\mathtt{Z})$,
\end{center}
where {\tt CPSIPG} is of type {\tt COMPLEX}, {\tt WPSIPG} is of type
{\tt COMPLEX*16}, and where {\tt Z} has the same type as the
function name. {\tt K} is of type {\tt INTEGER}.
\Method
The method for $\psi(z)$ described in Ref. 1 is adapted accordingly.
\Accuracy
{\tt CPSIPG} (except on CDC and Cray computers)
has full single-precision accuracy.
For most values of the argument {\tt Z}, {\tt WPSIPG}
(and {\tt CPSIPG} on CDC and Cray computers) has an accuracy of
approximately two significant digit less than the machine precision.
\Errorh
Error {\tt C317.1:} $\mathtt{K < 0}$ or $\mathtt{K > 4}$. \\
Error {\tt C317.2:} $\mathtt{X} = -n, (n=0,1,2,\ldots)$. \\
In both cases, the function value is set to zero, and a message is
written on {\tt Unit 6}, unless subroutine {\tt MTLSET} (N002) has
been called.
\Refer
\begin{enumerate}
\item K.S. K\"olbig, Programs for computing the logarithm of the gamma
function, and the digamma function, for complex arguments,
Computer Phys. Comm. {\bf 4} (1972) 221-226.
\end{enumerate}
$\bullet$
