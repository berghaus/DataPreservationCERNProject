% 05 jan 95  ksk
\Version{RBZEJY}                \Routid{C345}
\Keywords{BESSEL FUNCTION ZEROS}
\Author{K.S. K\"olbig}           \Library{MATHLIB}
\Submitter{}                \Submitted{01.08.1989}
\Language{Fortran}                \Revised{01.12.1994}
\Cernhead{Zeros of Bessel Functions J and Y}
Subroutine subprograms {\tt RBZEJY} and {\tt DBZEJY} calculate,
for real order $a \ge 0$, the first $N > 0$ zeros
$$j_{a,n}, \ y_{a,n}, \ j'_{a,n}, \ y'_{a,n} \qquad (n=1,2,\ldots,N)$$
of the Bessel functions $J_a(x),\,Y_a(x),\,J'_a(x),\,Y'_a(x)$,
respectively. The prime denotes the derivative of the function with
respect to $x$.
\par
On CDC and Cray computers, the double-precision version
{\tt DBZEJY} is not available.
\Structure
{\tt SUBROUTINE} subprograms \\
User Entry Names: \Rdef{RBZEJY}, \Rdef{DBZEJY} \\
Obsolete User Entry Names: \Rdef{BZEJY} $\equiv$ \Rdef{RBZEJY} \\
Files Referenced: {\tt Unit 6} \\
External References: \Rind{MTLMTR}{N002}, \Rind{ABEND}{Z035}
\Usage
For $\mathtt{t=R}$ (type {\tt REAL}), $\mathtt{t=D}$ (type
{\tt DOUBLE PRECISION}),
\begin{verbatim}
    CALL tBZEJY(A,N,MODE,REL,X)
\end{verbatim}
\begin{DLtt}{123456}
\item[A] (type according to {\tt t}) Order $a$.
\item[N]({\tt INTEGER}) Number $N$ of zeros wanted.
\item[MODE]({\tt INTEGER}) defines the function for which the zeros
are to be calculated:
\begin{DLtt}{1234}
\item[1] zeros of $J_a(x)$,
\item[2] zeros of $Y_a(x)$,
\item[3] zeros of $J'_a(x)$,
\item[4] zeros of $Y'_a(x)$.
\end{DLtt}
\item[REL] (type according to {\tt t}) The requested relative accuracy.
\item[X] (type according to {\tt t}) One-dimensional
array of length {\tt N} at least.
On exit, {\tt X(n)}, ($\mathtt{n=1,2,\ldots,N}$) contains
the first {\tt N} positive (in the case $\mathtt{A=0}$ and
$\mathtt{MODE=3}$,
non-negative) zeros of the function defined by {\tt MODE}.
\end{DLtt}
\Method
Initial approximations to the zeros are computed from asymptotic
expansions. These values are improved by higher-order Newton iteration
making use of the differential equation for the Bessel functions. (For
details see Ref. 1).
\Errorh
Error {\tt C345.1}: $\mathtt{A<0}.$
A message is written on {\tt Unit 6}, unless subroutine {\tt MTLSET}
(N002) has been called. The contents of {\tt X} is left unchanged.
$\mathtt{N \le 0}$ acts as do nothing.
\newpage
\Source
The subroutine is based on Algol procedures published in the References.
\Refer
\begin{enumerate}
\item N.M. Temme, An algorithm with Algol60 program for the computation
of the zeros of ordinary Bessel functions and those of their
derivatives, J. Comput. Phys. {\bf 32} (1979) 270--279.
\item N.M. Temme, On the numerical evaluation of the ordinary Bessel
function of the second kind, J. Comput. Phys. {\bf 21} (1976) 343--350.
\end{enumerate}
$\bullet$
