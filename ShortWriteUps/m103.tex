\Version{FLPSOR}                         \Routid{M103}
\Keywords{ARRAY SORT IN-SITU ITSELF}
\Author{H. von Eicken}                     \Library{KERNLIB}
\Submitter{}                                \Submitted{15.09.1978}
\Language{Fortran}    %\Revised{}
\Cernhead{Sort One-Dimensional Array into Itself}
The {\tt FLPSOR} package contains two entry points for sorting a
one-dimensional array, containing either floating point number or
integers, into itself. The sort is done in ascending order.
\Structure
{\tt SUBROUTINE} subprogram \\
User Entry Names: \Rdef{FLPSOR}, \Rdef{INTSOR}
\Usage
\begin{verbatim}
    CALL FLPSOR(A,N)
\end{verbatim}
sorts the first {\tt N} elements of the {\tt REAL} array {\tt A} in
ascending order into itself.
\begin{verbatim}
    CALL INTSOR(IA,N)
\end{verbatim}
sorts the first {\tt N} elements of the {\tt INTEGER} array {\tt IA} in
ascending order into itself.
\par
For more details, see {\bf Long Write-up} for {\tt SORTZV} (M101).
\Source
Based on an Algol procedure described in Ref. 1.
\Refer
\begin{enumerate}
\item R.S. Scowen, Algorithm 271 QUICKERSORT, Collected Algorithms
from CACM (1965).
\end{enumerate}
$\bullet$
