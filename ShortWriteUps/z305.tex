% 20 may 1992  mg
\Version {VMPACK}                    \Routid{Z305}
\Keywords{IBM INTERFACE SYSTEM VM/CMS VMBEEP VMCMS VMPACK VMREXX VMSATN VMTATN}
\Keywords{FORTRAN REXX VMRTRM}
\Author{R. Matthews, A. Cass}        \Library{KERNLIB, IBM  VM/CMS only}
\Submitter{}                \Submitted{08.06.1989}
\Language{IBM Assembler}           %   %\Revised{}
\Cernhead {IBM VM/CMS System Interface}
{\tt VMPACK} is a package of Fortran--callable subprograms
providing an interface to the VM/CMS operating system.
\Structure
{\tt SUBROUTINE} subprograms \\
User Entry Names:
\Rdef{VMBEEP}, \Rdef{VMCMS},  \Rdef{VMREXX}, \Rdef{VMRTRM},
\Rdef{VMSATN}, \Rdef{VMTATN} \\
Files Referenced: Virtual console
\Usage
{\bf Subroutine VMBEEP:}
\begin{verbatim}
    CALL VMBEEP
\end{verbatim}
will sound the alarm on the virtual console. \\[3mm]
{\bf Subroutine VMCMS:} \\[2mm]
This subroutine can be used to issue CMS or CP commands from a
Fortran program. They are processed in the same way as commands
issued from a {\tt REXX} exec in which {\tt 'ADDRESS COMMAND'}
has been specified, i.e. the {\tt EXEC} command must be used to invoke
execs and CP commands must be prefixed by the CMS command prefix CP.
\begin{verbatim}
    CALL VMCMS(COMAND,IRC)
\end{verbatim}
\begin{DLtt}{12345678}
\item[COMAND]({\tt CHARACTER}) A command.
\item[IRC] ({\tt INTEGER}) Contains, on exit, the return code from the
command.
\end{DLtt}
Refer to the appropriate IBM publications for a list and explanation of
the return codes which may be produced by the command.
\Examples
\begin{verbatim}
    CALL VMCMS('FILEDEF 27 TERMINAL',IRC)
    MSG='EXEC TELL JIM THE FILEDEF HAS BEEN ISSUED'
    CALL VMCMS(MSG,IRC)
\end{verbatim}
\Restrict
It is not permitted to issue a CMS command which runs in the user area,
nor a command which invokes another program, as this will overwrite
the program which is executing.
The commonly used commands which run in the user area are: \\[2mm]
\begin{tabular}{l*{8}{@{\hspace{4pt}}l}}
{\tt ASSEMBLE}, & {\tt COPYFILE}, & {\tt EXECUPDT}, & {\tt FORMAT}, &
{\tt LISTDS},   & {\tt LKED},     & {\tt LOADLIB},  {\tt MACLIB},   &
{\tt MACLIST},  \\
{\tt MODMAP},   & {\tt MOVEFILE}, &  {\tt OSRUN},   & {\tt SORT},   &
{\tt TAPEMAC},  & {\tt TAPPDS},   & {\tt TXTLIB},   & {\tt UPDATE}.
\end{tabular} \\[2mm]
\newpage
{\bf Subroutine VMREXX:} \\[2mm]
This subroutine can be called from a program which has been invoked
from a {\tt REXX} exec to set, inspect or drop any of the exec's
variables.
\begin{verbatim}
    CALL VMREXX(CODE,NAME,VALUE,IRC)
\end{verbatim}
\begin{DLtt}{123456}
\item[CODE] ({\tt CHARACTER*1}) Indicates the function to be performed:
\begin{DLtt}{12}
\item[A] Move {\tt REXX} stem variables to Fortran array.
See {\bf Examples}.
\item[B] Move Fortran array to {\tt REX} stem variables.
See {\bf Examples}.
\item[D] Drop the variable. If the name given is a stem, all
variables with that stem are dropped.
\item[F] Fetch the value of the variable.
\item[S] Set the value of the variable.
\item[N] Fetch the next variable. This function code be used
to search through all the variables of the exec file.
\item[P] Fetch 'private' information. When this function code is used
the variable name refers to certain fixed information:
\begin{DLtt}{12345678}
\item[ARG] Fetch the primary argument list.
\item[SOURCE] Fetch the source string as provided by the
{\tt REXX PARSE SOURCE} instruction.
\item[VERSION] Fetch the version string as provided by the
{\tt REXX PARSE VERSION} instruction.
\end{DLtt}
\end{DLtt}
\item[NAME] ({\tt CHARACTER}) Name of the {\tt REXX} variable as a
constant or variable (in UPPER case).
\item[VALUE] ({\tt CHARACTER}) Value of the {\tt REXX} variable as a
constant or variable.
\item[IRC] ({\tt INTEGER}) Return code. Contains, on exit,
one of the following values to indicate the result of the operation:
\begin{DLtt}{1234}
\item[\ \ 0] Execution was successful.
\item[\ \ 1] The variable did not exist.
\item[\ \ 2] Last variable transferred for function code {\tt 'N'}.
\item[\ \ 4] Truncation occurred during fetch.
\item[\ \ 8] Invalid variable name.
\item[\ 16] Invalid integer in {\tt stem.0} for function codes
{\tt 'A'} and {\tt 'B'}.
\item[128] Invalid function code.
\end{DLtt}
\end{DLtt}
\Examples
\begin{verbatim}
    CALL VMREXX('P','ARG',CHARS,IRC)
\end{verbatim}
will store the value of the primary argument list in
the 80-character variable {\tt CHARS}. If the argument list contains
more than 80 characters, then only the first 80 will be transferred and
the value of the return code will be $\mathtt{IRC = 4}$.
\par
Fetching and Setting Stem variables: Function codes {\tt 'A'} and
{\tt 'B'} allow transfer of data between Fortran arrays and
{\tt REXX} stem variables. The {\tt REXX} variable {\tt stem.0} must
contain the number of elements to be transferred. Then {\tt 'A'} moves
data from {\tt REXX} to Fortran, {\tt 'B'} from Fortran
to {\tt REXX}:
\newline
\begin{verbatim}
    /**/
    line.0 = 5
    line.1 = 'Hello'
    line.2 = 33
    line.3 = 'alpha'
    line.4 = 'Beta'
    line.5 = 8567
    'LOAD TEST(START'
    Do l = 1 To 5
    Say mine.l
    End
 
    PROGRAM TEST
    CHARACTER*5 GET(MAX),SET(MAX)
    DATA SET/'DUMMY','LINES',' TO  ','GIVE ','REXX '/
    CALL VMREXX('A','LINE.',GET,IRC)
    WRITE (6,'(1X,A10)') GET
    CALL VMREXX('S','MINE.0','5',IRC)
    CALL VMREXX('B','MINE.',SET,IRC)
    END
\end{verbatim}
If an error is detected accessing the {\tt stem.0} variable then the
return code is set and processing stops -- no data is transferred.
Otherwise all elements are processed regardless of truncation or
existence errors and the return code is set to zero.
Return code {\tt 16} indicates that the contents of {\tt stem.0}
were not a valid integer. Return code {\tt 4} implies the same or that
the integer was larger than {\tt 999,999,999}.
\Restrict
Note that the {\tt REXX} variables within an Exec which calls a
second Exec are not visible within the second Exec. This means
that, for example, {\tt VMREXX} has no knowledge of the variables within
an Exec which invokes program execution by using the {\tt GO} option
of the {\tt VFORT} exec.
\par
The maximum number of elements that can be transferred with
codes {\tt 'A'} and {\tt 'B'} is {\tt 999,999,999}. \\[3mm]
{\bf Subroutine VMRTRM:} \\[2mm]
This subroutine will read a line from the console stack or, if the
stack is empty, directly from the console (i.e. the user must type
in a line):
\begin{verbatim}
    CALL VMRTRM(LINE,LENGTH)
\end{verbatim}
\begin{DLtt}{12345678}
\item[LINE] ({\tt CHARACTER}) Contains, on exit, the line
read from the console.
\item[LENGTH] ({\tt INTEGER}) Number of characters placed in {\tt LINE}.
\end{DLtt}
If the data read from the console exceeds the length of {\tt LINE}
then it is truncated. \\[3mm]
\newpage
{\bf Subroutines VMSATN and VMTATN:} \\[2mm]
These subroutines can be used by interactive programs to process
attention interrupts from the virtual console. An attention interrupt
occurs when the user enters a string of characters on the command line
and presses the enter (or return) key at a time when the program has not
issued a read to the console, i.e. the user wishes to bring something to
the attention of the program even though the program has not requested
any input. This mechanism may be used, for example, to temporarily halt
a long-running program in order to query its status.
 
{\tt VMSATN} is used to establish the attention interrupt
mechanism and must be called before any attention
interrupts can be detected:
\begin{verbatim}
    CALL VMSATN(IRC)
\end{verbatim}
\begin{DLtt}{1234}
\item[IRC] ({\tt INTEGER}) Contains, on exit, the value zero if the
attention mechanism has been established, and a non-zero value otherwise.
\end{DLtt}
{\tt VMTATN} is used to determine whether an attention
interrupt has occurred. A Fortran {\tt READ} issued to the
virtual console (Fortran logical unit 5) can then be used
to obtain the character string that was entered:
\begin{verbatim}
    CALL VMTATN(IRC)
\end{verbatim}
\begin{DLtt}{1234}
\item[IRC] ({\tt INTEGER}) Contains, on exit, a non-zero
value if an interrupt has occured, and the value zero otherwise.
\end{DLtt}
\Examples
The following example shows a program which establishes the
attention mechanism and then performs an iterative process.
\par
{\tt VMTATN} is called at the beginning of each iteration
to determine whether an attention interrupt has occurred:
\begin{verbatim}
*...  ESTABLISH THE ATTENTION MECHANISM
      CALL VMSATN(IRC)
      IF(IRC .NE. 0) THEN
        process the error
      ENDIF
*...  PERFORM ITERATIVE PROCESSING
    1 CALL VMTATN(IRC)
      IF(IRC .NE. 0) THEN
        READ '(A)',LINE
        process the attention interrupt
      ELSE
        perform normal processing
      ENDIF
      GOTO 1
      END
\end{verbatim}
\Restrict
CMS support for immediate commands (e.g. HX) requires that CMS
handles attention interrupts from the virtual console. Immediate
commands will therefore not be recognized by CMS when {\tt VMSATN} and
{\tt VMTATN} are used to handle attention interrupts. The PA1 key may,
however, be used to interrupt execution of the virtual machine and
put the virtual console into 'CP READ' status.
\\ $\bullet$
