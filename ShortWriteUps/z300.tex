\Version{IOPACK}                           \Routid{Z300}
\Keywords{INPUT IOBACK IODD IOGET IOMARK IOOPTN IOPACK IOPUT IOREAD IORITE}
\Keywords{IORWND IOS IOSTOP OUTPUT PACK IOXOPN IO I/O IOXCLS}
\Author{R. Matthews}                      \Library{PACKLIB, IBM only}
\Submitter{}                              \Submitted{01.12. 1981}
\Language{Assembler}                     %\Revised{}
\Cernhead{IBM-Dependent Input/Output Package}
The standard input and output I/O-facilities available to IBM
Fortran programs contain a number of deficiencies. In particular,
only data sets which have a record format of {\tt VS} or {\tt VBS} can
be processed without the use of a {\tt FORMAT} statement. If a data set
is not of this type one must resort either to time-consuming formatted
I/O or to the use of special-purpose subprograms. {\tt IOPACK} attempts
to overcome the deficiencies by providing a set of subprograms for
reading, writing and manipulating data sets.
\par
The subprograms can process sequential data sets or members of
partitioned data sets on any type of device and one may refer to several
streams concurrently. Facilities are also provided for the specification
of {\tt DD} statements during program execution.
\Structure
{\tt SUBROUTINE} subprograms \\
User Entry Names:
\begin{tabular}[t]{l*{7}{@{\hspace{4pt}}l}}
\Rdef{IOBACK}, & \Rdef{IODD},   & \Rdef{IOGET},  & \Rdef{IOMARK}, &
\Rdef{IOOPTN}, & \Rdef{IOPUT},  & \Rdef{IOREAD}, & \Rdef{IORITE}, \\
\Rdef{IORWND}, & \Rdef{IOSTOP}
\end{tabular} \\[1mm]
Internal Entry Names:
{\tt IOXALOC}, {\tt IOXALOD}, {\tt IOXCLS}, {\tt IOXMSG}, {\tt IOXOPN},
{\tt IOXTBL}, {\tt IOXTED}, {\tt IOCTRL}
\Usage
See {\bf Long Write-up}.
\\ $\bullet$
