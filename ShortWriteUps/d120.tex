%   07 nov 95   ksk
\Version{RADMUL}                          \Routid{D120}
\Keywords{NUMERIC MUMERICAL INEGRATION MULTI DIMENSIONAL ADAPTIVE}
\Authors{A.C. Genz, A.A. Malik}        \Library{MATHLIB}
\Submitter{K.S. K\"olbig}                \Submitted{15.11.1995}
\Language{Fortran}                   %   \Revised{}
\Cernhead{Adaptive Quadrature for Multiple Integrals over $N$-Dimensional Rectangular Regions}
Subroutine subprograms {\tt RADMUL} and {\tt DADMUL} compute, to an
attempted specified accuracy, the value of the integral
$$ I_n \ = \ \displaystyle
\int_{a_n}^{b_n} \int_{a_{n-1}}^{b_{n-1}} \cdots
\int_{a_1}^{b_1} f(x_1,x_2,\ldots,x_n)\,dx_1\,dx_2\,\cdots\,dx_n $$
over an $n$-dimensional rectangular region, where
$a_i,\,b_i$, ($i=1,2,\ldots,n$) are constants.
\par
On computers other than CDC and Cray, only the double-precision version
{\tt DADMUL} is available. On CDC and Cray computers,
only the single-precision version {\tt RADMUL} is available.
\Structure
{\tt SUBROUTINE} subprograms \\
User Entry Names : \Rdef{RADMUL}, \Rdef{DADMUL}\\
External References: User-supplied {\tt SUBROUTINE} subprogram
\Usage
For $\mathtt{t=R}$ (type {\tt REAL}), $\mathtt{t=D}$ (type
{\tt DOUBLE PRECISION}),
\begin{verbatim}
    CALL tADMUL(F,N,A,B,MINPTS,MAXPTS,EPS,WK,IWK,RESULT,RELERR,NFNEVL,IFAIL)
\end{verbatim}
\begin{DLtt}{123456}
\item[F] (type according to {\tt t}) Name of a user-supplied
{\tt FUNCTION} subprogram, declared {\tt EXTERNAL} in the calling
program.
\item[N] ({\tt INTEGER}) Number $n$ of dimensions
($2 \le \mathtt{N} \le 15$).
\item[A,B] (type according to {\tt t}) One-dimensional arrays of length
$\mathtt{\ge N}$. On entry, $\mathtt{A(i)}$ and $\mathtt{B(i)}$, \\
($\mathtt{i=1,\ldots,N}$), contain the lower and upper limits of
integration, respectively. Note that $a_i,\,b_i$ correspond to $x_i$.
\item[MINPTS] ({\tt INTEGER}) Minimum number of function
evaluations requested. Must not exceed {\tt MAXPTS}.
\item[MAXPTS] ({\tt INTEGER}) Maximum number
($\ge$ {\tt 2**N} $+$ {\tt 2N(N} $+$ {\tt 1)} $+$ {\tt 1})
of function evaluations to be allowed.
\item[EPS] (type according to {\tt t}) Specified {\it relative} accuracy.
\item[WK] (type according to {\tt t}) One-dimensional array of length
{\tt IWK}, used as working space.
\item[IWK] ({\tt INTEGER}) Length ($\ge$ ({\tt 2N} $+$ {\tt 3}) $*$
({\tt 1} $+$ {\tt MAXPTS}$/$({\tt 2**N} $+$ {\tt 2N(N} $+$ {\tt 1)}
$+$ {\tt 1}))$/${\tt 2}) of {\tt WK}.
\item[RESULT] (type according to {\tt t}) Contains, on exit, an
approximate value of the integral $I_n$.
\item[RELERR] (type according to {\tt t}) Contains, on exit, an
estimation of the relative accuray of {\tt RESULT}.
\item[NFNEVL] ({\tt INTEGER}) Contains, on exit, the number
of function evaluations performed.
\newpage
\item[IFAIL] ({\tt INTEGER}) On exit:
\begin{DLtt}{12}
\item[0] Normal exit. $\mathtt{RELERR < EPS}$. At least {\tt MINPTS}
and at most {\tt MAXPTS} calls to the function {\tt F} were performed.
\item[1] {\tt MAXPTS} is too small for the specified accuracy {\tt EPS}.
{\tt RESULT} and {\tt RELERR} contain the values obtainable
for the specified value of {\tt MAXPTS}.
\item[2] {\tt IWK} is too small for the specified number {\tt MAXPTS}
of function evaluations. \\
{\tt RESULT} and {\tt RELERR} contain
the values obtainable for the specified value of {\tt IRK}.
\item[3] $\mathtt{N<2}$, or $\mathtt{N>15}$, or
$\mathtt{MINPTS>MAXPTS}$, or {\tt MAXPTS} $<$
{\tt 2**N} $+$ {\tt 2N(N} $+$ {\tt 1)} $+$ {\tt 1}. \\
{\tt RESULT} and {\tt RELERR} are set equal to zero.
\end{DLtt}
\end{DLtt}
\par
The user-supplied {\tt FUNCTION} subprogram {\tt F} should be of
the form
\begin{DLtt}{123456}
\item[] {\tt FUNCTION F(N,X)}
\item[] {\tt DIMENSION X(*)}
\item[] {\tt ...}
\item[] $\mathtt{F = }f(\mathtt{X(1),...,X(N)})$.
\item[] {\tt RETURN}
\item[] {\tt END}
\end{DLtt}
where {\tt X} and {\tt F} are of type {\tt t}.
\Method
An integration rule of degree seven is used together
with a certain strategy of subdivision.
For a more detailed description of the method see {\bf References}.
\Errorh
See description of argument {\tt IFAIL}.
\Notes
\begin{enumerate}
\item Multi-dimensional integration is time-consuming.
For each rectangular subregion, the routine requires $2^n+2n^2+2n+1$
function evaluations. Careful
programming of the integrand might result in substantial saving of time.
\item Numerical integration usually works best for smooth functions.
Some analysis or suitable transformations of the integral
prior to numerical work may contribute to numerical efficiency.
\end{enumerate}
\Source
This subroutine is an adapted version of Fortran program {\tt ADAPT}
published in Ref. 1.
\Refer
\begin{enumerate}
\item  A.C. Genz and A.A. Malik, Remarks on algorithm 006: An adaptive
algorithm for numerical integration over an $N$-dimensional rectangular
region, J. Comput. Appl. Math. {\bf 6} (1980) 295--302.
\item  A. van Doren and L. de Ridder, An adaptive
algorithm for numerical integration over an $n$-dimensional cube,
J. Comput. Appl. Math. {\bf 2} (1976) 207--217.
\end{enumerate}
A copy of the text part of the References is available.
\\ $\bullet$
