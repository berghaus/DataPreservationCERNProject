\Version{FUMILI}                \Routid{D510}
\Keywords{CHISQUARE FIT FITTING FUNCTION LIKELIHOOD}
\Author{I. Silin}                \Library{MATHLIB}
\Submitter{A. Kobine}             \Submitted{05.04.1971}
\Language{Fortran}                 \Revised{18.11.1985}
\Cernhead {Fitting Chisquare and Likelihood Functions}
\begin{center}
\fbox{\parbox{120mm}{\OBSOLETE
Please note that this routine has been obsoleted in CNL 211. Users are
advised not to use it any longer and to replace it in older programs.
No maintenance for it will take place and it will eventually disappear.
\\[3mm]
Suggested replacement: {\tt LEAMAX} (D501)
}}
\end{center}
{\tt FUMILI} minimizes the objective functions $\chi^2/2 $
and ML defined by:
$$ \frac{1}{2}\chi^2 \ = \ \frac{1}{2} \sum_{j=1}^N \left[
\frac{Y_j^*-
Y(X_j^{(1)},\ldots,X_j^{(L)};A_1,\ldots,A_M)}
{\Delta Y_j^*}\right ]^2 $$
and
$$ ML \ = \ \sum_{j=1}^N -\ln[Y(X_j^{(1)},\ldots,X_j^{(L)};A_1,
\ldots,A_M)]$$
with respect to the $M$ parameters $A$ where, for each $j$,
$ 1\leq j\leq N, Y_j^*$ is a data-point with user estimated error,
$\pm\Delta Y_j$, the $ X_j $ are $L$ co-ordinates of that point and $Y$
is a theoretical function predicting $ Y_j^*$ for a given set of $X_j$
and $A$.
\par
The method makes use of a particular property concerning the
dependence of the objective function ($\chi^2/2$ or $ML$) on the
theoretical function ($Y$) for faster convergence.
\Structure
{\tt SUBROUTINE} subprograms \\
User Entry Names: \Rdef{FUMILI}, \Rdef{LIKELM}, \Rdef{ERRORF}\\
Internal Entry Names: {\tt ARITHM}, {\tt D510BD}, {\tt FUNCT},
{\tt MCONV}, {\tt MONITO}, {\tt SCAL}, {\tt SGZ}\\
Files Referenced: Printer  \\
External References: User-supplied  {\tt FUNCTION} and
(optional) {\tt SUBROUTINE} subprograms \\
{\tt COMMON} Block Names and Lengths: \parbox[t]{90mm}{
{\tt /A/ 100}, {\tt /AL/ 100}, {\tt /AU/ 100}, {\tt /DA/ 100},
{\tt /DF/ 100}, \\
{\tt /ENDFLG/ 7}, {\tt /ERROR/ 500}, {\tt /EXDA/ 1500}, {\tt /G/ 100},\\
{\tt /NED/ 2}, {\tt /PL/ 100}, {\tt /PLU/ 100}, {\tt /R/ 100}, \\
{\tt /SIGMA/ 100}, {\tt /X/ 10}, {\tt /Z/ 2485}, {\tt /Z0/ 2485}}
\Usage
See {\bf Long Write-up}.
\Refer
\begin{enumerate}
\item Preprint YINDR-810, 1961 (Dubna) (CERN Library, preprint P. 810).
\end{enumerate}
$\bullet$
