%  18 oct 94   ksk
\Version{RGQUAD}                             \Routid{D107}
\Keywords{GAUSS GAUSSIAN NUMERICAL INTEGRATION QUADRATURE}
\Author{G.A. Erskine}                       \Library{MATHLIB}
\Submitter{K.S. K\"olbig}                   \Submitted{07.06.1992}
\Language{Fortran}                          \Revised{ }
\Cernhead{N-Point Gaussian Quadrature}
Function subprograms {\tt RGQUAD} and {\tt DGQUAD} calculate the
approximate value of the integral
 $$ \int_a^b f(t)dt $$
using the $N$-point Gauss-Legendre quadrature formula corresponding to
the interval $[a,b]$.
\par
Subroutine subprograms {\tt RGSET} and {\tt DGSET} store, for
subsequent use, the abscissae $x_i$ and the weights $w_i$ of the
$N$-point Gauss-Legendre quadrature formula corresponding to the
interval $[a,b]$.
\par
The following values of $N$ may be used:
2, 3, 4, 5, 6, 7, 8, 9, 10, 11, 12, 13, 14, 15, 16,
20, 24, 32, 40, 48, 64, 80, 96.
\par
{\tt RGQUAD}, {\tt RGSET} and {\tt DGQUAD}, {\tt DGSET} are independent
subprograms: it is not necessary, for instance, to call {\tt DGSET} in
order to use {\tt DGQUAD}, or vice-versa.
\par
On CDC and Cray computers, the double-precision versions {\tt DGQUAD}
and {\tt DGSET} are not provided.
\Structure
{\tt SUBROUTINE} and {\tt FUNCTION} subprograms \\
User Entry Names: \Rdef{RGQUAD}, \Rdef{RGSET}, \Rdef{DGQUAD},
\Rdef{DGSET} \\
Internal Entry Names: {\tt D107R1}, {\tt D107D1}  \\
Files Referenced: {\tt Unit 6} \\
External References: \Rind{MTLMTR}{N002}, \Rind{ABEND}{Z035},
User-supplied {\tt FUNCTION} subprogram
\Usage
{\bf To calculate the integral:}  \\[2mm]
For $\mathtt{t=R}$ (type {\tt REAL}), $\mathtt{t=D}$ (type
{\tt DOUBLE PRECISION}),
\begin{verbatim}
    tGQUAD(F,A,B,N)
\end{verbatim}
has, in any arithmetic expression, the value
$\displaystyle \sum_{i=1}^N w_if(x_i)$
which is an approximation to the given integral. \\[3mm]
{\bf To store the abscissae $x_i$ and the weights $w_i$:}
\begin{verbatim}
    CALL tGSET(A,B,N,X,W)
\end{verbatim}
\begin{DLtt}{1234}
\item[F] (type according to {\tt t}) Name of a user-supplied
{\tt FUNCTION} subprogram, declared {\tt EXTERNAL} in the calling
program. This subprogram must set $\mathtt{F(X)} = f(\mathtt{X})$.
\item [A,B] (type according to {\tt t}) End-points $a$ and $b$ of
the integration interval.
\item [N] ({\tt INTEGER}) Number $N$ of quadrature points.
\item[X,W] (type according to {\tt t}) One-dimensional arrays. On exit,
{\tt X(i)} and {\tt W(i)} contain $x_i$ and $w_i,\,(i=1,2,\ldots,N)$,
respectively.
\end{DLtt}
\Method
The values of $x_i$ and $w_i$ are computed by linearly scaling values
obtained from a stored table corresponding to the interval $[-1,+1]$.
\newpage
\Accuracy
The absolute error of {\tt RGQUAD} and {\tt DGQUAD} is proportional to
the value of the $(2N)$th derivative of $f$ at some internal point
of the interval $[a,b]$ (see Ref. 1).
\Errorh
Error {\tt D107.1:} The value {\tt N} does not appear in the list given
above. A message is written on {\tt Unit 6}, unless subroutine
{\tt MTLSET} (N002) has been called. If the subprogram referenced is
{\tt RGQUAD} or {\tt DGQUAD}, the function value is set equal to zero.
\Refer
\begin{enumerate}
\item A.H. Stroud and D. Secrest, Gaussian quadrature formulas,
(Prentice-Hall, Englewood Cliffs 1966).
\end{enumerate}
$\bullet$
