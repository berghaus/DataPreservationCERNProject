\Version{ROT}                             \Routid{F118}
\Keywords{ROTATION VECTOR}
\Author{CERN TC Division}                   \Library{KERNLIB}
\Submitter{C. Letertre}                     \Submitted{01.09.1969}
\Language{Fortran}                    %\Revised{}
\Cernhead{Rotating a 3-Vector}
Subroutine subprogram {\tt ROT} rotates a 3-vector $(a_1,a_2,a_3)$
by a given angle $\theta$ around the $z-$axis.
\Structure
{\tt SUBROUTINE} subprogram \\
User Entry Names: \Rdef{ROT}\\
{\tt COMMON} Block Names and Lengths: {\tt /SLATE/ 40}
\Usage
\begin{verbatim}
    CALL ROT(A,TH,B)
\end{verbatim}
\begin{DLtt}{1234}
\item [A] ({\tt REAL}) One-dimensional array of length {\tt 3},
containing $(a_1,a_2,a_3)$.
\item [TH] ({\tt REAL}) Angle $\theta$ given in radians.
\item [B]  ({\tt REAL}) One-dimensional array of length {\tt 3}. On exit,
{\tt B} contains the components $(b_1,b_2,b_3)$ of the rotated vector,
i.e.
\item [] $b_1=a_1\cos \theta -a_2\sin \theta$
\item [] $b_2=a_1\sin \theta +a_2\cos \theta$
\item [] $b_3=a_3$.
\end{DLtt}
{\tt B} may overlap {\tt A}.
\\ $\bullet$
