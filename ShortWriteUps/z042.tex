%  20 feb 1995   ksk
\Version{JUMPXN}                                 \Routid{Z042}
\Keywords{JUMPAD JUMPST JUMPX}
\Author{J.Zoll, R.Brun et al.}            \Library{KERNLIB}
\Submitter{J. Zoll}                        \Submitted{27.04.1988}
\Language{Fortran or C or Assembler}        \Revised{20.02.1995}
\Cernhead{Calling a Subroutine by its Address}
The purpose of this package is to provide a (limited) tool to connect
what is called a user-routine with an arbitrary name to a {\tt CALL} in a
package, pre-existing on a library.
\par
Because on most machines {\tt JUMPXn} is implemented in Fortran
or C, separate entries are needed for calling the user-routine
with zero, one, two, ..., nine parameters.
\Structure
{\tt SUBROUTINE} subprogram \\
User Entry Names: \Rdef{JUMPAD}, \Rdef{JUMPST}, \Rdef{JUMPXn},
($\mathtt{n = 0,1,\ldots,9}$) \\
Internal Entry Names: \Rind{JUMPYn}{Z042} ($\mathtt{n = 0,1,\ldots,9}$)
(if not Assembler or C)
\Usage
Three steps are necessary:
\begin{DLtt}{1234}
\item[1)] Get the transfer address {\tt IAD} of the routine
(for example {\tt TARGET}) to be called:
\begin{verbatim}
    EXTERNAL TARGET
    IAD=JUMPAD(TARGET)
\end{verbatim}
\item[2)]  Set the transfer address for the next transfer(s):
\begin{verbatim}
    CALL JUMPST(IAD)
\end{verbatim}
\item[3)] Execute a transfer, for a call with $\mathtt{n=0,1,\ldots,9}$
parameters:
\begin{verbatim}
        CALL JUMPX0
    or  CALL JUMPX1(P1)
        ...
    or  CALL JUMPX9(P1,P2,P3,P4,P5,P6,P7,P8,P9)
\end{verbatim}
\end{DLtt}
\Restrict
\par
Since on most machines {\tt JUMPXn} is written in Fortran or C,
the call to {\tt JUMPXn} will be found in the trace-back of routine
{\tt TARGET}, and {\tt RETURN} from {\tt TARGET} will pass through
{\tt JUMPXn}. Hence, normally (i.e. unless recursion is handled by a
particular machine), {\tt TARGET} or any of its called routines may not
again call {\tt JUMPXn}.
\\ $\bullet$
