%  18 apr 96  ksk
\Version{RIWIAD}                     \Routid{D114}
\Keywords{ADAPTIVE NUMERICAL MULTIDIMENSIONAL INTEGRATION CUBATURE}
\Author{B. Lautrup}                  \Library{MATHLIB}
\Submitter{}                         \Submitted{23.07.1971}
\Language{Fortran}                     \Revised{10.01.1986}
\Cernhead{Adaptive Multidimensional Monte-Carlo Integration }
\begin{center}
\fbox{\parbox{120mm}{\OBSOLETE
Please note that this routine has been obsoleted in CNL 223. Users are
advised not to use it any longer and to replace it in older programs.
No maintenance for it will take place and it will eventually disappear.
\\[3mm]
Suggested replacement: (in part) {\tt RADMUL} (D120) }}
\end{center}
{\tt RIWIAD} is an adaptive multidimensional integration subroutine
based on an original by G. Sheppey. It permits numerical integration
of a large class of functions, in particular those that are irregular
at the border of the integration region. The integral is always
performed over the unit hypercube.
\Structure
{\tt SUBROUTINE} subprogram\\
User Entry Names: \Rdef{RIWIAD}\\
Files Referenced: {\tt Unit 6}\\
External References: \Rind{RNDM}{V104}
user-supplied {\tt FUNCTION} subprogram \\
{\tt COMMON} Block Names and Lengths:
    \begin{tabular}[t]{@{}llll}
        \texttt{/ANSWER/ 2},   &\texttt{/INTERN/ 7},&
          \texttt {/OPTION/ 3},  &\texttt{/PARAMS/ 4}, \\
        \texttt{/RANDOM/ 1},   &\texttt{/STORE/ 77},&
           \texttt{/STORE1/ 10001}
    \end{tabular}
\Usage
See {\bf Long Write-up} for a description of all features.
Here only the standard use is described.
\par
The {\tt COMMON} block {\tt PARAMS} must always
be set by the user:
\begin{center}
{\tt COMMON /PARAMS/ ACC,NDIM,NSUB,ITER}
\end{center}
\begin{DLtt}{123456}
\item [ACC] ({\tt REAL}) Relative accuracy desired.
\item [NDIM] ({\tt INTEGER}) Number of dimension parameters.
\item [NSUB] ({\tt INTEGER}) Number of subvolumes allowed.
\item [ITER] ({\tt INTEGER}) Maximal number of iterations.
\end{DLtt}
The integrand is defined by a user-supplied {\tt FUNCTION}
subprogram having the array {\tt Q(NDIM)} as parameter, for example
\begin{verbatim}
    FUNCTION EXAMPLE(Q)
    REAL EXAMPLE,Q
    DIMENSION Q(7)
    ...
    END
\end{verbatim}
\newpage
This program defines {\tt EXAMPLE} as a function of the 7
variables $\mathtt{Q(1),\,Q(2),\,\ldots,\,Q(7)}$. The sequence
\begin{verbatim}
    EXTERNAL EXAMPLE
    COMMON /PARAMS/ ACC,NDIM,NSUB,ITER
    ACC=0.01
    NDIM=7
    NSUB=10000
    ITER=5
    CALL RIWIAD(EXAMPLE)
    ...
\end{verbatim}
will then integrate {\tt EXAMPLE} over the 7 variables
$\mathtt{Q(1),\,\ldots,\,Q(7)}$,
all in the interval from 0 to 1, i.e. over the 7-dimensional
unit hypercube. The result will be printed in detail in a readily
understandable form.
\par
The program allows extensive user control via the {\tt COMMON} blocks.
See {\bf Long Write-up} for details.
\Method
{\tt RIWIAD} is iterative and in a given
iteration it divides the unit hypercube into a certain number of
subvolumes by means of a given set of intervals on each axis. Within
each subvolume it estimates the mean value and variance of the
integrand by random sampling, and then calculates the Riemann sum over
the subvolumes. Using the variances found projected onto each axis it
calculates a set of new interval divisions to be used in the next
iteration. It returns when the desired accuracy is obtained or when
the maximum number of iterations has been performed.
\Restrict
There is, in principle, no limitations on the number of
dimensions, although the present version only allows up to
9-dimensional integrals. The maximal dimensionality can easily be
increased.
\Notes
\begin{enumerate}
\item The program is rather slow and should preferably be used only
when other methods (for instance Gaussian quadrature) fail due to
the irregular behaviour of the integrand. The time consumption is
essentially proportional to the product of {\tt NSUB} and {\tt ITER}.
\item The non-CDC/Cray implementation of {\tt RIWIAD} has
{\tt IMPLICIT DOUBLE PRECISION(A-H,O-Z)},
i.e. all non-{\tt INTEGER} variables are {\tt DOUBLE PRECISION},
including the user-supplied external function.
\end{enumerate}
$\bullet$
