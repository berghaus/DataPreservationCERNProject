%  13 mar 1996
\Version{RNHRAN}                               \Routid{V149}
\Keywords{DISTRIBUTION HISTOGRAM NUMBER RANDOM}
\Author{F. James, K.S. K\"olbig}                 \Library{MATHLIB}
\Submitter{}                                     \Submitted{20.03.1996}
\Language{Fortran}                      %\Revised{}
\Cernhead{Random Numbers According to Any Histogram}
{\tt RNHRAN} generates random numbers distributed according to
any empirical (one-dimensional) distribution. The distribution is
supplied in the form of a histogram. If the distribution is known
in functional form, {\tt FUNLUX} (V152) should be used instead.
\Structure
{\tt SUBROUTINE} subprograms \\
User Entry Names: \Rdef{RNHRAN}, \Rdef{RNHPRE}\\
Files Referenced: Printer\\
External References: \Rind{LOCATR}{E106}, \Rind{RANLUX}{V115}
\Usage
\begin{tabular}{@{\hspace*{8mm}}l@{\qquad}l}
{\tt CALL RNHPRE(Y,NBINS)}               & (once for each histogram) \\
{\tt CALL RNHRAN(Y,NBINS,XLO,XWID,XRAN)} & (for each random number)
\end{tabular}
\begin{DLtt}{12345678}
\item [Y] Array of length {\tt NBINS} at least containing the desired
distribution as histogram bin contents on input to {\tt RNHPRE}.
\item [NBINS] Number of bins.
\item [XLO] Lower edge of first bin.
\item [XWID] Bin width.
\item [XRAN] A random number returned by {\tt RNHRAN}.
\end{DLtt}
\Method
A uniform random number is generated using {\tt RANLUX} (V115).
The uniform number is then transformed to the
user's distribution using the cumulative probability distribution
constructed from his histogram. The cumulative distribution is
inverted using a binary search for the nearest bin boundary and a
linear interpolation within the bin. {\tt RNHRAN} therefore generates a
constant density within each bin.
\Notes
{\tt RNHPRE} changes the values {\tt Y} to form the cumulative
distribution which is needed by {\tt RNHRAN}. If {\tt Y} already
contains the cumulative distribution rather than the probability
density, then {\tt RNHPRE} should not be called, but in that case
{\tt Y(NBINS)} must be exactly equal to one. Numbers may be drawn from
several different distributions in the same run by calling {\tt RNHRAN}
with different arrays {\tt Y1}, {\tt Y2}, etc. and (if desired)
different values of {\tt NBINS},
{\tt XLO}, {\tt XWID} (but always the same values for a given array
{\tt Y}). The routine {\tt  RNHPRE} should be used to initialize each
array {\tt Yi}.
\par
The performance of the above method is nearly independent of the shape
of the function or number of bins.
\Errorh
If the the input data to {\tt RNHPRE} are not valid (some values
negative or all values zero), an error message is printed, the input
values are printed, and zero is returned instead of a random number.
As many as five such messages may be printed to allow for possible
errors in more than one distribution.
\par
If {\tt RNHPRE} is not called, and the input data are not already in
cumulative form, {\tt RNHRAN} performs the initialization itself and
prints a warning message. {\tt RNHRAN} recognizes that the data are not
in cumulative form if $\mathtt{Y(NBINS) \ne 1}$.
\\ $\bullet$
