\Version {RG32}                            \Routid{V108}
\Keywords{DISTRIBUTION GAUSSIAN GENERATOR NUMBER PORTABLE RANDOM}
\Author{F. James}                     \Library{MATHLIB,IBM and VAX only}
\Submitter{T. Lindel\"of}              \Submitted{15.09.1978}
\Language{CDC: Fortran, IBM: Assembler}   %\Revised{}
\Cernhead{Portable Gaussian Random Number Generator}
{\tt RG32} generates random numbers distributed according to a
Gaussian distribution of mean zero and variance one. It produces
exactly the same {\tt REAL} numbers on any computer of word length at
least 32 bits. On computers with longer words, the lower bits will
be zero. It is intended for testing calculations across different
machines and for other applications where it is desired to have
exactly the same numbers generated on different machines. It is
not recommended for delicate Monte Carlo calculations involving
more than $10^5$ random Gaussian variates.
\Structure
{\tt SUBROUTINE} and {\tt FUNCTION} subprograms \\
User Entry Names: \Rdef{RG32}, \Rdef{RG32IN}, \Rdef{RG32OT}
\Usage
\begin{verbatim}
    X = RG32(DUMMY)
\end{verbatim}
assigns to {\tt X} the next number in a sequence of pseudorandom real
numbers in a Gaussian distribution of mean zero and variance one.
The generator does not require initialization, but it may be started
with a particular odd integer {\tt ISEED} by calling:
\begin{verbatim}
    CALL RG32IN(ISEED)
\end{verbatim}
where {\tt ISEED} should be between 5 and 10 decimal digits long.
The current value of the seed may be found by calling:
\begin{verbatim}
    CALL RG32OT(ISEED)
\end{verbatim}
where the value in {\tt ISEED} will be the current seed value.
With the default starting value of the seed ({\tt 875949887})
the first three 'Gaussian' pseudorandom numbers generated are:
\begin{verbatim}
    1.    1.613800048828
    2.    -.931945800781
    3.     .363372802734
\end{verbatim}
\Method
Pseudorandom integers are generated according to the multiplicative
congruential method using the same multiplier ({\tt 69069})
as {\tt RN32} (V106). Twelve consecutive integers are added, and the sum
appropriately rounded, truncated, floated, and normalized. The
resulting distribution is of course only approximately Gaussian, but
the difference is not statistically significant for sequences of
reasonable length (a few million). The method assures that no
values will be returned outside the interval $(-6.,+6.)$, whereas a true
Gaussian distribution would have a finite but very small probability
outside that interval.
\Notes
Although {\tt RG32} uses the same method and multiplier as {\tt RN32},
the {\tt RG32} default starting seed corresponds to the 10003-rd value
coming from {\tt RN32} (with its default starting seed), so that the two
can be considered as independent for sequences that are not too long.
This means, however, that care must be exercised if the user wishes to
reset the seeds using {\tt RN32IN} and {\tt RG32IN}, in order to avoid
possible correlations.
\\ $\bullet$
