%  18.10.94  ksk
\Version{RFRDH1}                             \Routid{D601}
\Keywords{LINEAR FREDHOLM INTEGRAL EQUATION SOLUTION SECON KIND}
\Author{G.A. Erskine and K.S. K\"olbig}        \Library{MATHLIB}
\Submitter{}                                \Submitted{07.06.1992}
\Language{Fortran}                          \Revised{ }
\Cernhead{Solution of a Linear Fredholm Integral Equation of Second Kind}
Subroutine subprograms {\tt RFRDH1}, {\tt DFRDH1} and function
subprograms {\tt RFRDH2}, {\tt DFRDH2} calculate an approximation to the
solution $y$ of the Fredholm integral equation
$$y(x) \ = \ F(x)+ \displaystyle \int_a^b G(x,t)\,y(t)\,dt
\hspace{20mm} (1) $$
over the interval $[a,b]$. The function $F$ must not be identically zero.
The interval $[a,b]$ may be divided into $m$ subintervals
$[t_{i-1},t_i], (i=1,2,\ldots,m)$, with
$a = t_0 < t_1 < \cdots < t_m = b$.
\par
The order $N_i$ (number of abscissae) of the Gaussian quadrature
formula used for integrating over $[t_{i-1},t_i]$ is specified
separately for each subinterval.
\par
Function subprograms {\tt RFRDH3} and {\tt DFRDH3} evaluate numerically
integrals of the form $\displaystyle \int_a^b H(t)\,y(t)\,dt$
where $H$ is an arbitrary function and $y$ is the solution of (1).
\par
The following values of $N_i$ may be used:
2, 3, 4, 5, 6, 7, 8, 9, 10, 11, 12, 13, 14, 15, 16,
20, 24, 32, 40, 48, 64, 80, 96.
\par
On computers other than CDC and Cray, only the double-precision versions
{\tt DFRDH1} etc. are available.
On CDC and Cray computers, only the single-precision versions
{\tt RFRDH1} etc. are available.
\Structure
{\tt SUBROUTINE} and {\tt FUNCTION} subprograms \\
User Entry Names: \Rdef{RFRDH1}, \Rdef{RFRDH2}, \Rdef{RFRDH3},
\Rdef{DFRDH1}, \Rdef{DFRDH2}, \Rdef{DFRDH3} \\
Files Referenced: {\tt Unit 6} \\
External References: \parbox[t]{128mm} {
\Rind{RGSET}{D107}, \Rind{DGSET}{D107}, \Rind{REQN}{F010},
\Rind{DEQN}{F010}, \Rind{MTLMTR}{N002}, \\ 
\Rind{ABEND}{Z035}, user-supplied {\tt FUNCTION} subprograms.}
\Usage
For $\mathtt{t=R}$ (type {\tt REAL}), $\mathtt{t=D}$ (type
{\tt DOUBLE PRECISION}), the first step in the solution of (1)
must be the execution of a statement of the form:
\begin{verbatim}
    CALL tFRDH1(F,G,M,T,NG,WS,IDIM,N)
\end{verbatim}
\begin{DLtt}{1234}
\item[F,G] (type according to {\tt t}) Names of user-supplied
{\tt FUNCTION} subprograms, declared {\tt EXTERNAL} in the calling
program.
Subprogram {\tt F} must set $\mathtt{F}(\mathtt{X})=F(\mathtt{X})$,
subprogram {\tt G} must set $\mathtt{G}(\mathtt{X,T})=G(\mathtt{X,T})$.
\item[M] ({\tt INTEGER}) Number $m \ge 1$  of subintervals in $[a,b]$.
\item[T] (type according to {\tt t}) One-dimensional array of
dimension {\tt (0:d)} where $\mathtt{d \ge M}$. On entry,
{\tt T} must contain the $m+1$ ordered points of subdivision
$t_i,(i=0,1,\ldots,m)$, with $t_0=a$ and $t_m=b$.
\item[NG] ({\tt INTEGER}) One-dimensional array of length
$\mathtt{\ge M}$. On entry, {\tt NG} must contain the order
(number of absissae) $N_i$ of the Gaussian quadrature formula to be used
in the interval $t_{i-1} \le t \le t_i, (i=1,2,\ldots,m)$.
\item[WS] (type according to {\tt t}) Two-dimensional array  of
dimensions $\mathtt{(IDIM,\ge IDIM+4)}$. Used as working space and for
communication between the subprograms.
\item[IDIM] ({\tt INTEGER}) A number $\ge \sum_{i=1}^m N_i$.
\item[N] ({\tt INTEGER}) On exit, $\mathtt{N} = \sum_{i=1}^m N_i$.
\end{DLtt}
\newpage
Once {\tt tFRDH1} has been called, the function subprograms
{\tt tFRDH2} and {\tt tFRDH3} may be referenced any number of times
without any further call to {\tt tFRDH1}.
\par
In any arithmetic expression,
\begin{verbatim}
    tFRDH2(F,G,X,WS,IDIM,N)
\end{verbatim}
has the value $y(\mathtt{X})$, where $y$ is the approximate solution of
(1).
\par
In any arithmetic expression,
\begin{verbatim}
    tFRDH3(H,WS,IDIM,N)
\end{verbatim}
has the approximate value of $\displaystyle \int_a^b H(t)\,y(t)\,dt$
where $y$ is the approximate solution of (1).
\par
{\tt H} (type according to {\tt t}) is the name of a user-defined
{\tt FUNCTION} subprogram, declared {\tt EXTERNAL} in the calling
program. This subprogram must set
$\mathtt{H}(\mathtt{X}) = H(\mathtt{X})$.
\par
\Method
Let the sets $\{w_k\}$ and $\{z_k\}$ be defined by
$$ \begin{array}{rcl}
\{w_k\} & = & \{w_1^{(1)},\ldots,w_{N_1}^{(1)},\ldots,
w_1^{(m)},\ldots,w_{N_m}^{(m)}\}, \\[3mm]
\{z_k\} & = & \{z_1^{(1)},\ldots,z_{N_1}^{(1)},\ldots,
z_1^{(m)},\ldots,z_{N_m}^{(m)}\}.
\end{array} $$
$w_j^{(i)}$ and $z_j^{(i)}$ are respectively the weights and
abscissae of the $N_i$-point Gaussian quadrature formulae corresponding
to the interval $[t_{i-1},t_i]$.
Subprograms {\tt RFRDH1} or {\tt DFRDH1} sets up and solves the following
system of
simultaneous linear equations with unknowns $y(z_k)$:
$$ y(z_k) \ = \ F(z_k) + \displaystyle
\sum_{j=1}^N \ w_j G(z_k,z_j) y(z_k) \qquad (k = 1,2,\ldots,N) $$
where $N = \sum_{i=1}^m N_i$.
\par
Function subprogram {\tt tFRDH2} calculates \qquad
$y(\mathtt{X}) \ = \ \displaystyle \sum_{k=1}^N
w_k\,G(\mathtt{X},z_k)\,y(z_k)$.
\par
Function subprogram {\tt tFRDH3} calculates \qquad
$I \ = \ \displaystyle \sum_{k=1}^N w_k\,H(z_k)\,y(z_k)$.
\Accuracy
The accuracy depends upon the extend to which the product $G(x,t)y(t)$
can be represented by a polynomial of degree $2N_i-1$ for all $x$ in the
interval $[t_{i-1},t_i],(i=1,2,\ldots,m)$.
\Errorh
Error {\tt D601.1:} In {\tt tFRDH1}, the system of
linear equations is singular. A message is written on {\tt Unit 6},
unless subroutine {\tt MTLSET} (N002) has been called. \\
If any of the values $N_i$ does not appear in the list given above,
a message is written on {\tt Unit 6} by {\tt RGSET} or {\tt DGSET}
(D107) unless subroutine {\tt MTLSET} (N002) has been called.
\\ $\bullet$
