% 20 may 1992  mg
\Version{DIVDIF}                           \Routid{E105}
\Keywords{FUNCTION INTERPOLATION}
\Author{F. James}                          \Library{KERNLIB}
\Submitter{G.A. Erskine}                    \Submitted{19.07.1973}
\Language{Fortran}                   \Revised{27.11.1984}
\Cernhead{Function Interpolation}
Function subprogram {\tt DIVDIF} interpolates in a table of arguments
$a_j$ and function values $f_j = f(a_j)$, using an interpolating
polynomial of specified degree which passes through tabular points which
are symmetrically-positioned around the interpolation argument. The
table arguments $a_j$ need not be equidistant.
\Structure
{\tt FUNCTION} subprogram \\
User Entry Names: \Rdef{DIVDIF}\\
Files Referenced: Printer\\
External References: \Rind{KERMTR}{N001}, \Rind{ABEND}{Z035}
\Usage
In any arithmetic expression,
\begin{center}
{\tt DIVDIF(F,A,N,X,M)}
\end{center}
has an approximate value of $f(\mathtt{X})$.
\begin{DLtt}{1234}
\item [F]({\tt REAL}) One-dimensional array. {\tt F(j)} must be
equal to the value at {\tt A(j)} of the function to be interpolated,
$(j=1,2,\ldots,\mathtt{N})$.
\item [A]({\tt REAL}) One-dimensional array. {\tt A(j)} must be
equal to the table argument $a_j,(j=1,2,\ldots,\mathtt{N})$.
\item [N]({\tt INTEGER}) Number of values in arrays {\tt F} and {\tt A}.
\item [X]({\tt REAL}) Argument at which the interpolating
polynomial is to be evaluated.
\item [M]({\tt INTEGER}) Requested degree of the interpolating
polynomial. If {\tt M} exceeds $M_{max}=\min(10,\mathtt{N}-1)$ the
interpolation is carried out using a polynomial of degree $M_{max}$
instead of {\tt M}. The original value of {\tt M} is unchanged on exit.
\end{DLtt}
\Method
Newton's divided difference formula is used.
Except when {\tt X} lies near one end of the table (in which
case unsymmetrically-situated interpolation points are used),
the interpolation procedure is as follows: \\
 {\tt M} {\bf odd}: \\
 An interpolating polynomial passing through
$\mathtt{M}+1$ successive points $(a_j,f_j)$ symmetrically placed with
respect to {\tt X} is used. \\
{\tt M} {\bf even}: \\
The mean of two interpolating polynomials is used, each passing through
$\mathtt{M}+1$ successive points $(a_j,f_j)$,
one polynomial having an extra point to the left of {\tt X}, the other
having an extra point to the right of {\tt X}. \\
If {\tt X} lies too close to either end of the table for
symmetrically-positioned tabular values to be used, the $\mathtt{M}+1$
values at the end of the table are used. If {\tt X} lies outside the
range of the table, the interpolation becomes an extrapolation, with
corresponding loss of accuracy.
\Restrict
The argument values $\mathtt{A(1),A(2)},\ldots$ must be in either
strictly increasing order or strictly decreasing order.
No error message is printed if this is not true.
\newpage
\Errorh
Error {\tt E105.1}:  $\mathtt{N<2}$ or $\mathtt{M<1}$.
{\tt DIVDIF} is set equal to zero and a message is printed
unless subroutine {\tt KERSET (N001)} has been called.
\Notes
See also the write-up for {\tt POLINT} (E100).
\\ $\bullet$
