%  20 feb 1995   ksk
\Version{KAPACK}                        \Routid{Z303}
\Keywords{KAADD KAADDM KACOPY KADEL KADELM KAFREE KAGET KAGETM
KAHOLD KALEN}
\Keywords{KALIST KALOC KAMAKE KAMSG KAOPTN KAPUT KAPUTM KARLSE KASEQ}
\Keywords{KASEQM KASTOP KEYWORD RANDOM RAND RNDM}
\Author{R. Matthews}                    \Library{PACKLIB}
\Submitter{}                            \Submitted{25.08.1983}
\Language{Fortran}                       \Revised{07.02.1986}
\Cernhead{Random Access I/O Using Keywords}
\begin{center}
\fbox{\parbox{120mm}{\OBSOLETE
Please note that this routine has been obsoleted in CNL 219. Users are
advised not to use it any longer and to replace it in older programs.
No maintenance for it will take place and it will eventually disappear.
\\[3mm]
Suggested replacement: {\tt ZEBRA} (Q100) or {\tt HEPDB} (Q180) }}
\end{center}
A package of Fortran-callable subprograms for manipulating a
random access file in which the records are of variable length and
identified by a two-component name. This package may be used as the
basis of a data base or bookkeeping system.
\Structure
{\tt SUBROUTINE} subprograms \\
User Entry Names:
\begin{htmlonly}
\begin{tabular}{llllllll}
\end{htmlonly}
\begin{latexonly}
\begin{tabular}[t]{l*{7}{@{\hspace{4pt}}l}}
\end{latexonly}
\Rdef{KAADD},  & \Rdef{KAADDM}, & \Rdef{KACOPY}, & \Rdef{KADEL}, &
\Rdef{KADELM}, & \Rdef{KAFREE}, & \Rdef{KAGET},  & \Rdef{KAGETM}, \\
\Rdef{KAHOLD}, & \Rdef{KALEN},  & \Rdef{KALIST}, & \Rdef{KALOC}, &
\Rdef{KAMAKE}, & \Rdef{KAMSG},  & \Rdef{KAOPTN}, & \Rdef{KAPRE}, \\
\Rdef{KAPREM}, & \Rdef{KAPRIK}, & \Rdef{KAPUT},  & \Rdef{KAPUTM}, &
\Rdef{KARLSE}, & \Rdef{KASEQ},  & \Rdef{KASEQM}, & \Rdef{KASTOP}
\end{tabular}
\Usage
See {\bf Long Write-up}.
\\ $\bullet$
