% 20 may 1992  mg
\Version{CHEB}                           \Routid{E221}
\Keywords{LINEAR SYSTEM CHEBYCHEV NORM OVERDETERMINED}
\Author{I. Barrodale, C. Phillips}        \Library{MATHLIB}
\Submitter{F. James}                       \Submitted{01.12.1981}
\Language{Fortran}                    %\Revised{}
\Cernhead{Solution of Overdetermined Linear System in the Chebychev Norm}
{\tt CHEB} finds the Chebyshev or minimax solution to a set of
overdetermined linear equations
$\mathbf{Ax=b}$, i.e. it finds the vector {\bf x} which minimizes:
$$ c \ = \ \max_{1\leq i\leq m} c_i \ = \ \max_{1\leq i\leq m}
\left| b_i - \sum_{j=1}^n a_{ij}x_j \right|.$$
\Structure
{\tt SUBROUTINE} subprogram \\
User Entry Names: \Rdef{CHEB}
\Usage
\begin{verbatim}
    CALL CHEB(M,N,MDIM,NDIM,A,B,TOL,RELERR,X,IRANK,RESMAX,ITER,ICODE)
\end{verbatim}
\begin{DLtt}{12345678}
\item [M]({\tt INTEGER}) Number of equations $m$.
\item [N] ({\tt INTEGER}) Number of unknowns $n (\leq m)$.
\item [MDIM]({\tt INTEGER}) Declared second dimension of array {\tt A},
with $\mathtt{MDIM}\geq m+1$.
\item [NDIM]({\tt INTEGER}) Declared first dimension of array {\tt A},
with $\mathtt{NDIM} \geq n+3$.
\item [A]({\tt REAL}) Two-dimensional array. On entry, {\tt A} must
contain the  {\it transpose} of the coefficient matrix, i.e.,
$\mathtt{A(i,j)}=a_{ij}$. The contents of {\tt A} is destroyed during
execution.
\item [B]({\tt REAL}) One-dimensional array of length $\geq m+1$.
On entry, the first $m$ elements of {\tt B} must contain the vector
{\bf b}. On exit, these elements contain the residuals $c_i$.
\item [TOL]({\tt REAL}) Tolerance parameter which should be set to a
value somewhat greater than the machine precision.
\item [RELERR]({\tt REAL}) On entry, {\tt RELERR} should be set to 0.0
if the true minimax solution is required. (For {\tt RELERR} non-zero
see {\bf Notes}).
\item [X]({\tt REAL}) One-dimensional array of length $\geq n+3$.
On exit, the first $n$ elements of {\tt X} contain the solution vector
{\bf x}.
\item [IRANK]({\tt INTEGER}) On exit, {\tt IRANK} contains an estimate of
the rank of the matrix {\tt A}. (This estimate may depend on {\tt TOL)}.
\item [RESMAX]({\tt REAL}) On exit, {\tt RESMAX} contains the value $c$
of the maximum residual.
\item [ITER]({\tt INTEGER}) On exit, {\tt ITER} contains the number of
simplex iterations performed.
\item [ICODE]({\tt INTEGER}) On exit, {\tt ICODE} contains one of the
following: \\
$\mathtt{= 0:}$ Solution {\bf x} is not unique, \\
$\mathtt{= 1:}$ Solution {\bf x} is unique,\\
$\mathtt{= 2:}$ Calculation terminated prematurely because of rounding
error.
\end{DLtt}
\Method
Modified simplex method of linear programming applied to the dual of
the stated minimax problem.
\newpage
\Notes
\begin{enumerate}
\item  If {\tt RELERR} on entry contains a non-zero positive value $r$,
{\tt RELERR} on exit contains a value $r'<r$,
and the computed solution {\bf x'} in {\tt X} and the maximum residual
$c'$ in {\tt RESMAX} are such that $(c'-c)/c < r'$,
where $c$ is the maximum residual corresponding to the true minimax
solution {\bf x}.  By setting {\tt RELERR} non-zero
(e.g. {\tt RELERR = 0.1}),
the number of simplex iterations is usually reduced.
\item If {\tt RESMAX} is within one or two orders of magnitude of
{\tt TOL}, the computed residuals in {\tt B}
on exit may contain few significant digits, and may have been set to
zero if $\mathtt{RESMAX < TOL}$.
\end{enumerate}
\Refer
\begin{enumerate}
\item I. Barrodale  C. Phillips,  Algorithm 495:
Solution of an overdetermined system of linear equations in the
Chebyshev norm, ACM Trans. Math. Software {\bf 1} (1975) 264--270.
\end{enumerate}
$\bullet$
