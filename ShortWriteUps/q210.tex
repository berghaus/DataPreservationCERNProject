%   20 feb 95   ksk
\Version {ZBOOK}                        \Routid{Q210}
\Keywords{DYNAMIC MANAGE MEMORY}
\Author{R. Brun, F. Carena, M. Hansroul, H. Grote, J.C. Lassalle,
W. Wojcik}                            \Library{PACKLIB}
\Submitter{}                             \Submitted{15.09.1978}
\Language{Fortran}                      \Revised{17.12.1991}
\Cernhead{Dynamic Memory Management}
\begin{center}
\fbox{\parbox{120mm}{\OBSOLETE
Please note that this routine has been obsoleted in CNL 219. Users are
advised not to use it any longer and to replace it in older programs.
No maintenance for it will take place and it will eventually disappear.
\\[3mm]
Suggested replacement: {\tt ZEBRA} (Q100) }}
\end{center}
{\tt ZBOOK} provides facilities to create (at execution time) memory
blocks of variable lengths, manage them and perform the following
operations on them:
\begin{itemize}
\item  create a block
\item  increase or decrease size of block
\item  set block to zero
\item  drop or delete block
\item  write block to file
\item  read from file
\item  print contents of block
\end{itemize}
Using {\tt ZBOOK}, the total size of all blocks together cannot exceed
the dimension of the array specified in the user's Fortran program.
Using a subpackage {\tt YBOOK} in connection with {\tt HBOOK} (Y250),
however, dynamic allocation of the total space is possible.
\Structure
{\tt SUBROUTINE} package \\
User Entry Names: \Rdef{ZBOOK}
\Usage
See {\bf Long Write-up}.
\\ $\bullet$
