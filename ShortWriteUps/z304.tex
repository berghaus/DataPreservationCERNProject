\Version {IOSPACK}                           \Routid{Z304}
\Keywords{SUBROUTINE PACKAGE ROUTINE}
\Keywords{IOSPACK IOSPUT IOSSIZ IOSCLR OUTPUT-ONLY OUTPUT}
\Keywords{IBM FULLSCREEN FULL-SCREEN SCREEN IO I/O}
\Keywords{7171 3270 X3270 VT100 COMPATIBLE SERIES/1}
\Keywords{GENERAL IBM INTERFACE}
\Keywords{PACK VM/CMS}
\Author{E. Perotto}                    \Library{KERNLIB, IBM VM/CMS only}
\Submitter{}                                \Submitted{03.07.1985}
\Language{IBM Assembler}                 %   %\Revised{}
\Cernhead {General Purpose IBM VM/CMS Non-Graphics Full Screen Interface
Package}
This package contains the following routines:
\begin{DLtt}{1234}
\item {\tt IOSCLR} to clear the screen.
\item {\tt IOSSIZ} to return the max number of lines supported
by the terminal.
\item {\tt IOSPUT} to write a line of text to any line on the screen.
\end{DLtt}
\Structure
{\tt FUNCTION} subprograms \\
User Entry Names: \Rdef{IOSCLR}, \Rdef{IOSSIZ}, \Rdef{IOSPUT}
\Usage
The variable {\tt IRC} and the functions {\tt IOSCLR}, {\tt IOSSIZ},
and {\tt IOSPUT} are of type {\tt INTEGER}.
\begin{verbatim}
    IRC = IOSCLR()
\end{verbatim}
will clear the screen of the logged-on VM-CMS terminal if it
supports IBM 3270 terminal protocols and set the next free line
(see {\tt IOSPUT} below) to be the first line on the screen, otherwise
no action is performed (including for batch execution and line-mode
terminals).
 \begin{DLtt}{1234}
\item[IRC] $\mathtt{= \ 0:}$ Operation successful. \\
           $\mathtt{= \ 1:}$ Line was busy, try again. \\
           $\mathtt{= -2:}$ CP error, not able to address the terminal.
\end{DLtt}
\begin{verbatim}
    IRC = IOSSIZ(LAST)
\end{verbatim}
will interrogate the terminal to find the number of addressable
lines. Note that {\tt IOSPUT} (below) can only address lines {\tt 1}
to $\mathtt{LAST-3}$, the last 3 lines having the special logical numbers of
{\tt 98, 99} and {\tt 100}.
\begin{DLtt}{123456}
\item [LAST] ({\tt INTEGER}) Contains, on exit, the size of the
screen in lines.
\item[IRC] $\mathtt{= 0:}$ Operation successful.  \\
           $\mathtt{= 1:}$ Line was busy, try again. \\
           $\mathtt{= 2:}$ CP error, not able to address the terminal.
\end{DLtt}
\begin{verbatim}
    IRC = IOSPUT(LINE,LENGTH,LINENO)
\end{verbatim}
will display a line of text to the logged-on full screen VM-CMS
terminal if it supports IBM 3270 terminal protocols, otherwise no
action is performed (including for batch execution and line mode
terminals). Any existing lines are entirely replaced when addressed
by {\tt IOSPUT}. {\tt IOSPUT} maintains internally a current line
counter. After a call to {\tt IOSCLR} this counter is set to one.
If {\tt IOSCLR} is not called first the value used is that last
addressed by the operating system $\mathtt{+1}$. Note that different
terminals may have different numbers
of lines available. A special set of numbers has been reserved for the
3 lines at the bottom of the screen: {\tt 100} for the last line,
{\tt 99} for the second to last line (the CMS command input line) and
{\tt 98} for the third to last line (this is normally a protected area
containing a CMS prompt or message).
\begin{DLtt}{12345678}
\item[LINE] ({\tt CHARACTER}) Contains, on entry, an character string
(or hollerith array) to be displayed.
\item[LENGTH] ({\tt INTEGER}) Contains, on entry, the length of
the character string (or the number of bytes in the hollerith array).
It may be zero. Values greater than the length of a single line
result on output as several lines. The "current line" is incremented
appropriately.
\item[LINENO] ({\tt INTEGER}) Controls, on entry,  the line to be
displayed.
\begin{DLtt}{12345678}
\item[$>$ 0:] Line number on which to display. Between
{\tt 1} and the last line less 3. A number which does not
correspond to a displayable line results in a {\bf more} prompt.
On keying in {\bf enter} or {\bf return}, the screen is cleared,
the text is displayed on line {\tt 1} and the current line is set to
line {\tt 2}.
\item[$<$ 0:] Display on the current line less this value. The
current line is reset to the line displayed plus one. If the
value is sufficiently negative as to be less than screen line {\tt 1}
the screen is cleared, text is displayed on line {\tt 1} and the current
line is set to line {\tt 2}.
\item[$=$ 98:] The third line from the bottom of the screen. This is
usually an  area containing a CMS prompt or comment.
\item[$=$ 99:] The second line from the bottom of the screen.
\item[$=$ 100:] The bottom of the screen (the last addressable line).
\item[$>$ 100:] A {\bf more} prompt is displayed. On keying in
{\bf enter} or {\bf return}, the screen is cleared, the text is
displayed on line {\tt 1} and the current line is set to line {\tt 2}.
\end{DLtt}
\end{DLtt}
$\bullet$
