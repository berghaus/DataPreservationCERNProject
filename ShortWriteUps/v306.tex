\Version {PROXIM}                       \Routid{V306}
\Keywords{ANGLE RANGE ADJUSTE}
\Author{J. Zoll, K.S. K\"olbig}          \Library{KERNLIB}
\Submitter{}                             \Submitted{15.03.1976}
\Language{Fortran}                 \Revised{15.02.1989}
\Cernhead {Adjusting an Angle to Another Angle}
Function subprogram {\tt PROXIM} computes, for two angles $\alpha,\beta$
given as arguments, and by adding a suitable multiple of $2\pi$ to
$\beta$, an angle $\beta^*$ such that
$$ \alpha -\pi \leq \beta^* \leq \alpha + \pi. $$
\Structure
{\tt FUNCTION} subprogram \\
User Entry Names: \Rdef{PROXIM}
\Usage
In any arithmetic expression,
\begin{verbatim}
    PROXIM(B,A)
\end{verbatim}
has the value $\beta^*$ for $\mathtt{B}=\beta$ and $\mathtt{A}=\alpha$.
{\tt PROXIM}, {\tt B} and {\tt A} are of type {\tt REAL} and in radians.
\Notes
The Fortran statement function
\begin{verbatim}
    PROXIM(B,A)=B+C1*ANINT(C2*(A-B))
\end{verbatim}
with $\mathtt{C1=2\pi,\,C2=1/C1}$ has the same effect.
\\ $\bullet$
