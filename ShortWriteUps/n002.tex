\Version{MTLSET}                          \Routid{N002}
\Keywords{ERROR MATHLIB SECTION MONITOR}
\Author{K.S. K\"olbig}                            \Library{MATHLIB}
\Submitter{}                                  \Submitted{07.06.1992}
\Language{Fortran}                      \Revised{15.03.1993}
\Cernhead{Error Processing for MATHLIB}
Subroutine {\tt MTLSET} allows the user to redefine the action to be
taken by {\bf certain} subprograms in MATHLIB
when certain specified error conditions are detected.
Error recovery may be performed either on each occurrence of the error,
or only a specified number of times. Messages may be written either on
each occurrence of the error, or only a specified number of times. Error
messages may be written (by default) onto the system output unit, or
may be re-routed to some other output file.
\Structure
{\tt SUBROUTINE} subprogram \\
User Entry Names: \Rdef{MTLSET}\\
Internal Entry Names: {\tt MTLMTR}\\
Files Referenced: Printer or user-defined\\
External References:  \Rind {ABEND}{Z035}
\Usage
\begin{verbatim}
    CALL MTLSET(ER,LGFILE,LM,LR)
\end{verbatim}
\begin{DLtt}{12345678}
\item [ER] ({\tt CHARACTER*6}) A character string that identifies the
range of error conditions for which action is to be redefined.
\item [LGFILE]({\tt INTEGER}) The logical unit number to be used
for error messages, or zero if error messages are to be written
onto the system output unit.
\item [LM] ({\tt INTEGER}) The number of occurrences of each error
condition in the range {\tt ER} for which an error message is to be
written. $\mathtt{LM<0}$ is ignored, $\mathtt{LM \ge 255}$ is treated
as infinity.
\item [LR] ({\tt INTEGER}) The number of times that error
recovery is to be performed for each error condition in the range
{\tt ER}. $\mathtt{LR<0} $ is ignored,
$\mathtt{LR \ge 255}$ is treated as infinity. If any error condition in
the range {\tt ER} occurs $\mathtt{LR+1}$ times a message is printed and
the run is terminated by calling {\tt ABEND} (Z035).
\end{DLtt}
\Notes
\begin{enumerate}
\item {\tt MTLSET} applies to those MATHLIB error conditions
which are specified by a six-character code (e.g. {\tt C204.2}) in the
{\bf Error handling} section of the Short Write-ups.
\item If the string {\tt ER} consists of six characters specifying a
single error condition \\
(e.g., {\tt ER}$=${\tt 'C204.2'}), {\tt LM} and
{\tt LR} apply only to this one error condition.
\par
If the six-character string {\tt ER} ends with one or more blanks,
{\tt LM} and {\tt LR} apply to all error conditions whose leftmost
characters match the non-blank characters of {\tt ER}. \\
Thus {\tt ER $=$ 'C2\ \ \ \ '}  (four blanks) applies to all error
conditions in packages C200 to C299, and {\tt ER $=$ '\ \ \ \ \ \ '}
(six blanks) applies to all error conditions under the control of
{\tt MTLSET}.
\item The value of {\tt LGFILE} applies to all error messages written
under the control of {\tt MTLSET}.
\end{enumerate}
$\bullet$
