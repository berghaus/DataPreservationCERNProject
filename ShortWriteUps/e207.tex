\Version{TRISUM}                              \Routid{E207}
\Keywords{SERIES SUM TRIGONOMETRIC}
\Author{T. H{\aa}vie}                             \Library{MATHLIB}
\Submitter{}                                 \Submitted{18.06.1969}
\Language{Fortran}                       %\Revised{}
\Cernhead{Summation of Trigonometric Series}
{\tt TRISUM} computes the sum of the trigonometric series
$$ f(x) = \frac{1}{2} a_0 + \sum^n_{k=1}a_k\cos(kx) + \sum^m_{k=1}
b_k\sin(kx) $$
for a given argument $x$ in the range $-\pi\leq x \leq \pi$ and
coefficients $a_k, b_k$.
\Structure
{\tt FUNCTION} subprogram\\
User Entry Names: \Rdef{TRISUM}
\Usage
In any arithmetic expression,
\begin{center}
{\tt TRISUM(X,A,N,B,M,IOP)}
\end{center}
has the value $f(x)$.
\begin{DLtt}{123456}
\item[X] ({\tt REAL}) The argument $x$.
\item[A] ({\tt REAL}) Array containing the cosine coefficients $a_k$,
where $\mathtt{A(K)}=a_{k-1}$.
\item[N] ({\tt INTEGER}) Equals $n+1$, the total number of coefficients
in the cosine series.
\item[B] ({\tt REAL}) Array containing the sine coefficients
$\mathtt{B(K)}=b_k$.
\item[M] ({\tt INTEGER}) Equals $m$, the total number of coefficients
in the sine series.
\item[IOP] ({\tt INTEGER}) An option number: \\
$\mathtt{= 1:}$ the general case, \\
$\mathtt{= 2:}$ only the cosine-terms are present, i.e., $f(x)=f(-x)$,\\
$\mathtt{= 3:}$ only the sine-terms are present, i.e., $f(x)=-f(-x)$.
\end{DLtt}
\Method
Standard recurrence relations are used for calculating the sum
(see Ref. 1).
\Notes
For a function $f(z)$ given in the range
$a\leq z\leq b$, use the transformation
$$\begin{array}{l@{\quad=\quad}ll}
\displaystyle x &
\displaystyle  \frac{2\pi}{b-a}\left( z -\frac{b+a}{2} \right)
&\displaystyle  \mbox{ for \tt IOP=1}\\[4mm]
\displaystyle  x &\displaystyle  \frac{\pi}{b-a}(z-a)
&\displaystyle  \mbox{ for {\tt IOP=2}  or {\tt IOP=3}}.
\end{array}$$
\Refer
\begin{enumerate}
\item W. Clenshaw,  MTAC (later renamed Math. Comp.) {\bf 9} (1955)
118--120.
\end{enumerate}
$\bullet$
