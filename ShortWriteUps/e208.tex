%  20 dec 94 ksk
\Version {LSQ}                \Routid{E208}
\Keywords{FIT LEAST SQUARES POLYNOMIAL}
\Author{E. Keil}              \Library{KERNLIB}
\Submitter{B. Schorr}          \Submitted{01.12.1969}
\Language{Fortran}             \Revised{27.11.1984}
\Cernhead {Least Squares Polynomial Fit}
\begin{center}
\fbox{\parbox{120mm}{\OBSOLETE
Please note that this routine has been obsoleted in CNL 218. Users are
advised not to use it any longer and to replace it in older programs.
No maintenance for it will take place and it will eventually disappear.
\\[3mm]
Suggested replacement: {\tt RLSQPM} (E201)}}
\end{center}
Subroutine {\tt LSQ} fits a polynomial of degree $m-1$ to $n$
equally-weighted data points ($x_i,y_i$). The computed coefficients
$a_j$ of the fitted polynomial have values which minimize
$$\sum^n_{i=1}\left( y_i-\sum^m_{j=1}a_jx_i^{j-1} \right) ^2.$$
For the case $m=2$ (straight line fit), subroutine {\tt LLSQ} is
faster and easier to use than {\tt LSQ}.
\par
Meaningful results can usually be obtained only for small values
of $m$ (typically less than 10).
\Structure
{\tt SUBROUTINE} subprograms \\
User Entry Names: \Rdef{LSQ}, \Rdef{LLSQ}\\
Files Referenced: Printer\\
External References:
\Rind{RVSUM}{F002}, \Rind{RSEQN}{F012}, \Rind{DSEQN}{F012},
\Rind{KERMTR}{N001}, \Rind{ABEND}{Z035}
\Usage
\begin{verbatim}
    CALL LSQ(N,X,Y,M,A)
    CALL LLSQ(N,X,Y,A1,A2,IFAIL)
\end{verbatim}
\begin{DLtt}{12345678}
\item[N] ({\tt INTEGER}) Number $n$ of data points.
\item[X] ({\tt REAL}) One-dimensional array. {\tt X(i)} must be equal
to the data coordinate $x_i$, \\ $(i=1,2,\ldots,\mathtt{N})$.
\item  [Y] ({\tt REAL}) One-dimensional array. {\tt Y(i)} must be equal
to the observed value $y_i$, \\  $(i=1,2,\ldots,\mathtt{N})$.
\item[M] ({\tt INTEGER}) On entry, {\tt M} must be equal to the number
$m$ of coefficients of the polynomial to be fitted. On exit, the value of
{\tt M} may differ from this (see {\bf Error Handling}).
\item[A] ({\tt REAL}) One-dimensional array. On exit from {\tt LSQ},
{\tt A(j)} is equal to the coefficient of $x^{j-1}$ in the fitted
polynomial, $(j=1,2,\ldots,\mathtt{M})$.
\item[A1,A2] ({\tt REAL}) On exit from {\tt LLSQ}, {\tt A1} and {\tt A2}
are equal to the coefficients of the fitted straight line $a_1+a_2 x$.
\item[IFAIL] ({\tt INTEGER}) On exit from {\tt LLSQ}, {\tt IFAIL}
is equal to {\tt -2} if $\mathtt{N < 2}$, to {\tt -1} if the matrix of
normal equations is numerically singular, and to zero otherwise.
\end{DLtt}
\Method
Normal equations.
\Errorh
Error {\tt E208.1:}  $\mathtt{M<1}$ or $\mathtt{M>N}$ or
$\mathtt{M>20}$ (subroutine {\tt LSQ}). {\tt M} is replaced by zero.\\
Error {\tt E208.2:}  The normal equations matrix is numerically singular
(subroutine {\tt LSQ}).\\
For each error, a message is printed unless subroutine
{\tt KERSET (N001)} has been called.
\Notes
On computers other than Cray and CDC double-precision arithmetic is used
internally.
\\ $\bullet$
