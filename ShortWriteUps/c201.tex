%   07 jun 95   ksk
\Version{RSNLEQ}                          \Routid{C201}
\Keywords{NUMERIC MUMERICAL SOLVE SOLUTION RESOLVE NON-LINEAR EQUATION}
\Authors{J.J. Mor\'e, M.Y. Cosnard}     \Library{MATHLIB}
\Submitter{K.S. K\"olbig}                \Submitted{01.06.1989}
\Language{Fortran}                      \Revised{01.12.1994}
\Cernhead{Numerical Solution of Systems of Nonlinear Equations}
Subroutine subprograms {\tt RSNLEQ} and {\tt DSNLEQ} compute a vector
$x_i,(i=1,2,\ldots,n)$, which approximates an exact
solution $x_i^*$ of the
system of n nonlinear equations with n unknowns
$$ F_i(x_1,\ldots,x_n)=0,\quad (i=1,2,\ldots,n).$$
These subroutines incorporate two convergence test, making use of
arguments {\tt FTOL} and {\tt XTOL} respectively.
If $x_i,(i=1,2,\ldots,n)$, denotes
the result of the current iteration, and $x_i'$ the result of the
previous iteration, the calculation is terminated if either of the
following tests is successful:
$$\begin{array}{ll}
\mbox{Test 1 :} & \quad \max |F_i(x_1,\ldots,x_n)|\le \mathtt{FTOL,}\\
\mbox{Test 2 :} & \quad \max|x_i-x_i'| \le \mathtt{XTOL}*\max|x_i|,
\end{array}$$
where the maxima are taken over $ 1\le i\le n.$

On computers other than CDC and Cray, only the double-precision version
{\tt DSNLEQ} is available. On CDC and Cray computers,
only the single-precision version {\tt RSNLEQ} is available.
\Structure
{\tt SUBROUTINE} subprograms \\
User Entry Names : \Rdef{RSNLEQ}, \Rdef{DSNLEQ}\\
Obsolete User Entry Names : \Rdef{SNLEQ} $\equiv$ {\tt RSNLEQ}\\
Files  Referenced : {\tt Unit 6} \\
External  References: User-supplied {\tt SUBROUTINE} subprogram
\Usage
For $\mathtt{t=R}$ (type {\tt REAL}), $\mathtt{t=D}$ (type
{\tt DOUBLE PRECISION}),
\begin{verbatim}
    CALL tSNLEQ(N,X,F,FTOL,XTOL,MAXF,IPRT,INFO,SUB,W)
\end{verbatim}
\begin{DLtt}{123456}
\item[N] ({\tt INTEGER}) Number $n$ of equations and variables.
\item[X] (type according to {\tt t}) One-dimensional array of length
$\ge $ {\tt N}. On entry, $\mathtt{X(i),(i=1,\ldots,N)}$, must
contain an estimate to a solution $x_i^*$ of the system of equations.
On exit, {\tt X(i)} contains the final estimate to $x_i^*$.
\item[F] (type according to {\tt t}) One-dimensional array of length
$\ge $ {\tt N}. On exit, $\mathtt{F(i),(i=1,\ldots,N)}$, contains the
final value of the residual $F_{\mathtt i}\mathtt{(X(1),\ldots,X(N))}$.
\item[FTOL] (type according to {\tt t}) Accuracy parameter for Test 1.
\item[XTOL] (type according to {\tt t}) Accuracy parameter for Test 2.
\item[MAXF] ({\tt INTEGER}) Maximum permitted number of iterations,
where each iteration involves N calls to the user-supplied
subroutine {\tt SUB}. The recommended value
for {\tt MAXF} is {\tt 50*(N+3)}.
\item[IPRT] ({\tt INTEGER}) If $\mathtt{IPRT=0}$ no intermediate
results are printed. \\
If $\mathtt{IPRT=1}$ the values of {\tt i} and
$\mathtt{X(i),(i=1,2,\ldots,n)}$, are printed after each iteration.
\item[INFO] ({\tt INTEGER}) On exit, the value of {\tt INFO} shows the
reason why execution was terminated as follows:
\begin{DLtt}{12}
\item[0] Unacceptable input arguments ($\mathtt{N < 1}$
or $\mathtt{FTOL \le 0}$ or $\mathtt{XTOL \le 0})$.
\item[1] Test 1 is successful.
\item[2] Test 2 is successful.
\item[3] Test 1 and Test 2 are both successful.
\item[4] Number of iterations is $\ge \mathtt{MAXF}$.
\item[5] Approximate (finite difference) Jacobian matrix is singular
\item[6] Iterations are not making good progress.
\item[7] Iterations are diverging.
\item[8] Iterations are converging, but either (i) {\tt XTOL} is too
small, or (ii) convergence is very slow because the Jacobian is
nearly singular near $x_i^*$ or because the variables $x_i$ are badly
scaled.
\end{DLtt}
\item[SUB]Name of a user-supplied {\tt SUBROUTINE} subprogram,
declared {\tt EXTERNAL} in the calling program.
\item[W] (type according to {\tt t}) Array containing at least
{\tt N*(N+3)} elements required as working-space.
\end{DLtt}
\par
The user-supplied {\tt SUBROUTINE} subprogram {\tt SUB} should
be of the form
\begin{DLtt}{123456}
 \item[] {\tt SUBROUTINE SUB(N,X,F,K)}
 \item[] {\tt DIMENSION X(*),F(*)}
 \item[] {\tt ...}
 \item[] Statements which set {\tt F(K)} equal to the value of
$F_{\mathtt{K}}(\mathtt{X(1),...,X(N)})$ without changing any other
element of array {\tt F}.
\item[] {\tt ...}
\item[] {\tt RETURN}
\item[] {\tt END}
\end{DLtt}
where {\tt X} and {\tt F} are of type {\tt t}.
\par
Subroutine {\tt SUB} should not change the value of the argument
{\tt K} unless the user wants to terminate the execution of
{\tt tSNLEQ}, in which case {\tt K} should be set equal to a negative
integer, whose value will be copied into argument {\tt INFO} of
{\tt tSNLEQ} before exit. \\ [4mm]
\Method
A modification of Brent's method as described in Ref. 1.
\Errorh
See description of argument {\tt INFO}.
\Notes
\begin{enumerate}
\item Whenever possible the equations $F_i=0$ should be numbered
in order of increasing nonlinearity.
\item These subroutines do not use any techniques which attempt to
obtain global convergence. Convergence may therefore fail to occur
if the initial estimate is too far from an exact solution.
\end{enumerate}
\Source
This subroutine has been adapted from the Fortran program
published in Ref. 2.
\Refer
\begin{enumerate}
\item  J.J. Mor\'e and M.Y. Cosnard, Numerical solution of nonlinear
equations, ACM Trans. Math. Software {\bf 5} (1979) 64--85.
\item  J.J. Mor\'e and M.Y. Cosnard,
Algorithm 554 BRENTM, A FORTRAN subroutine for the numerical
solution of systems of nonlinear equations,
Collected Algorithms from CACM (1980).
\end{enumerate}
$\bullet$
