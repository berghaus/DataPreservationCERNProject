%   18.10.94  ksk
\Version{RMINFC}                    \Routid{D503}
\Keywords{FUNCTION MINIMUM}
\Author{K.S. K\"olbig}                 \Library{MATHLIB}
\Submitter{}                          \Submitted{15.11.1993}
\Language{Fortran}                   %  \Revised{}
\Cernhead{Minimum of a Function of One Variable}
Subroutine subprograms {\tt RMINFC} and {\tt DMINFC} calculate,
to a limited specified accuracy, the abscissa of a single local minimum
of a real-valued function $f(x)$ lying in a given interval $(a,b)$,
together with the function value at the minimum. Although this
subprogram may find a minimum under other conditions (see {\bf Notes}),
the search interval should contain exactly one local minimum point $x$
with $a<x<b$.
\par
On CDC and Cray computers, the double-precision version {\tt DMINFC}
is not available.
\Structure
{\tt SUBROUTINE} subprograms \\
User Entry  Names: \Rdef{RMINFC}, \Rdef{DMINFC} \\
External References: User-supplied {\tt FUNCTION} subprogram
\Usage
For $\mathtt{t=R}$ (type {\tt REAL}), $\mathtt{t=D}$ (type
{\tt DOUBLE PRECISION}),
\begin{verbatim}
    CALL tMINFC(F,A,B,EPS,DELTA,X,Y,LLM)
\end{verbatim}
\begin{DLtt}{123456}
\item[F] (type according to {\tt t}) Name of a user-supplied
{\tt FUNCTION} subprogram, declared {\tt EXTERNAL} in the calling
program. This function must set $\mathtt{F(X)}=f(\mathtt{X})$.
\item[A,B] (type according to {\tt t}) On entry, {\tt A} and {\tt B}
must specify the end-points $a,b$ of the search interval.
\item[EPS] (type according to {\tt t}) On entry, {\tt EPS} must be
equal to the accuracy parameter $\varepsilon$ (see {\bf Accuracy}).
\item[DELTA] (type according to {\tt t}) On entry, {\tt DELTA} must
be equal to the parameter $\delta$ specifying a tolerance interval
near {\tt A} and {\tt B} (see {\bf Accuracy}).
\item[X] (type according to {\tt t}) On exit, {\tt X} is the computed
approximation to the abscissa of a minimum of the function $f(x)$.
\item[Y] (type according to {\tt t})
Contains, on exit, the value of $f(\mathtt{X})$.
\item[LLM] ({\tt LOGICAL}) On exit, {\tt LLM} is {\tt .TRUE.} if
the relations $\mathtt{|X-A|}>\delta$ and $\mathtt{|X-B|}>\delta$
are both true (i.e. if {\tt X} is the abscissa of a local minimum
lying inside the interval $\mathtt{[A,B]}$), and {\tt .FALSE.}
otherwise (see {\bf Notes}).
\end{DLtt}
\Method
The so-called {\it golden section search} is applied (see
{\bf References}). This method uses a fixed number $n$
of function evaluations, where
$n=[\,2.08 \times \ln\,(|a-b|/\varepsilon)+\frac{1}{2}\,]+1$.
\Accuracy
The accuracy depends on the behaviour of the function and is difficult
to measure. For example, a flat minimum results in poor accuracy.
This implies that the subprograms are not intended to replace
the usual procedures when a minimum of a function
is needed in the exact mathematical sense.
In any case, a choice of $\varepsilon > 10^{-8}$ in double-precision
and of $\varepsilon > 10^{-4}$ in single-precision mode usually
results in a relative error of {\tt X} which is smaller than
or in the order of $\varepsilon$. A suggested value of $\delta$ is
$\delta=10\varepsilon$.
\newpage
\Notes
\begin{enumerate}
\item As a rule, the specified interval $(a,b)$ should contain strictly
one
local minimum.
\item If this is not the case, and if $f(x)$ is monotonous in $(a,b)$,
the subprograms find the minimum at the correct endpoint $a$ or $b$.
{\tt LLM} is set to {\tt .FALSE.} in this case.
\item In all other possible cases, the behaviour of the subprograms is
not easy to predict. In particular, in the case of several minimal
points inside $(a,b)$, one of them is found, but not necessarily
the one with the smallest value of the function.
\end{enumerate}
\Refer
\begin{enumerate}
\item R. Fletcher, Practical methods of optimization (John Wiley \& Sons,
Chichester 1987) 39--40.
\item W. Krabs, Einf\"uhrung in die lineare und nichtlineare
Optimierung f\"ur Ingenieure (BSB B.G. Teubner, Leipzig 1983) 84--86
\end{enumerate}
$\bullet$
