\Version{PROBKL}               \Routid{G102}
\Keywords{DISTRIBUTION KOLMOGOROV}
\Author{F. James, K.S. K\"olbig}      \Library{MATHLIB}
      \Submitter{}                 \Submitted{15.10.1976}
\Language{Fortran}                     \Revised{15.03.1993}
\Cernhead{Kolmogorov Distribution}
Function subprogram {\tt PROBKL} calculates the Kolmogorov distribution
function
$$P(X) \ = \
-2 \ \sum_{j=1}^\infty \ (-1)^j \exp(-2j^2X^2) $$
for real arguments $X$.
\Structure
{\tt FUNCTION} subprogram \\
User Entry Name: \Rdef{PROBKL}
\Usage
In any arithmetic expression,
\begin{center}
{\tt PROBKL(X)} \quad has the value \quad $P(\mathtt{X})$,
\end{center}
where {\tt PROBKL} and {\tt X} are of type {\tt REAL}.
\Method
Direct evaluation or using functional relations.
\Accuracy
Approximately seven digits are correct.
Results smaller than $10^{-40}$ (corresponding to $X > 6.8116$)
are set to zero. Note that the above formula has a statistical meaning
only for "large" $N (>10)$.
\Notes
\begin{enumerate}
\item For an experimental distribution with $N$ events and a maximum
deviation $\Delta N$ from a hypothetical distribution,
$P({\tt X})$ with $\mathtt{X} = \Delta N \sqrt{N}$ gives the confidence
level for the null hypothesis.
\item To compare two experimental distributions with $M$ and $N$ events,
respectively, one may use
$\mathtt{X} = \sqrt{M N/(M + N)} \Delta N$.
\end{enumerate}
$\bullet$
