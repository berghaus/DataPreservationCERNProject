% 20 may 1992  mg
\Version{CHEBQU}                                  \Routid{D115}
\Keywords{NUMERICAL INTEGRATION CLENSHAW CURTIS QUADRATURE}
\Author{T. H{\aa}vie}                                 \Library{MATHLIB}
\Submitter{}                                       \Submitted{29.11.1971}
\Language{Fortran}                         \Revised{01.01.1975}
\Cernhead{Double-Precision Clenshaw-Curtis Integration}
{\tt CHEBQU} computes an approximation to the integral
        $$ I=\int_a^b f(x)dx $$
in double-precision mode using the modified Clenshaw-Curtis algorithm
(see {\tt GPINDP} (D111) with {\tt IOP=3}),
but written as an ordinary quadrature
formula. The routine is faster and also much more stable with respect
to accumulation of round-off error than {\tt GPINDP} (D111),
when a very high precision is wanted. It also provides the experienced
user with a possibility to print intermediate results.
\Structure
{\tt FUNCTION} subprogram \\
User Entry Names: \Rdef{CHEBQU}  \\
Internal Entry Names:  {\tt D115BD}\\
Files Referenced: {\tt Unit 6}\\
External  References: User supplied {\tt FUNCTION} subprogram \\
{\tt COMMON} Block Names and Lengths:
\Rind{/DPCHEB/ 4107}, \Rind{/CHEINT/ 7}
\Usage
In any arithmetic expression,
\begin{center}
{\tt CHEBQU(A,B,EPSIN,EPSOUT,JOP,F)}
\end{center}
has the value of the above integral. {\tt CHEBQU} is of type
{\tt DOUBLE PRECISION}.
\begin{DLtt}{12345678}
\item [A,B] ({\tt DOUBLE PRECISION}) Limits of integration.
\item [EPSIN] ({\tt DOUBLE PRECISION}) Relative accuracy required.
\item [EPSOUT] ({\tt DOUBLE PRECISION}) Relative accuracy obtained.
\item [JOP] ({\tt INTEGER)} An option parameter: \\
$\mathtt{= 0:}$ No printing of intermediate calculations, \\
$\mathtt{= 1:}$ Print intermediate calculations.
\item[F] ({\tt DOUBLE PRECISION}) Name of a user-supplied {\tt FUNCTION}
subprogram declared {\tt EXTERNAL} in the calling program. This
subprogram must set $\mathtt{F(X)} = f(\mathtt{X})$.
\end{DLtt}
\Method
The calculation terminates when the
specified accuracy, {\tt EPSIN}, is obtained or after 1025 evaluations of
the integrand, whichever comes first. \\
The {\tt COMMON} block \Lit{/DPCHEB/} contains the weights and
abscissae for the quadrature formula. \\
The {\tt COMMON} block \Lit{/CHEINT/} contains
$ T_N^*,U_N^*, \Delta_N^*$ (see References) and the number N of
integrand values used.
\newpage
\Refer
\begin{enumerate}
\item T. H\aa vie, On a modification of
the Clenshaw-Curtis quadrature formula, BIT 9 (1969) 338--350.
\item T. H\aa vie, Some methods for automatic integration
and their implementation on the CERN CDC 65/6600 computers, CERN
71-26 (1971).
\item T. H\aa vie, Further remarks on the modified
Clenshaw-Curtis quadrature formula, Report CERN DD/71/21 (1971).
\end{enumerate}
$\bullet$
