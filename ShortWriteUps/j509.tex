\Version {CONPRT}                       \Routid{J509}
\Keywords{CONTOUR FUNCTION PRINT OUTPUT}
\Author{A. Sorenssen}                    \Library{MATHLIB}
\Submitter{}                             \Submitted{18.06.1969}
\Language{Fortran}                        \Revised{27.11.1984}
\Cernhead {Print Function Contours in Two Variables}
{\tt CONPRT} is an interface to {\tt CONTPL} (see {\bf Long Write-up})
which
produces a contour plot of an arbitrary (real or integer) function
of two variables on the line printer.
\Structure
{\tt SUBROUTINE} and {\tt FUNCTION} subprograms \\
User Entry Names: \Rdef{CONPRT}, \Rdef{PAPER}, \Rdef{NAMES},
\Rdef{FRAME}, \Rdef{SETUP}, \Rdef{FINDEM}, \Rdef{CONT} \\
Internal Entry Names: {\tt FUPLFT}, {\tt FREARG}, {\tt INTER},
{\tt J509BD}, {\tt ORDRE2}\\
Files Referenced: Printer\\
External References: \Rind{FINARG} \\
{\tt COMMON} Block Names and Lengths: {\tt /J509C1/ 20},
{\tt /PLOTSC/ 24}, {\tt /SCALES/ 8}
\Usage
Simplified call:
\begin{verbatim}
    CALL CONPRT(F,IDM1,IDM2,M,N,NC,FMIN,FMAX)
\end{verbatim}
\begin{DLtt}{123456}
\item [F] ({\tt REAL} or {\tt INTEGER}) Two-dimensional array of
functional values $f(m,n)$ for which the contours are desired.
\item [IDM1] ({\tt INTEGER}) The first dimension of {\tt F} in
the calling program.
\item [IDM2] ({\tt INTEGER}) The second dimension of {\tt F} in
the calling program. $\mathtt{IDM1,IDM2 \leq 121}$.
\item [M] ({\tt INTEGER}) Number of points to be used for the contour
printout in the vertical direction.
$\mathtt{2 \leq M \leq 121,\,M \leq IDM1}$.
\item [N] ({\tt INTEGER}) Number of points to be used for the contour
printout in the horizontal direction.
$\mathtt{2 \leq N \leq 121,\,N \leq IDM1}$.
\item [NC] ({\tt INTEGER}) Number of contours to be printed.
$\mathtt{2 \leq NC \leq 10}$.
\item [FMIN] ({\tt INTEGER}) Minimum functional value for the contour
range.
\item [FMAX] ({\tt INTEGER}) Maximum functional value for the contour
range.
\item[] If {\tt FMIN} and {\tt FMAX} are both zero, the minimum and
maximum functional values of the actual array {\tt F(M,N)} are
assumed for {\tt FMIN} and {\tt FMAX}, respectively.
\end{DLtt}
\Notes
{\tt CONPRT} calls routines from {\tt CONTPL} providing them with
parameters derived from the above arguments and a number of default
values. See {\bf Long Write-up} for greater flexibility by calling the
{\tt CONTPL} routines directly.
\\ $\bullet$
