%  20 feb 1995  ksk
\Version{IARGC}                        \Routid{Z264}
\Keywords{FORTRAN FUNCTION IARGV}
\Author{F. Carminati, M. Marquina}
\Library{KERNLIB or Fortran Run-Time Library}
\Submitter{}                             \Submitted{13.07.1988}
\Language{Fortran + C}                   \Revised{15.03.1993}
\Cernhead{Returns Command Line Arguments}
{\tt IARGC} is used to return arguments that the user has given
to an executable module on the command line.
\Structure
{\tt FUNCTION} subprograms\\
User Entry Names: \Rdef{GETARG}, \Rdef{IARGC}
\Usage
\begin{verbatim}
    NPAR = IARGC()
\end{verbatim}
sets {\tt NPAR} to the number of blank delimited arguments present after
the program name on the command line. {\tt NPAR} and {\tt IARGC} are
of type {\tt INTEGER}.
\begin{verbatim}
    CALL GETARG(IARG,GOTEXT)
\end{verbatim}
\begin{DLtt}{123456}
\item [IARG] ({\tt INTEGER}) Contains, on entry, the number of the
argument to retrieve. Unchanged on exit.
\item [GOTEXT] ({\tt CHARACTER}) Contains, on exit, the {\tt IARG}-th
argument.
\end{DLtt}
\Notes
\begin{enumerate}
\item Arguments surrounded by double quotes ({\tt "}) are treated as
single, e.g.
\begin{verbatim}
    "a variable here"
\end{verbatim}
is equivalent to one argument.
\item On VM/CMS, due to technical restrictions, at least one of the
routines must be called before any I/O (typically a PRINT statement).
\item {\tt GETARG(0,GOTEXT)} returns name of executing program (not VM).
\end{enumerate}
\Example
\begin{verbatim}
    CHARACTER*100 STRING
C
C-- Retrieve the number of arguments given to this program
C
    NPAR=IARGC()
C-- and then get one by one, storing it in STRING
    DO 10 N = 1,NPAR
    CALL GETARG(N,STRING)
    PRINT *, STRING(1:LENOCC(STRING))
 10 CONTINUE
    END
\end{verbatim}
$\bullet$
