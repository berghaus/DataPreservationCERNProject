% 20 may 1992  mg
\Version {CVTCDC}                          \Routid{M218}
\Keywords{CONVERT TRANSLATE CDC IBM FLOAT NUMBER FORMAT}
\Author{F. Carminati}                       \Library{KERNLIB, IBM only}
\Submitter{}                                 \Submitted{05.10.1986}
\Language{Fortran}                       %\Revised{}
\Cernhead {Convert Between CDC and IBM Floating-Point Number Formats}
{\tt CVTCDC} converts floating point numbers between host machine and CDC
formats. Both Short (60-bit) and Long (120-bit) CDC formats are
catered for.
\Structure
{\tt SUBROUTINE} subprograms \\
User Entry Names:
\begin{tabular}[t]{l*{3}{@{\hspace{4pt}}l}}
\Rdef{SCDDX}, & \Rdef{SCDSX}, & \Rdef{SXDCD}, & \Rdef{SXSCD} \\
\Rdef{DCDDX}, & \Rdef{DCDSX}, & \Rdef{DXDCD}, & \Rdef{DXSCD}
\end{tabular}
\Usage
\begin{verbatim}
    CALL SCDDX(DWORDS,NWORDS)
\end{verbatim}
\begin{DLtt}{12345678}
\item[DWORDS] ({\tt DOUBLE PRECISION}) Array dimensioned to at least
{\tt NWORDS} in the calling program and containing on input CDC
short (60-bit) normalised floating point numbers, stored in the elements
of {\tt DWORDS}, right-adjusted with zero-fill. After the call the first
{\tt NWORDS} elements will contain converted normalised
{\tt DOUBLE PRECISION} floating point numbers in the host machine
format. This routine is only available on IBM.
\item[NWORDS] ({\tt INTEGER}) Constant or variable containing on input
the number of CDC numbers to be converted. Unchanged on output. A
value $\mathtt{< 1}$ causes a do-nothing return.
\end{DLtt}
\begin{verbatim}
    CALL DXSCD(DWORDS,NWORDS)
\end{verbatim}
\begin{DLtt}{12345678}
\item[DWORDS] ({\tt DOUBLE PRECISION}) Array dimensioned to at least
{\tt NWORDS} in the calling program and containing on input in the first
{\tt NWORDS} elements {\tt DOUBLE PRECISION} floating point numbers
in the host machine format. After the call the first {\tt NWORDS}
elements will contain converted CDC {\tt SINGLE PRECISION} (60-bit)
floating point numbers, in the elements of {\tt DWORDS}, right-adjusted
with zero-fill. This routine is only available on IBM.
\item[NWORDS] ({\tt INTEGER}) Constant or variable containing on input
the number of host numbers to be converted. Unchanged on output. A
value $\mathtt{< 1}$ causes a do-nothing return.
\end{DLtt}
\begin{verbatim}
    CALL DCDSX(DWORDS,NWORDS)
\end{verbatim}
\begin{DLtt}{12345678}
\item[DWORDS] ({\tt SINGLE PRECISION}) Array dimensioned to at
least {\tt 4*NWORDS} in the calling program and containing on input CDC
long (120-bit) normalised floating point numbers, stored in 2
consecutive pairs of elements of {\tt DWORDS}, 60 bits in each pair,
right-adjusted with zero-fill. After the call the first {\tt NWORDS}
elements will contain converted normalised {\tt SINGLE PRECISION}
floating point numbers in the host machine format. This routine is only
available on IBM.
\item[NWORDS] ({\tt INTEGER}) Constant or variable containing on input
the number of CDC numbers to be converted. Unchanged on output. A
value $\mathtt{< 1}$ causes a do-nothing return.
\end{DLtt}
\newpage
\begin{verbatim}
    CALL SXDCD(DWORDS,NWORDS)
\end{verbatim}
\begin{DLtt}{12345678}
\item[DWORDS] ({\tt SINGLE PRECISION}) Array dimensioned to at
least {\tt 4*NWORDS} in the calling program and containing on input
in the first {\tt NWORDS} elements {\tt SINGLE PRECISION} floating point
numbers in the host machine format. After the call the first
{\tt 4*NWORDS} elements will contain converted CDC
{\tt DOUBLE PRECISION} (120-bit) floating point numbers, each in 2
consecutive pairs of elements of {\tt DWORDS}, 60 bits per pair,
right-adjusted with zero-fill. This routine is only available on IBM.
\item[NWORDS] ({\tt INTEGER}) Constant or variable containing on input
the number of host numbers to be converted. Unchanged on output. A
value $\mathtt{< 1}$ causes a do-nothing return.
\end{DLtt}
\begin{verbatim}
    CALL DCDDX(DWORDS,NWORDS)
\end{verbatim}
\begin{DLtt}{12345678}
\item[DWORDS] ({\tt DOUBLE PRECISION}) Array dimensioned to at least
{\tt 2*NWORDS} in the calling program and containing on input CDC
long (120-bit) normalised floating point numbers, stored in consecutive
pairs of elements of {\tt DWORDS}, 60 bits per element, right-adjusted
with zero-fill. After the call the first {\tt NWORDS} elements will
contain converted normalised {\tt DOUBLE PRECISION} floating point
numbers in the host machine format. This routine is only available on
IBM.
\item[NWORDS] ({\tt INTEGER}) Constant or variable containing on input
the number of CDC numbers to be converted. Unchanged on output. A
value $\mathtt{< 1}$ causes a do-nothing return.
\end{DLtt}
\begin{verbatim}
    CALL DXDCD(DWORDS,NWORDS)
\end{verbatim}
\begin{DLtt}{12345678}
\item[DWORDS] ({\tt DOUBLE PRECISION}) Array dimensioned to at
least {\tt 2*NWORDS} in the calling program and containing on input
in the first {\tt NWORDS} elements {\tt DOUBLE PRECISION} floating point
numbers in the host machine format. After the call the first
{\tt 2*NWORDS} elements will contain converted CDC
{\tt DOUBLE PRECISION} (120-bit) floating point numbers in each pair of
consecutive elements of {\tt DWORDS}, 60 bits per element,
right-adjusted with zero-fill. This routine is only available on IBM.
\item[NWORDS] ({\tt INTEGER}) Constant or variable containing on input
the number of host numbers to be converted. Unchanged on output. A
value $\mathtt{< 1}$ causes a do-nothing return.
\end{DLtt}
\begin{verbatim}
    CALL SCDSX(DWORDS,NWORDS)
\end{verbatim}
\begin{DLtt}{12345678}
\item[DWORDS] ({\tt SINGLE PRECISION}) array dimensioned to at least
{\tt 2*NWORDS} in the calling program and containing on input CDC
short (60-bit) normalised floating point numbers, stored in two
consecutive elements of {\tt DWORDS} right-adjusted with zero-fill.
After the call the first {\tt NWORDS} elements will contain
converted normalised {\tt SINGLE PRECISION} floating point numbers in
the host machine format. This routine is only available on IBM.
\item[NWORDS] ({\tt INTEGER}) constant or variable containing on input
the number of CDC numbers to be converted. Unchanged on output. A
value $\mathtt{< 1}$ causes a do-nothing return.
\end{DLtt}
\begin{verbatim}
    CALL SXSCD(DWORDS,NWORDS)
\end{verbatim}
\begin{DLtt}{12345678}
\item[DWORDS] ({\tt SINGLE PRECISION}) Array dimensioned to at least
{\tt 2*NWORDS} in the calling program and containing on input in the
first {\tt NWORDS} elements {\tt SINGLE PRECISION} floating point
numbers in the host machine format. After the call the first
{\tt 2*NWORDS} elements will contain converted CDC
{\tt SINGLE PRECISION} (60-bit) floating point numbers, in two
consecutive elements of {\tt DWORDS} right-adjusted with zero-fill.
This routine is only available on IBM.
\item[NWORDS] ({\tt INTEGER}) Constant or variable containing on input
the number of host numbers to be converted. Unchanged on output. A
value $\mathtt{< 1}$ causes a do-nothing return.
\end{DLtt}
\newpage
\Accuracy
Precision in the mantissa will be lost by truncation
of the least significant bits when converting from a source format to a
target format with fewer bits in the mantissa. Note that the mantissa
lengths are 48 bits for CDC short, 96 bits for CDC long, 24 bits for
IBM short, 56 bits for IBM long.
\par
Exponent ranges also differ among the machines. The rule
followed on conversion is that when a source machine value is out of
range for the target machine the value set is the limiting value for
the target machine, i.e., the largest or smallest possible floating
point number on that machine. Note that CDC overflow, underflow and
indefinite numbers are converted into the largest, smallest and largest
possible target IBM numbers respectively. The sign of the source
number is preserved during these out-of-range conversions. The
exponent ranges are $10^{-78}$ to $10^{76}$ for IBM short and long and
$10^{-293}$ to $10^{322}$ for CDC short floating point numbers. Hence all
CDC numbers greater than $10^{76}$ will be set to $10^{76}$ when
converted to IBM {\tt DOUBLE} and {\tt SINGLE PRECISION} while any CDC
numbers smaller than $10^{-78}$ would be set to $10^{-78}$ when converted
to IBM format.
\Notes
In the calling sequences above {\tt S} stands for short
representation i.e., 32 and 60 bits on IBM and CDC
respectively, while {\tt D} stands for long representation, i.e., 64
and 120 bits IBM and CDC, respectively. The default lengths
on IBM and CDC are short. The long forms must be explicitly
requested by a {\tt DOUBLE PRECISION} statement. {\tt X} stands for
the host machine and the position of {\tt CD} and {\tt X} implies the
direction of processing. Hence {\tt DCDSX} implies convert long CDC
format (120-bits) to short host machine format.
\\ $\bullet$
