%24 mar 94 ksk
\Version{WHOAMI}                           \Routid{Z266}
\Keywords{NAME EXECUTING MODULE}
\Author{F. Carminati, J. Zoll}             \Library{KERNLIB, VAX only}
\Submitter{}                                 \Submitted{01.04.1994}
\Language{Fortran}                       \Revised{}
\Cernhead{Get the name of the executing module}
This routine will figure out the path-name of the executing image.
On the VAX this is done with a system call, on UNIX by scanning the
search path until it finds the module whose name is in {\tt argv[0]}.
\Structure
{\tt SUBROUTINE} subprograms \\
User Entry Names: \Rdef{WHOAMI} \\
Common Blocks: {\tt COMMON /SLATE/ ND,NE,NF,DUMMY(37)}
\Usage
\begin{verbatim}
    CALL WHOAMI(NAME)
\end{verbatim}
On exit, {\tt NAME} contains the full path-name of the module.
 
Status and various lengths are returned in {\tt /SLATE/}:
\begin{verbatim}
    ND = 0  if the call failed,
       > 0  the number of characters in the path-name
\end{verbatim}
On the VAX:
\begin{verbatim}
ND =  number of characters in the path-name with .EXE;n stripped
NE =  number of characters before the semicolon,
NF =  number of characters in the complete name.
\end{verbatim}
For example:
\begin{verbatim}
      if NAME is  DISK:[CERN]WYLBUR.EXE;4
                  _:.=+=.: 1_:.=+=.: 2_:.=+=
 
      we will get  ND=17, NE=21, NF=23.
\end{verbatim}
{\bf Note}: At the moment this is available only on the VAX; the code
exists for UNIX but is not yet in the library.
$\bullet$
