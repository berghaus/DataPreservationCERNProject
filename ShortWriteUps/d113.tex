\Version{CGAUSS}                    \Routid{D113}
\Keywords{ADAPTIVE NUMERICAL COMPLEX INTEGRATION QUADRATURE}
\Author{G.A. Erskine}             \Library{MATHLIB}
\Submitter{K.S. K\"olbig}          \Submitted{07.12.1970}
\Language{Fortran}                  \Revised{15.03.1993}
\Cernhead{Adaptive Complex Integration Along a Line Segment}
Function subprograms {\tt CGAUSS} and {\tt WGAUSS} compute,
to an attempted specified accuracy, the value of the complex integral
 $$ I=\int^B_A f(z)dz.$$
The path of integration is the directed line segment $AB$ in the complex
$z$-plane. The function $f(z)$ must be single-valued on this segment.
\par
The double-precision version {\tt WGAUSS} is available only on computers
which support a {\tt COMPLEX*16} Fortran data type.
\Structure
{\tt FUNCTION} subprograms  \\
User Entry Names: \Rdef{CGAUSS}, \Rdef{WGAUSS}\\
Files Referenced: {\tt Unit 6} \\
External References: \Rind{MTLMTR}{N002}, \Rind{ABEND}{Z035},
user-supplied {\tt FUNCTION} subprogram
\Usage
In any arithmetic expression,
\begin{center}
{\tt CGAUSS(F,A,B,EPS)} \qquad or \qquad {\tt WGAUSS(F,A,B,EPS)}
\end{center}
has the approximate value of the integral $I$.
\begin{DLtt}{123456}
\item[F] Name of a user-supplied {\tt FUNCTION} subprogram, declared
{\tt EXTERNAL} in the calling program. This subroutine must set
$\mathtt{F(Z)} = f(\mathtt{Z})$.
\item [A,B]End-points of integration interval.
\item[EPS]Accuracy parameter (see {\bf Accuracy}).
\end{DLtt}
{\tt CGAUSS} is of type {\tt COMPLEX}, {\tt WGAUSS} is of type
{\tt COMPLEX*16}, and the arguments {\tt F}, {\tt A},
{\tt B}, and {\tt Z} (in {\tt F}) have the same type as the
function name. {\tt EPS} is of type {\tt REAL} for {\tt CGAUSS}
and of type {\tt DOUBLE PRECISION} for {\tt WGAUSS}.
\Method
For any line segment $[a,b]$ we define $g_8(a,b)$ and $g_{16}(a,b)$ to
be the 8-point and 16-point Gaussian quadrature approximations to
     $$\int^b_a f(z)dz $$
and define
$$ r(a,b) =\frac{|g_{16} (a,b) - g_8(a,b)|}{1+|g_{16}(a,b)|}. $$
\newpage
Then, with $G$ = {\tt CGAUSS} or {\tt WGAUSS},
$$ G =\sum_{i=1}^kg_{16}(z_{i-1},z_i),$$
where, starting with $ z_0=A $ and finishing with $z_k=B$,
the subdivision points $ z_i \ (i=1,2,\ldots) $ are given by
$$ z_i = z_{i-1} + \lambda (B-z_{i-1}), $$
with $\lambda$ equal to the first member of the sequence
$1,1/2,1/4,\ldots $ for which $r(z_{i-1},z_i) < \mathtt{EPS} $.
If, at any stage in the process of subdivision, the ratio
  $$ q=\left|\frac{z_i-z_{i-1}}{B-A}\right| $$
is so small that $1+0.005q$ is indistinguishable from 1 to
machine accuracy, an error exit occurs with the function value
set equal to zero.
\Accuracy
Unless there is severe cancellation of positive and negative
values of $f(z)$ over the interval $[A,B]$, the argument {\tt EPS}
may be considered as specifying a bound on the {\it relative} error of
$I$ in the case $|I| > 1$, and a bound on the {\it absolute} error in
the case $|I|<1$. More precisely, if $k$ is the number of sub-intervals
contributing to the approximation (see {\bf Method}), and if
$$ I_{abs} = \int^B_A|f(z)|dz,$$
then the relation
$$ \frac{|G - I|}{I_{abs}+k}< \mathtt{EPS} $$
will nearly always be true, provided the routine terminates
without printing an error message. For functions
$f$ having no singularities in the closed interval $[A,B]$
the accuracy will usually be much higher than this.
\Errorh
Error {\tt D113.1}: The requested accuracy (see {\bf Method})
cannot be obtained.
The function value is set equal to zero, and a message is written on
{\tt Unit 6}, unless subroutine {\tt MTLSET} (N002) has been called.
\Notes
Values of the function $f(z)$ at the  end-points of the line segment
$A$ and $B$ are not required. The subprogram may therefore
be used when these values are undefined.
\\ $\bullet$
