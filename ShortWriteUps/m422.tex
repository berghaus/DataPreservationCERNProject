\Version {PACBYT}                          \Routid{M422}
\Keywords{MANIPULATION HANDLING PACK VECTOR BYTE}
\Author{J. Zoll, C. Letertre}              \Library{KERNLIB}
\Submitter{}                               \Submitted{28.01.1971}
\Language{Fortran or Assembler}      \Revised{16.09.1991}
\Cernhead {Handling Packed Vectors of Bytes}
{\tt PACBYT} allows handling of packed vectors of bytes. Any such
vector consists of a string of bytes, all of {\tt NBITS} bits,
{\tt INWORD} of them packed into one computer word,
stored from right to left.
\Structure
{\tt SUBROUTINE} and {\tt FUNCTION} subprograms \\
User Entry Names: \Rdef{PKBYT}, \Rdef{UPKBYT}, \Rdef{JBYTPK},
\Rdef{SBYTPK}\\
External References:  \Rind{JBYT}{M421}, \Rind{SBYT}{M421} (Fortran
version)
\Usage
The 2--word vector {\tt MPACK} specifies the packing parameters:
\begin{verbatim}
    MPACK(1) = NBITS
    MPACK(2) = INWORD
\end{verbatim}
$\mathtt{MPACK(1)=0}$ is accepted as specifying both $\mathtt{NBITS=1}$ and
{\tt INWORD} equal to the number of bits per word on the given computer.
\begin{verbatim}
    CALL PKBYT(IB,MX,JX,N,MPACK)
\end{verbatim}
packs the {\tt N}--word vector {\tt IB} of small integers into the bytes
{\tt JX,JX+1,\ldots,JX+N-1} of the byte-vector {\tt MX}.
\begin{verbatim}
    CALL UPKBYT(MA,JA,IY,N,MPACK)
\end{verbatim}
unpacks the {\tt N} bytes {\tt JA,JA+1,\ldots,JA+N-1} of the packed
byte-vector {\tt MA} into the vector {\tt IY} of small integers.
\begin{verbatim}
    IX = JBYTPK(MA,JA,MPACK)
\end{verbatim}
fetches the {\tt JA}-th byte from the packed byte-vector {\tt MA}.
\begin{verbatim}
    CALL SBYTPK(IT,MX,JX,MPACK)
\end{verbatim}
sets the first byte from {\tt IT} into the {\tt JX}'th byte of the packed
byte vector {\tt MX}.
\Notes
\begin{enumerate}
\item These routines, and the manner of packing byte-vectors, is
compatible with the routines {\tt JBYT} and {\tt SBYT} (M421),
except that there the {\it location} of a byte in the word is specified,
whereas here the {\it ordinal number} of a byte in the vector has to be
given. The conversion is as follows:
\par
The byte with ordinal number {\tt J} is found in word
$\mathtt{JW=(J-1)/INWORD+1}$, \\
on byte $\mathtt{JB=J-(JW-1)*INWORD}$
starting at bit $\mathtt{L=(JB-1)*NBITS+1}$.
\item  Bits and bytes are numbered from right to left within one and
the same computer word; across a word boundary there is a jump from
the most significant part of the current word to the least
significant part of the next word.
\end{enumerate}
$\bullet$
