\Version {DEFLS}                   \Routid{F230}
\Keywords{DEFLATE MATRIX EIGEN VALUE VECTOR}
\Author{E. Edberg}                  \Library{MATHLIB}
\Submitter{}                         \Submitted{01.07.1974}
\Language{Fortran}              %\Revised{}
\Cernhead {Deflate Matrix with Known Eigenvalue/Eigenvector}
Given a matrix of order $n$ and an eigenvalue and a corresponding
eigenvector (both real) the routines {\tt DEFL}, {\tt DEFLS} and
{\tt DEFLSS} compute a deflated matrix, i.e. a matrix of order $n-1$
with the same eigenvalues as the given matrix except for the known
eigenvalue. The eigenvectors will in general not be the same.
\begin{DLtt}{12345678}
\item[DEFL] expects no special properties of the input matrix except
that it is real and has a known real eigenvalue and eigenvector.
\item[DEFLS] is written for symmetric real matrices.
\item[DEFLSS] expects the input matrix to be symmetric and stored in a
one-dimensional array as described below.
\end{DLtt}
\Structure
{\tt SUBROUTINE} subprograms \\
User Entry Names: \Rdef{DEFL}, \Rdef{DEFLS}, \Rdef{DEFLSS}\\
Internal Entry Names: {\tt SKAL}
\Usage
\begin{verbatim}
    CALL DEFL  (A,N,NDIM,EIGVAL,EIGVEC)
    CALL DEFLS (A,N,NDIM,EIGVAL,EIGVEC)
    CALL DEFLSS(A,N,EIGVAL,EIGVEC)
\end{verbatim}
\begin{DLtt}{12345678}
\item [A] ({\tt REAL}) The input matrix. In the first two cases {\tt A}
is a two-dimensional array, but for {\tt DEFLSS}, {\tt A} is
one-dimensional and should contain the elements {\tt A(1,1)},
{\tt A(2,1)}, {\tt A(2,2)}, {\tt A(3,1)},\ldots,{\tt A(N,N)},
stored in that order. On return, {\tt A} will contain the deflated
matrix.
\item [N] ({\tt INTEGER}) Order of the matrix {\tt A}.
\item [NDIM] ({\tt INTEGER}) First dimension parameter of {\tt A} in the
calling program.
\item [EIGVAL] ({\tt REAL}) The known eigenvalue.
\item [EIGVEC] ({\tt REAL}) One-dimensional array containing the
eigenvector corresponding to {\tt EIGVAL}.
\end{DLtt}
\Method
Householder transformations.
\Notes
The user should be aware of the fact that the accuracy of the deflated
matrix depends on the accuracy of the given eigenvalue and eigenvector.
\Refer
\begin{enumerate}
\item J.H. Wilkinson, The Algebraic Eigenvalue Problem, 585.
\end{enumerate}
$\bullet$
