% 24 oct 94   ksk
\Version{RASLGF}                              \Routid{C330}
\Keywords{ASSOCIATE FUNCTION LEGENDRE}
\Author{K.S. K\"olbig}                        \Library{MATHLIB}
\Submitter{}                                  \Submitted{15.05.1987}
\Language{Fortran}                        \Revised{01.12.1994}
\Cernhead{Legendre and Associated Legendre Functions}
Subroutine subprograms {\tt RASLGF} and {\tt DASLGF} calculate,
for a given real argument
$x,(-1 \le x \le 1)$, and a given integer value of the order $m$,
a sequence of either unnormalized or normalized Legendre $(m=0)$ or
Associated Legendre $(m \ne 0)$ functions of
degree $n = 0,1,2,\ldots,N$, defined by
$$\begin{array}{l@{\quad=\quad}ll@{\qquad}r}
P_n^m(x) & \displaystyle(1-x^2)^{m/2} \
\frac{d^m}{dx^m}P_n(x) & (m \ge 0) & {\rm (1a)} \\[5mm]
P_n^m(x) & \displaystyle\frac{(n+m)!}{(n-m)!} \ P_n^{-m}(x) &
(m<0) & {\rm (1b)} \\[5mm]
\overline{P_n^m}(x) &
\displaystyle \sqrt{\frac{2n+1}{2}\frac{(n-m)!}{(n+m)!}}
\ P_n^m(x), & & (2)
\end{array}$$
respectively, where
$$ P_n(x) \ \equiv \ P_n^0(x) \ = \
\displaystyle\frac{1}{2^n n!} \ \frac{d^n}{dx^n}(x^2-1)^n $$
is the Legendre polynominal of degree $n$.
Note that some authors use an
additional factor $(-1)^m$ in the definition (1).
\par
On CDC and Cray computers,
the double-precision version {\tt DASLGF} is not available.
\Structure
{\tt SUBROUTINE} subprograms \\
User Entry Names: \Rdef{RASLGF}, \Rdef{DASLGF}\\
Obsolete User Entry Names: \Rdef{ASLGF} $\equiv$ {\tt RASLGF}\\
Files Referenced: {\tt Unit 6} \\
External References: \Rind{MTLMTR}{N002}, \Rind{ABEND}{Z035}
\Usage
For $\mathtt{t=R}$ (type {\tt REAL}), $\mathtt{t=D}$ (type
{\tt DOUBLE PRECISION}),
\begin{verbatim}
    CALL tASLGF(MODE,X,M,NL,P)
\end{verbatim}
\begin{DLtt}{123456}
\item[MODE] ({\tt INTEGER}) $\mathtt{= 1:}$ Unnormalized functions (1),\\
\phantom{({\tt INTEGER})} $\mathtt{= 2:}$ Normalized functions (2).
\item[X] (type according to {\tt t}) The argument $x$.
\item[M]({\tt INTEGER}) The order $m$ (upper index) of all functions in
the computed sequence. It is permissible for {\tt M} to be negative.
\item[NL]({\tt INTEGER}) Specifies the degree $N$ of the last
function in the computed sequences.
\item[P] (type according to {\tt t}) One-dimensional array of
dimension ({\tt 0:d}) where $\mathtt{d} \ge \mathtt{NL}$. \\
On exit, $\mathtt{P(n),(n=0,1,\ldots,NL)}$, contains $P_n^m
(\mathtt{X})$ or $\overline{P_n^m}(\mathtt{X})$ as
specified by {\tt MODE}. (See {\bf Notes}).
\end{DLtt}
\Method
The functions $P_n^m(x)$ are for $ m>0 $ calculated by means of
the standard recurrence relation.
\newpage
\Restrict
\begin{enumerate}
\item $\mathtt{-1 \le X \le 1}$.
\item $\mathtt{MODE = 1}$ or {\tt 2}.
\item If $\mathtt{M = 0: 0 \le NL \le 100}$: \\
if $\mathtt{M \ne 0: |M| \le 27}$
and $\mathtt{0 \le NL \le 55 - |M|}$;\,
($\mathtt{0 \le NL \le 33 - |M|}$ on VAX/VMS).
\end{enumerate}
\Accuracy
{\tt RASLGF} (except on CDC and Cray computers)
has full single-precision accuracy.
For most values of the argument {\tt X}, {\tt DASLGF}
(and {\tt RASLGF} on CDC and Cray computers) has an accuracy of
approximately two significant digits less than the machine precision.
\Notes
In accordance with the definitions,
$\mathtt{P(n) = 0}$ for $\mathtt{n = 0,1,\cdots,|M|-1}$.
\Errorh
Error {\tt C330.1}: $\mathtt{|X|>1}$. \\
Error {\tt C330.2}: $\mathtt{MODE \ne 1}$ and $\mathtt{MODE \ne 2}$. \\
Error {\tt C330.3}: $\mathtt{M}$ and $\mathtt{NL}$ incompatible. \\
In all cases, a message is written on
{\tt Unit 6}, unless subroutine {\tt MTLSET} (N002) has been called.
The initial contents of array {\tt P(n)} is left unchanged.
\\ $\bullet$
