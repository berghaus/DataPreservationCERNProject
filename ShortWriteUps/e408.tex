%  18 oct 1994 ksk
\Version {RCHPWS}                           \Routid{E408}
\Keywords{CONVERSION CHEBYSHEV POWER SERIES}
\Author{K.S. K\"olbig}            \Library{MATHLIB}
\Submitter{}         \Submitted{15.02.1994}
\Language{Fortran}                    %\Revised{}
\Cernhead {Conversion of Chebyshev to Power and Power to
Chebyshev Series}
Subroutine subprograms {\tt RCHPWS}, {\tt RPWCHS} and
{\tt DCHPWS}, {\tt DPWCHS} perform the conversion of a finite
Chebyshev series to a finite power series (i.e. a polynomial) and
{\it vice versa}.
\par
Thus, given the coefficients $c_j$,
$(j=0,1,\ldots,n)$ of a finite Chebyshev series,
{\tt RCHPWS} and {\tt DCHWPS} calculate the coefficients
$a_j$, $(j=0,1,\ldots,n)$ of the equivalent polynomial:
$$ c_0+c_1T_1(x)+\cdots+c_nT_n(x) \ = \ a_0+a_1x+\cdots+a_nx^n. $$
Conversely, given the coefficients $a_j$,
$(j=0,1,\ldots,n)$ of a power series,
{\tt RPWCHS} and {\tt DPWCHS} calculate the coefficients
$c_j$, $(j=0,1,\ldots,n)$ of the equivalent finite Chebyshev series:
$$ a_0+a_1x+\cdots+a_nx^n \ = \ c_0+c_1T_1(x)+\cdots+c_nT_n(x). $$
In both cases, $T_j(x)$ is the Chebyshev polynomial of degree $j$.
\par
Note that sometimes the constant term in the Chebyshev series is
defined differently, i.e. $c_0/2$ instead of $c_0$.
Here, the definition of Ref. 1 is used.
\par
On computers other than CDC or Cray, only the double-precision versions
{\tt DCHPWS} and {\tt DPWCHS} are available.
On CDC and Cray computers, only the single-precision versions
{\tt RCHPWS} and {\tt RPWCHS} are available.
\Structure
{\tt SUBROUTINE} subprograms \\
User Entry Names: \Rdef{RCHPWS}, \Rdef{RPWCHS},
                  \Rdef{DCHPWS}, \Rdef{DPWCHS} \\
Files referenced: {\tt Unit 6}
\Usage
For $\mathtt{t=R}$ (type {\tt REAL}), $\mathtt{t=D}$ (type
{\tt DOUBLE PRECISION}),
\begin{verbatim}
    CALL tCHPWS(N,C,A)
\end{verbatim}
\begin{DLtt}{1234}
\item[N] ({\tt INTEGER}) Degree $n$ of last Chebyshev polynomial
in the expansion.
\item[C] (type according to {\tt t}) One-dimensional array of dimension
{\tt (0:d)}, where $\mathtt{d \ge N}$. On entry, {\tt C} must contain
the coefficients $c_j$, $(j=0,1,\ldots,n)$ of the Chebyshev expansion.
\item[A] (type according to {\tt t}) One-dimensional array of dimension
{\tt (0:d)}, where $\mathtt{d \ge N}$. On exit, {\tt A} contains the
coefficients $a_j$, $(j=0,1,\ldots,n)$ of the power series expansion.
\end{DLtt}
\begin{verbatim}
    CALL tPWCHS(N,A,C)
\end{verbatim}
\begin{DLtt}{1234}
\item[N] ({\tt INTEGER}) Degree $n$ of the polynomial.
\item[A] (type according to {\tt t}) One-dimensional array of dimension
{\tt (0:d)}, where $\mathtt{0 \ge N}$. On entry, {\tt A} must contain
the coefficients $a_j$, $(j=0,1,\ldots,n)$ of the polynomial.
\item[C] (type according to {\tt t}) One-dimensional array of dimension
{\tt (0:d)}, where $\mathtt{0 \ge N}$. On exit, {\tt C} contains the
coefficients $c_j$, $(j=0,1,\ldots,n)$ of the Chebyshev expansion.
\end{DLtt}
\Errorh
Error {\tt E408.1}: $\mathtt{N<0}$ or $\mathtt{N>100}$. \\
A message is written on {\tt Unit 6}, unless subroutine {\tt MTLSET}
(N002) has been called.
\Refer
\begin{enumerate}
\item Y.L. Luke, Mathematical functions and their approximations,
(Academic Press, New York 1975)
\end{enumerate}
$\bullet$
