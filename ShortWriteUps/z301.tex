\Version{VAXTIO}                           \Routid{Z301}
\Keywords{FORTRAN INTERFACE READ TAPE VAX VAXTIO}
\Author{C. Ciapetti, J. Zoll}             \Library{KERNLIB, VAX only}
\Submitter{}                                \Submitted{01.09.1983}
\Language{VAX Fortran}                  %\Revised{}
\Cernhead{VAX Fortran Interface for Reading and Writing 'Foreign' Tapes}
{\tt VAXTIO} handles non-native tapes on the VAX; it is needed
because VAX Fortran does not provide a {\tt U} format.
\par
If the tape to be handled is on logical unit {\tt 11}, mounted on
{\tt MTAO}, with physical records of 3600 bytes maximum, for example, the
following commands have to be given:
\begin{verbatim}
    $ MOUNT MTAO:/FOREIGN/BLOCKSIZE=3600/RECORDSIZE=3600
    $ ASSIGN MTA0: QIOUNIT11
\end{verbatim}
\Structure
{\tt SUBROUTINE} subprogram \\
User Entry Names: \Rdef{VAXTIO}\\
Internal Entry Names: {\tt WAIT2S} \\
Files Referenced: User defined parameter\\
{\tt COMMON} Block Names and Lengths: {\tt /VAXTIO/ 240}
\Usage
\begin{verbatim}
   CALL VAXTIO(LUN,MODOP,IBUF,NDO,NDONE,NCODE,LUNMSG)
\end{verbatim}
{\bf Input} parameters:
\begin{DLtt}{12345678}
\item [LUN] Logical unit number ($\mathtt{0 < LUN < 61}$).
\item [MODOP] Operation mode, indicating the kind of operation to be
performed; for details, see below.
\item [IBUF] Data area for read and write.
\item [NDO] Number of units to be done.
\item [LUNMSG] Fortran logical unit number for printing
diagnostic messages; if zero, printing is suppressed.
\end{DLtt}
{\bf Output} parameters:
\begin{DLtt}{1234567890}
\item [NDONE] Number of units done; error if negative.
\item [NCODE] {\tt QIO} System status code.
\end{DLtt}
The following operations are provided at present: \\[2mm]
\begin{tabular}[t]{lll}
$\mathtt{MODOP=-2:}$ & \multicolumn{2}{l} {Write {\tt EOF} (3 tape marks
are written and the tape is positioned after the first}\\
& \multicolumn{2}{l} {tape mark).} \\
& $\mathtt{NDONE = 1}$ & Successful. \\
& $\mathtt{NDONE = 0}$ & End-of-tape. \\
& $\mathtt{NDONE =-7}$ & Trouble. \\
$\mathtt{MODOP=-1:}$ & \multicolumn{2}{l} {Write one record, tranfer
{\tt NDO} bytes from {\tt IBUF} to tape.}\\
& $\mathtt{NDONE > 1}$ & Number of bytes written. \\
& $\mathtt{NDONE = 0}$ & End-of-tape, but record written. \\
& $\mathtt{NDONE =-7}$ & Trouble.
\end{tabular}  \\
\begin{tabular}[t]{lll}
$\mathtt{MODOP=0:}$ & \multicolumn{2}{l} {Read one record, transfer at most
{\tt NDO} bytes from tape to {\tt IBUF}, excess data}\\
& \multicolumn{2}{l} {are lost.} \\
& $\mathtt{NDONE > 0}$ & Number of bytes transferred. \\
& $\mathtt{NDONE = 0}$ & {\tt EOF}, end-of-tape. \\
& $\mathtt{NDONE =-1}$ & Read error, record skipped. \\
& $\mathtt{NDONE =-7}$ & Trouble. \\
$\mathtt{MODOP=1:}$ & \multicolumn{2}{l} {Assign a channel for logical unit
(if not done explicitly, assignment occurs on}\\
& \multicolumn{2}{l} {first contact).} \\
& $\mathtt{NDONE = 1}$ & Successful. \\
& $\mathtt{NDONE = 0}$ & Channel already assigned. \\
& $\mathtt{NDONE =-7}$ & Trouble. \\
$\mathtt{MODOP=2:}$ & \multicolumn{2}{l} {Skip $\mathtt{|NDO|}$
records, forward if $\mathtt{NDO>0}$, reverse if $\mathtt{NDO < 0}$.}
($\mathtt{|NDO| < 32768}$) \\
& $\mathtt{NDONE > 0}$   & Number of records skipped. \\
& $\mathtt{NDONE < NDO}$ & {\tt EOF} seen, skipped, counted. \\
& $\mathtt{NDONE =-7}$   & Trouble. \\
$\mathtt{MODOP=3:}$ &
\multicolumn{2}{l} {Skip $\mathtt{|NDO|}$ files, forward
if $\mathtt{NDO > 0}$, reverse if $\mathtt{NDO < 0}$.}\\
& $\mathtt{NDONE > 0}$   & Number of files skipped. \\
& $\mathtt{NDONE < NDO}$ & End-of-tape seen. \\
& $\mathtt{NDONE =-7}$   & Trouble.  \\
$\mathtt{MODOP=4:}$ & Rewind.   \\
$\mathtt{MODOP=5:}$ & \multicolumn{2}{l} {Rewind and unload.} \\
& $\mathtt{NDONE = 1}$ & Successful. \\
& $\mathtt{NDONE =-7}$ & Trouble.   \\
$\mathtt{MODOP=6:}$ & \multicolumn{2}{l} {De-assign channel; this should
be done if a logical unit is no longer needed.}\\
& $\mathtt{NDONE = 1}$ & Successful. \\
& $\mathtt{NDONE =-7}$ & Trouble.
\end{tabular}
\\ $\bullet$
