% 07 nov 94  ksk
\Version{SORTR}                                        \Routid{M107}
\Keywords{SORT ROW MATRIX}
\Author{F. Carminati}                                 \Library{KERNLIB}
\Submitter{}                                     \Submitted{09.02.1989}
\Language{Fortran}                  %\Revised{}
\Cernhead{Sort Rows of a Matrix}
{\tt SORTR} re-arranges the row order of a matrix in such a way that the
elements of a selected column are either in increasing or decreasing
order as described. When these elements are equal, the rows are kept
in their original order.
\Structure
{\tt SUBROUTINE} subprogram \\
User Entry Names: \Rdef{SORTR}, \Rdef{SORTI}, \Rdef{SORTD}\\
External References: \Rind{VECMAN}{F121}, \Rind{USWOP}{V301}
(not on all machines)
\Usage
For $\mathtt{t=I}$ (type {\tt INTEGER}),
$\mathtt{t=R}$ (type {\tt REAL}), $\mathtt{t=D}$ (type
{\tt DOUBLE PRECISION}),
\begin{verbatim}
    CALL SORTt(MX,NC,NR,NCS)
\end{verbatim}
performs an ordering operation on the matrix {\tt MX} of type {\tt t},
dimensioned {\tt (NC,NR)}, using the {\tt NCS}-th element of each
row as ordering criterion.
\par
The matrix {\tt MX} is stored by rows, the first element of a row
following immediately after the last element of the preceding row.
\par
Obviously, $\mathtt{1 \le |NCS| \le NC}$ is a condition. If this is not
met or if $\mathtt{NR \le 1}$, {\tt SORTX} will do nothing.
\par
If $\mathtt{NCS > 0}$, the subroutine re-orders the rows of
{\tt MX} in such
a way that the {\tt NCS}-th element of each row is greater than or
equal to the {\tt NCS}-th element of the preceding row. If
$\mathtt{NCS < 0}$, the rows of {\tt MX} are re-ordered in such a way that
the {\tt NCS}-th element of each row is smaller than or equal to the
{\tt NCS}-th element of the preceding row.
\\ $\bullet$
