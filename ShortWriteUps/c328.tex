\Version{CWHITM}                     \Routid{C328}
\Keywords{COMPLEX WHITTAKER FUNCTION}
\Author{K.S. K\"olbig}                \Library{MATHLIB}
\Submitter{}                          \Submitted{15.01.1988}
\Language{Fortran}                     \Revised{15.03.1993}
\Cernhead{Whittaker Function M of Complex Argument and Complex Indices}
Function subprograms {\tt CWHITM} and {\tt WWHITM}
compute the Whittacker function
$$ M_{\kappa,\mu}(z) \ = \ e^{-\frac{1}{2}z} z^{\frac{1}{2}+\mu}
M(\textstyle \frac{1}{2}+\mu-\kappa,1+2\mu,z) $$
for complex arguments $z$ and complex indices
$\kappa,\mu$, where $M(a,b,z)$ is Kummer's function (See Ref. 1).
The $z$-plane is cut along the negative real axis.
\par
The double-precision version {\tt WWHITM} is available only on computers
which support a {\tt COMPLEX*16} Fortran data type.
\Structure
{\tt FUNCTION} subprograms \\
User Entry Names: \Rdef{CWHITM}, \Rdef{WWHITM}\\
Files Referenced: {\tt Unit 6} \\
External References:
   \begin{tabular}[t]{@{}llll}
       \Rind{CLGAMA}{C306},&\Rind{WLGAMA}{C306},&
       \Rind{CCLBES}{C309},&\Rind{WCLBES}{C309},\\
       \Rind{MTLMTR}{N002},&\Rind{ABEND}{Z035}
   \end{tabular}
\Usage
In any arithmetic expression,
\begin{center}
{\tt CWHITM(Z,KA,MU)} \quad or \quad {\tt WWHITM(Z,KA,MU)} \quad has the
value \quad $M_{\mathtt{KA,MU}}(\mathtt{Z}),$
\end{center}
where $\mathtt{KA} = \kappa$ and $\mathtt{MU} = \mu$.
{\tt CWHITM} is of type {\tt COMPLEX}, {\tt WWHITM} is of type
{\tt COMPLEX*16}, and {\tt Z}, {\tt KA} and
{\tt MU} have the same type as the function name.
\Method
For $\mu-\kappa+\frac{1}{2}$ or $\mu+\kappa+\frac{1}{2}$ equal to a
negative
integer, $M_{\kappa,\mu}(z)$ reduces to a polynomial in $z$. For other
values, a regular Coulomb wave function $F_0(\nu,\rho)$
is computed by using subprogram {\tt CCLBES} (C309)
in conjunction with functional relations.
\Restrict
$\mu \neq -\frac{1}{2},-\frac{3}{2},\ldots$;
Re $z \geq 0$ if Im $z=0$.
\Accuracy
{\tt CWHITM} (except on CDC and Cray computers)
has full single-precision accuracy.
For most values of the arguments, {\tt WWHITM} (and {\tt CWHITM}
on CDC and Cray computers) has an accuracy of approximately two to
three decimal digits less than the machine precision.
\Errorh
Error {\tt C328.1}: $\mathtt{Z=X+iY}$ with $\mathtt{X<0}$ and
$\mathtt{Y=0}$. \\
Error {\tt C328.2}: $2*\mathtt{MU=}-n,\,(n=1,2,\ldots)$. \\
In both cases, the function value is set equal to zero,
and a message is written on
{\tt Unit 6}, unless subroutine {\tt MTLSET} (N002) has been called.
An error message is also written on {\tt Unit 6} if the internal call to
{\tt CCLBES} or {\tt WCLBES} returns $\mathtt{JFAIL} \ne 0$
(see Short write-up for {\tt CCLBES} (C309)).
\newpage
\Refer
\begin{enumerate}
\item  M. Abramowitz and I.A. Stegun (Eds.), Handbook
of Mathematical Functions,
Chapter 13, Confluent Hypergeometric Functions,
9th printing with corrections, (Dover, New York 1972).
\item L.J. Slater, Confluent hypergeometric functions, (University Press,
Cambridge 1960)
\end{enumerate}
$\bullet$
