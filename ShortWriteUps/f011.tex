% 27 May 1992 mg
\Version{RFACT}                               \Routid{F011}
\Keywords{LINEAR EQUATION MATRIX INVERSION SOLUTION DETERMINANT}
\Author{G.A. Erskine, H. Lipps}                 \Library{KERNLIB}
\Submitter{}                                    \Submitted{18.12.1979}
\Language{Fortran or Assembler or COMPASS}   \Revised{27.11.1984}
\Cernhead{Repeated Solution of Linear Equations, Matrix Inversion, Determinant}
These subroutines provide a two-step procedure for solving sets of
linear equations
$$\mathbf{AX=B} \mbox{\hspace{2cm}(*)}$$
which is faster than the library programs {\tt RINV} (F010) when (*)
must be solved repeatedly for the same matrix {\bf A} with different
sets of right-hand sides. The inverse matrix $\mathbf{A^{-1}}$ and the
determinant det({\bf A}) may also be calculated.
\Structure
{\tt SUBROUTINE} subprograms\\
User Entry Names: \Rdef{RFACT}, \Rdef{RFEQN}, \Rdef{RFINV},
\Rdef{DFACT}, \Rdef{DFEQN}, \Rdef{DFINV},
\Rdef{CFACT}, \Rdef{CFEQN}, \Rdef{CFINV} \\
Internal Entry Names:  {\tt TMPRNT}\\
Files Referenced: Printer\\
External References: \Rind{KERMTR}{N001}, \Rind{ABEND}{Z035}
\Usage
For $\mathtt{t=R}$ (type {\tt REAL}), $\mathtt{t=D}$ (type
{\tt DOUBLE PRECISION}), $\mathtt{t=C}$ (type {\tt COMPLEX}):
\begin{verbatim}
    CALL tFACT(N,A,IDIM,IR,IFAIL,DET,JFAIL)
    CALL tFEQN(N,A,IDIM,IR,K,B)
    CALL tFINV(N,A,IDIM,IR)
\end{verbatim}
\begin{DLtt}{123456}
\item [N] ({\tt INTEGER}) Order of the square matrix {\bf A}.
\item [A] (Type according to {\tt t}) Two-dimensional array
whose first dimension has the value {\tt IDIM}.
\item [IDIM] ({\tt INTEGER}) First dimension of array {\tt A}
(and of array {\tt B} if $\mathtt{K > 1}$).
\item [IR] ({\tt INTEGER}) Array of at least {\tt N}
elements, required as working space.
\item [IFAIL] ({\tt INTEGER}) On exit, {\tt IFAIL} will be set to
$\mathtt{-1}$ if {\bf A} is found to be singular, and to {\tt 0}
otherwise.
(Singularity will often go undetected because of rounding errors during
factorization even if the elements of {\bf A} have integral values.)
\item [DET] (Type according to {\tt t}) On exit, {\tt DET} will be set
to the value det({\bf A}) unless {\tt JFAIL} returns a non-zero value.
\item [JFAIL] ({\tt INTEGER}) On exit, {\tt JFAIL} will be set to zero if
det({\bf A}) can be safely evaluated. Otherwise {\tt JFAIL} is set as
follows: \\
$\mathtt{= -1}$ if det({\bf A}) is probably too small, \\
$\mathtt{= +1}$ if det({\bf A}) is probably too large.
\item [K] ({\tt INTEGER}) Number of columns of the matrices {\bf B}
and {\bf X}.
\item [B] (Type according to {\tt t}) In general, a two-dimensional
array whose first dimension has the value {\tt IDIM}.
{\tt B} may be one-dimensional if $\mathtt{K = 1}$.
\end{DLtt}
\newpage
Subroutine {\tt tFACT} must be called with matrix {\bf A} in array
{\tt A} prior to any calls to {\tt tFEQN} and {\tt tFINV}. On return the
situation is as follows:
\begin{enumerate}
\item Provided {\bf A} is non-singular, {\tt IFAIL} will be set
to {\tt 0}, and {\tt A} and {\tt R} will be set in preparation for calls
to {\tt tFEQN} and {\tt tFINV}.
\par
If {\bf A} is singular, {\tt IFAIL} will be set to $\mathtt{-1}$,
in which case any subsequent call to {\tt tFEQN} or {\tt tFINV} will give
unpredictable results.
\item Provided det({\bf A}) can be safely evaluated within the range
of the computer, {\tt JFAIL} will be set to {\tt 0} and and {\tt DET}
will be set to det({\bf A}). In particular, if {\bf A} is singular,
both {\tt JFAIL} and {\tt DET} will be set to zero.
\par
If the evaluation of det({\bf A}) would probably cause underflow,
{\tt JFAIL} will be set to $\mathtt{-1}$ and {\tt DET} will be set to
zero.
\par
If the evaluation of det({\bf A}) would probably cause overflow,
{\tt JFAIL} will be set to $\mathtt{+1}$ and {\tt DET} will be incorrect.
\par
Execution continues, and subsequent calls to {\tt tFEQN} and {\tt tFINV}
will give correct results.
\end{enumerate}
Subroutine {\tt tFEQN} may be called only after {\tt tFACT} has been
called, with the contents of {\tt A} and {\tt R} unchanged, and with
matrix {\bf B} in array {\tt B}. On return, {\tt B} will contain
the solution {\bf X}, with {\tt A} and {\tt R} unchanged. Therefore a
single call to {\tt tFACT} may be followed by several calls to
{\tt tFEQN} with differing {\bf B}.
\par
Subroutine {\tt tFINV} may be called only after {\tt tFACT} has been
called, with the contents of {\tt A} and {\tt R} unchanged. On return,
{\tt A} will contain the inverse $\mathbf{A^{-1}}$ of {\bf A}. Therefore,
once {\tt tFINV} has been called, it is no longer meaningful to call
{\tt tFEQN} with {\tt A} as parameter.
\Method
Triangular factorization with row interchanges. The inverse
matrix $\mathbf{A^{-1}}$ is the product, in reverse order, of the
in-place inverses of the triangular factors. The array {\tt R} holds
information specifying the row interchanges.
\Accuracy
On computers with IBM 370 architecture, inner products are accumulated
using double-precision arithmetic internally for arrays of type
{\tt REAL} and {\tt COMPLEX}.
\Errorh
If $\mathtt{N < 1}$ or $\mathtt{IDIM < N}$ or $\mathtt{K < 1}$,
a message is printed and program execution is terminated by
calling {\tt ABEND} (Z035).
\Examples
Assume that the $10 \times 10$ matrix {\bf A}, the $10 \times 3$ matrix
{\bf B}, and the 10-element vector {\bf z} are stored according to
the Fortran convention in arrays {\tt A}, {\tt B} and {\tt Z}
respectively of a program containing the declarations
\begin{verbatim}
    DIMENSION IR(25)
    COMPLEX A(25,30),B(25,10),Z(25),DET
\end{verbatim}
Then, unless {\bf A} is singular (which is to cause a jump to
statement {\tt 100}), the following statements will set
$\mathtt{DET}=\det(\mathbf{A})$, replace {\bf B} by $\mathbf{A^{-1}B}$,
replace
{\bf z} by $\mathbf{A^{-1}z}$, and replace {\bf A} by  $\mathbf{A^{-1}}$:
\begin{verbatim}
    CALL CFACT (10,A,25,IR,IFAIL,DET,JFAIL)
    IF(IFAIL .NE. 0) GO TO 100
    CALL CFEQN(10,A,25,IR,3,B)
    CALL CFEQN(10,A,25,IR,1,Z)
    CALL CFINV(10,A,25,IR)
\end{verbatim}
$\bullet$
