%  ksk 09 may 96
\Version{CFT}                         \Routid{D702}
\Keywords{COMPLEX FAST FOURIER TRANSFORM}
\Author{R.C. Singeton (Stanford)}      \Library{MATHLIB}
\Submitter{B. Fornberg}                \Submitted{03.05.1968}
\Language{Fortran}                     \Revised{01.10.1974}
\Cernhead{Complex Fast Fourier Transform}
A discrete Fourier transform is defined by:
$$ Y(n) \ = \
\sum^{N-1}_{j=0}X(j) \ \exp\left(\frac{-2\pi ijn}{N}\right),
\qquad (n=0,1,\ldots,N-1). $$
and the inverse
$$ Z(j) \ = \
\sum^{N-1}_{n=0}Y(n) \ \exp\left(\frac{2\pi ijn}{N}\right),
 \qquad (j=0,1,\ldots,N-1) $$
satisfying $Z(j) = N X(j), (j = 0,1,\ldots,N-1)$.
{\tt CFT} evaluates these sums using fast Fourier technique. It
is not required that $N$ is a power of 2. One-, two- and
three-dimensional transforms can be performed.
\Structure
{\tt SUBROUTINE} subprogram  \\
User Entry Names: \Rdef{CFT} \\
Files Referenced: Printer
\Usage
\begin{verbatim}
    CALL CFT(A,B,NTOT,N,NSPAN,ISN)
\end{verbatim}
Arrays {\tt A} and {\tt B} originally hold the real and imaginary
components
of the data, and return the real and imaginary components of the
resulting Fourier coefficients.
\par
Multivariate data is indexed according to the Fortran array element
successor function, without limit on the number of implied multiple
subscripts. The {\tt SUBROUTINE} is called once for each variate.
The calls for a multivariate transform may be in any order.
{\tt NTOT} is the total number of complex data values.
{\tt N} is the dimension of the current variable.
{\tt NSPAN/N} is the spacing of consecutive data values while
indexing the current variable. The sign of {\tt ISN} determines
the sign of the complex exponential, and the magnitude of {\tt ISN}
is normally one.
\par
For a single-variate transform,
$\mathtt{NTOT=N=NSPAN=}$ (number of complex data values), e.g.
\begin{verbatim}
    CALL CFT(A,B,N,N,N,1)
\end{verbatim}
A tri-variate transform with {\tt A(N1,N2,N3)}, {\tt B(N1,N2,N3)} is
computed by
\begin{DLtt}{1234}
\item[] {\tt CALL CFT(A,B,N1*N2*N3,N1,N1,1)}
\item[] {\tt CALL CFT(A,B,N1*N2*N3,N2,N1*N2,1)} \qquad and
\item[] {\tt CALL CFT(A,B,N1*N2*N3,N3,N1*N2*N3,1)}
\end{DLtt}
The data may alternatively be stored in a single {\tt COMPLEX}
array {\tt A}, then the magnitude of {\tt ISN} changed to two to give the
correct indexing increment and the second parameter used to pass
the initial address for the sequence of imaginary values, for
example:
\begin{verbatim}
    REAL
    EQUIVALENCE (A,S)
    ...
    CALL CFT (A,S(2),NTOT,N,NSPAN,2)
\end{verbatim}
Arrays {\tt AT(MAXF)}, {\tt CK(MAXF)}, {\tt BT(MAXF)}, {\tt SK(MAXF)},
and {\tt NP(MAXP)} are used for temporary storage. If the available
storage is insufficient the program is terminated by a {\tt STOP}.
\par
{\tt MAXF}
must be $\geq $ the maximum prime factor of {\tt N}. {\tt MAXP} must be
$>$ the number of prime factors of {\tt N}. In addition, if the
square-free portion {\tt K} of {\tt N} has two or more prime factors,
then {\tt MAXP} must be $\geq \mathtt{K}-1$. Storage in {\tt NFAC} allows
for a maximum of 11 factors of {\tt N}. If {\tt N} has more than one
square-free factor, the product of the square-free factors must be
$\leq$ 210.
\Notes
{\tt CFT} is very general since
the number of points is not restricted to powers of two, as is the
case for {\tt RFT} (D700) and {\tt FFTRC} (D701).
For $\mathtt{N=16,32,64,128}$ the routines in {\tt FFTRC} (D701) are
considerably faster.
\Refer
\begin{enumerate}
\item R.C. Singleton, An Algorithm for Computing the
Mixed Radix F.F.T., IEEE Trans. Audio Electroacoust., AU--1(1969)
93--107.
\item Reprinted in: L.R. Rabiner and C.M. Rader: Digital
Signal Processing,  IEEE Press New York (1972) 294.
\end{enumerate}
$\bullet$
