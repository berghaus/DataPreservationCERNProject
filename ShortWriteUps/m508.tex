\Version{BITPOS}                        \Routid{M508}
\Keywords{BIT FIND LOCATE STRING}
\Author{M. Metcalf, R. Matthews}        \Library{KERNLIB}
\Submitter{}                            \Submitted{01.02.1982}
\Language{Fortran and CDC: COMPASS, IBM: Assembler}
\Revised{20.06.1985}
\Cernhead{Find One-Bits in a String}
{\tt BITPOS} locates and counts the {\tt 1}-bits in the right-most
{\tt NBITS} bits in a word or in a full-word array, returning their
positions. Bit numbering is right-to-left, bit number 0 being the least
significant bit in the first full word, bit number {\tt NBPW} being the
least significant bit in the second full word etc., where {\tt NBPW}
is the number of bits per machine word; this numbering is compatible
with {\tt BITPAK} (M441).
\Structure
{\tt SUBROUTINE} subprogram \\
User Entry Names: \Rdef{BITPOS}\\
External  Entry Names: \Rind{URKBYT}{M422} (Fortran only)\\
{\tt COMMON} Block Names and Lengths: {\tt /SLATE/ 40}
(Fortran only)
\Usage
\begin{verbatim}
    CALL BITPOS(IWORDS,NBITS,IXV,NX)
\end{verbatim}
\begin{DLtt}{12345678}
\item [IWORDS] Word or full-word array to be analysed.
\item [NBITS] The first {\tt NBITS} of array {\tt IWORDS} are inspected.
\item [IXV] Bit positions of the {\tt 1}-bits in {\tt IWORD} are placed
into {\tt IXV(1)} through {\tt IXV(NX)} in increasing order. {\tt IXV}
must be dimensioned to {\tt NBITS} at least. The positions are numbered
from 0.
\item [NX] Number of {\tt 1}-bits found.
\end{DLtt}
\Notes
The Fortran version contains a symbolic constant whose value must
be set equal to the number of bits in a word (default {\tt 32}).
\Examples
\begin{verbatim}
      DIMENSION IXV(9)
      IWORD = 1676
C     1676 in base 2 is 11010001100
      CALL BITPOS(IWORD,9,IXV,NX)
\end{verbatim}
sets
\begin{verbatim}
      NX = 3, IXV(1) = 2, IXV(2) = 3, IXV(3) = 7.
\end{verbatim}
$\bullet$
