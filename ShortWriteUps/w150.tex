\Version{TRSPRT}                             \Routid{W150}
\Keywords{BEAM OPTICS ORDER SECOND TRANSPORT}
\Author{C.H. Moore, D.C. Carey}             \Library{PGMLIB}
\Submitter{C. Iselin}                       \Submitted{27.11.1984}
\Language{Fortran 4}                        %\Revised{}
\Cernhead{Transport, Second-Order Beam Optics}
{\tt TRSPRT} is a first- and second-order matrix multiplication program
for the design of magnetic beam transport systems. It has been in
use in various versions since 1963. The present version, written by
D.C. Carey at FNAL and extensively modified at CERN is described in
CERN 80-04, NAL 91 and SLAC 91. It includes both first- and
second-order fitting capabilities. A beam line is described as a
sequence of elements. Such elements may represent magnets or the
intervals separating them, but also specify calculations to be done, or
special conditions to be applied. The program works in six-dimensional
phase space $(x,x',y,y',l,dp/p)$; it is therefore also capable of
calculating coupling between planes.
\Structure
Complete {\tt PROGRAM}\\
User Entry Names: \Rdef{TRSPRT}\\
Files Referenced: {\tt INPUT}, {\tt OUTPUT}, \\
External References: \Rind{UBUNCH}{M409}, \Rind{ABEND}{Z035},
\Rind{DATIMH}{Z007}
\Usage
See {\bf Long Write-up}. {\tt TRSPRT} is accessed from
{\tt PGMLIB} as described in section 'Execution of Complete Programs,
{\tt PGMLIB}' in Chapter 1 of the Program Library Manual.
\Source
SLAC and FNAL, USA
\Refer
\begin{enumerate}
\item K.L. Brown, D.C. Carey, C. Iselin and F. Rothacker,
Designing Charged Particle Beam Transport Systems, CERN 80-04 (1980)
\end{enumerate}
A copy of Ref. 1 is available as {\bf Long Write-up}.
\\ $\bullet$
