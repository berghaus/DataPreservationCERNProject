% 5 jun 1992
\Version{BTMOVE}                    \Routid{M438}
\Keywords{BIT MOVE STRING}
\Author{H. Grote}                     \Library{KERNLIB}
\Submitter{}                          \Submitted{01.12.1980}
\Language{CDC: Fortran and Compass, IBM: Assembler}  %\Revised{}
\Cernhead{Move Bit String}
{\tt BTMOVE} moves a contiguous string of {\tt N} bits from any position
in memory to any other position.
Bits are numbered from left to right (most significant to least
significant within words) and may be across word boundaries.
\Structure
{\tt SUBROUTINE} subprogram \\
User Entry Names: \Rdef{BTMOVE} \\
External References: \Rind {UCOPY}{V301} (CDC only)
\Usage
\begin{verbatim}
    CALL BTMOVE(SOURCE,ISRC,TARGET,ITGT,N)
\end{verbatim}
moves the string of {\tt N} contiguous bits starting at position
{\tt ISRC} in word or array {\tt SOURCE} to position {\tt ITGT} in word
or array {\tt TARGET}. The other bits in {\tt TARGET} are not changed,
nor is {\tt SOURCE}.
\Notes
Source and target strings must not overlap in storage, else the results
may be unpredictable.
\Examples
IBM: \\
Move the highest bit (sign-bit) in word {\tt A} to the lowest
position of {\tt I} so that it can be treated as an integer:
\begin{verbatim}
    REAL A
    INTEGER*4 I
    I=0
    CALL BTMOVE(A,1,I,32,1)
\end{verbatim}
CDC: \\
Pack the five integers of array I5(5) into one word {\tt IPACK}, using
12 bits per packed integer:
\begin{verbatim}
    DIMENSION I5(5)
    IPOS=1
    DO 1 I = 1,5
    CALL BTMOVE(I5(I),49,IPACK,IPOS,12)
  1 IPOS=IPOS+12
\end{verbatim}
Move a string of 20 characters from positions 41-60 in array {\tt A} to
positions 7-26 in array {\tt B}:
\begin{verbatim}
    DIMENSION A(6),B(3)
    CALL BTMOVE(A,241,B,37,120)
\end{verbatim}
$\bullet$
