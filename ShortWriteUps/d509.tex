% 20 may 1992  mg
\Version {MINVAR}                    \Routid{D509}
\Keywords{FUNCTION MINIMUM}
\Author{T. Pomentale}                 \Library{KERNLIB}
\Submitter{}                          \Submitted{28.02.1972}
\Language{Fortran}                     \Revised{27.11 1984}
\Cernhead {Minimum of a Function of One Variable}
Subroutine {\tt MINVAR} computes, to an attempted specified accuracy,
the abscissa of a local minimum of a real-valued function $f(x)$
lying in a given interval $[A,B]$, together with the value of the
function at the minimum.
\Structure
{\tt SUBROUTINE} subprogram \\
User Entry Names: \Rdef{MINVAR}  \\
Internal Entry  Names: {\tt D509HI}\\
Files Referenced: Printer\\
External References: \Rind{KERMTR} (N001), \Rind{ABEND} (Z035),
user-supplied {\tt FUNCTION} subprogram
\Usage
\begin{verbatim}
    CALL MINVAR(X,Y,R,EPS,STEP,MAXF,A,B,F)
\end{verbatim}
\begin{DLtt}{123456}
\item[X]({\tt REAL}) On entry, {\tt X} must be equal to an estimate
of the abscissa of a minimum of $f$. On exit, {\tt X} is the computed
approximation to the abscissa of the minimum.
\item [Y]({\tt REAL}) On exit, $\mathtt{Y}=f(\mathtt{X})$.
\item [R]({\tt REAL}) On exit, the value of {\tt R} is such that
$|\mathtt{X} - x_0|<\mathtt{R}$, where $x_0$ is the exact abscissa of
a minimum of $f$.
\item [EPS]({\tt REAL}) On entry, {\tt EPS} must be equal to the accuracy
parameter (see Accuracy). Unchanged on exit.
\item [STEP]({\tt REAL}) On entry, {\tt STEP} must specify a suggested
initial search step. Unchanged on exit.
\item [MAXF] ({\tt INTEGER}) On entry, {\tt MAXF} must be equal to the
maximum permitted number of references to the function {\tt F}.
Unchanged on exit.
\item [A,B]({\tt REAL}) On entry, {\tt A} and {\tt B} must specify the
end-points of the search interval. Unchanged on exit.
\item [F]({\tt REAL}) Name of a user-supplied {\tt FUNCTION}
subprogram, declared {\tt EXTERNAL} in the calling program.
\end{DLtt}
The user-supplied {\tt FUNCTION} subprogram {\tt F} must be of the form
\begin{verbatim}
    FUNCTION F(X,I)
\end{verbatim}
where {\tt X} is {\tt REAL}, and must set $\mathtt{F}=f(\mathtt{X})$.
The {\tt INTEGER} argument {\tt I} is set by {\tt MINVAR} before each
reference to {\tt F} as follows:
\begin{DLtt}{1234}
\item[] $\mathtt{I = 0:}$ \quad  First reference.
\item[] $\mathtt{I = 1:}$ \quad  Subsequent reference.
\end{DLtt}
\Method
An iterative method is used. Each iteration consists
of finding the minimum of an interpolating parabola, followed
(if necessary) by a down-hill search.
\newpage
\Accuracy
If {\tt MINVAR} terminates without producing an error message,
the following relation holds:
$$ \frac{|\mathtt{X} - x_0|}{1+|\mathtt{X}|} \leq \mathtt{EPS} $$
\Errorh
Error {\tt D509.1:}  More than {\tt MAXF} references to the function
{\tt F} are required. A message is printed unless subroutine
{\tt KERSET} (N001) has been called.
\\ $\bullet$
