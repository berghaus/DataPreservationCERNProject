% 20 jan 95   ksk
\Version {RANLUX}                         \Routid{V115}
\Keywords{UNIFORM DISTRIBUTION NUMBER RANDOM GENERATOR GARANTEED
QUALITY}
\Author{F. James}             \Library{MATHLIB}
\Submitter{}          \Submitted{15.03.1994}
\Language{Fortran}                   %\Revised{}
\Cernhead{Uniform Random Numbers of Guaranteed Quality}
{\tt RANLUX} generates pseudorandom numbers uniformly distributed
in the interval (0,1), the end points excluded.
Each call produces an array of single-precision real numbers
of which 24 bits of mantissa are random.
The user can choose a {\bf luxury level} which guarantees the quality
required for his application.  The lowest luxury level (zero) gives
a fast generator which will fail some sophisticated tests of randomness;
The highest level (four) is about five times slower but guarantees
complete randomness. In all cases the period is greater than $10^{165}$.
Independent subsequences can be generated.  Entries are provided for
initialization and checkpointing.
\Structure
{\tt SUBROUTINE} Subprograms\\
User Entry Names: \Rdef{RANLUX}, \Rdef{RLUXGO}, \Rdef{RLUXAT},
\Rdef{RLUXIN}, \Rdef{RLUXUT}
\Usage
\begin{verbatim}
    CALL RANLUX(RVEC,LEN)
\end{verbatim}
returns a vector {\tt RVEC} of {\tt LEN} 32-bit random floating
point numbers in the interval (0,1), the end points excluded.
{\tt RVEC} is an array of type {\tt REAL} and
of length {\tt LEN} at least.
\par
{\bf Luxury levels:}
\par
For simplicity, five standard luxury levels may be chosen ($t$ is the
time factor relative to level zero; for the definition of $p$, see
{\bf References}). Ref. 1. explains the method, Ref. 2.
describes the Fortran implementation in more detail.
\begin{center}
\begin{tabular}{|c|c|c|l|}
\cline{1-3}
Level & $p$ & $t$ \\
\hline
0 & 24 & 1 & \parbox[t]{100mm}{
Equivalent to the original {\tt RCARRY} of Marsaglia and Zaman,
very long period, but fails many tests.} \\
1 & 48 & 1.5 & \parbox[t]{100mm}{
Considerable improvement in quality over level 0,
now passes the gap test, but still fails spectral test.} \\
2 & 97 & 2 & \parbox[t]{100mm}{
Passes all known tests, but theoretically still defective.} \\
3 & 223 & 3 & \parbox[t]{100mm}{
{\tt DEFAULT VALUE}. Any theoretically possible
correlations have very small chance of being observed.} \\
4 & 389 & 5 & \parbox[t]{100mm}{
Highest possible luxury, all 24 bits chaotic.} \\
\hline
\end{tabular}
\end{center}
As a rough indication of timing, {\tt RNDM} (V104) is about $t$=0.5,
{\tt RANMAR} (V113) $t$=1, and {\tt RANECU} (V114) $t$=2.
Concerning the quality scale, {\tt RNDM} is maybe good enough
for moving fish around on a screen saver (if you are not afraid of
getting some diagonal lines on your screen),
{\tt RANMAR} and {\tt RANECU} both have
quality which probably corresponds to a luxury level between 1 and 2, but
this is based only on empirical testing and true quality may be lower.
\par
No initialization is necessary if the user wants default values.
Otherwise the following are available:
\begin{verbatim}
    CALL RLUXGO(LUX,INT,K1,K2)
\end{verbatim}
When $\mathtt{K1=K2=0}$, this call initializes
the {\tt RANLUX} generator from one 32-bit integer {\tt INT} and sets
the Luxury Level. If {\tt LUX} is an integer between {\tt 0} and
{\tt 4}, it sets the luxury level as defined above.
If $\mathtt{LUX > 24}$, it is taken as the value of $p$,
which then can take on other values than those given in the table.
If $\mathtt{INT=0}$, default initialization is used and only the
luxury level is set by {\tt LUX}.  Otherwise, every possible
value of {\tt INT} gives rise to a valid, independent sequence
which will not overlap any sequence initialized with any
other value of {\tt INT}.  The integers {\tt K1} and {\tt K2} are used
for restarting the generator from a break point saved by {\tt RLUXAT}.
\begin{verbatim}
    CALL RLUXAT(LUX,INT,K1,K2)
\end{verbatim}
dumps the four integers which can be used
to restart the generator at this point by calling {\tt RLUXGO}.
{\tt RANLUX} will then skip over $\mathtt{K1}+10^9\mathtt{*K2}$ numbers
to reach the break point. A more efficient but less convenient
method for restarting is offered by {\tt RLUXIN} and {\tt RLUXUT}.
\begin{verbatim}
    CALL RLUXIN(IVEC)
\end{verbatim}
restarts the generator from vector {\tt IVEC} of
25 32-bit integers (see {\tt RLUXUT}).
{\tt IVEC} is an array of type {\tt INTEGER} and
of length {\tt 25} at least.
\begin{verbatim}
    CALL RLUXUT(IVEC)
\end{verbatim}
outputs the current values of the 25 32-bit
integer seeds, to be used for restarting.
\Refer
\begin{enumerate}
\item
M. L\"uscher, A portable high-quality random number generator for
lattice field theory simulations,
Computer Phys. Commun. {\bf 79} (1994), 100--110.
\item F. James, RANLUX: A Fortran implementation of the high-quality
pseudorandom number generator of L\"uscher,
Computer Phys. Commun. {\bf 79} (1994) 111--114.
\end{enumerate}
$\bullet$
 
 
