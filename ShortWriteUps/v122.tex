% ksk 22.02.94
\Version{CORSET}                       \Routid{V122}
\Keywords{CORRELATION GAUSSIAN DISTRIBUTION NUMBER RANDOM}
\Author{F. James}             \Library{MATHLIB}
\Submitter{}          \Submitted{15.03.1994}
\Language{Fortran}                   %\Revised{}
\Cernhead{Correlated Gaussian-distributed Random Numbers}
 
{\tt CORGEN} generates vectors of single-precision random numbers
in a Gaussian distribution of mean zero and covariance matrix {\tt V}.
The generator must first be set up by a call to {\tt CORSET} which
transforms the covariance matrix {\tt V} to an appropriate
{\it square root} matrix {\tt C} which is then used by {\tt CORGEN}.
{\tt CORGEN} uses the Gaussian generator {\tt RNORML} (V120) underneath,
which in turn uses the uniform generator {\tt RANMAR} (V113) underneath,
so initialization is performed as in V113, but beware that this
also initializes both {\tt RANMAR} and {\tt RNORML}!
The code is portable Fortran, but the results are not guaranteed
to be identical on all platforms as explained in {\tt RNORML} (V120).
\Structure
{\tt SUBROUTINE} Subprograms\\
User Entry Names: \Rdef{CORSET}, \Rdef{CORGEN}
\Usage
\begin{verbatim}
    DIMENSION V(n,n), C(n,n), X(n)
    CALL CORSET(V,C,n)
    DO ...
      CALL CORGEN(C,X,n)
\end{verbatim}
The call to {\tt CORSET} transforms covariance matrix {\tt V} to {\tt C}.
The call to {\tt CORGEN} uses {\tt C} to generate vector {\tt X} of
correlated Gaussian variables with covariance matrix {\tt V}.
\par
The limitation $\mathtt{n \le 100}$ is imposed by the dimension of an
intermediate storage vector in {\tt CORSET}.
\par
Note that {\tt CORSET} takes longer than {\tt CORGEN}
(for medium to large matrices).
If it is desired to generate numbers according to a few different
matrices, then each pair {\tt Vi}, {\tt Ci} must be separately
dimensioned and saved as long as it is needed.
\Method
The square root method seems to be an old one whose origins are not
known to the author (Ref. 1, p. 1182).
\Refer
\begin{enumerate}
\item F. James, Monte Carlo theory and practice,
Rep. Prog. Phys. {\bf 43} (1980) 1145--1189.
\end{enumerate}
$\bullet$
